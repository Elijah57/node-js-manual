\section{Net}\label{net}

\begin{quote}
Stability: 2 - Stable
\end{quote}

The \texttt{node:net} module provides an asynchronous network API for
creating stream-based TCP or \hyperref[ipc-support]{IPC} servers
(\hyperref[netcreateserveroptions-connectionlistener]{\texttt{net.createServer()}})
and clients
(\hyperref[netcreateconnection]{\texttt{net.createConnection()}}).

It can be accessed using:

\begin{Shaded}
\begin{Highlighting}[]
\KeywordTok{const}\NormalTok{ net }\OperatorTok{=} \PreprocessorTok{require}\NormalTok{(}\StringTok{\textquotesingle{}node:net\textquotesingle{}}\NormalTok{)}\OperatorTok{;}
\end{Highlighting}
\end{Shaded}

\subsection{IPC support}\label{ipc-support}

The \texttt{node:net} module supports IPC with named pipes on Windows,
and Unix domain sockets on other operating systems.

\subsubsection{Identifying paths for IPC
connections}\label{identifying-paths-for-ipc-connections}

\hyperref[netconnect]{\texttt{net.connect()}},
\hyperref[netcreateconnection]{\texttt{net.createConnection()}},
\hyperref[serverlisten]{\texttt{server.listen()}}, and
\hyperref[socketconnect]{\texttt{socket.connect()}} take a \texttt{path}
parameter to identify IPC endpoints.

On Unix, the local domain is also known as the Unix domain. The path is
a file system pathname. It gets truncated to an OS-dependent length of
\texttt{sizeof(sockaddr\_un.sun\_path)\ -\ 1}. Typical values are 107
bytes on Linux and 103 bytes on macOS. If a Node.js API abstraction
creates the Unix domain socket, it will unlink the Unix domain socket as
well. For example,
\hyperref[netcreateserveroptions-connectionlistener]{\texttt{net.createServer()}}
may create a Unix domain socket and
\hyperref[serverclosecallback]{\texttt{server.close()}} will unlink it.
But if a user creates the Unix domain socket outside of these
abstractions, the user will need to remove it. The same applies when a
Node.js API creates a Unix domain socket but the program then crashes.
In short, a Unix domain socket will be visible in the file system and
will persist until unlinked. On Linux, You can use Unix abstract socket
by adding \texttt{\textbackslash{}0} to the beginning of the path, such
as \texttt{\textbackslash{}0abstract}. The path to the Unix abstract
socket is not visible in the file system and it will disappear
automatically when all open references to the socket are closed.

On Windows, the local domain is implemented using a named pipe. The path
\emph{must} refer to an entry in
\texttt{\textbackslash{}\textbackslash{}?\textbackslash{}pipe\textbackslash{}}
or
\texttt{\textbackslash{}\textbackslash{}.\textbackslash{}pipe\textbackslash{}}.
Any characters are permitted, but the latter may do some processing of
pipe names, such as resolving \texttt{..} sequences. Despite how it
might look, the pipe namespace is flat. Pipes will \emph{not persist}.
They are removed when the last reference to them is closed. Unlike Unix
domain sockets, Windows will close and remove the pipe when the owning
process exits.

JavaScript string escaping requires paths to be specified with extra
backslash escaping such as:

\begin{Shaded}
\begin{Highlighting}[]
\NormalTok{net}\OperatorTok{.}\FunctionTok{createServer}\NormalTok{()}\OperatorTok{.}\FunctionTok{listen}\NormalTok{(}
\NormalTok{  path}\OperatorTok{.}\FunctionTok{join}\NormalTok{(}\StringTok{\textquotesingle{}}\SpecialCharTok{\textbackslash{}\textbackslash{}\textbackslash{}\textbackslash{}}\StringTok{?}\SpecialCharTok{\textbackslash{}\textbackslash{}}\StringTok{pipe\textquotesingle{}}\OperatorTok{,} \BuiltInTok{process}\OperatorTok{.}\FunctionTok{cwd}\NormalTok{()}\OperatorTok{,} \StringTok{\textquotesingle{}myctl\textquotesingle{}}\NormalTok{))}\OperatorTok{;}
\end{Highlighting}
\end{Shaded}

\subsection{\texorpdfstring{Class:
\texttt{net.BlockList}}{Class: net.BlockList}}\label{class-net.blocklist}

The \texttt{BlockList} object can be used with some network APIs to
specify rules for disabling inbound or outbound access to specific IP
addresses, IP ranges, or IP subnets.

\subsubsection{\texorpdfstring{\texttt{blockList.addAddress(address{[},\ type{]})}}{blockList.addAddress(address{[}, type{]})}}\label{blocklist.addaddressaddress-type}

\begin{itemize}
\tightlist
\item
  \texttt{address} \{string\textbar net.SocketAddress\} An IPv4 or IPv6
  address.
\item
  \texttt{type} \{string\} Either
  \texttt{\textquotesingle{}ipv4\textquotesingle{}} or
  \texttt{\textquotesingle{}ipv6\textquotesingle{}}. \textbf{Default:}
  \texttt{\textquotesingle{}ipv4\textquotesingle{}}.
\end{itemize}

Adds a rule to block the given IP address.

\subsubsection{\texorpdfstring{\texttt{blockList.addRange(start,\ end{[},\ type{]})}}{blockList.addRange(start, end{[}, type{]})}}\label{blocklist.addrangestart-end-type}

\begin{itemize}
\tightlist
\item
  \texttt{start} \{string\textbar net.SocketAddress\} The starting IPv4
  or IPv6 address in the range.
\item
  \texttt{end} \{string\textbar net.SocketAddress\} The ending IPv4 or
  IPv6 address in the range.
\item
  \texttt{type} \{string\} Either
  \texttt{\textquotesingle{}ipv4\textquotesingle{}} or
  \texttt{\textquotesingle{}ipv6\textquotesingle{}}. \textbf{Default:}
  \texttt{\textquotesingle{}ipv4\textquotesingle{}}.
\end{itemize}

Adds a rule to block a range of IP addresses from \texttt{start}
(inclusive) to \texttt{end} (inclusive).

\subsubsection{\texorpdfstring{\texttt{blockList.addSubnet(net,\ prefix{[},\ type{]})}}{blockList.addSubnet(net, prefix{[}, type{]})}}\label{blocklist.addsubnetnet-prefix-type}

\begin{itemize}
\tightlist
\item
  \texttt{net} \{string\textbar net.SocketAddress\} The network IPv4 or
  IPv6 address.
\item
  \texttt{prefix} \{number\} The number of CIDR prefix bits. For IPv4,
  this must be a value between \texttt{0} and \texttt{32}. For IPv6,
  this must be between \texttt{0} and \texttt{128}.
\item
  \texttt{type} \{string\} Either
  \texttt{\textquotesingle{}ipv4\textquotesingle{}} or
  \texttt{\textquotesingle{}ipv6\textquotesingle{}}. \textbf{Default:}
  \texttt{\textquotesingle{}ipv4\textquotesingle{}}.
\end{itemize}

Adds a rule to block a range of IP addresses specified as a subnet mask.

\subsubsection{\texorpdfstring{\texttt{blockList.check(address{[},\ type{]})}}{blockList.check(address{[}, type{]})}}\label{blocklist.checkaddress-type}

\begin{itemize}
\tightlist
\item
  \texttt{address} \{string\textbar net.SocketAddress\} The IP address
  to check
\item
  \texttt{type} \{string\} Either
  \texttt{\textquotesingle{}ipv4\textquotesingle{}} or
  \texttt{\textquotesingle{}ipv6\textquotesingle{}}. \textbf{Default:}
  \texttt{\textquotesingle{}ipv4\textquotesingle{}}.
\item
  Returns: \{boolean\}
\end{itemize}

Returns \texttt{true} if the given IP address matches any of the rules
added to the \texttt{BlockList}.

\begin{Shaded}
\begin{Highlighting}[]
\KeywordTok{const}\NormalTok{ blockList }\OperatorTok{=} \KeywordTok{new}\NormalTok{ net}\OperatorTok{.}\FunctionTok{BlockList}\NormalTok{()}\OperatorTok{;}
\NormalTok{blockList}\OperatorTok{.}\FunctionTok{addAddress}\NormalTok{(}\StringTok{\textquotesingle{}123.123.123.123\textquotesingle{}}\NormalTok{)}\OperatorTok{;}
\NormalTok{blockList}\OperatorTok{.}\FunctionTok{addRange}\NormalTok{(}\StringTok{\textquotesingle{}10.0.0.1\textquotesingle{}}\OperatorTok{,} \StringTok{\textquotesingle{}10.0.0.10\textquotesingle{}}\NormalTok{)}\OperatorTok{;}
\NormalTok{blockList}\OperatorTok{.}\FunctionTok{addSubnet}\NormalTok{(}\StringTok{\textquotesingle{}8592:757c:efae:4e45::\textquotesingle{}}\OperatorTok{,} \DecValTok{64}\OperatorTok{,} \StringTok{\textquotesingle{}ipv6\textquotesingle{}}\NormalTok{)}\OperatorTok{;}

\BuiltInTok{console}\OperatorTok{.}\FunctionTok{log}\NormalTok{(blockList}\OperatorTok{.}\FunctionTok{check}\NormalTok{(}\StringTok{\textquotesingle{}123.123.123.123\textquotesingle{}}\NormalTok{))}\OperatorTok{;}  \CommentTok{// Prints: true}
\BuiltInTok{console}\OperatorTok{.}\FunctionTok{log}\NormalTok{(blockList}\OperatorTok{.}\FunctionTok{check}\NormalTok{(}\StringTok{\textquotesingle{}10.0.0.3\textquotesingle{}}\NormalTok{))}\OperatorTok{;}  \CommentTok{// Prints: true}
\BuiltInTok{console}\OperatorTok{.}\FunctionTok{log}\NormalTok{(blockList}\OperatorTok{.}\FunctionTok{check}\NormalTok{(}\StringTok{\textquotesingle{}222.111.111.222\textquotesingle{}}\NormalTok{))}\OperatorTok{;}  \CommentTok{// Prints: false}

\CommentTok{// IPv6 notation for IPv4 addresses works:}
\BuiltInTok{console}\OperatorTok{.}\FunctionTok{log}\NormalTok{(blockList}\OperatorTok{.}\FunctionTok{check}\NormalTok{(}\StringTok{\textquotesingle{}::ffff:7b7b:7b7b\textquotesingle{}}\OperatorTok{,} \StringTok{\textquotesingle{}ipv6\textquotesingle{}}\NormalTok{))}\OperatorTok{;} \CommentTok{// Prints: true}
\BuiltInTok{console}\OperatorTok{.}\FunctionTok{log}\NormalTok{(blockList}\OperatorTok{.}\FunctionTok{check}\NormalTok{(}\StringTok{\textquotesingle{}::ffff:123.123.123.123\textquotesingle{}}\OperatorTok{,} \StringTok{\textquotesingle{}ipv6\textquotesingle{}}\NormalTok{))}\OperatorTok{;} \CommentTok{// Prints: true}
\end{Highlighting}
\end{Shaded}

\subsubsection{\texorpdfstring{\texttt{blockList.rules}}{blockList.rules}}\label{blocklist.rules}

\begin{itemize}
\tightlist
\item
  Type: \{string{[}{]}\}
\end{itemize}

The list of rules added to the blocklist.

\subsection{\texorpdfstring{Class:
\texttt{net.SocketAddress}}{Class: net.SocketAddress}}\label{class-net.socketaddress}

\subsubsection{\texorpdfstring{\texttt{new\ net.SocketAddress({[}options{]})}}{new net.SocketAddress({[}options{]})}}\label{new-net.socketaddressoptions}

\begin{itemize}
\tightlist
\item
  \texttt{options} \{Object\}

  \begin{itemize}
  \tightlist
  \item
    \texttt{address} \{string\} The network address as either an IPv4 or
    IPv6 string. \textbf{Default}:
    \texttt{\textquotesingle{}127.0.0.1\textquotesingle{}} if
    \texttt{family} is
    \texttt{\textquotesingle{}ipv4\textquotesingle{}};
    \texttt{\textquotesingle{}::\textquotesingle{}} if \texttt{family}
    is \texttt{\textquotesingle{}ipv6\textquotesingle{}}.
  \item
    \texttt{family} \{string\} One of either
    \texttt{\textquotesingle{}ipv4\textquotesingle{}} or
    \texttt{\textquotesingle{}ipv6\textquotesingle{}}. \textbf{Default}:
    \texttt{\textquotesingle{}ipv4\textquotesingle{}}.
  \item
    \texttt{flowlabel} \{number\} An IPv6 flow-label used only if
    \texttt{family} is
    \texttt{\textquotesingle{}ipv6\textquotesingle{}}.
  \item
    \texttt{port} \{number\} An IP port.
  \end{itemize}
\end{itemize}

\subsubsection{\texorpdfstring{\texttt{socketaddress.address}}{socketaddress.address}}\label{socketaddress.address}

\begin{itemize}
\tightlist
\item
  Type \{string\}
\end{itemize}

\subsubsection{\texorpdfstring{\texttt{socketaddress.family}}{socketaddress.family}}\label{socketaddress.family}

\begin{itemize}
\tightlist
\item
  Type \{string\} Either
  \texttt{\textquotesingle{}ipv4\textquotesingle{}} or
  \texttt{\textquotesingle{}ipv6\textquotesingle{}}.
\end{itemize}

\subsubsection{\texorpdfstring{\texttt{socketaddress.flowlabel}}{socketaddress.flowlabel}}\label{socketaddress.flowlabel}

\begin{itemize}
\tightlist
\item
  Type \{number\}
\end{itemize}

\subsubsection{\texorpdfstring{\texttt{socketaddress.port}}{socketaddress.port}}\label{socketaddress.port}

\begin{itemize}
\tightlist
\item
  Type \{number\}
\end{itemize}

\subsection{\texorpdfstring{Class:
\texttt{net.Server}}{Class: net.Server}}\label{class-net.server}

\begin{itemize}
\tightlist
\item
  Extends: \{EventEmitter\}
\end{itemize}

This class is used to create a TCP or \hyperref[ipc-support]{IPC}
server.

\subsubsection{\texorpdfstring{\texttt{new\ net.Server({[}options{]}{[},\ connectionListener{]})}}{new net.Server({[}options{]}{[}, connectionListener{]})}}\label{new-net.serveroptions-connectionlistener}

\begin{itemize}
\tightlist
\item
  \texttt{options} \{Object\} See
  \hyperref[netcreateserveroptions-connectionlistener]{\texttt{net.createServer({[}options{]}{[},\ connectionListener{]})}}.
\item
  \texttt{connectionListener} \{Function\} Automatically set as a
  listener for the
  \hyperref[event-connection]{\texttt{\textquotesingle{}connection\textquotesingle{}}}
  event.
\item
  Returns: \{net.Server\}
\end{itemize}

\texttt{net.Server} is an
\href{events.md\#class-eventemitter}{\texttt{EventEmitter}} with the
following events:

\subsubsection{\texorpdfstring{Event:
\texttt{\textquotesingle{}close\textquotesingle{}}}{Event: \textquotesingle close\textquotesingle{}}}\label{event-close}

Emitted when the server closes. If connections exist, this event is not
emitted until all connections are ended.

\subsubsection{\texorpdfstring{Event:
\texttt{\textquotesingle{}connection\textquotesingle{}}}{Event: \textquotesingle connection\textquotesingle{}}}\label{event-connection}

\begin{itemize}
\tightlist
\item
  \{net.Socket\} The connection object
\end{itemize}

Emitted when a new connection is made. \texttt{socket} is an instance of
\texttt{net.Socket}.

\subsubsection{\texorpdfstring{Event:
\texttt{\textquotesingle{}error\textquotesingle{}}}{Event: \textquotesingle error\textquotesingle{}}}\label{event-error}

\begin{itemize}
\tightlist
\item
  \{Error\}
\end{itemize}

Emitted when an error occurs. Unlike
\hyperref[class-netsocket]{\texttt{net.Socket}}, the
\hyperref[event-close]{\texttt{\textquotesingle{}close\textquotesingle{}}}
event will \textbf{not} be emitted directly following this event unless
\hyperref[serverclosecallback]{\texttt{server.close()}} is manually
called. See the example in discussion of
\hyperref[serverlisten]{\texttt{server.listen()}}.

\subsubsection{\texorpdfstring{Event:
\texttt{\textquotesingle{}listening\textquotesingle{}}}{Event: \textquotesingle listening\textquotesingle{}}}\label{event-listening}

Emitted when the server has been bound after calling
\hyperref[serverlisten]{\texttt{server.listen()}}.

\subsubsection{\texorpdfstring{Event:
\texttt{\textquotesingle{}drop\textquotesingle{}}}{Event: \textquotesingle drop\textquotesingle{}}}\label{event-drop}

When the number of connections reaches the threshold of
\texttt{server.maxConnections}, the server will drop new connections and
emit \texttt{\textquotesingle{}drop\textquotesingle{}} event instead. If
it is a TCP server, the argument is as follows, otherwise the argument
is \texttt{undefined}.

\begin{itemize}
\tightlist
\item
  \texttt{data} \{Object\} The argument passed to event listener.

  \begin{itemize}
  \tightlist
  \item
    \texttt{localAddress} \{string\} Local address.
  \item
    \texttt{localPort} \{number\} Local port.
  \item
    \texttt{localFamily} \{string\} Local family.
  \item
    \texttt{remoteAddress} \{string\} Remote address.
  \item
    \texttt{remotePort} \{number\} Remote port.
  \item
    \texttt{remoteFamily} \{string\} Remote IP family.
    \texttt{\textquotesingle{}IPv4\textquotesingle{}} or
    \texttt{\textquotesingle{}IPv6\textquotesingle{}}.
  \end{itemize}
\end{itemize}

\subsubsection{\texorpdfstring{\texttt{server.address()}}{server.address()}}\label{server.address}

\begin{itemize}
\tightlist
\item
  Returns: \{Object\textbar string\textbar null\}
\end{itemize}

Returns the bound \texttt{address}, the address \texttt{family} name,
and \texttt{port} of the server as reported by the operating system if
listening on an IP socket (useful to find which port was assigned when
getting an OS-assigned address):
\texttt{\{\ port:\ 12346,\ family:\ \textquotesingle{}IPv4\textquotesingle{},\ address:\ \textquotesingle{}127.0.0.1\textquotesingle{}\ \}}.

For a server listening on a pipe or Unix domain socket, the name is
returned as a string.

\begin{Shaded}
\begin{Highlighting}[]
\KeywordTok{const}\NormalTok{ server }\OperatorTok{=}\NormalTok{ net}\OperatorTok{.}\FunctionTok{createServer}\NormalTok{((socket) }\KeywordTok{=\textgreater{}}\NormalTok{ \{}
\NormalTok{  socket}\OperatorTok{.}\FunctionTok{end}\NormalTok{(}\StringTok{\textquotesingle{}goodbye}\SpecialCharTok{\textbackslash{}n}\StringTok{\textquotesingle{}}\NormalTok{)}\OperatorTok{;}
\NormalTok{\})}\OperatorTok{.}\FunctionTok{on}\NormalTok{(}\StringTok{\textquotesingle{}error\textquotesingle{}}\OperatorTok{,}\NormalTok{ (err) }\KeywordTok{=\textgreater{}}\NormalTok{ \{}
  \CommentTok{// Handle errors here.}
  \ControlFlowTok{throw}\NormalTok{ err}\OperatorTok{;}
\NormalTok{\})}\OperatorTok{;}

\CommentTok{// Grab an arbitrary unused port.}
\NormalTok{server}\OperatorTok{.}\FunctionTok{listen}\NormalTok{(() }\KeywordTok{=\textgreater{}}\NormalTok{ \{}
  \BuiltInTok{console}\OperatorTok{.}\FunctionTok{log}\NormalTok{(}\StringTok{\textquotesingle{}opened server on\textquotesingle{}}\OperatorTok{,}\NormalTok{ server}\OperatorTok{.}\FunctionTok{address}\NormalTok{())}\OperatorTok{;}
\NormalTok{\})}\OperatorTok{;}
\end{Highlighting}
\end{Shaded}

\texttt{server.address()} returns \texttt{null} before the
\texttt{\textquotesingle{}listening\textquotesingle{}} event has been
emitted or after calling \texttt{server.close()}.

\subsubsection{\texorpdfstring{\texttt{server.close({[}callback{]})}}{server.close({[}callback{]})}}\label{server.closecallback}

\begin{itemize}
\tightlist
\item
  \texttt{callback} \{Function\} Called when the server is closed.
\item
  Returns: \{net.Server\}
\end{itemize}

Stops the server from accepting new connections and keeps existing
connections. This function is asynchronous, the server is finally closed
when all connections are ended and the server emits a
\hyperref[event-close]{\texttt{\textquotesingle{}close\textquotesingle{}}}
event. The optional \texttt{callback} will be called once the
\texttt{\textquotesingle{}close\textquotesingle{}} event occurs. Unlike
that event, it will be called with an \texttt{Error} as its only
argument if the server was not open when it was closed.

\subsubsection{\texorpdfstring{\texttt{server{[}Symbol.asyncDispose{]}()}}{server{[}Symbol.asyncDispose{]}()}}\label{serversymbol.asyncdispose}

\begin{quote}
Stability: 1 - Experimental
\end{quote}

Calls \hyperref[serverclosecallback]{\texttt{server.close()}} and
returns a promise that fulfills when the server has closed.

\subsubsection{\texorpdfstring{\texttt{server.getConnections(callback)}}{server.getConnections(callback)}}\label{server.getconnectionscallback}

\begin{itemize}
\tightlist
\item
  \texttt{callback} \{Function\}
\item
  Returns: \{net.Server\}
\end{itemize}

Asynchronously get the number of concurrent connections on the server.
Works when sockets were sent to forks.

Callback should take two arguments \texttt{err} and \texttt{count}.

\subsubsection{\texorpdfstring{\texttt{server.listen()}}{server.listen()}}\label{server.listen}

Start a server listening for connections. A \texttt{net.Server} can be a
TCP or an \hyperref[ipc-support]{IPC} server depending on what it
listens to.

Possible signatures:

\begin{itemize}
\tightlist
\item
  \hyperref[serverlistenhandle-backlog-callback]{\texttt{server.listen(handle{[},\ backlog{]}{[},\ callback{]})}}
\item
  \hyperref[serverlistenoptions-callback]{\texttt{server.listen(options{[},\ callback{]})}}
\item
  \hyperref[serverlistenpath-backlog-callback]{\texttt{server.listen(path{[},\ backlog{]}{[},\ callback{]})}}
  for \hyperref[ipc-support]{IPC} servers
\item
  \hyperref[serverlistenport-host-backlog-callback]{\texttt{server.listen({[}port{[},\ host{[},\ backlog{]}{]}{]}{[},\ callback{]})}}
  for TCP servers
\end{itemize}

This function is asynchronous. When the server starts listening, the
\hyperref[event-listening]{\texttt{\textquotesingle{}listening\textquotesingle{}}}
event will be emitted. The last parameter \texttt{callback} will be
added as a listener for the
\hyperref[event-listening]{\texttt{\textquotesingle{}listening\textquotesingle{}}}
event.

All \texttt{listen()} methods can take a \texttt{backlog} parameter to
specify the maximum length of the queue of pending connections. The
actual length will be determined by the OS through sysctl settings such
as \texttt{tcp\_max\_syn\_backlog} and \texttt{somaxconn} on Linux. The
default value of this parameter is 511 (not 512).

All \hyperref[class-netsocket]{\texttt{net.Socket}} are set to
\texttt{SO\_REUSEADDR} (see
\href{https://man7.org/linux/man-pages/man7/socket.7.html}{\texttt{socket(7)}}
for details).

The \texttt{server.listen()} method can be called again if and only if
there was an error during the first \texttt{server.listen()} call or
\texttt{server.close()} has been called. Otherwise, an
\texttt{ERR\_SERVER\_ALREADY\_LISTEN} error will be thrown.

One of the most common errors raised when listening is
\texttt{EADDRINUSE}. This happens when another server is already
listening on the requested \texttt{port}/\texttt{path}/\texttt{handle}.
One way to handle this would be to retry after a certain amount of time:

\begin{Shaded}
\begin{Highlighting}[]
\NormalTok{server}\OperatorTok{.}\FunctionTok{on}\NormalTok{(}\StringTok{\textquotesingle{}error\textquotesingle{}}\OperatorTok{,}\NormalTok{ (e) }\KeywordTok{=\textgreater{}}\NormalTok{ \{}
  \ControlFlowTok{if}\NormalTok{ (e}\OperatorTok{.}\AttributeTok{code} \OperatorTok{===} \StringTok{\textquotesingle{}EADDRINUSE\textquotesingle{}}\NormalTok{) \{}
    \BuiltInTok{console}\OperatorTok{.}\FunctionTok{error}\NormalTok{(}\StringTok{\textquotesingle{}Address in use, retrying...\textquotesingle{}}\NormalTok{)}\OperatorTok{;}
    \PreprocessorTok{setTimeout}\NormalTok{(() }\KeywordTok{=\textgreater{}}\NormalTok{ \{}
\NormalTok{      server}\OperatorTok{.}\FunctionTok{close}\NormalTok{()}\OperatorTok{;}
\NormalTok{      server}\OperatorTok{.}\FunctionTok{listen}\NormalTok{(PORT}\OperatorTok{,}\NormalTok{ HOST)}\OperatorTok{;}
\NormalTok{    \}}\OperatorTok{,} \DecValTok{1000}\NormalTok{)}\OperatorTok{;}
\NormalTok{  \}}
\NormalTok{\})}\OperatorTok{;}
\end{Highlighting}
\end{Shaded}

\paragraph{\texorpdfstring{\texttt{server.listen(handle{[},\ backlog{]}{[},\ callback{]})}}{server.listen(handle{[}, backlog{]}{[}, callback{]})}}\label{server.listenhandle-backlog-callback}

\begin{itemize}
\tightlist
\item
  \texttt{handle} \{Object\}
\item
  \texttt{backlog} \{number\} Common parameter of
  \hyperref[serverlisten]{\texttt{server.listen()}} functions
\item
  \texttt{callback} \{Function\}
\item
  Returns: \{net.Server\}
\end{itemize}

Start a server listening for connections on a given \texttt{handle} that
has already been bound to a port, a Unix domain socket, or a Windows
named pipe.

The \texttt{handle} object can be either a server, a socket (anything
with an underlying \texttt{\_handle} member), or an object with an
\texttt{fd} member that is a valid file descriptor.

Listening on a file descriptor is not supported on Windows.

\paragraph{\texorpdfstring{\texttt{server.listen(options{[},\ callback{]})}}{server.listen(options{[}, callback{]})}}\label{server.listenoptions-callback}

\begin{itemize}
\tightlist
\item
  \texttt{options} \{Object\} Required. Supports the following
  properties:

  \begin{itemize}
  \tightlist
  \item
    \texttt{port} \{number\}
  \item
    \texttt{host} \{string\}
  \item
    \texttt{path} \{string\} Will be ignored if \texttt{port} is
    specified. See
    \hyperref[identifying-paths-for-ipc-connections]{Identifying paths
    for IPC connections}.
  \item
    \texttt{backlog} \{number\} Common parameter of
    \hyperref[serverlisten]{\texttt{server.listen()}} functions.
  \item
    \texttt{exclusive} \{boolean\} \textbf{Default:} \texttt{false}
  \item
    \texttt{readableAll} \{boolean\} For IPC servers makes the pipe
    readable for all users. \textbf{Default:} \texttt{false}.
  \item
    \texttt{writableAll} \{boolean\} For IPC servers makes the pipe
    writable for all users. \textbf{Default:} \texttt{false}.
  \item
    \texttt{ipv6Only} \{boolean\} For TCP servers, setting
    \texttt{ipv6Only} to \texttt{true} will disable dual-stack support,
    i.e., binding to host \texttt{::} won't make \texttt{0.0.0.0} be
    bound. \textbf{Default:} \texttt{false}.
  \item
    \texttt{signal} \{AbortSignal\} An AbortSignal that may be used to
    close a listening server.
  \end{itemize}
\item
  \texttt{callback} \{Function\} functions.
\item
  Returns: \{net.Server\}
\end{itemize}

If \texttt{port} is specified, it behaves the same as
\hyperref[serverlistenport-host-backlog-callback]{\texttt{server.listen({[}port{[},\ host{[},\ backlog{]}{]}{]}{[},\ callback{]})}}.
Otherwise, if \texttt{path} is specified, it behaves the same as
\hyperref[serverlistenpath-backlog-callback]{\texttt{server.listen(path{[},\ backlog{]}{[},\ callback{]})}}.
If none of them is specified, an error will be thrown.

If \texttt{exclusive} is \texttt{false} (default), then cluster workers
will use the same underlying handle, allowing connection handling duties
to be shared. When \texttt{exclusive} is \texttt{true}, the handle is
not shared, and attempted port sharing results in an error. An example
which listens on an exclusive port is shown below.

\begin{Shaded}
\begin{Highlighting}[]
\NormalTok{server}\OperatorTok{.}\FunctionTok{listen}\NormalTok{(\{}
  \DataTypeTok{host}\OperatorTok{:} \StringTok{\textquotesingle{}localhost\textquotesingle{}}\OperatorTok{,}
  \DataTypeTok{port}\OperatorTok{:} \DecValTok{80}\OperatorTok{,}
  \DataTypeTok{exclusive}\OperatorTok{:} \KeywordTok{true}\OperatorTok{,}
\NormalTok{\})}\OperatorTok{;}
\end{Highlighting}
\end{Shaded}

When \texttt{exclusive} is \texttt{true} and the underlying handle is
shared, it is possible that several workers query a handle with
different backlogs. In this case, the first \texttt{backlog} passed to
the master process will be used.

Starting an IPC server as root may cause the server path to be
inaccessible for unprivileged users. Using \texttt{readableAll} and
\texttt{writableAll} will make the server accessible for all users.

If the \texttt{signal} option is enabled, calling \texttt{.abort()} on
the corresponding \texttt{AbortController} is similar to calling
\texttt{.close()} on the server:

\begin{Shaded}
\begin{Highlighting}[]
\KeywordTok{const}\NormalTok{ controller }\OperatorTok{=} \KeywordTok{new} \FunctionTok{AbortController}\NormalTok{()}\OperatorTok{;}
\NormalTok{server}\OperatorTok{.}\FunctionTok{listen}\NormalTok{(\{}
  \DataTypeTok{host}\OperatorTok{:} \StringTok{\textquotesingle{}localhost\textquotesingle{}}\OperatorTok{,}
  \DataTypeTok{port}\OperatorTok{:} \DecValTok{80}\OperatorTok{,}
  \DataTypeTok{signal}\OperatorTok{:}\NormalTok{ controller}\OperatorTok{.}\AttributeTok{signal}\OperatorTok{,}
\NormalTok{\})}\OperatorTok{;}
\CommentTok{// Later, when you want to close the server.}
\NormalTok{controller}\OperatorTok{.}\FunctionTok{abort}\NormalTok{()}\OperatorTok{;}
\end{Highlighting}
\end{Shaded}

\paragraph{\texorpdfstring{\texttt{server.listen(path{[},\ backlog{]}{[},\ callback{]})}}{server.listen(path{[}, backlog{]}{[}, callback{]})}}\label{server.listenpath-backlog-callback}

\begin{itemize}
\tightlist
\item
  \texttt{path} \{string\} Path the server should listen to. See
  \hyperref[identifying-paths-for-ipc-connections]{Identifying paths for
  IPC connections}.
\item
  \texttt{backlog} \{number\} Common parameter of
  \hyperref[serverlisten]{\texttt{server.listen()}} functions.
\item
  \texttt{callback} \{Function\}.
\item
  Returns: \{net.Server\}
\end{itemize}

Start an \hyperref[ipc-support]{IPC} server listening for connections on
the given \texttt{path}.

\paragraph{\texorpdfstring{\texttt{server.listen({[}port{[},\ host{[},\ backlog{]}{]}{]}{[},\ callback{]})}}{server.listen({[}port{[}, host{[}, backlog{]}{]}{]}{[}, callback{]})}}\label{server.listenport-host-backlog-callback}

\begin{itemize}
\tightlist
\item
  \texttt{port} \{number\}
\item
  \texttt{host} \{string\}
\item
  \texttt{backlog} \{number\} Common parameter of
  \hyperref[serverlisten]{\texttt{server.listen()}} functions.
\item
  \texttt{callback} \{Function\}.
\item
  Returns: \{net.Server\}
\end{itemize}

Start a TCP server listening for connections on the given \texttt{port}
and \texttt{host}.

If \texttt{port} is omitted or is 0, the operating system will assign an
arbitrary unused port, which can be retrieved by using
\texttt{server.address().port} after the
\hyperref[event-listening]{\texttt{\textquotesingle{}listening\textquotesingle{}}}
event has been emitted.

If \texttt{host} is omitted, the server will accept connections on the
\href{https://en.wikipedia.org/wiki/IPv6_address\#Unspecified_address}{unspecified
IPv6 address} (\texttt{::}) when IPv6 is available, or the
\href{https://en.wikipedia.org/wiki/0.0.0.0}{unspecified IPv4 address}
(\texttt{0.0.0.0}) otherwise.

In most operating systems, listening to the
\href{https://en.wikipedia.org/wiki/IPv6_address\#Unspecified_address}{unspecified
IPv6 address} (\texttt{::}) may cause the \texttt{net.Server} to also
listen on the \href{https://en.wikipedia.org/wiki/0.0.0.0}{unspecified
IPv4 address} (\texttt{0.0.0.0}).

\subsubsection{\texorpdfstring{\texttt{server.listening}}{server.listening}}\label{server.listening}

\begin{itemize}
\tightlist
\item
  \{boolean\} Indicates whether or not the server is listening for
  connections.
\end{itemize}

\subsubsection{\texorpdfstring{\texttt{server.maxConnections}}{server.maxConnections}}\label{server.maxconnections}

\begin{itemize}
\tightlist
\item
  \{integer\}
\end{itemize}

Set this property to reject connections when the server's connection
count gets high.

It is not recommended to use this option once a socket has been sent to
a child with
\href{child_process.md\#child_processforkmodulepath-args-options}{\texttt{child\_process.fork()}}.

\subsubsection{\texorpdfstring{\texttt{server.ref()}}{server.ref()}}\label{server.ref}

\begin{itemize}
\tightlist
\item
  Returns: \{net.Server\}
\end{itemize}

Opposite of \texttt{unref()}, calling \texttt{ref()} on a previously
\texttt{unref}ed server will \emph{not} let the program exit if it's the
only server left (the default behavior). If the server is \texttt{ref}ed
calling \texttt{ref()} again will have no effect.

\subsubsection{\texorpdfstring{\texttt{server.unref()}}{server.unref()}}\label{server.unref}

\begin{itemize}
\tightlist
\item
  Returns: \{net.Server\}
\end{itemize}

Calling \texttt{unref()} on a server will allow the program to exit if
this is the only active server in the event system. If the server is
already \texttt{unref}ed calling \texttt{unref()} again will have no
effect.

\subsection{\texorpdfstring{Class:
\texttt{net.Socket}}{Class: net.Socket}}\label{class-net.socket}

\begin{itemize}
\tightlist
\item
  Extends: \{stream.Duplex\}
\end{itemize}

This class is an abstraction of a TCP socket or a streaming
\hyperref[ipc-support]{IPC} endpoint (uses named pipes on Windows, and
Unix domain sockets otherwise). It is also an
\href{events.md\#class-eventemitter}{\texttt{EventEmitter}}.

A \texttt{net.Socket} can be created by the user and used directly to
interact with a server. For example, it is returned by
\hyperref[netcreateconnection]{\texttt{net.createConnection()}}, so the
user can use it to talk to the server.

It can also be created by Node.js and passed to the user when a
connection is received. For example, it is passed to the listeners of a
\hyperref[event-connection]{\texttt{\textquotesingle{}connection\textquotesingle{}}}
event emitted on a \hyperref[class-netserver]{\texttt{net.Server}}, so
the user can use it to interact with the client.

\subsubsection{\texorpdfstring{\texttt{new\ net.Socket({[}options{]})}}{new net.Socket({[}options{]})}}\label{new-net.socketoptions}

\begin{itemize}
\tightlist
\item
  \texttt{options} \{Object\} Available options are:

  \begin{itemize}
  \tightlist
  \item
    \texttt{fd} \{number\} If specified, wrap around an existing socket
    with the given file descriptor, otherwise a new socket will be
    created.
  \item
    \texttt{allowHalfOpen} \{boolean\} If set to \texttt{false}, then
    the socket will automatically end the writable side when the
    readable side ends. See
    \hyperref[netcreateserveroptions-connectionlistener]{\texttt{net.createServer()}}
    and the
    \hyperref[event-end]{\texttt{\textquotesingle{}end\textquotesingle{}}}
    event for details. \textbf{Default:} \texttt{false}.
  \item
    \texttt{readable} \{boolean\} Allow reads on the socket when an
    \texttt{fd} is passed, otherwise ignored. \textbf{Default:}
    \texttt{false}.
  \item
    \texttt{writable} \{boolean\} Allow writes on the socket when an
    \texttt{fd} is passed, otherwise ignored. \textbf{Default:}
    \texttt{false}.
  \item
    \texttt{signal} \{AbortSignal\} An Abort signal that may be used to
    destroy the socket.
  \end{itemize}
\item
  Returns: \{net.Socket\}
\end{itemize}

Creates a new socket object.

The newly created socket can be either a TCP socket or a streaming
\hyperref[ipc-support]{IPC} endpoint, depending on what it
\hyperref[socketconnect]{\texttt{connect()}} to.

\subsubsection{\texorpdfstring{Event:
\texttt{\textquotesingle{}close\textquotesingle{}}}{Event: \textquotesingle close\textquotesingle{}}}\label{event-close-1}

\begin{itemize}
\tightlist
\item
  \texttt{hadError} \{boolean\} \texttt{true} if the socket had a
  transmission error.
\end{itemize}

Emitted once the socket is fully closed. The argument \texttt{hadError}
is a boolean which says if the socket was closed due to a transmission
error.

\subsubsection{\texorpdfstring{Event:
\texttt{\textquotesingle{}connect\textquotesingle{}}}{Event: \textquotesingle connect\textquotesingle{}}}\label{event-connect}

Emitted when a socket connection is successfully established. See
\hyperref[netcreateconnection]{\texttt{net.createConnection()}}.

\subsubsection{\texorpdfstring{Event:
\texttt{\textquotesingle{}data\textquotesingle{}}}{Event: \textquotesingle data\textquotesingle{}}}\label{event-data}

\begin{itemize}
\tightlist
\item
  \{Buffer\textbar string\}
\end{itemize}

Emitted when data is received. The argument \texttt{data} will be a
\texttt{Buffer} or \texttt{String}. Encoding of data is set by
\hyperref[socketsetencodingencoding]{\texttt{socket.setEncoding()}}.

The data will be lost if there is no listener when a \texttt{Socket}
emits a \texttt{\textquotesingle{}data\textquotesingle{}} event.

\subsubsection{\texorpdfstring{Event:
\texttt{\textquotesingle{}drain\textquotesingle{}}}{Event: \textquotesingle drain\textquotesingle{}}}\label{event-drain}

Emitted when the write buffer becomes empty. Can be used to throttle
uploads.

See also: the return values of \texttt{socket.write()}.

\subsubsection{\texorpdfstring{Event:
\texttt{\textquotesingle{}end\textquotesingle{}}}{Event: \textquotesingle end\textquotesingle{}}}\label{event-end}

Emitted when the other end of the socket signals the end of
transmission, thus ending the readable side of the socket.

By default (\texttt{allowHalfOpen} is \texttt{false}) the socket will
send an end of transmission packet back and destroy its file descriptor
once it has written out its pending write queue. However, if
\texttt{allowHalfOpen} is set to \texttt{true}, the socket will not
automatically \hyperref[socketenddata-encoding-callback]{\texttt{end()}}
its writable side, allowing the user to write arbitrary amounts of data.
The user must call
\hyperref[socketenddata-encoding-callback]{\texttt{end()}} explicitly to
close the connection (i.e.~sending a FIN packet back).

\subsubsection{\texorpdfstring{Event:
\texttt{\textquotesingle{}error\textquotesingle{}}}{Event: \textquotesingle error\textquotesingle{}}}\label{event-error-1}

\begin{itemize}
\tightlist
\item
  \{Error\}
\end{itemize}

Emitted when an error occurs. The
\texttt{\textquotesingle{}close\textquotesingle{}} event will be called
directly following this event.

\subsubsection{\texorpdfstring{Event:
\texttt{\textquotesingle{}lookup\textquotesingle{}}}{Event: \textquotesingle lookup\textquotesingle{}}}\label{event-lookup}

Emitted after resolving the host name but before connecting. Not
applicable to Unix sockets.

\begin{itemize}
\tightlist
\item
  \texttt{err} \{Error\textbar null\} The error object. See
  \href{dns.md\#dnslookuphostname-options-callback}{\texttt{dns.lookup()}}.
\item
  \texttt{address} \{string\} The IP address.
\item
  \texttt{family} \{number\textbar null\} The address type. See
  \href{dns.md\#dnslookuphostname-options-callback}{\texttt{dns.lookup()}}.
\item
  \texttt{host} \{string\} The host name.
\end{itemize}

\subsubsection{\texorpdfstring{Event:
\texttt{\textquotesingle{}ready\textquotesingle{}}}{Event: \textquotesingle ready\textquotesingle{}}}\label{event-ready}

Emitted when a socket is ready to be used.

Triggered immediately after
\texttt{\textquotesingle{}connect\textquotesingle{}}.

\subsubsection{\texorpdfstring{Event:
\texttt{\textquotesingle{}timeout\textquotesingle{}}}{Event: \textquotesingle timeout\textquotesingle{}}}\label{event-timeout}

Emitted if the socket times out from inactivity. This is only to notify
that the socket has been idle. The user must manually close the
connection.

See also:
\hyperref[socketsettimeouttimeout-callback]{\texttt{socket.setTimeout()}}.

\subsubsection{\texorpdfstring{\texttt{socket.address()}}{socket.address()}}\label{socket.address}

\begin{itemize}
\tightlist
\item
  Returns: \{Object\}
\end{itemize}

Returns the bound \texttt{address}, the address \texttt{family} name and
\texttt{port} of the socket as reported by the operating system:
\texttt{\{\ port:\ 12346,\ family:\ \textquotesingle{}IPv4\textquotesingle{},\ address:\ \textquotesingle{}127.0.0.1\textquotesingle{}\ \}}

\subsubsection{\texorpdfstring{\texttt{socket.autoSelectFamilyAttemptedAddresses}}{socket.autoSelectFamilyAttemptedAddresses}}\label{socket.autoselectfamilyattemptedaddresses}

\begin{itemize}
\tightlist
\item
  \{string{[}{]}\}
\end{itemize}

This property is only present if the family autoselection algorithm is
enabled in
\hyperref[socketconnectoptions-connectlistener]{\texttt{socket.connect(options)}}
and it is an array of the addresses that have been attempted.

Each address is a string in the form of \texttt{\$IP:\$PORT}. If the
connection was successful, then the last address is the one that the
socket is currently connected to.

\subsubsection{\texorpdfstring{\texttt{socket.bufferSize}}{socket.bufferSize}}\label{socket.buffersize}

\begin{quote}
Stability: 0 - Deprecated: Use
\href{stream.md\#writablewritablelength}{\texttt{writable.writableLength}}
instead.
\end{quote}

\begin{itemize}
\tightlist
\item
  \{integer\}
\end{itemize}

This property shows the number of characters buffered for writing. The
buffer may contain strings whose length after encoding is not yet known.
So this number is only an approximation of the number of bytes in the
buffer.

\texttt{net.Socket} has the property that \texttt{socket.write()} always
works. This is to help users get up and running quickly. The computer
cannot always keep up with the amount of data that is written to a
socket. The network connection simply might be too slow. Node.js will
internally queue up the data written to a socket and send it out over
the wire when it is possible.

The consequence of this internal buffering is that memory may grow.
Users who experience large or growing \texttt{bufferSize} should attempt
to ``throttle'' the data flows in their program with
\hyperref[socketpause]{\texttt{socket.pause()}} and
\hyperref[socketresume]{\texttt{socket.resume()}}.

\subsubsection{\texorpdfstring{\texttt{socket.bytesRead}}{socket.bytesRead}}\label{socket.bytesread}

\begin{itemize}
\tightlist
\item
  \{integer\}
\end{itemize}

The amount of received bytes.

\subsubsection{\texorpdfstring{\texttt{socket.bytesWritten}}{socket.bytesWritten}}\label{socket.byteswritten}

\begin{itemize}
\tightlist
\item
  \{integer\}
\end{itemize}

The amount of bytes sent.

\subsubsection{\texorpdfstring{\texttt{socket.connect()}}{socket.connect()}}\label{socket.connect}

Initiate a connection on a given socket.

Possible signatures:

\begin{itemize}
\tightlist
\item
  \hyperref[socketconnectoptions-connectlistener]{\texttt{socket.connect(options{[},\ connectListener{]})}}
\item
  \hyperref[socketconnectpath-connectlistener]{\texttt{socket.connect(path{[},\ connectListener{]})}}
  for \hyperref[ipc-support]{IPC} connections.
\item
  \hyperref[socketconnectport-host-connectlistener]{\texttt{socket.connect(port{[},\ host{]}{[},\ connectListener{]})}}
  for TCP connections.
\item
  Returns: \{net.Socket\} The socket itself.
\end{itemize}

This function is asynchronous. When the connection is established, the
\hyperref[event-connect]{\texttt{\textquotesingle{}connect\textquotesingle{}}}
event will be emitted. If there is a problem connecting, instead of a
\hyperref[event-connect]{\texttt{\textquotesingle{}connect\textquotesingle{}}}
event, an
\hyperref[event-error_1]{\texttt{\textquotesingle{}error\textquotesingle{}}}
event will be emitted with the error passed to the
\hyperref[event-error_1]{\texttt{\textquotesingle{}error\textquotesingle{}}}
listener. The last parameter \texttt{connectListener}, if supplied, will
be added as a listener for the
\hyperref[event-connect]{\texttt{\textquotesingle{}connect\textquotesingle{}}}
event \textbf{once}.

This function should only be used for reconnecting a socket after
\texttt{\textquotesingle{}close\textquotesingle{}} has been emitted or
otherwise it may lead to undefined behavior.

\paragraph{\texorpdfstring{\texttt{socket.connect(options{[},\ connectListener{]})}}{socket.connect(options{[}, connectListener{]})}}\label{socket.connectoptions-connectlistener}

\begin{itemize}
\tightlist
\item
  \texttt{options} \{Object\}
\item
  \texttt{connectListener} \{Function\} Common parameter of
  \hyperref[socketconnect]{\texttt{socket.connect()}} methods. Will be
  added as a listener for the
  \hyperref[event-connect]{\texttt{\textquotesingle{}connect\textquotesingle{}}}
  event once.
\item
  Returns: \{net.Socket\} The socket itself.
\end{itemize}

Initiate a connection on a given socket. Normally this method is not
needed, the socket should be created and opened with
\hyperref[netcreateconnection]{\texttt{net.createConnection()}}. Use
this only when implementing a custom Socket.

For TCP connections, available \texttt{options} are:

\begin{itemize}
\tightlist
\item
  \texttt{port} \{number\} Required. Port the socket should connect to.
\item
  \texttt{host} \{string\} Host the socket should connect to.
  \textbf{Default:}
  \texttt{\textquotesingle{}localhost\textquotesingle{}}.
\item
  \texttt{localAddress} \{string\} Local address the socket should
  connect from.
\item
  \texttt{localPort} \{number\} Local port the socket should connect
  from.
\item
  \texttt{family} \{number\}: Version of IP stack. Must be \texttt{4},
  \texttt{6}, or \texttt{0}. The value \texttt{0} indicates that both
  IPv4 and IPv6 addresses are allowed. \textbf{Default:} \texttt{0}.
\item
  \texttt{hints} \{number\} Optional
  \href{dns.md\#supported-getaddrinfo-flags}{\texttt{dns.lookup()}
  hints}.
\item
  \texttt{lookup} \{Function\} Custom lookup function. \textbf{Default:}
  \href{dns.md\#dnslookuphostname-options-callback}{\texttt{dns.lookup()}}.
\item
  \texttt{noDelay} \{boolean\} If set to \texttt{true}, it disables the
  use of Nagle's algorithm immediately after the socket is established.
  \textbf{Default:} \texttt{false}.
\item
  \texttt{keepAlive} \{boolean\} If set to \texttt{true}, it enables
  keep-alive functionality on the socket immediately after the
  connection is established, similarly on what is done in
  \hyperref[socketsetkeepaliveenable-initialdelay]{\texttt{socket.setKeepAlive({[}enable{]}{[},\ initialDelay{]})}}.
  \textbf{Default:} \texttt{false}.
\item
  \texttt{keepAliveInitialDelay} \{number\} If set to a positive number,
  it sets the initial delay before the first keepalive probe is sent on
  an idle socket.\textbf{Default:} \texttt{0}.
\item
  \texttt{autoSelectFamily} \{boolean\}: If set to \texttt{true}, it
  enables a family autodetection algorithm that loosely implements
  section 5 of \href{https://www.rfc-editor.org/rfc/rfc8305.txt}{RFC
  8305}. The \texttt{all} option passed to lookup is set to
  \texttt{true} and the sockets attempts to connect to all obtained IPv6
  and IPv4 addresses, in sequence, until a connection is established.
  The first returned AAAA address is tried first, then the first
  returned A address, then the second returned AAAA address and so on.
  Each connection attempt is given the amount of time specified by the
  \texttt{autoSelectFamilyAttemptTimeout} option before timing out and
  trying the next address. Ignored if the \texttt{family} option is not
  \texttt{0} or if \texttt{localAddress} is set. Connection errors are
  not emitted if at least one connection succeeds. If all connections
  attempts fails, a single \texttt{AggregateError} with all failed
  attempts is emitted. \textbf{Default:}
  \hyperref[netgetdefaultautoselectfamily]{\texttt{net.getDefaultAutoSelectFamily()}}
\item
  \texttt{autoSelectFamilyAttemptTimeout} \{number\}: The amount of time
  in milliseconds to wait for a connection attempt to finish before
  trying the next address when using the \texttt{autoSelectFamily}
  option. If set to a positive integer less than \texttt{10}, then the
  value \texttt{10} will be used instead. \textbf{Default:}
  \hyperref[netgetdefaultautoselectfamilyattempttimeout]{\texttt{net.getDefaultAutoSelectFamilyAttemptTimeout()}}
\end{itemize}

For \hyperref[ipc-support]{IPC} connections, available \texttt{options}
are:

\begin{itemize}
\tightlist
\item
  \texttt{path} \{string\} Required. Path the client should connect to.
  See \hyperref[identifying-paths-for-ipc-connections]{Identifying paths
  for IPC connections}. If provided, the TCP-specific options above are
  ignored.
\end{itemize}

For both types, available \texttt{options} include:

\begin{itemize}
\tightlist
\item
  \texttt{onread} \{Object\} If specified, incoming data is stored in a
  single \texttt{buffer} and passed to the supplied \texttt{callback}
  when data arrives on the socket. This will cause the streaming
  functionality to not provide any data. The socket will emit events
  like \texttt{\textquotesingle{}error\textquotesingle{}},
  \texttt{\textquotesingle{}end\textquotesingle{}}, and
  \texttt{\textquotesingle{}close\textquotesingle{}} as usual. Methods
  like \texttt{pause()} and \texttt{resume()} will also behave as
  expected.

  \begin{itemize}
  \tightlist
  \item
    \texttt{buffer} \{Buffer\textbar Uint8Array\textbar Function\}
    Either a reusable chunk of memory to use for storing incoming data
    or a function that returns such.
  \item
    \texttt{callback} \{Function\} This function is called for every
    chunk of incoming data. Two arguments are passed to it: the number
    of bytes written to \texttt{buffer} and a reference to
    \texttt{buffer}. Return \texttt{false} from this function to
    implicitly \texttt{pause()} the socket. This function will be
    executed in the global context.
  \end{itemize}
\end{itemize}

Following is an example of a client using the \texttt{onread} option:

\begin{Shaded}
\begin{Highlighting}[]
\KeywordTok{const}\NormalTok{ net }\OperatorTok{=} \PreprocessorTok{require}\NormalTok{(}\StringTok{\textquotesingle{}node:net\textquotesingle{}}\NormalTok{)}\OperatorTok{;}
\NormalTok{net}\OperatorTok{.}\FunctionTok{connect}\NormalTok{(\{}
  \DataTypeTok{port}\OperatorTok{:} \DecValTok{80}\OperatorTok{,}
  \DataTypeTok{onread}\OperatorTok{:}\NormalTok{ \{}
    \CommentTok{// Reuses a 4KiB Buffer for every read from the socket.}
    \DataTypeTok{buffer}\OperatorTok{:} \BuiltInTok{Buffer}\OperatorTok{.}\FunctionTok{alloc}\NormalTok{(}\DecValTok{4} \OperatorTok{*} \DecValTok{1024}\NormalTok{)}\OperatorTok{,}
    \DataTypeTok{callback}\OperatorTok{:} \KeywordTok{function}\NormalTok{(nread}\OperatorTok{,}\NormalTok{ buf) \{}
      \CommentTok{// Received data is available in \textasciigrave{}buf\textasciigrave{} from 0 to \textasciigrave{}nread\textasciigrave{}.}
      \BuiltInTok{console}\OperatorTok{.}\FunctionTok{log}\NormalTok{(buf}\OperatorTok{.}\FunctionTok{toString}\NormalTok{(}\StringTok{\textquotesingle{}utf8\textquotesingle{}}\OperatorTok{,} \DecValTok{0}\OperatorTok{,}\NormalTok{ nread))}\OperatorTok{;}
\NormalTok{    \}}\OperatorTok{,}
\NormalTok{  \}}\OperatorTok{,}
\NormalTok{\})}\OperatorTok{;}
\end{Highlighting}
\end{Shaded}

\paragraph{\texorpdfstring{\texttt{socket.connect(path{[},\ connectListener{]})}}{socket.connect(path{[}, connectListener{]})}}\label{socket.connectpath-connectlistener}

\begin{itemize}
\tightlist
\item
  \texttt{path} \{string\} Path the client should connect to. See
  \hyperref[identifying-paths-for-ipc-connections]{Identifying paths for
  IPC connections}.
\item
  \texttt{connectListener} \{Function\} Common parameter of
  \hyperref[socketconnect]{\texttt{socket.connect()}} methods. Will be
  added as a listener for the
  \hyperref[event-connect]{\texttt{\textquotesingle{}connect\textquotesingle{}}}
  event once.
\item
  Returns: \{net.Socket\} The socket itself.
\end{itemize}

Initiate an \hyperref[ipc-support]{IPC} connection on the given socket.

Alias to
\hyperref[socketconnectoptions-connectlistener]{\texttt{socket.connect(options{[},\ connectListener{]})}}
called with \texttt{\{\ path:\ path\ \}} as \texttt{options}.

\paragraph{\texorpdfstring{\texttt{socket.connect(port{[},\ host{]}{[},\ connectListener{]})}}{socket.connect(port{[}, host{]}{[}, connectListener{]})}}\label{socket.connectport-host-connectlistener}

\begin{itemize}
\tightlist
\item
  \texttt{port} \{number\} Port the client should connect to.
\item
  \texttt{host} \{string\} Host the client should connect to.
\item
  \texttt{connectListener} \{Function\} Common parameter of
  \hyperref[socketconnect]{\texttt{socket.connect()}} methods. Will be
  added as a listener for the
  \hyperref[event-connect]{\texttt{\textquotesingle{}connect\textquotesingle{}}}
  event once.
\item
  Returns: \{net.Socket\} The socket itself.
\end{itemize}

Initiate a TCP connection on the given socket.

Alias to
\hyperref[socketconnectoptions-connectlistener]{\texttt{socket.connect(options{[},\ connectListener{]})}}
called with \texttt{\{port:\ port,\ host:\ host\}} as \texttt{options}.

\subsubsection{\texorpdfstring{\texttt{socket.connecting}}{socket.connecting}}\label{socket.connecting}

\begin{itemize}
\tightlist
\item
  \{boolean\}
\end{itemize}

If \texttt{true},
\hyperref[socketconnectoptions-connectlistener]{\texttt{socket.connect(options{[},\ connectListener{]})}}
was called and has not yet finished. It will stay \texttt{true} until
the socket becomes connected, then it is set to \texttt{false} and the
\texttt{\textquotesingle{}connect\textquotesingle{}} event is emitted.
Note that the
\hyperref[socketconnectoptions-connectlistener]{\texttt{socket.connect(options{[},\ connectListener{]})}}
callback is a listener for the
\texttt{\textquotesingle{}connect\textquotesingle{}} event.

\subsubsection{\texorpdfstring{\texttt{socket.destroy({[}error{]})}}{socket.destroy({[}error{]})}}\label{socket.destroyerror}

\begin{itemize}
\tightlist
\item
  \texttt{error} \{Object\}
\item
  Returns: \{net.Socket\}
\end{itemize}

Ensures that no more I/O activity happens on this socket. Destroys the
stream and closes the connection.

See \href{stream.md\#writabledestroyerror}{\texttt{writable.destroy()}}
for further details.

\subsubsection{\texorpdfstring{\texttt{socket.destroyed}}{socket.destroyed}}\label{socket.destroyed}

\begin{itemize}
\tightlist
\item
  \{boolean\} Indicates if the connection is destroyed or not. Once a
  connection is destroyed no further data can be transferred using it.
\end{itemize}

See \href{stream.md\#writabledestroyed}{\texttt{writable.destroyed}} for
further details.

\subsubsection{\texorpdfstring{\texttt{socket.destroySoon()}}{socket.destroySoon()}}\label{socket.destroysoon}

Destroys the socket after all data is written. If the
\texttt{\textquotesingle{}finish\textquotesingle{}} event was already
emitted the socket is destroyed immediately. If the socket is still
writable it implicitly calls \texttt{socket.end()}.

\subsubsection{\texorpdfstring{\texttt{socket.end({[}data{[},\ encoding{]}{]}{[},\ callback{]})}}{socket.end({[}data{[}, encoding{]}{]}{[}, callback{]})}}\label{socket.enddata-encoding-callback}

\begin{itemize}
\tightlist
\item
  \texttt{data} \{string\textbar Buffer\textbar Uint8Array\}
\item
  \texttt{encoding} \{string\} Only used when data is \texttt{string}.
  \textbf{Default:} \texttt{\textquotesingle{}utf8\textquotesingle{}}.
\item
  \texttt{callback} \{Function\} Optional callback for when the socket
  is finished.
\item
  Returns: \{net.Socket\} The socket itself.
\end{itemize}

Half-closes the socket. i.e., it sends a FIN packet. It is possible the
server will still send some data.

See
\href{stream.md\#writableendchunk-encoding-callback}{\texttt{writable.end()}}
for further details.

\subsubsection{\texorpdfstring{\texttt{socket.localAddress}}{socket.localAddress}}\label{socket.localaddress}

\begin{itemize}
\tightlist
\item
  \{string\}
\end{itemize}

The string representation of the local IP address the remote client is
connecting on. For example, in a server listening on
\texttt{\textquotesingle{}0.0.0.0\textquotesingle{}}, if a client
connects on \texttt{\textquotesingle{}192.168.1.1\textquotesingle{}},
the value of \texttt{socket.localAddress} would be
\texttt{\textquotesingle{}192.168.1.1\textquotesingle{}}.

\subsubsection{\texorpdfstring{\texttt{socket.localPort}}{socket.localPort}}\label{socket.localport}

\begin{itemize}
\tightlist
\item
  \{integer\}
\end{itemize}

The numeric representation of the local port. For example, \texttt{80}
or \texttt{21}.

\subsubsection{\texorpdfstring{\texttt{socket.localFamily}}{socket.localFamily}}\label{socket.localfamily}

\begin{itemize}
\tightlist
\item
  \{string\}
\end{itemize}

The string representation of the local IP family.
\texttt{\textquotesingle{}IPv4\textquotesingle{}} or
\texttt{\textquotesingle{}IPv6\textquotesingle{}}.

\subsubsection{\texorpdfstring{\texttt{socket.pause()}}{socket.pause()}}\label{socket.pause}

\begin{itemize}
\tightlist
\item
  Returns: \{net.Socket\} The socket itself.
\end{itemize}

Pauses the reading of data. That is,
\hyperref[event-data]{\texttt{\textquotesingle{}data\textquotesingle{}}}
events will not be emitted. Useful to throttle back an upload.

\subsubsection{\texorpdfstring{\texttt{socket.pending}}{socket.pending}}\label{socket.pending}

\begin{itemize}
\tightlist
\item
  \{boolean\}
\end{itemize}

This is \texttt{true} if the socket is not connected yet, either because
\texttt{.connect()} has not yet been called or because it is still in
the process of connecting (see
\hyperref[socketconnecting]{\texttt{socket.connecting}}).

\subsubsection{\texorpdfstring{\texttt{socket.ref()}}{socket.ref()}}\label{socket.ref}

\begin{itemize}
\tightlist
\item
  Returns: \{net.Socket\} The socket itself.
\end{itemize}

Opposite of \texttt{unref()}, calling \texttt{ref()} on a previously
\texttt{unref}ed socket will \emph{not} let the program exit if it's the
only socket left (the default behavior). If the socket is \texttt{ref}ed
calling \texttt{ref} again will have no effect.

\subsubsection{\texorpdfstring{\texttt{socket.remoteAddress}}{socket.remoteAddress}}\label{socket.remoteaddress}

\begin{itemize}
\tightlist
\item
  \{string\}
\end{itemize}

The string representation of the remote IP address. For example,
\texttt{\textquotesingle{}74.125.127.100\textquotesingle{}} or
\texttt{\textquotesingle{}2001:4860:a005::68\textquotesingle{}}. Value
may be \texttt{undefined} if the socket is destroyed (for example, if
the client disconnected).

\subsubsection{\texorpdfstring{\texttt{socket.remoteFamily}}{socket.remoteFamily}}\label{socket.remotefamily}

\begin{itemize}
\tightlist
\item
  \{string\}
\end{itemize}

The string representation of the remote IP family.
\texttt{\textquotesingle{}IPv4\textquotesingle{}} or
\texttt{\textquotesingle{}IPv6\textquotesingle{}}. Value may be
\texttt{undefined} if the socket is destroyed (for example, if the
client disconnected).

\subsubsection{\texorpdfstring{\texttt{socket.remotePort}}{socket.remotePort}}\label{socket.remoteport}

\begin{itemize}
\tightlist
\item
  \{integer\}
\end{itemize}

The numeric representation of the remote port. For example, \texttt{80}
or \texttt{21}. Value may be \texttt{undefined} if the socket is
destroyed (for example, if the client disconnected).

\subsubsection{\texorpdfstring{\texttt{socket.resetAndDestroy()}}{socket.resetAndDestroy()}}\label{socket.resetanddestroy}

\begin{itemize}
\tightlist
\item
  Returns: \{net.Socket\}
\end{itemize}

Close the TCP connection by sending an RST packet and destroy the
stream. If this TCP socket is in connecting status, it will send an RST
packet and destroy this TCP socket once it is connected. Otherwise, it
will call \texttt{socket.destroy} with an \texttt{ERR\_SOCKET\_CLOSED}
Error. If this is not a TCP socket (for example, a pipe), calling this
method will immediately throw an \texttt{ERR\_INVALID\_HANDLE\_TYPE}
Error.

\subsubsection{\texorpdfstring{\texttt{socket.resume()}}{socket.resume()}}\label{socket.resume}

\begin{itemize}
\tightlist
\item
  Returns: \{net.Socket\} The socket itself.
\end{itemize}

Resumes reading after a call to
\hyperref[socketpause]{\texttt{socket.pause()}}.

\subsubsection{\texorpdfstring{\texttt{socket.setEncoding({[}encoding{]})}}{socket.setEncoding({[}encoding{]})}}\label{socket.setencodingencoding}

\begin{itemize}
\tightlist
\item
  \texttt{encoding} \{string\}
\item
  Returns: \{net.Socket\} The socket itself.
\end{itemize}

Set the encoding for the socket as a
\href{stream.md\#class-streamreadable}{Readable Stream}. See
\href{stream.md\#readablesetencodingencoding}{\texttt{readable.setEncoding()}}
for more information.

\subsubsection{\texorpdfstring{\texttt{socket.setKeepAlive({[}enable{]}{[},\ initialDelay{]})}}{socket.setKeepAlive({[}enable{]}{[}, initialDelay{]})}}\label{socket.setkeepaliveenable-initialdelay}

\begin{itemize}
\tightlist
\item
  \texttt{enable} \{boolean\} \textbf{Default:} \texttt{false}
\item
  \texttt{initialDelay} \{number\} \textbf{Default:} \texttt{0}
\item
  Returns: \{net.Socket\} The socket itself.
\end{itemize}

Enable/disable keep-alive functionality, and optionally set the initial
delay before the first keepalive probe is sent on an idle socket.

Set \texttt{initialDelay} (in milliseconds) to set the delay between the
last data packet received and the first keepalive probe. Setting
\texttt{0} for \texttt{initialDelay} will leave the value unchanged from
the default (or previous) setting.

Enabling the keep-alive functionality will set the following socket
options:

\begin{itemize}
\tightlist
\item
  \texttt{SO\_KEEPALIVE=1}
\item
  \texttt{TCP\_KEEPIDLE=initialDelay}
\item
  \texttt{TCP\_KEEPCNT=10}
\item
  \texttt{TCP\_KEEPINTVL=1}
\end{itemize}

\subsubsection{\texorpdfstring{\texttt{socket.setNoDelay({[}noDelay{]})}}{socket.setNoDelay({[}noDelay{]})}}\label{socket.setnodelaynodelay}

\begin{itemize}
\tightlist
\item
  \texttt{noDelay} \{boolean\} \textbf{Default:} \texttt{true}
\item
  Returns: \{net.Socket\} The socket itself.
\end{itemize}

Enable/disable the use of Nagle's algorithm.

When a TCP connection is created, it will have Nagle's algorithm
enabled.

Nagle's algorithm delays data before it is sent via the network. It
attempts to optimize throughput at the expense of latency.

Passing \texttt{true} for \texttt{noDelay} or not passing an argument
will disable Nagle's algorithm for the socket. Passing \texttt{false}
for \texttt{noDelay} will enable Nagle's algorithm.

\subsubsection{\texorpdfstring{\texttt{socket.setTimeout(timeout{[},\ callback{]})}}{socket.setTimeout(timeout{[}, callback{]})}}\label{socket.settimeouttimeout-callback}

\begin{itemize}
\tightlist
\item
  \texttt{timeout} \{number\}
\item
  \texttt{callback} \{Function\}
\item
  Returns: \{net.Socket\} The socket itself.
\end{itemize}

Sets the socket to timeout after \texttt{timeout} milliseconds of
inactivity on the socket. By default \texttt{net.Socket} do not have a
timeout.

When an idle timeout is triggered the socket will receive a
\hyperref[event-timeout]{\texttt{\textquotesingle{}timeout\textquotesingle{}}}
event but the connection will not be severed. The user must manually
call \hyperref[socketenddata-encoding-callback]{\texttt{socket.end()}}
or \hyperref[socketdestroyerror]{\texttt{socket.destroy()}} to end the
connection.

\begin{Shaded}
\begin{Highlighting}[]
\NormalTok{socket}\OperatorTok{.}\FunctionTok{setTimeout}\NormalTok{(}\DecValTok{3000}\NormalTok{)}\OperatorTok{;}
\NormalTok{socket}\OperatorTok{.}\FunctionTok{on}\NormalTok{(}\StringTok{\textquotesingle{}timeout\textquotesingle{}}\OperatorTok{,}\NormalTok{ () }\KeywordTok{=\textgreater{}}\NormalTok{ \{}
  \BuiltInTok{console}\OperatorTok{.}\FunctionTok{log}\NormalTok{(}\StringTok{\textquotesingle{}socket timeout\textquotesingle{}}\NormalTok{)}\OperatorTok{;}
\NormalTok{  socket}\OperatorTok{.}\FunctionTok{end}\NormalTok{()}\OperatorTok{;}
\NormalTok{\})}\OperatorTok{;}
\end{Highlighting}
\end{Shaded}

If \texttt{timeout} is 0, then the existing idle timeout is disabled.

The optional \texttt{callback} parameter will be added as a one-time
listener for the
\hyperref[event-timeout]{\texttt{\textquotesingle{}timeout\textquotesingle{}}}
event.

\subsubsection{\texorpdfstring{\texttt{socket.timeout}}{socket.timeout}}\label{socket.timeout}

\begin{itemize}
\tightlist
\item
  \{number\textbar undefined\}
\end{itemize}

The socket timeout in milliseconds as set by
\hyperref[socketsettimeouttimeout-callback]{\texttt{socket.setTimeout()}}.
It is \texttt{undefined} if a timeout has not been set.

\subsubsection{\texorpdfstring{\texttt{socket.unref()}}{socket.unref()}}\label{socket.unref}

\begin{itemize}
\tightlist
\item
  Returns: \{net.Socket\} The socket itself.
\end{itemize}

Calling \texttt{unref()} on a socket will allow the program to exit if
this is the only active socket in the event system. If the socket is
already \texttt{unref}ed calling \texttt{unref()} again will have no
effect.

\subsubsection{\texorpdfstring{\texttt{socket.write(data{[},\ encoding{]}{[},\ callback{]})}}{socket.write(data{[}, encoding{]}{[}, callback{]})}}\label{socket.writedata-encoding-callback}

\begin{itemize}
\tightlist
\item
  \texttt{data} \{string\textbar Buffer\textbar Uint8Array\}
\item
  \texttt{encoding} \{string\} Only used when data is \texttt{string}.
  \textbf{Default:} \texttt{utf8}.
\item
  \texttt{callback} \{Function\}
\item
  Returns: \{boolean\}
\end{itemize}

Sends data on the socket. The second parameter specifies the encoding in
the case of a string. It defaults to UTF8 encoding.

Returns \texttt{true} if the entire data was flushed successfully to the
kernel buffer. Returns \texttt{false} if all or part of the data was
queued in user memory.
\hyperref[event-drain]{\texttt{\textquotesingle{}drain\textquotesingle{}}}
will be emitted when the buffer is again free.

The optional \texttt{callback} parameter will be executed when the data
is finally written out, which may not be immediately.

See \texttt{Writable} stream
\href{stream.md\#writablewritechunk-encoding-callback}{\texttt{write()}}
method for more information.

\subsubsection{\texorpdfstring{\texttt{socket.readyState}}{socket.readyState}}\label{socket.readystate}

\begin{itemize}
\tightlist
\item
  \{string\}
\end{itemize}

This property represents the state of the connection as a string.

\begin{itemize}
\tightlist
\item
  If the stream is connecting \texttt{socket.readyState} is
  \texttt{opening}.
\item
  If the stream is readable and writable, it is \texttt{open}.
\item
  If the stream is readable and not writable, it is \texttt{readOnly}.
\item
  If the stream is not readable and writable, it is \texttt{writeOnly}.
\end{itemize}

\subsection{\texorpdfstring{\texttt{net.connect()}}{net.connect()}}\label{net.connect}

Aliases to
\hyperref[netcreateconnection]{\texttt{net.createConnection()}}.

Possible signatures:

\begin{itemize}
\tightlist
\item
  \hyperref[netconnectoptions-connectlistener]{\texttt{net.connect(options{[},\ connectListener{]})}}
\item
  \hyperref[netconnectpath-connectlistener]{\texttt{net.connect(path{[},\ connectListener{]})}}
  for \hyperref[ipc-support]{IPC} connections.
\item
  \hyperref[netconnectport-host-connectlistener]{\texttt{net.connect(port{[},\ host{]}{[},\ connectListener{]})}}
  for TCP connections.
\end{itemize}

\subsubsection{\texorpdfstring{\texttt{net.connect(options{[},\ connectListener{]})}}{net.connect(options{[}, connectListener{]})}}\label{net.connectoptions-connectlistener}

\begin{itemize}
\tightlist
\item
  \texttt{options} \{Object\}
\item
  \texttt{connectListener} \{Function\}
\item
  Returns: \{net.Socket\}
\end{itemize}

Alias to
\hyperref[netcreateconnectionoptions-connectlistener]{\texttt{net.createConnection(options{[},\ connectListener{]})}}.

\subsubsection{\texorpdfstring{\texttt{net.connect(path{[},\ connectListener{]})}}{net.connect(path{[}, connectListener{]})}}\label{net.connectpath-connectlistener}

\begin{itemize}
\tightlist
\item
  \texttt{path} \{string\}
\item
  \texttt{connectListener} \{Function\}
\item
  Returns: \{net.Socket\}
\end{itemize}

Alias to
\hyperref[netcreateconnectionpath-connectlistener]{\texttt{net.createConnection(path{[},\ connectListener{]})}}.

\subsubsection{\texorpdfstring{\texttt{net.connect(port{[},\ host{]}{[},\ connectListener{]})}}{net.connect(port{[}, host{]}{[}, connectListener{]})}}\label{net.connectport-host-connectlistener}

\begin{itemize}
\tightlist
\item
  \texttt{port} \{number\}
\item
  \texttt{host} \{string\}
\item
  \texttt{connectListener} \{Function\}
\item
  Returns: \{net.Socket\}
\end{itemize}

Alias to
\hyperref[netcreateconnectionport-host-connectlistener]{\texttt{net.createConnection(port{[},\ host{]}{[},\ connectListener{]})}}.

\subsection{\texorpdfstring{\texttt{net.createConnection()}}{net.createConnection()}}\label{net.createconnection}

A factory function, which creates a new
\hyperref[class-netsocket]{\texttt{net.Socket}}, immediately initiates
connection with \hyperref[socketconnect]{\texttt{socket.connect()}},
then returns the \texttt{net.Socket} that starts the connection.

When the connection is established, a
\hyperref[event-connect]{\texttt{\textquotesingle{}connect\textquotesingle{}}}
event will be emitted on the returned socket. The last parameter
\texttt{connectListener}, if supplied, will be added as a listener for
the
\hyperref[event-connect]{\texttt{\textquotesingle{}connect\textquotesingle{}}}
event \textbf{once}.

Possible signatures:

\begin{itemize}
\tightlist
\item
  \hyperref[netcreateconnectionoptions-connectlistener]{\texttt{net.createConnection(options{[},\ connectListener{]})}}
\item
  \hyperref[netcreateconnectionpath-connectlistener]{\texttt{net.createConnection(path{[},\ connectListener{]})}}
  for \hyperref[ipc-support]{IPC} connections.
\item
  \hyperref[netcreateconnectionport-host-connectlistener]{\texttt{net.createConnection(port{[},\ host{]}{[},\ connectListener{]})}}
  for TCP connections.
\end{itemize}

The \hyperref[netconnect]{\texttt{net.connect()}} function is an alias
to this function.

\subsubsection{\texorpdfstring{\texttt{net.createConnection(options{[},\ connectListener{]})}}{net.createConnection(options{[}, connectListener{]})}}\label{net.createconnectionoptions-connectlistener}

\begin{itemize}
\tightlist
\item
  \texttt{options} \{Object\} Required. Will be passed to both the
  \hyperref[new-netsocketoptions]{\texttt{new\ net.Socket({[}options{]})}}
  call and the
  \hyperref[socketconnectoptions-connectlistener]{\texttt{socket.connect(options{[},\ connectListener{]})}}
  method.
\item
  \texttt{connectListener} \{Function\} Common parameter of the
  \hyperref[netcreateconnection]{\texttt{net.createConnection()}}
  functions. If supplied, will be added as a listener for the
  \hyperref[event-connect]{\texttt{\textquotesingle{}connect\textquotesingle{}}}
  event on the returned socket once.
\item
  Returns: \{net.Socket\} The newly created socket used to start the
  connection.
\end{itemize}

For available options, see
\hyperref[new-netsocketoptions]{\texttt{new\ net.Socket({[}options{]})}}
and
\hyperref[socketconnectoptions-connectlistener]{\texttt{socket.connect(options{[},\ connectListener{]})}}.

Additional options:

\begin{itemize}
\tightlist
\item
  \texttt{timeout} \{number\} If set, will be used to call
  \hyperref[socketsettimeouttimeout-callback]{\texttt{socket.setTimeout(timeout)}}
  after the socket is created, but before it starts the connection.
\end{itemize}

Following is an example of a client of the echo server described in the
\hyperref[netcreateserveroptions-connectionlistener]{\texttt{net.createServer()}}
section:

\begin{Shaded}
\begin{Highlighting}[]
\KeywordTok{const}\NormalTok{ net }\OperatorTok{=} \PreprocessorTok{require}\NormalTok{(}\StringTok{\textquotesingle{}node:net\textquotesingle{}}\NormalTok{)}\OperatorTok{;}
\KeywordTok{const}\NormalTok{ client }\OperatorTok{=}\NormalTok{ net}\OperatorTok{.}\FunctionTok{createConnection}\NormalTok{(\{ }\DataTypeTok{port}\OperatorTok{:} \DecValTok{8124}\NormalTok{ \}}\OperatorTok{,}\NormalTok{ () }\KeywordTok{=\textgreater{}}\NormalTok{ \{}
  \CommentTok{// \textquotesingle{}connect\textquotesingle{} listener.}
  \BuiltInTok{console}\OperatorTok{.}\FunctionTok{log}\NormalTok{(}\StringTok{\textquotesingle{}connected to server!\textquotesingle{}}\NormalTok{)}\OperatorTok{;}
\NormalTok{  client}\OperatorTok{.}\FunctionTok{write}\NormalTok{(}\StringTok{\textquotesingle{}world!}\SpecialCharTok{\textbackslash{}r\textbackslash{}n}\StringTok{\textquotesingle{}}\NormalTok{)}\OperatorTok{;}
\NormalTok{\})}\OperatorTok{;}
\NormalTok{client}\OperatorTok{.}\FunctionTok{on}\NormalTok{(}\StringTok{\textquotesingle{}data\textquotesingle{}}\OperatorTok{,}\NormalTok{ (data) }\KeywordTok{=\textgreater{}}\NormalTok{ \{}
  \BuiltInTok{console}\OperatorTok{.}\FunctionTok{log}\NormalTok{(data}\OperatorTok{.}\FunctionTok{toString}\NormalTok{())}\OperatorTok{;}
\NormalTok{  client}\OperatorTok{.}\FunctionTok{end}\NormalTok{()}\OperatorTok{;}
\NormalTok{\})}\OperatorTok{;}
\NormalTok{client}\OperatorTok{.}\FunctionTok{on}\NormalTok{(}\StringTok{\textquotesingle{}end\textquotesingle{}}\OperatorTok{,}\NormalTok{ () }\KeywordTok{=\textgreater{}}\NormalTok{ \{}
  \BuiltInTok{console}\OperatorTok{.}\FunctionTok{log}\NormalTok{(}\StringTok{\textquotesingle{}disconnected from server\textquotesingle{}}\NormalTok{)}\OperatorTok{;}
\NormalTok{\})}\OperatorTok{;}
\end{Highlighting}
\end{Shaded}

To connect on the socket \texttt{/tmp/echo.sock}:

\begin{Shaded}
\begin{Highlighting}[]
\KeywordTok{const}\NormalTok{ client }\OperatorTok{=}\NormalTok{ net}\OperatorTok{.}\FunctionTok{createConnection}\NormalTok{(\{ }\DataTypeTok{path}\OperatorTok{:} \StringTok{\textquotesingle{}/tmp/echo.sock\textquotesingle{}}\NormalTok{ \})}\OperatorTok{;}
\end{Highlighting}
\end{Shaded}

\subsubsection{\texorpdfstring{\texttt{net.createConnection(path{[},\ connectListener{]})}}{net.createConnection(path{[}, connectListener{]})}}\label{net.createconnectionpath-connectlistener}

\begin{itemize}
\tightlist
\item
  \texttt{path} \{string\} Path the socket should connect to. Will be
  passed to
  \hyperref[socketconnectpath-connectlistener]{\texttt{socket.connect(path{[},\ connectListener{]})}}.
  See \hyperref[identifying-paths-for-ipc-connections]{Identifying paths
  for IPC connections}.
\item
  \texttt{connectListener} \{Function\} Common parameter of the
  \hyperref[netcreateconnection]{\texttt{net.createConnection()}}
  functions, an ``once'' listener for the
  \texttt{\textquotesingle{}connect\textquotesingle{}} event on the
  initiating socket. Will be passed to
  \hyperref[socketconnectpath-connectlistener]{\texttt{socket.connect(path{[},\ connectListener{]})}}.
\item
  Returns: \{net.Socket\} The newly created socket used to start the
  connection.
\end{itemize}

Initiates an \hyperref[ipc-support]{IPC} connection.

This function creates a new
\hyperref[class-netsocket]{\texttt{net.Socket}} with all options set to
default, immediately initiates connection with
\hyperref[socketconnectpath-connectlistener]{\texttt{socket.connect(path{[},\ connectListener{]})}},
then returns the \texttt{net.Socket} that starts the connection.

\subsubsection{\texorpdfstring{\texttt{net.createConnection(port{[},\ host{]}{[},\ connectListener{]})}}{net.createConnection(port{[}, host{]}{[}, connectListener{]})}}\label{net.createconnectionport-host-connectlistener}

\begin{itemize}
\tightlist
\item
  \texttt{port} \{number\} Port the socket should connect to. Will be
  passed to
  \hyperref[socketconnectport-host-connectlistener]{\texttt{socket.connect(port{[},\ host{]}{[},\ connectListener{]})}}.
\item
  \texttt{host} \{string\} Host the socket should connect to. Will be
  passed to
  \hyperref[socketconnectport-host-connectlistener]{\texttt{socket.connect(port{[},\ host{]}{[},\ connectListener{]})}}.
  \textbf{Default:}
  \texttt{\textquotesingle{}localhost\textquotesingle{}}.
\item
  \texttt{connectListener} \{Function\} Common parameter of the
  \hyperref[netcreateconnection]{\texttt{net.createConnection()}}
  functions, an ``once'' listener for the
  \texttt{\textquotesingle{}connect\textquotesingle{}} event on the
  initiating socket. Will be passed to
  \hyperref[socketconnectport-host-connectlistener]{\texttt{socket.connect(port{[},\ host{]}{[},\ connectListener{]})}}.
\item
  Returns: \{net.Socket\} The newly created socket used to start the
  connection.
\end{itemize}

Initiates a TCP connection.

This function creates a new
\hyperref[class-netsocket]{\texttt{net.Socket}} with all options set to
default, immediately initiates connection with
\hyperref[socketconnectport-host-connectlistener]{\texttt{socket.connect(port{[},\ host{]}{[},\ connectListener{]})}},
then returns the \texttt{net.Socket} that starts the connection.

\subsection{\texorpdfstring{\texttt{net.createServer({[}options{]}{[},\ connectionListener{]})}}{net.createServer({[}options{]}{[}, connectionListener{]})}}\label{net.createserveroptions-connectionlistener}

\begin{itemize}
\item
  \texttt{options} \{Object\}

  \begin{itemize}
  \tightlist
  \item
    \texttt{allowHalfOpen} \{boolean\} If set to \texttt{false}, then
    the socket will automatically end the writable side when the
    readable side ends. \textbf{Default:} \texttt{false}.
  \item
    \texttt{highWaterMark} \{number\} Optionally overrides all
    \hyperref[class-netsocket]{\texttt{net.Socket}}s'
    \texttt{readableHighWaterMark} and \texttt{writableHighWaterMark}.
    \textbf{Default:} See
    \href{stream.md\#streamgetdefaulthighwatermarkobjectmode}{\texttt{stream.getDefaultHighWaterMark()}}.
  \item
    \texttt{pauseOnConnect} \{boolean\} Indicates whether the socket
    should be paused on incoming connections. \textbf{Default:}
    \texttt{false}.
  \item
    \texttt{noDelay} \{boolean\} If set to \texttt{true}, it disables
    the use of Nagle's algorithm immediately after a new incoming
    connection is received. \textbf{Default:} \texttt{false}.
  \item
    \texttt{keepAlive} \{boolean\} If set to \texttt{true}, it enables
    keep-alive functionality on the socket immediately after a new
    incoming connection is received, similarly on what is done in
    \hyperref[socketsetkeepaliveenable-initialdelay]{\texttt{socket.setKeepAlive({[}enable{]}{[},\ initialDelay{]})}}.
    \textbf{Default:} \texttt{false}.
  \item
    \texttt{keepAliveInitialDelay} \{number\} If set to a positive
    number, it sets the initial delay before the first keepalive probe
    is sent on an idle socket.\textbf{Default:} \texttt{0}.
  \end{itemize}
\item
  \texttt{connectionListener} \{Function\} Automatically set as a
  listener for the
  \hyperref[event-connection]{\texttt{\textquotesingle{}connection\textquotesingle{}}}
  event.
\item
  Returns: \{net.Server\}
\end{itemize}

Creates a new TCP or \hyperref[ipc-support]{IPC} server.

If \texttt{allowHalfOpen} is set to \texttt{true}, when the other end of
the socket signals the end of transmission, the server will only send
back the end of transmission when
\hyperref[socketenddata-encoding-callback]{\texttt{socket.end()}} is
explicitly called. For example, in the context of TCP, when a FIN packed
is received, a FIN packed is sent back only when
\hyperref[socketenddata-encoding-callback]{\texttt{socket.end()}} is
explicitly called. Until then the connection is half-closed
(non-readable but still writable). See
\hyperref[event-end]{\texttt{\textquotesingle{}end\textquotesingle{}}}
event and \href{https://tools.ietf.org/html/rfc1122}{RFC 1122} (section
4.2.2.13) for more information.

If \texttt{pauseOnConnect} is set to \texttt{true}, then the socket
associated with each incoming connection will be paused, and no data
will be read from its handle. This allows connections to be passed
between processes without any data being read by the original process.
To begin reading data from a paused socket, call
\hyperref[socketresume]{\texttt{socket.resume()}}.

The server can be a TCP server or an \hyperref[ipc-support]{IPC} server,
depending on what it \hyperref[serverlisten]{\texttt{listen()}} to.

Here is an example of a TCP echo server which listens for connections on
port 8124:

\begin{Shaded}
\begin{Highlighting}[]
\KeywordTok{const}\NormalTok{ net }\OperatorTok{=} \PreprocessorTok{require}\NormalTok{(}\StringTok{\textquotesingle{}node:net\textquotesingle{}}\NormalTok{)}\OperatorTok{;}
\KeywordTok{const}\NormalTok{ server }\OperatorTok{=}\NormalTok{ net}\OperatorTok{.}\FunctionTok{createServer}\NormalTok{((c) }\KeywordTok{=\textgreater{}}\NormalTok{ \{}
  \CommentTok{// \textquotesingle{}connection\textquotesingle{} listener.}
  \BuiltInTok{console}\OperatorTok{.}\FunctionTok{log}\NormalTok{(}\StringTok{\textquotesingle{}client connected\textquotesingle{}}\NormalTok{)}\OperatorTok{;}
\NormalTok{  c}\OperatorTok{.}\FunctionTok{on}\NormalTok{(}\StringTok{\textquotesingle{}end\textquotesingle{}}\OperatorTok{,}\NormalTok{ () }\KeywordTok{=\textgreater{}}\NormalTok{ \{}
    \BuiltInTok{console}\OperatorTok{.}\FunctionTok{log}\NormalTok{(}\StringTok{\textquotesingle{}client disconnected\textquotesingle{}}\NormalTok{)}\OperatorTok{;}
\NormalTok{  \})}\OperatorTok{;}
\NormalTok{  c}\OperatorTok{.}\FunctionTok{write}\NormalTok{(}\StringTok{\textquotesingle{}hello}\SpecialCharTok{\textbackslash{}r\textbackslash{}n}\StringTok{\textquotesingle{}}\NormalTok{)}\OperatorTok{;}
\NormalTok{  c}\OperatorTok{.}\FunctionTok{pipe}\NormalTok{(c)}\OperatorTok{;}
\NormalTok{\})}\OperatorTok{;}
\NormalTok{server}\OperatorTok{.}\FunctionTok{on}\NormalTok{(}\StringTok{\textquotesingle{}error\textquotesingle{}}\OperatorTok{,}\NormalTok{ (err) }\KeywordTok{=\textgreater{}}\NormalTok{ \{}
  \ControlFlowTok{throw}\NormalTok{ err}\OperatorTok{;}
\NormalTok{\})}\OperatorTok{;}
\NormalTok{server}\OperatorTok{.}\FunctionTok{listen}\NormalTok{(}\DecValTok{8124}\OperatorTok{,}\NormalTok{ () }\KeywordTok{=\textgreater{}}\NormalTok{ \{}
  \BuiltInTok{console}\OperatorTok{.}\FunctionTok{log}\NormalTok{(}\StringTok{\textquotesingle{}server bound\textquotesingle{}}\NormalTok{)}\OperatorTok{;}
\NormalTok{\})}\OperatorTok{;}
\end{Highlighting}
\end{Shaded}

Test this by using \texttt{telnet}:

\begin{Shaded}
\begin{Highlighting}[]
\ExtensionTok{telnet}\NormalTok{ localhost 8124}
\end{Highlighting}
\end{Shaded}

To listen on the socket \texttt{/tmp/echo.sock}:

\begin{Shaded}
\begin{Highlighting}[]
\NormalTok{server}\OperatorTok{.}\FunctionTok{listen}\NormalTok{(}\StringTok{\textquotesingle{}/tmp/echo.sock\textquotesingle{}}\OperatorTok{,}\NormalTok{ () }\KeywordTok{=\textgreater{}}\NormalTok{ \{}
  \BuiltInTok{console}\OperatorTok{.}\FunctionTok{log}\NormalTok{(}\StringTok{\textquotesingle{}server bound\textquotesingle{}}\NormalTok{)}\OperatorTok{;}
\NormalTok{\})}\OperatorTok{;}
\end{Highlighting}
\end{Shaded}

Use \texttt{nc} to connect to a Unix domain socket server:

\begin{Shaded}
\begin{Highlighting}[]
\ExtensionTok{nc} \AttributeTok{{-}U}\NormalTok{ /tmp/echo.sock}
\end{Highlighting}
\end{Shaded}

\subsection{\texorpdfstring{\texttt{net.getDefaultAutoSelectFamily()}}{net.getDefaultAutoSelectFamily()}}\label{net.getdefaultautoselectfamily}

Gets the current default value of the \texttt{autoSelectFamily}~option
of
\hyperref[socketconnectoptions-connectlistener]{\texttt{socket.connect(options)}}.
The initial default value is \texttt{true}, unless the command line
option \texttt{-\/-no-network-family-autoselection} is provided.

\begin{itemize}
\tightlist
\item
  Returns: \{boolean\} The current default value of the
  \texttt{autoSelectFamily} option.
\end{itemize}

\subsection{\texorpdfstring{\texttt{net.setDefaultAutoSelectFamily(value)}}{net.setDefaultAutoSelectFamily(value)}}\label{net.setdefaultautoselectfamilyvalue}

Sets the default value of the \texttt{autoSelectFamily}~option of
\hyperref[socketconnectoptions-connectlistener]{\texttt{socket.connect(options)}}.

\begin{itemize}
\tightlist
\item
  \texttt{value} \{boolean\}~The new default value. The initial default
  value is \texttt{false}.
\end{itemize}

\subsection{\texorpdfstring{\texttt{net.getDefaultAutoSelectFamilyAttemptTimeout()}}{net.getDefaultAutoSelectFamilyAttemptTimeout()}}\label{net.getdefaultautoselectfamilyattempttimeout}

Gets the current default value of the
\texttt{autoSelectFamilyAttemptTimeout}~option of
\hyperref[socketconnectoptions-connectlistener]{\texttt{socket.connect(options)}}.
The initial default value is \texttt{250}.

\begin{itemize}
\tightlist
\item
  Returns: \{number\} The current default value of the
  \texttt{autoSelectFamilyAttemptTimeout} option.
\end{itemize}

\subsection{\texorpdfstring{\texttt{net.setDefaultAutoSelectFamilyAttemptTimeout(value)}}{net.setDefaultAutoSelectFamilyAttemptTimeout(value)}}\label{net.setdefaultautoselectfamilyattempttimeoutvalue}

Sets the default value of the
\texttt{autoSelectFamilyAttemptTimeout}~option of
\hyperref[socketconnectoptions-connectlistener]{\texttt{socket.connect(options)}}.

\begin{itemize}
\tightlist
\item
  \texttt{value} \{number\}~The new default value, which must be a
  positive number. If the number is less than \texttt{10}, the value
  \texttt{10} is used instead. The initial default value is
  \texttt{250}.
\end{itemize}

\subsection{\texorpdfstring{\texttt{net.isIP(input)}}{net.isIP(input)}}\label{net.isipinput}

\begin{itemize}
\tightlist
\item
  \texttt{input} \{string\}
\item
  Returns: \{integer\}
\end{itemize}

Returns \texttt{6} if \texttt{input} is an IPv6 address. Returns
\texttt{4} if \texttt{input} is an IPv4 address in
\href{https://en.wikipedia.org/wiki/Dot-decimal_notation}{dot-decimal
notation} with no leading zeroes. Otherwise, returns \texttt{0}.

\begin{Shaded}
\begin{Highlighting}[]
\NormalTok{net}\OperatorTok{.}\FunctionTok{isIP}\NormalTok{(}\StringTok{\textquotesingle{}::1\textquotesingle{}}\NormalTok{)}\OperatorTok{;} \CommentTok{// returns 6}
\NormalTok{net}\OperatorTok{.}\FunctionTok{isIP}\NormalTok{(}\StringTok{\textquotesingle{}127.0.0.1\textquotesingle{}}\NormalTok{)}\OperatorTok{;} \CommentTok{// returns 4}
\NormalTok{net}\OperatorTok{.}\FunctionTok{isIP}\NormalTok{(}\StringTok{\textquotesingle{}127.000.000.001\textquotesingle{}}\NormalTok{)}\OperatorTok{;} \CommentTok{// returns 0}
\NormalTok{net}\OperatorTok{.}\FunctionTok{isIP}\NormalTok{(}\StringTok{\textquotesingle{}127.0.0.1/24\textquotesingle{}}\NormalTok{)}\OperatorTok{;} \CommentTok{// returns 0}
\NormalTok{net}\OperatorTok{.}\FunctionTok{isIP}\NormalTok{(}\StringTok{\textquotesingle{}fhqwhgads\textquotesingle{}}\NormalTok{)}\OperatorTok{;} \CommentTok{// returns 0}
\end{Highlighting}
\end{Shaded}

\subsection{\texorpdfstring{\texttt{net.isIPv4(input)}}{net.isIPv4(input)}}\label{net.isipv4input}

\begin{itemize}
\tightlist
\item
  \texttt{input} \{string\}
\item
  Returns: \{boolean\}
\end{itemize}

Returns \texttt{true} if \texttt{input} is an IPv4 address in
\href{https://en.wikipedia.org/wiki/Dot-decimal_notation}{dot-decimal
notation} with no leading zeroes. Otherwise, returns \texttt{false}.

\begin{Shaded}
\begin{Highlighting}[]
\NormalTok{net}\OperatorTok{.}\FunctionTok{isIPv4}\NormalTok{(}\StringTok{\textquotesingle{}127.0.0.1\textquotesingle{}}\NormalTok{)}\OperatorTok{;} \CommentTok{// returns true}
\NormalTok{net}\OperatorTok{.}\FunctionTok{isIPv4}\NormalTok{(}\StringTok{\textquotesingle{}127.000.000.001\textquotesingle{}}\NormalTok{)}\OperatorTok{;} \CommentTok{// returns false}
\NormalTok{net}\OperatorTok{.}\FunctionTok{isIPv4}\NormalTok{(}\StringTok{\textquotesingle{}127.0.0.1/24\textquotesingle{}}\NormalTok{)}\OperatorTok{;} \CommentTok{// returns false}
\NormalTok{net}\OperatorTok{.}\FunctionTok{isIPv4}\NormalTok{(}\StringTok{\textquotesingle{}fhqwhgads\textquotesingle{}}\NormalTok{)}\OperatorTok{;} \CommentTok{// returns false}
\end{Highlighting}
\end{Shaded}

\subsection{\texorpdfstring{\texttt{net.isIPv6(input)}}{net.isIPv6(input)}}\label{net.isipv6input}

\begin{itemize}
\tightlist
\item
  \texttt{input} \{string\}
\item
  Returns: \{boolean\}
\end{itemize}

Returns \texttt{true} if \texttt{input} is an IPv6 address. Otherwise,
returns \texttt{false}.

\begin{Shaded}
\begin{Highlighting}[]
\NormalTok{net}\OperatorTok{.}\FunctionTok{isIPv6}\NormalTok{(}\StringTok{\textquotesingle{}::1\textquotesingle{}}\NormalTok{)}\OperatorTok{;} \CommentTok{// returns true}
\NormalTok{net}\OperatorTok{.}\FunctionTok{isIPv6}\NormalTok{(}\StringTok{\textquotesingle{}fhqwhgads\textquotesingle{}}\NormalTok{)}\OperatorTok{;} \CommentTok{// returns false}
\end{Highlighting}
\end{Shaded}
