\section{Child process}\label{child-process}

\begin{quote}
Stability: 2 - Stable
\end{quote}

The \texttt{node:child\_process} module provides the ability to spawn
subprocesses in a manner that is similar, but not identical, to
popen(3). This capability is primarily provided by the
\hyperref[child_processspawncommand-args-options]{\texttt{child\_process.spawn()}}
function:

\begin{Shaded}
\begin{Highlighting}[]
\KeywordTok{const}\NormalTok{ \{ spawn \} }\OperatorTok{=} \PreprocessorTok{require}\NormalTok{(}\StringTok{\textquotesingle{}node:child\_process\textquotesingle{}}\NormalTok{)}\OperatorTok{;}
\KeywordTok{const}\NormalTok{ ls }\OperatorTok{=} \FunctionTok{spawn}\NormalTok{(}\StringTok{\textquotesingle{}ls\textquotesingle{}}\OperatorTok{,}\NormalTok{ [}\StringTok{\textquotesingle{}{-}lh\textquotesingle{}}\OperatorTok{,} \StringTok{\textquotesingle{}/usr\textquotesingle{}}\NormalTok{])}\OperatorTok{;}

\NormalTok{ls}\OperatorTok{.}\AttributeTok{stdout}\OperatorTok{.}\FunctionTok{on}\NormalTok{(}\StringTok{\textquotesingle{}data\textquotesingle{}}\OperatorTok{,}\NormalTok{ (data) }\KeywordTok{=\textgreater{}}\NormalTok{ \{}
  \BuiltInTok{console}\OperatorTok{.}\FunctionTok{log}\NormalTok{(}\VerbatimStringTok{\textasciigrave{}stdout: }\SpecialCharTok{$\{}\NormalTok{data}\SpecialCharTok{\}}\VerbatimStringTok{\textasciigrave{}}\NormalTok{)}\OperatorTok{;}
\NormalTok{\})}\OperatorTok{;}

\NormalTok{ls}\OperatorTok{.}\AttributeTok{stderr}\OperatorTok{.}\FunctionTok{on}\NormalTok{(}\StringTok{\textquotesingle{}data\textquotesingle{}}\OperatorTok{,}\NormalTok{ (data) }\KeywordTok{=\textgreater{}}\NormalTok{ \{}
  \BuiltInTok{console}\OperatorTok{.}\FunctionTok{error}\NormalTok{(}\VerbatimStringTok{\textasciigrave{}stderr: }\SpecialCharTok{$\{}\NormalTok{data}\SpecialCharTok{\}}\VerbatimStringTok{\textasciigrave{}}\NormalTok{)}\OperatorTok{;}
\NormalTok{\})}\OperatorTok{;}

\NormalTok{ls}\OperatorTok{.}\FunctionTok{on}\NormalTok{(}\StringTok{\textquotesingle{}close\textquotesingle{}}\OperatorTok{,}\NormalTok{ (code) }\KeywordTok{=\textgreater{}}\NormalTok{ \{}
  \BuiltInTok{console}\OperatorTok{.}\FunctionTok{log}\NormalTok{(}\VerbatimStringTok{\textasciigrave{}child process exited with code }\SpecialCharTok{$\{}\NormalTok{code}\SpecialCharTok{\}}\VerbatimStringTok{\textasciigrave{}}\NormalTok{)}\OperatorTok{;}
\NormalTok{\})}\OperatorTok{;}
\end{Highlighting}
\end{Shaded}

By default, pipes for \texttt{stdin}, \texttt{stdout}, and
\texttt{stderr} are established between the parent Node.js process and
the spawned subprocess. These pipes have limited (and platform-specific)
capacity. If the subprocess writes to stdout in excess of that limit
without the output being captured, the subprocess blocks waiting for the
pipe buffer to accept more data. This is identical to the behavior of
pipes in the shell. Use the
\texttt{\{\ stdio:\ \textquotesingle{}ignore\textquotesingle{}\ \}}
option if the output will not be consumed.

The command lookup is performed using the \texttt{options.env.PATH}
environment variable if \texttt{env} is in the \texttt{options} object.
Otherwise, \texttt{process.env.PATH} is used. If \texttt{options.env} is
set without \texttt{PATH}, lookup on Unix is performed on a default
search path search of \texttt{/usr/bin:/bin} (see your operating
system's manual for execvpe/execvp), on Windows the current processes
environment variable \texttt{PATH} is used.

On Windows, environment variables are case-insensitive. Node.js
lexicographically sorts the \texttt{env} keys and uses the first one
that case-insensitively matches. Only first (in lexicographic order)
entry will be passed to the subprocess. This might lead to issues on
Windows when passing objects to the \texttt{env} option that have
multiple variants of the same key, such as \texttt{PATH} and
\texttt{Path}.

The
\hyperref[child_processspawncommand-args-options]{\texttt{child\_process.spawn()}}
method spawns the child process asynchronously, without blocking the
Node.js event loop. The
\hyperref[child_processspawnsynccommand-args-options]{\texttt{child\_process.spawnSync()}}
function provides equivalent functionality in a synchronous manner that
blocks the event loop until the spawned process either exits or is
terminated.

For convenience, the \texttt{node:child\_process} module provides a
handful of synchronous and asynchronous alternatives to
\hyperref[child_processspawncommand-args-options]{\texttt{child\_process.spawn()}}
and
\hyperref[child_processspawnsynccommand-args-options]{\texttt{child\_process.spawnSync()}}.
Each of these alternatives are implemented on top of
\hyperref[child_processspawncommand-args-options]{\texttt{child\_process.spawn()}}
or
\hyperref[child_processspawnsynccommand-args-options]{\texttt{child\_process.spawnSync()}}.

\begin{itemize}
\tightlist
\item
  \hyperref[child_processexeccommand-options-callback]{\texttt{child\_process.exec()}}:
  spawns a shell and runs a command within that shell, passing the
  \texttt{stdout} and \texttt{stderr} to a callback function when
  complete.
\item
  \hyperref[child_processexecfilefile-args-options-callback]{\texttt{child\_process.execFile()}}:
  similar to
  \hyperref[child_processexeccommand-options-callback]{\texttt{child\_process.exec()}}
  except that it spawns the command directly without first spawning a
  shell by default.
\item
  \hyperref[child_processforkmodulepath-args-options]{\texttt{child\_process.fork()}}:
  spawns a new Node.js process and invokes a specified module with an
  IPC communication channel established that allows sending messages
  between parent and child.
\item
  \hyperref[child_processexecsynccommand-options]{\texttt{child\_process.execSync()}}:
  a synchronous version of
  \hyperref[child_processexeccommand-options-callback]{\texttt{child\_process.exec()}}
  that will block the Node.js event loop.
\item
  \hyperref[child_processexecfilesyncfile-args-options]{\texttt{child\_process.execFileSync()}}:
  a synchronous version of
  \hyperref[child_processexecfilefile-args-options-callback]{\texttt{child\_process.execFile()}}
  that will block the Node.js event loop.
\end{itemize}

For certain use cases, such as automating shell scripts, the
\hyperref[synchronous-process-creation]{synchronous counterparts} may be
more convenient. In many cases, however, the synchronous methods can
have significant impact on performance due to stalling the event loop
while spawned processes complete.

\subsection{Asynchronous process
creation}\label{asynchronous-process-creation}

The
\hyperref[child_processspawncommand-args-options]{\texttt{child\_process.spawn()}},
\hyperref[child_processforkmodulepath-args-options]{\texttt{child\_process.fork()}},
\hyperref[child_processexeccommand-options-callback]{\texttt{child\_process.exec()}},
and
\hyperref[child_processexecfilefile-args-options-callback]{\texttt{child\_process.execFile()}}
methods all follow the idiomatic asynchronous programming pattern
typical of other Node.js APIs.

Each of the methods returns a
\hyperref[class-childprocess]{\texttt{ChildProcess}} instance. These
objects implement the Node.js
\href{events.md\#class-eventemitter}{\texttt{EventEmitter}} API,
allowing the parent process to register listener functions that are
called when certain events occur during the life cycle of the child
process.

The
\hyperref[child_processexeccommand-options-callback]{\texttt{child\_process.exec()}}
and
\hyperref[child_processexecfilefile-args-options-callback]{\texttt{child\_process.execFile()}}
methods additionally allow for an optional \texttt{callback} function to
be specified that is invoked when the child process terminates.

\subsubsection{\texorpdfstring{Spawning \texttt{.bat} and \texttt{.cmd}
files on
Windows}{Spawning .bat and .cmd files on Windows}}\label{spawning-.bat-and-.cmd-files-on-windows}

The importance of the distinction between
\hyperref[child_processexeccommand-options-callback]{\texttt{child\_process.exec()}}
and
\hyperref[child_processexecfilefile-args-options-callback]{\texttt{child\_process.execFile()}}
can vary based on platform. On Unix-type operating systems (Unix, Linux,
macOS)
\hyperref[child_processexecfilefile-args-options-callback]{\texttt{child\_process.execFile()}}
can be more efficient because it does not spawn a shell by default. On
Windows, however, \texttt{.bat} and \texttt{.cmd} files are not
executable on their own without a terminal, and therefore cannot be
launched using
\hyperref[child_processexecfilefile-args-options-callback]{\texttt{child\_process.execFile()}}.
When running on Windows, \texttt{.bat} and \texttt{.cmd} files can be
invoked using
\hyperref[child_processspawncommand-args-options]{\texttt{child\_process.spawn()}}
with the \texttt{shell} option set, with
\hyperref[child_processexeccommand-options-callback]{\texttt{child\_process.exec()}},
or by spawning \texttt{cmd.exe} and passing the \texttt{.bat} or
\texttt{.cmd} file as an argument (which is what the \texttt{shell}
option and
\hyperref[child_processexeccommand-options-callback]{\texttt{child\_process.exec()}}
do). In any case, if the script filename contains spaces it needs to be
quoted.

\begin{Shaded}
\begin{Highlighting}[]
\CommentTok{// On Windows Only...}
\KeywordTok{const}\NormalTok{ \{ spawn \} }\OperatorTok{=} \PreprocessorTok{require}\NormalTok{(}\StringTok{\textquotesingle{}node:child\_process\textquotesingle{}}\NormalTok{)}\OperatorTok{;}
\KeywordTok{const}\NormalTok{ bat }\OperatorTok{=} \FunctionTok{spawn}\NormalTok{(}\StringTok{\textquotesingle{}cmd.exe\textquotesingle{}}\OperatorTok{,}\NormalTok{ [}\StringTok{\textquotesingle{}/c\textquotesingle{}}\OperatorTok{,} \StringTok{\textquotesingle{}my.bat\textquotesingle{}}\NormalTok{])}\OperatorTok{;}

\NormalTok{bat}\OperatorTok{.}\AttributeTok{stdout}\OperatorTok{.}\FunctionTok{on}\NormalTok{(}\StringTok{\textquotesingle{}data\textquotesingle{}}\OperatorTok{,}\NormalTok{ (data) }\KeywordTok{=\textgreater{}}\NormalTok{ \{}
  \BuiltInTok{console}\OperatorTok{.}\FunctionTok{log}\NormalTok{(data}\OperatorTok{.}\FunctionTok{toString}\NormalTok{())}\OperatorTok{;}
\NormalTok{\})}\OperatorTok{;}

\NormalTok{bat}\OperatorTok{.}\AttributeTok{stderr}\OperatorTok{.}\FunctionTok{on}\NormalTok{(}\StringTok{\textquotesingle{}data\textquotesingle{}}\OperatorTok{,}\NormalTok{ (data) }\KeywordTok{=\textgreater{}}\NormalTok{ \{}
  \BuiltInTok{console}\OperatorTok{.}\FunctionTok{error}\NormalTok{(data}\OperatorTok{.}\FunctionTok{toString}\NormalTok{())}\OperatorTok{;}
\NormalTok{\})}\OperatorTok{;}

\NormalTok{bat}\OperatorTok{.}\FunctionTok{on}\NormalTok{(}\StringTok{\textquotesingle{}exit\textquotesingle{}}\OperatorTok{,}\NormalTok{ (code) }\KeywordTok{=\textgreater{}}\NormalTok{ \{}
  \BuiltInTok{console}\OperatorTok{.}\FunctionTok{log}\NormalTok{(}\VerbatimStringTok{\textasciigrave{}Child exited with code }\SpecialCharTok{$\{}\NormalTok{code}\SpecialCharTok{\}}\VerbatimStringTok{\textasciigrave{}}\NormalTok{)}\OperatorTok{;}
\NormalTok{\})}\OperatorTok{;}
\end{Highlighting}
\end{Shaded}

\begin{Shaded}
\begin{Highlighting}[]
\CommentTok{// OR...}
\KeywordTok{const}\NormalTok{ \{ exec}\OperatorTok{,}\NormalTok{ spawn \} }\OperatorTok{=} \PreprocessorTok{require}\NormalTok{(}\StringTok{\textquotesingle{}node:child\_process\textquotesingle{}}\NormalTok{)}\OperatorTok{;}
\FunctionTok{exec}\NormalTok{(}\StringTok{\textquotesingle{}my.bat\textquotesingle{}}\OperatorTok{,}\NormalTok{ (err}\OperatorTok{,}\NormalTok{ stdout}\OperatorTok{,}\NormalTok{ stderr) }\KeywordTok{=\textgreater{}}\NormalTok{ \{}
  \ControlFlowTok{if}\NormalTok{ (err) \{}
    \BuiltInTok{console}\OperatorTok{.}\FunctionTok{error}\NormalTok{(err)}\OperatorTok{;}
    \ControlFlowTok{return}\OperatorTok{;}
\NormalTok{  \}}
  \BuiltInTok{console}\OperatorTok{.}\FunctionTok{log}\NormalTok{(stdout)}\OperatorTok{;}
\NormalTok{\})}\OperatorTok{;}

\CommentTok{// Script with spaces in the filename:}
\KeywordTok{const}\NormalTok{ bat }\OperatorTok{=} \FunctionTok{spawn}\NormalTok{(}\StringTok{\textquotesingle{}"my script.cmd"\textquotesingle{}}\OperatorTok{,}\NormalTok{ [}\StringTok{\textquotesingle{}a\textquotesingle{}}\OperatorTok{,} \StringTok{\textquotesingle{}b\textquotesingle{}}\NormalTok{]}\OperatorTok{,}\NormalTok{ \{ }\DataTypeTok{shell}\OperatorTok{:} \KeywordTok{true}\NormalTok{ \})}\OperatorTok{;}
\CommentTok{// or:}
\FunctionTok{exec}\NormalTok{(}\StringTok{\textquotesingle{}"my script.cmd" a b\textquotesingle{}}\OperatorTok{,}\NormalTok{ (err}\OperatorTok{,}\NormalTok{ stdout}\OperatorTok{,}\NormalTok{ stderr) }\KeywordTok{=\textgreater{}}\NormalTok{ \{}
  \CommentTok{// ...}
\NormalTok{\})}\OperatorTok{;}
\end{Highlighting}
\end{Shaded}

\subsubsection{\texorpdfstring{\texttt{child\_process.exec(command{[},\ options{]}{[},\ callback{]})}}{child\_process.exec(command{[}, options{]}{[}, callback{]})}}\label{child_process.execcommand-options-callback}

\begin{itemize}
\tightlist
\item
  \texttt{command} \{string\} The command to run, with space-separated
  arguments.
\item
  \texttt{options} \{Object\}

  \begin{itemize}
  \tightlist
  \item
    \texttt{cwd} \{string\textbar URL\} Current working directory of the
    child process. \textbf{Default:} \texttt{process.cwd()}.
  \item
    \texttt{env} \{Object\} Environment key-value pairs.
    \textbf{Default:} \texttt{process.env}.
  \item
    \texttt{encoding} \{string\} \textbf{Default:}
    \texttt{\textquotesingle{}utf8\textquotesingle{}}
  \item
    \texttt{shell} \{string\} Shell to execute the command with. See
    \hyperref[shell-requirements]{Shell requirements} and
    \hyperref[default-windows-shell]{Default Windows shell}.
    \textbf{Default:}
    \texttt{\textquotesingle{}/bin/sh\textquotesingle{}} on Unix,
    \texttt{process.env.ComSpec} on Windows.
  \item
    \texttt{signal} \{AbortSignal\} allows aborting the child process
    using an AbortSignal.
  \item
    \texttt{timeout} \{number\} \textbf{Default:} \texttt{0}
  \item
    \texttt{maxBuffer} \{number\} Largest amount of data in bytes
    allowed on stdout or stderr. If exceeded, the child process is
    terminated and any output is truncated. See caveat at
    \hyperref[maxbuffer-and-unicode]{\texttt{maxBuffer} and Unicode}.
    \textbf{Default:} \texttt{1024\ *\ 1024}.
  \item
    \texttt{killSignal} \{string\textbar integer\} \textbf{Default:}
    \texttt{\textquotesingle{}SIGTERM\textquotesingle{}}
  \item
    \texttt{uid} \{number\} Sets the user identity of the process (see
    setuid(2)).
  \item
    \texttt{gid} \{number\} Sets the group identity of the process (see
    setgid(2)).
  \item
    \texttt{windowsHide} \{boolean\} Hide the subprocess console window
    that would normally be created on Windows systems. \textbf{Default:}
    \texttt{false}.
  \end{itemize}
\item
  \texttt{callback} \{Function\} called with the output when process
  terminates.

  \begin{itemize}
  \tightlist
  \item
    \texttt{error} \{Error\}
  \item
    \texttt{stdout} \{string\textbar Buffer\}
  \item
    \texttt{stderr} \{string\textbar Buffer\}
  \end{itemize}
\item
  Returns: \{ChildProcess\}
\end{itemize}

Spawns a shell then executes the \texttt{command} within that shell,
buffering any generated output. The \texttt{command} string passed to
the exec function is processed directly by the shell and special
characters (vary based on
\href{https://en.wikipedia.org/wiki/List_of_command-line_interpreters}{shell})
need to be dealt with accordingly:

\begin{Shaded}
\begin{Highlighting}[]
\KeywordTok{const}\NormalTok{ \{ exec \} }\OperatorTok{=} \PreprocessorTok{require}\NormalTok{(}\StringTok{\textquotesingle{}node:child\_process\textquotesingle{}}\NormalTok{)}\OperatorTok{;}

\FunctionTok{exec}\NormalTok{(}\StringTok{\textquotesingle{}"/path/to/test file/test.sh" arg1 arg2\textquotesingle{}}\NormalTok{)}\OperatorTok{;}
\CommentTok{// Double quotes are used so that the space in the path is not interpreted as}
\CommentTok{// a delimiter of multiple arguments.}

\FunctionTok{exec}\NormalTok{(}\StringTok{\textquotesingle{}echo "The }\SpecialCharTok{\textbackslash{}\textbackslash{}}\StringTok{$HOME variable is $HOME"\textquotesingle{}}\NormalTok{)}\OperatorTok{;}
\CommentTok{// The $HOME variable is escaped in the first instance, but not in the second.}
\end{Highlighting}
\end{Shaded}

\textbf{Never pass unsanitized user input to this function. Any input
containing shell metacharacters may be used to trigger arbitrary command
execution.}

If a \texttt{callback} function is provided, it is called with the
arguments \texttt{(error,\ stdout,\ stderr)}. On success, \texttt{error}
will be \texttt{null}. On error, \texttt{error} will be an instance of
\href{errors.md\#class-error}{\texttt{Error}}. The \texttt{error.code}
property will be the exit code of the process. By convention, any exit
code other than \texttt{0} indicates an error. \texttt{error.signal}
will be the signal that terminated the process.

The \texttt{stdout} and \texttt{stderr} arguments passed to the callback
will contain the stdout and stderr output of the child process. By
default, Node.js will decode the output as UTF-8 and pass strings to the
callback. The \texttt{encoding} option can be used to specify the
character encoding used to decode the stdout and stderr output. If
\texttt{encoding} is
\texttt{\textquotesingle{}buffer\textquotesingle{}}, or an unrecognized
character encoding, \texttt{Buffer} objects will be passed to the
callback instead.

\begin{Shaded}
\begin{Highlighting}[]
\KeywordTok{const}\NormalTok{ \{ exec \} }\OperatorTok{=} \PreprocessorTok{require}\NormalTok{(}\StringTok{\textquotesingle{}node:child\_process\textquotesingle{}}\NormalTok{)}\OperatorTok{;}
\FunctionTok{exec}\NormalTok{(}\StringTok{\textquotesingle{}cat *.js missing\_file | wc {-}l\textquotesingle{}}\OperatorTok{,}\NormalTok{ (error}\OperatorTok{,}\NormalTok{ stdout}\OperatorTok{,}\NormalTok{ stderr) }\KeywordTok{=\textgreater{}}\NormalTok{ \{}
  \ControlFlowTok{if}\NormalTok{ (error) \{}
    \BuiltInTok{console}\OperatorTok{.}\FunctionTok{error}\NormalTok{(}\VerbatimStringTok{\textasciigrave{}exec error: }\SpecialCharTok{$\{}\NormalTok{error}\SpecialCharTok{\}}\VerbatimStringTok{\textasciigrave{}}\NormalTok{)}\OperatorTok{;}
    \ControlFlowTok{return}\OperatorTok{;}
\NormalTok{  \}}
  \BuiltInTok{console}\OperatorTok{.}\FunctionTok{log}\NormalTok{(}\VerbatimStringTok{\textasciigrave{}stdout: }\SpecialCharTok{$\{}\NormalTok{stdout}\SpecialCharTok{\}}\VerbatimStringTok{\textasciigrave{}}\NormalTok{)}\OperatorTok{;}
  \BuiltInTok{console}\OperatorTok{.}\FunctionTok{error}\NormalTok{(}\VerbatimStringTok{\textasciigrave{}stderr: }\SpecialCharTok{$\{}\NormalTok{stderr}\SpecialCharTok{\}}\VerbatimStringTok{\textasciigrave{}}\NormalTok{)}\OperatorTok{;}
\NormalTok{\})}\OperatorTok{;}
\end{Highlighting}
\end{Shaded}

If \texttt{timeout} is greater than \texttt{0}, the parent will send the
signal identified by the \texttt{killSignal} property (the default is
\texttt{\textquotesingle{}SIGTERM\textquotesingle{}}) if the child runs
longer than \texttt{timeout} milliseconds.

Unlike the exec(3) POSIX system call, \texttt{child\_process.exec()}
does not replace the existing process and uses a shell to execute the
command.

If this method is invoked as its
\href{util.md\#utilpromisifyoriginal}{\texttt{util.promisify()}}ed
version, it returns a \texttt{Promise} for an \texttt{Object} with
\texttt{stdout} and \texttt{stderr} properties. The returned
\texttt{ChildProcess} instance is attached to the \texttt{Promise} as a
\texttt{child} property. In case of an error (including any error
resulting in an exit code other than 0), a rejected promise is returned,
with the same \texttt{error} object given in the callback, but with two
additional properties \texttt{stdout} and \texttt{stderr}.

\begin{Shaded}
\begin{Highlighting}[]
\KeywordTok{const}\NormalTok{ util }\OperatorTok{=} \PreprocessorTok{require}\NormalTok{(}\StringTok{\textquotesingle{}node:util\textquotesingle{}}\NormalTok{)}\OperatorTok{;}
\KeywordTok{const}\NormalTok{ exec }\OperatorTok{=}\NormalTok{ util}\OperatorTok{.}\FunctionTok{promisify}\NormalTok{(}\PreprocessorTok{require}\NormalTok{(}\StringTok{\textquotesingle{}node:child\_process\textquotesingle{}}\NormalTok{)}\OperatorTok{.}\AttributeTok{exec}\NormalTok{)}\OperatorTok{;}

\KeywordTok{async} \KeywordTok{function} \FunctionTok{lsExample}\NormalTok{() \{}
  \KeywordTok{const}\NormalTok{ \{ stdout}\OperatorTok{,}\NormalTok{ stderr \} }\OperatorTok{=} \ControlFlowTok{await} \FunctionTok{exec}\NormalTok{(}\StringTok{\textquotesingle{}ls\textquotesingle{}}\NormalTok{)}\OperatorTok{;}
  \BuiltInTok{console}\OperatorTok{.}\FunctionTok{log}\NormalTok{(}\StringTok{\textquotesingle{}stdout:\textquotesingle{}}\OperatorTok{,}\NormalTok{ stdout)}\OperatorTok{;}
  \BuiltInTok{console}\OperatorTok{.}\FunctionTok{error}\NormalTok{(}\StringTok{\textquotesingle{}stderr:\textquotesingle{}}\OperatorTok{,}\NormalTok{ stderr)}\OperatorTok{;}
\NormalTok{\}}
\FunctionTok{lsExample}\NormalTok{()}\OperatorTok{;}
\end{Highlighting}
\end{Shaded}

If the \texttt{signal} option is enabled, calling \texttt{.abort()} on
the corresponding \texttt{AbortController} is similar to calling
\texttt{.kill()} on the child process except the error passed to the
callback will be an \texttt{AbortError}:

\begin{Shaded}
\begin{Highlighting}[]
\KeywordTok{const}\NormalTok{ \{ exec \} }\OperatorTok{=} \PreprocessorTok{require}\NormalTok{(}\StringTok{\textquotesingle{}node:child\_process\textquotesingle{}}\NormalTok{)}\OperatorTok{;}
\KeywordTok{const}\NormalTok{ controller }\OperatorTok{=} \KeywordTok{new} \FunctionTok{AbortController}\NormalTok{()}\OperatorTok{;}
\KeywordTok{const}\NormalTok{ \{ signal \} }\OperatorTok{=}\NormalTok{ controller}\OperatorTok{;}
\KeywordTok{const}\NormalTok{ child }\OperatorTok{=} \FunctionTok{exec}\NormalTok{(}\StringTok{\textquotesingle{}grep ssh\textquotesingle{}}\OperatorTok{,}\NormalTok{ \{ signal \}}\OperatorTok{,}\NormalTok{ (error) }\KeywordTok{=\textgreater{}}\NormalTok{ \{}
  \BuiltInTok{console}\OperatorTok{.}\FunctionTok{error}\NormalTok{(error)}\OperatorTok{;} \CommentTok{// an AbortError}
\NormalTok{\})}\OperatorTok{;}
\NormalTok{controller}\OperatorTok{.}\FunctionTok{abort}\NormalTok{()}\OperatorTok{;}
\end{Highlighting}
\end{Shaded}

\subsubsection{\texorpdfstring{\texttt{child\_process.execFile(file{[},\ args{]}{[},\ options{]}{[},\ callback{]})}}{child\_process.execFile(file{[}, args{]}{[}, options{]}{[}, callback{]})}}\label{child_process.execfilefile-args-options-callback}

\begin{itemize}
\tightlist
\item
  \texttt{file} \{string\} The name or path of the executable file to
  run.
\item
  \texttt{args} \{string{[}{]}\} List of string arguments.
\item
  \texttt{options} \{Object\}

  \begin{itemize}
  \tightlist
  \item
    \texttt{cwd} \{string\textbar URL\} Current working directory of the
    child process.
  \item
    \texttt{env} \{Object\} Environment key-value pairs.
    \textbf{Default:} \texttt{process.env}.
  \item
    \texttt{encoding} \{string\} \textbf{Default:}
    \texttt{\textquotesingle{}utf8\textquotesingle{}}
  \item
    \texttt{timeout} \{number\} \textbf{Default:} \texttt{0}
  \item
    \texttt{maxBuffer} \{number\} Largest amount of data in bytes
    allowed on stdout or stderr. If exceeded, the child process is
    terminated and any output is truncated. See caveat at
    \hyperref[maxbuffer-and-unicode]{\texttt{maxBuffer} and Unicode}.
    \textbf{Default:} \texttt{1024\ *\ 1024}.
  \item
    \texttt{killSignal} \{string\textbar integer\} \textbf{Default:}
    \texttt{\textquotesingle{}SIGTERM\textquotesingle{}}
  \item
    \texttt{uid} \{number\} Sets the user identity of the process (see
    setuid(2)).
  \item
    \texttt{gid} \{number\} Sets the group identity of the process (see
    setgid(2)).
  \item
    \texttt{windowsHide} \{boolean\} Hide the subprocess console window
    that would normally be created on Windows systems. \textbf{Default:}
    \texttt{false}.
  \item
    \texttt{windowsVerbatimArguments} \{boolean\} No quoting or escaping
    of arguments is done on Windows. Ignored on Unix. \textbf{Default:}
    \texttt{false}.
  \item
    \texttt{shell} \{boolean\textbar string\} If \texttt{true}, runs
    \texttt{command} inside of a shell. Uses
    \texttt{\textquotesingle{}/bin/sh\textquotesingle{}} on Unix, and
    \texttt{process.env.ComSpec} on Windows. A different shell can be
    specified as a string. See \hyperref[shell-requirements]{Shell
    requirements} and \hyperref[default-windows-shell]{Default Windows
    shell}. \textbf{Default:} \texttt{false} (no shell).
  \item
    \texttt{signal} \{AbortSignal\} allows aborting the child process
    using an AbortSignal.
  \end{itemize}
\item
  \texttt{callback} \{Function\} Called with the output when process
  terminates.

  \begin{itemize}
  \tightlist
  \item
    \texttt{error} \{Error\}
  \item
    \texttt{stdout} \{string\textbar Buffer\}
  \item
    \texttt{stderr} \{string\textbar Buffer\}
  \end{itemize}
\item
  Returns: \{ChildProcess\}
\end{itemize}

The \texttt{child\_process.execFile()} function is similar to
\hyperref[child_processexeccommand-options-callback]{\texttt{child\_process.exec()}}
except that it does not spawn a shell by default. Rather, the specified
executable \texttt{file} is spawned directly as a new process making it
slightly more efficient than
\hyperref[child_processexeccommand-options-callback]{\texttt{child\_process.exec()}}.

The same options as
\hyperref[child_processexeccommand-options-callback]{\texttt{child\_process.exec()}}
are supported. Since a shell is not spawned, behaviors such as I/O
redirection and file globbing are not supported.

\begin{Shaded}
\begin{Highlighting}[]
\KeywordTok{const}\NormalTok{ \{ execFile \} }\OperatorTok{=} \PreprocessorTok{require}\NormalTok{(}\StringTok{\textquotesingle{}node:child\_process\textquotesingle{}}\NormalTok{)}\OperatorTok{;}
\KeywordTok{const}\NormalTok{ child }\OperatorTok{=} \FunctionTok{execFile}\NormalTok{(}\StringTok{\textquotesingle{}node\textquotesingle{}}\OperatorTok{,}\NormalTok{ [}\StringTok{\textquotesingle{}{-}{-}version\textquotesingle{}}\NormalTok{]}\OperatorTok{,}\NormalTok{ (error}\OperatorTok{,}\NormalTok{ stdout}\OperatorTok{,}\NormalTok{ stderr) }\KeywordTok{=\textgreater{}}\NormalTok{ \{}
  \ControlFlowTok{if}\NormalTok{ (error) \{}
    \ControlFlowTok{throw}\NormalTok{ error}\OperatorTok{;}
\NormalTok{  \}}
  \BuiltInTok{console}\OperatorTok{.}\FunctionTok{log}\NormalTok{(stdout)}\OperatorTok{;}
\NormalTok{\})}\OperatorTok{;}
\end{Highlighting}
\end{Shaded}

The \texttt{stdout} and \texttt{stderr} arguments passed to the callback
will contain the stdout and stderr output of the child process. By
default, Node.js will decode the output as UTF-8 and pass strings to the
callback. The \texttt{encoding} option can be used to specify the
character encoding used to decode the stdout and stderr output. If
\texttt{encoding} is
\texttt{\textquotesingle{}buffer\textquotesingle{}}, or an unrecognized
character encoding, \texttt{Buffer} objects will be passed to the
callback instead.

If this method is invoked as its
\href{util.md\#utilpromisifyoriginal}{\texttt{util.promisify()}}ed
version, it returns a \texttt{Promise} for an \texttt{Object} with
\texttt{stdout} and \texttt{stderr} properties. The returned
\texttt{ChildProcess} instance is attached to the \texttt{Promise} as a
\texttt{child} property. In case of an error (including any error
resulting in an exit code other than 0), a rejected promise is returned,
with the same \texttt{error} object given in the callback, but with two
additional properties \texttt{stdout} and \texttt{stderr}.

\begin{Shaded}
\begin{Highlighting}[]
\KeywordTok{const}\NormalTok{ util }\OperatorTok{=} \PreprocessorTok{require}\NormalTok{(}\StringTok{\textquotesingle{}node:util\textquotesingle{}}\NormalTok{)}\OperatorTok{;}
\KeywordTok{const}\NormalTok{ execFile }\OperatorTok{=}\NormalTok{ util}\OperatorTok{.}\FunctionTok{promisify}\NormalTok{(}\PreprocessorTok{require}\NormalTok{(}\StringTok{\textquotesingle{}node:child\_process\textquotesingle{}}\NormalTok{)}\OperatorTok{.}\AttributeTok{execFile}\NormalTok{)}\OperatorTok{;}
\KeywordTok{async} \KeywordTok{function} \FunctionTok{getVersion}\NormalTok{() \{}
  \KeywordTok{const}\NormalTok{ \{ stdout \} }\OperatorTok{=} \ControlFlowTok{await} \FunctionTok{execFile}\NormalTok{(}\StringTok{\textquotesingle{}node\textquotesingle{}}\OperatorTok{,}\NormalTok{ [}\StringTok{\textquotesingle{}{-}{-}version\textquotesingle{}}\NormalTok{])}\OperatorTok{;}
  \BuiltInTok{console}\OperatorTok{.}\FunctionTok{log}\NormalTok{(stdout)}\OperatorTok{;}
\NormalTok{\}}
\FunctionTok{getVersion}\NormalTok{()}\OperatorTok{;}
\end{Highlighting}
\end{Shaded}

\textbf{If the \texttt{shell} option is enabled, do not pass unsanitized
user input to this function. Any input containing shell metacharacters
may be used to trigger arbitrary command execution.}

If the \texttt{signal} option is enabled, calling \texttt{.abort()} on
the corresponding \texttt{AbortController} is similar to calling
\texttt{.kill()} on the child process except the error passed to the
callback will be an \texttt{AbortError}:

\begin{Shaded}
\begin{Highlighting}[]
\KeywordTok{const}\NormalTok{ \{ execFile \} }\OperatorTok{=} \PreprocessorTok{require}\NormalTok{(}\StringTok{\textquotesingle{}node:child\_process\textquotesingle{}}\NormalTok{)}\OperatorTok{;}
\KeywordTok{const}\NormalTok{ controller }\OperatorTok{=} \KeywordTok{new} \FunctionTok{AbortController}\NormalTok{()}\OperatorTok{;}
\KeywordTok{const}\NormalTok{ \{ signal \} }\OperatorTok{=}\NormalTok{ controller}\OperatorTok{;}
\KeywordTok{const}\NormalTok{ child }\OperatorTok{=} \FunctionTok{execFile}\NormalTok{(}\StringTok{\textquotesingle{}node\textquotesingle{}}\OperatorTok{,}\NormalTok{ [}\StringTok{\textquotesingle{}{-}{-}version\textquotesingle{}}\NormalTok{]}\OperatorTok{,}\NormalTok{ \{ signal \}}\OperatorTok{,}\NormalTok{ (error) }\KeywordTok{=\textgreater{}}\NormalTok{ \{}
  \BuiltInTok{console}\OperatorTok{.}\FunctionTok{error}\NormalTok{(error)}\OperatorTok{;} \CommentTok{// an AbortError}
\NormalTok{\})}\OperatorTok{;}
\NormalTok{controller}\OperatorTok{.}\FunctionTok{abort}\NormalTok{()}\OperatorTok{;}
\end{Highlighting}
\end{Shaded}

\subsubsection{\texorpdfstring{\texttt{child\_process.fork(modulePath{[},\ args{]}{[},\ options{]})}}{child\_process.fork(modulePath{[}, args{]}{[}, options{]})}}\label{child_process.forkmodulepath-args-options}

\begin{itemize}
\tightlist
\item
  \texttt{modulePath} \{string\textbar URL\} The module to run in the
  child.
\item
  \texttt{args} \{string{[}{]}\} List of string arguments.
\item
  \texttt{options} \{Object\}

  \begin{itemize}
  \tightlist
  \item
    \texttt{cwd} \{string\textbar URL\} Current working directory of the
    child process.
  \item
    \texttt{detached} \{boolean\} Prepare child to run independently of
    its parent process. Specific behavior depends on the platform, see
    \hyperref[optionsdetached]{\texttt{options.detached}}).
  \item
    \texttt{env} \{Object\} Environment key-value pairs.
    \textbf{Default:} \texttt{process.env}.
  \item
    \texttt{execPath} \{string\} Executable used to create the child
    process.
  \item
    \texttt{execArgv} \{string{[}{]}\} List of string arguments passed
    to the executable. \textbf{Default:} \texttt{process.execArgv}.
  \item
    \texttt{gid} \{number\} Sets the group identity of the process (see
    setgid(2)).
  \item
    \texttt{serialization} \{string\} Specify the kind of serialization
    used for sending messages between processes. Possible values are
    \texttt{\textquotesingle{}json\textquotesingle{}} and
    \texttt{\textquotesingle{}advanced\textquotesingle{}}. See
    \hyperref[advanced-serialization]{Advanced serialization} for more
    details. \textbf{Default:}
    \texttt{\textquotesingle{}json\textquotesingle{}}.
  \item
    \texttt{signal} \{AbortSignal\} Allows closing the child process
    using an AbortSignal.
  \item
    \texttt{killSignal} \{string\textbar integer\} The signal value to
    be used when the spawned process will be killed by timeout or abort
    signal. \textbf{Default:}
    \texttt{\textquotesingle{}SIGTERM\textquotesingle{}}.
  \item
    \texttt{silent} \{boolean\} If \texttt{true}, stdin, stdout, and
    stderr of the child will be piped to the parent, otherwise they will
    be inherited from the parent, see the
    \texttt{\textquotesingle{}pipe\textquotesingle{}} and
    \texttt{\textquotesingle{}inherit\textquotesingle{}} options for
    \hyperref[child_processspawncommand-args-options]{\texttt{child\_process.spawn()}}'s
    \hyperref[optionsstdio]{\texttt{stdio}} for more details.
    \textbf{Default:} \texttt{false}.
  \item
    \texttt{stdio} \{Array\textbar string\} See
    \hyperref[child_processspawncommand-args-options]{\texttt{child\_process.spawn()}}'s
    \hyperref[optionsstdio]{\texttt{stdio}}. When this option is
    provided, it overrides \texttt{silent}. If the array variant is
    used, it must contain exactly one item with value
    \texttt{\textquotesingle{}ipc\textquotesingle{}} or an error will be
    thrown. For instance
    \texttt{{[}0,\ 1,\ 2,\ \textquotesingle{}ipc\textquotesingle{}{]}}.
  \item
    \texttt{uid} \{number\} Sets the user identity of the process (see
    setuid(2)).
  \item
    \texttt{windowsVerbatimArguments} \{boolean\} No quoting or escaping
    of arguments is done on Windows. Ignored on Unix. \textbf{Default:}
    \texttt{false}.
  \item
    \texttt{timeout} \{number\} In milliseconds the maximum amount of
    time the process is allowed to run. \textbf{Default:}
    \texttt{undefined}.
  \end{itemize}
\item
  Returns: \{ChildProcess\}
\end{itemize}

The \texttt{child\_process.fork()} method is a special case of
\hyperref[child_processspawncommand-args-options]{\texttt{child\_process.spawn()}}
used specifically to spawn new Node.js processes. Like
\hyperref[child_processspawncommand-args-options]{\texttt{child\_process.spawn()}},
a \hyperref[class-childprocess]{\texttt{ChildProcess}} object is
returned. The returned
\hyperref[class-childprocess]{\texttt{ChildProcess}} will have an
additional communication channel built-in that allows messages to be
passed back and forth between the parent and child. See
\hyperref[subprocesssendmessage-sendhandle-options-callback]{\texttt{subprocess.send()}}
for details.

Keep in mind that spawned Node.js child processes are independent of the
parent with exception of the IPC communication channel that is
established between the two. Each process has its own memory, with their
own V8 instances. Because of the additional resource allocations
required, spawning a large number of child Node.js processes is not
recommended.

By default, \texttt{child\_process.fork()} will spawn new Node.js
instances using the
\href{process.md\#processexecpath}{\texttt{process.execPath}} of the
parent process. The \texttt{execPath} property in the \texttt{options}
object allows for an alternative execution path to be used.

Node.js processes launched with a custom \texttt{execPath} will
communicate with the parent process using the file descriptor (fd)
identified using the environment variable \texttt{NODE\_CHANNEL\_FD} on
the child process.

Unlike the fork(2) POSIX system call, \texttt{child\_process.fork()}
does not clone the current process.

The \texttt{shell} option available in
\hyperref[child_processspawncommand-args-options]{\texttt{child\_process.spawn()}}
is not supported by \texttt{child\_process.fork()} and will be ignored
if set.

If the \texttt{signal} option is enabled, calling \texttt{.abort()} on
the corresponding \texttt{AbortController} is similar to calling
\texttt{.kill()} on the child process except the error passed to the
callback will be an \texttt{AbortError}:

\begin{Shaded}
\begin{Highlighting}[]
\ControlFlowTok{if}\NormalTok{ (}\BuiltInTok{process}\OperatorTok{.}\AttributeTok{argv}\NormalTok{[}\DecValTok{2}\NormalTok{] }\OperatorTok{===} \StringTok{\textquotesingle{}child\textquotesingle{}}\NormalTok{) \{}
  \PreprocessorTok{setTimeout}\NormalTok{(() }\KeywordTok{=\textgreater{}}\NormalTok{ \{}
    \BuiltInTok{console}\OperatorTok{.}\FunctionTok{log}\NormalTok{(}\VerbatimStringTok{\textasciigrave{}Hello from }\SpecialCharTok{$\{}\BuiltInTok{process}\OperatorTok{.}\AttributeTok{argv}\NormalTok{[}\DecValTok{2}\NormalTok{]}\SpecialCharTok{\}}\VerbatimStringTok{!\textasciigrave{}}\NormalTok{)}\OperatorTok{;}
\NormalTok{  \}}\OperatorTok{,} \DecValTok{1\_000}\NormalTok{)}\OperatorTok{;}
\NormalTok{\} }\ControlFlowTok{else}\NormalTok{ \{}
  \KeywordTok{const}\NormalTok{ \{ fork \} }\OperatorTok{=} \PreprocessorTok{require}\NormalTok{(}\StringTok{\textquotesingle{}node:child\_process\textquotesingle{}}\NormalTok{)}\OperatorTok{;}
  \KeywordTok{const}\NormalTok{ controller }\OperatorTok{=} \KeywordTok{new} \FunctionTok{AbortController}\NormalTok{()}\OperatorTok{;}
  \KeywordTok{const}\NormalTok{ \{ signal \} }\OperatorTok{=}\NormalTok{ controller}\OperatorTok{;}
  \KeywordTok{const}\NormalTok{ child }\OperatorTok{=} \FunctionTok{fork}\NormalTok{(}\BuiltInTok{\_\_filename}\OperatorTok{,}\NormalTok{ [}\StringTok{\textquotesingle{}child\textquotesingle{}}\NormalTok{]}\OperatorTok{,}\NormalTok{ \{ signal \})}\OperatorTok{;}
\NormalTok{  child}\OperatorTok{.}\FunctionTok{on}\NormalTok{(}\StringTok{\textquotesingle{}error\textquotesingle{}}\OperatorTok{,}\NormalTok{ (err) }\KeywordTok{=\textgreater{}}\NormalTok{ \{}
    \CommentTok{// This will be called with err being an AbortError if the controller aborts}
\NormalTok{  \})}\OperatorTok{;}
\NormalTok{  controller}\OperatorTok{.}\FunctionTok{abort}\NormalTok{()}\OperatorTok{;} \CommentTok{// Stops the child process}
\NormalTok{\}}
\end{Highlighting}
\end{Shaded}

\subsubsection{\texorpdfstring{\texttt{child\_process.spawn(command{[},\ args{]}{[},\ options{]})}}{child\_process.spawn(command{[}, args{]}{[}, options{]})}}\label{child_process.spawncommand-args-options}

\begin{itemize}
\tightlist
\item
  \texttt{command} \{string\} The command to run.
\item
  \texttt{args} \{string{[}{]}\} List of string arguments.
\item
  \texttt{options} \{Object\}

  \begin{itemize}
  \tightlist
  \item
    \texttt{cwd} \{string\textbar URL\} Current working directory of the
    child process.
  \item
    \texttt{env} \{Object\} Environment key-value pairs.
    \textbf{Default:} \texttt{process.env}.
  \item
    \texttt{argv0} \{string\} Explicitly set the value of
    \texttt{argv{[}0{]}} sent to the child process. This will be set to
    \texttt{command} if not specified.
  \item
    \texttt{stdio} \{Array\textbar string\} Child's stdio configuration
    (see \hyperref[optionsstdio]{\texttt{options.stdio}}).
  \item
    \texttt{detached} \{boolean\} Prepare child to run independently of
    its parent process. Specific behavior depends on the platform, see
    \hyperref[optionsdetached]{\texttt{options.detached}}).
  \item
    \texttt{uid} \{number\} Sets the user identity of the process (see
    setuid(2)).
  \item
    \texttt{gid} \{number\} Sets the group identity of the process (see
    setgid(2)).
  \item
    \texttt{serialization} \{string\} Specify the kind of serialization
    used for sending messages between processes. Possible values are
    \texttt{\textquotesingle{}json\textquotesingle{}} and
    \texttt{\textquotesingle{}advanced\textquotesingle{}}. See
    \hyperref[advanced-serialization]{Advanced serialization} for more
    details. \textbf{Default:}
    \texttt{\textquotesingle{}json\textquotesingle{}}.
  \item
    \texttt{shell} \{boolean\textbar string\} If \texttt{true}, runs
    \texttt{command} inside of a shell. Uses
    \texttt{\textquotesingle{}/bin/sh\textquotesingle{}} on Unix, and
    \texttt{process.env.ComSpec} on Windows. A different shell can be
    specified as a string. See \hyperref[shell-requirements]{Shell
    requirements} and \hyperref[default-windows-shell]{Default Windows
    shell}. \textbf{Default:} \texttt{false} (no shell).
  \item
    \texttt{windowsVerbatimArguments} \{boolean\} No quoting or escaping
    of arguments is done on Windows. Ignored on Unix. This is set to
    \texttt{true} automatically when \texttt{shell} is specified and is
    CMD. \textbf{Default:} \texttt{false}.
  \item
    \texttt{windowsHide} \{boolean\} Hide the subprocess console window
    that would normally be created on Windows systems. \textbf{Default:}
    \texttt{false}.
  \item
    \texttt{signal} \{AbortSignal\} allows aborting the child process
    using an AbortSignal.
  \item
    \texttt{timeout} \{number\} In milliseconds the maximum amount of
    time the process is allowed to run. \textbf{Default:}
    \texttt{undefined}.
  \item
    \texttt{killSignal} \{string\textbar integer\} The signal value to
    be used when the spawned process will be killed by timeout or abort
    signal. \textbf{Default:}
    \texttt{\textquotesingle{}SIGTERM\textquotesingle{}}.
  \end{itemize}
\item
  Returns: \{ChildProcess\}
\end{itemize}

The \texttt{child\_process.spawn()} method spawns a new process using
the given \texttt{command}, with command-line arguments in
\texttt{args}. If omitted, \texttt{args} defaults to an empty array.

\textbf{If the \texttt{shell} option is enabled, do not pass unsanitized
user input to this function. Any input containing shell metacharacters
may be used to trigger arbitrary command execution.}

A third argument may be used to specify additional options, with these
defaults:

\begin{Shaded}
\begin{Highlighting}[]
\KeywordTok{const}\NormalTok{ defaults }\OperatorTok{=}\NormalTok{ \{}
  \DataTypeTok{cwd}\OperatorTok{:} \KeywordTok{undefined}\OperatorTok{,}
  \DataTypeTok{env}\OperatorTok{:} \BuiltInTok{process}\OperatorTok{.}\AttributeTok{env}\OperatorTok{,}
\NormalTok{\}}\OperatorTok{;}
\end{Highlighting}
\end{Shaded}

Use \texttt{cwd} to specify the working directory from which the process
is spawned. If not given, the default is to inherit the current working
directory. If given, but the path does not exist, the child process
emits an \texttt{ENOENT} error and exits immediately. \texttt{ENOENT} is
also emitted when the command does not exist.

Use \texttt{env} to specify environment variables that will be visible
to the new process, the default is
\href{process.md\#processenv}{\texttt{process.env}}.

\texttt{undefined} values in \texttt{env} will be ignored.

Example of running \texttt{ls\ -lh\ /usr}, capturing \texttt{stdout},
\texttt{stderr}, and the exit code:

\begin{Shaded}
\begin{Highlighting}[]
\KeywordTok{const}\NormalTok{ \{ spawn \} }\OperatorTok{=} \PreprocessorTok{require}\NormalTok{(}\StringTok{\textquotesingle{}node:child\_process\textquotesingle{}}\NormalTok{)}\OperatorTok{;}
\KeywordTok{const}\NormalTok{ ls }\OperatorTok{=} \FunctionTok{spawn}\NormalTok{(}\StringTok{\textquotesingle{}ls\textquotesingle{}}\OperatorTok{,}\NormalTok{ [}\StringTok{\textquotesingle{}{-}lh\textquotesingle{}}\OperatorTok{,} \StringTok{\textquotesingle{}/usr\textquotesingle{}}\NormalTok{])}\OperatorTok{;}

\NormalTok{ls}\OperatorTok{.}\AttributeTok{stdout}\OperatorTok{.}\FunctionTok{on}\NormalTok{(}\StringTok{\textquotesingle{}data\textquotesingle{}}\OperatorTok{,}\NormalTok{ (data) }\KeywordTok{=\textgreater{}}\NormalTok{ \{}
  \BuiltInTok{console}\OperatorTok{.}\FunctionTok{log}\NormalTok{(}\VerbatimStringTok{\textasciigrave{}stdout: }\SpecialCharTok{$\{}\NormalTok{data}\SpecialCharTok{\}}\VerbatimStringTok{\textasciigrave{}}\NormalTok{)}\OperatorTok{;}
\NormalTok{\})}\OperatorTok{;}

\NormalTok{ls}\OperatorTok{.}\AttributeTok{stderr}\OperatorTok{.}\FunctionTok{on}\NormalTok{(}\StringTok{\textquotesingle{}data\textquotesingle{}}\OperatorTok{,}\NormalTok{ (data) }\KeywordTok{=\textgreater{}}\NormalTok{ \{}
  \BuiltInTok{console}\OperatorTok{.}\FunctionTok{error}\NormalTok{(}\VerbatimStringTok{\textasciigrave{}stderr: }\SpecialCharTok{$\{}\NormalTok{data}\SpecialCharTok{\}}\VerbatimStringTok{\textasciigrave{}}\NormalTok{)}\OperatorTok{;}
\NormalTok{\})}\OperatorTok{;}

\NormalTok{ls}\OperatorTok{.}\FunctionTok{on}\NormalTok{(}\StringTok{\textquotesingle{}close\textquotesingle{}}\OperatorTok{,}\NormalTok{ (code) }\KeywordTok{=\textgreater{}}\NormalTok{ \{}
  \BuiltInTok{console}\OperatorTok{.}\FunctionTok{log}\NormalTok{(}\VerbatimStringTok{\textasciigrave{}child process exited with code }\SpecialCharTok{$\{}\NormalTok{code}\SpecialCharTok{\}}\VerbatimStringTok{\textasciigrave{}}\NormalTok{)}\OperatorTok{;}
\NormalTok{\})}\OperatorTok{;}
\end{Highlighting}
\end{Shaded}

Example: A very elaborate way to run
\texttt{ps\ ax\ \textbar{}\ grep\ ssh}

\begin{Shaded}
\begin{Highlighting}[]
\KeywordTok{const}\NormalTok{ \{ spawn \} }\OperatorTok{=} \PreprocessorTok{require}\NormalTok{(}\StringTok{\textquotesingle{}node:child\_process\textquotesingle{}}\NormalTok{)}\OperatorTok{;}
\KeywordTok{const}\NormalTok{ ps }\OperatorTok{=} \FunctionTok{spawn}\NormalTok{(}\StringTok{\textquotesingle{}ps\textquotesingle{}}\OperatorTok{,}\NormalTok{ [}\StringTok{\textquotesingle{}ax\textquotesingle{}}\NormalTok{])}\OperatorTok{;}
\KeywordTok{const}\NormalTok{ grep }\OperatorTok{=} \FunctionTok{spawn}\NormalTok{(}\StringTok{\textquotesingle{}grep\textquotesingle{}}\OperatorTok{,}\NormalTok{ [}\StringTok{\textquotesingle{}ssh\textquotesingle{}}\NormalTok{])}\OperatorTok{;}

\NormalTok{ps}\OperatorTok{.}\AttributeTok{stdout}\OperatorTok{.}\FunctionTok{on}\NormalTok{(}\StringTok{\textquotesingle{}data\textquotesingle{}}\OperatorTok{,}\NormalTok{ (data) }\KeywordTok{=\textgreater{}}\NormalTok{ \{}
\NormalTok{  grep}\OperatorTok{.}\AttributeTok{stdin}\OperatorTok{.}\FunctionTok{write}\NormalTok{(data)}\OperatorTok{;}
\NormalTok{\})}\OperatorTok{;}

\NormalTok{ps}\OperatorTok{.}\AttributeTok{stderr}\OperatorTok{.}\FunctionTok{on}\NormalTok{(}\StringTok{\textquotesingle{}data\textquotesingle{}}\OperatorTok{,}\NormalTok{ (data) }\KeywordTok{=\textgreater{}}\NormalTok{ \{}
  \BuiltInTok{console}\OperatorTok{.}\FunctionTok{error}\NormalTok{(}\VerbatimStringTok{\textasciigrave{}ps stderr: }\SpecialCharTok{$\{}\NormalTok{data}\SpecialCharTok{\}}\VerbatimStringTok{\textasciigrave{}}\NormalTok{)}\OperatorTok{;}
\NormalTok{\})}\OperatorTok{;}

\NormalTok{ps}\OperatorTok{.}\FunctionTok{on}\NormalTok{(}\StringTok{\textquotesingle{}close\textquotesingle{}}\OperatorTok{,}\NormalTok{ (code) }\KeywordTok{=\textgreater{}}\NormalTok{ \{}
  \ControlFlowTok{if}\NormalTok{ (code }\OperatorTok{!==} \DecValTok{0}\NormalTok{) \{}
    \BuiltInTok{console}\OperatorTok{.}\FunctionTok{log}\NormalTok{(}\VerbatimStringTok{\textasciigrave{}ps process exited with code }\SpecialCharTok{$\{}\NormalTok{code}\SpecialCharTok{\}}\VerbatimStringTok{\textasciigrave{}}\NormalTok{)}\OperatorTok{;}
\NormalTok{  \}}
\NormalTok{  grep}\OperatorTok{.}\AttributeTok{stdin}\OperatorTok{.}\FunctionTok{end}\NormalTok{()}\OperatorTok{;}
\NormalTok{\})}\OperatorTok{;}

\NormalTok{grep}\OperatorTok{.}\AttributeTok{stdout}\OperatorTok{.}\FunctionTok{on}\NormalTok{(}\StringTok{\textquotesingle{}data\textquotesingle{}}\OperatorTok{,}\NormalTok{ (data) }\KeywordTok{=\textgreater{}}\NormalTok{ \{}
  \BuiltInTok{console}\OperatorTok{.}\FunctionTok{log}\NormalTok{(data}\OperatorTok{.}\FunctionTok{toString}\NormalTok{())}\OperatorTok{;}
\NormalTok{\})}\OperatorTok{;}

\NormalTok{grep}\OperatorTok{.}\AttributeTok{stderr}\OperatorTok{.}\FunctionTok{on}\NormalTok{(}\StringTok{\textquotesingle{}data\textquotesingle{}}\OperatorTok{,}\NormalTok{ (data) }\KeywordTok{=\textgreater{}}\NormalTok{ \{}
  \BuiltInTok{console}\OperatorTok{.}\FunctionTok{error}\NormalTok{(}\VerbatimStringTok{\textasciigrave{}grep stderr: }\SpecialCharTok{$\{}\NormalTok{data}\SpecialCharTok{\}}\VerbatimStringTok{\textasciigrave{}}\NormalTok{)}\OperatorTok{;}
\NormalTok{\})}\OperatorTok{;}

\NormalTok{grep}\OperatorTok{.}\FunctionTok{on}\NormalTok{(}\StringTok{\textquotesingle{}close\textquotesingle{}}\OperatorTok{,}\NormalTok{ (code) }\KeywordTok{=\textgreater{}}\NormalTok{ \{}
  \ControlFlowTok{if}\NormalTok{ (code }\OperatorTok{!==} \DecValTok{0}\NormalTok{) \{}
    \BuiltInTok{console}\OperatorTok{.}\FunctionTok{log}\NormalTok{(}\VerbatimStringTok{\textasciigrave{}grep process exited with code }\SpecialCharTok{$\{}\NormalTok{code}\SpecialCharTok{\}}\VerbatimStringTok{\textasciigrave{}}\NormalTok{)}\OperatorTok{;}
\NormalTok{  \}}
\NormalTok{\})}\OperatorTok{;}
\end{Highlighting}
\end{Shaded}

Example of checking for failed \texttt{spawn}:

\begin{Shaded}
\begin{Highlighting}[]
\KeywordTok{const}\NormalTok{ \{ spawn \} }\OperatorTok{=} \PreprocessorTok{require}\NormalTok{(}\StringTok{\textquotesingle{}node:child\_process\textquotesingle{}}\NormalTok{)}\OperatorTok{;}
\KeywordTok{const}\NormalTok{ subprocess }\OperatorTok{=} \FunctionTok{spawn}\NormalTok{(}\StringTok{\textquotesingle{}bad\_command\textquotesingle{}}\NormalTok{)}\OperatorTok{;}

\NormalTok{subprocess}\OperatorTok{.}\FunctionTok{on}\NormalTok{(}\StringTok{\textquotesingle{}error\textquotesingle{}}\OperatorTok{,}\NormalTok{ (err) }\KeywordTok{=\textgreater{}}\NormalTok{ \{}
  \BuiltInTok{console}\OperatorTok{.}\FunctionTok{error}\NormalTok{(}\StringTok{\textquotesingle{}Failed to start subprocess.\textquotesingle{}}\NormalTok{)}\OperatorTok{;}
\NormalTok{\})}\OperatorTok{;}
\end{Highlighting}
\end{Shaded}

Certain platforms (macOS, Linux) will use the value of
\texttt{argv{[}0{]}} for the process title while others (Windows, SunOS)
will use \texttt{command}.

Node.js overwrites \texttt{argv{[}0{]}} with \texttt{process.execPath}
on startup, so \texttt{process.argv{[}0{]}} in a Node.js child process
will not match the \texttt{argv0} parameter passed to \texttt{spawn}
from the parent. Retrieve it with the \texttt{process.argv0} property
instead.

If the \texttt{signal} option is enabled, calling \texttt{.abort()} on
the corresponding \texttt{AbortController} is similar to calling
\texttt{.kill()} on the child process except the error passed to the
callback will be an \texttt{AbortError}:

\begin{Shaded}
\begin{Highlighting}[]
\KeywordTok{const}\NormalTok{ \{ spawn \} }\OperatorTok{=} \PreprocessorTok{require}\NormalTok{(}\StringTok{\textquotesingle{}node:child\_process\textquotesingle{}}\NormalTok{)}\OperatorTok{;}
\KeywordTok{const}\NormalTok{ controller }\OperatorTok{=} \KeywordTok{new} \FunctionTok{AbortController}\NormalTok{()}\OperatorTok{;}
\KeywordTok{const}\NormalTok{ \{ signal \} }\OperatorTok{=}\NormalTok{ controller}\OperatorTok{;}
\KeywordTok{const}\NormalTok{ grep }\OperatorTok{=} \FunctionTok{spawn}\NormalTok{(}\StringTok{\textquotesingle{}grep\textquotesingle{}}\OperatorTok{,}\NormalTok{ [}\StringTok{\textquotesingle{}ssh\textquotesingle{}}\NormalTok{]}\OperatorTok{,}\NormalTok{ \{ signal \})}\OperatorTok{;}
\NormalTok{grep}\OperatorTok{.}\FunctionTok{on}\NormalTok{(}\StringTok{\textquotesingle{}error\textquotesingle{}}\OperatorTok{,}\NormalTok{ (err) }\KeywordTok{=\textgreater{}}\NormalTok{ \{}
  \CommentTok{// This will be called with err being an AbortError if the controller aborts}
\NormalTok{\})}\OperatorTok{;}
\NormalTok{controller}\OperatorTok{.}\FunctionTok{abort}\NormalTok{()}\OperatorTok{;} \CommentTok{// Stops the child process}
\end{Highlighting}
\end{Shaded}

\paragraph{\texorpdfstring{\texttt{options.detached}}{options.detached}}\label{options.detached}

On Windows, setting \texttt{options.detached} to \texttt{true} makes it
possible for the child process to continue running after the parent
exits. The child will have its own console window. Once enabled for a
child process, it cannot be disabled.

On non-Windows platforms, if \texttt{options.detached} is set to
\texttt{true}, the child process will be made the leader of a new
process group and session. Child processes may continue running after
the parent exits regardless of whether they are detached or not. See
setsid(2) for more information.

By default, the parent will wait for the detached child to exit. To
prevent the parent from waiting for a given \texttt{subprocess} to exit,
use the \texttt{subprocess.unref()} method. Doing so will cause the
parent's event loop to not include the child in its reference count,
allowing the parent to exit independently of the child, unless there is
an established IPC channel between the child and the parent.

When using the \texttt{detached} option to start a long-running process,
the process will not stay running in the background after the parent
exits unless it is provided with a \texttt{stdio} configuration that is
not connected to the parent. If the parent's \texttt{stdio} is
inherited, the child will remain attached to the controlling terminal.

Example of a long-running process, by detaching and also ignoring its
parent \texttt{stdio} file descriptors, in order to ignore the parent's
termination:

\begin{Shaded}
\begin{Highlighting}[]
\KeywordTok{const}\NormalTok{ \{ spawn \} }\OperatorTok{=} \PreprocessorTok{require}\NormalTok{(}\StringTok{\textquotesingle{}node:child\_process\textquotesingle{}}\NormalTok{)}\OperatorTok{;}

\KeywordTok{const}\NormalTok{ subprocess }\OperatorTok{=} \FunctionTok{spawn}\NormalTok{(}\BuiltInTok{process}\OperatorTok{.}\AttributeTok{argv}\NormalTok{[}\DecValTok{0}\NormalTok{]}\OperatorTok{,}\NormalTok{ [}\StringTok{\textquotesingle{}child\_program.js\textquotesingle{}}\NormalTok{]}\OperatorTok{,}\NormalTok{ \{}
  \DataTypeTok{detached}\OperatorTok{:} \KeywordTok{true}\OperatorTok{,}
  \DataTypeTok{stdio}\OperatorTok{:} \StringTok{\textquotesingle{}ignore\textquotesingle{}}\OperatorTok{,}
\NormalTok{\})}\OperatorTok{;}

\NormalTok{subprocess}\OperatorTok{.}\FunctionTok{unref}\NormalTok{()}\OperatorTok{;}
\end{Highlighting}
\end{Shaded}

Alternatively one can redirect the child process' output into files:

\begin{Shaded}
\begin{Highlighting}[]
\KeywordTok{const}\NormalTok{ fs }\OperatorTok{=} \PreprocessorTok{require}\NormalTok{(}\StringTok{\textquotesingle{}node:fs\textquotesingle{}}\NormalTok{)}\OperatorTok{;}
\KeywordTok{const}\NormalTok{ \{ spawn \} }\OperatorTok{=} \PreprocessorTok{require}\NormalTok{(}\StringTok{\textquotesingle{}node:child\_process\textquotesingle{}}\NormalTok{)}\OperatorTok{;}
\KeywordTok{const}\NormalTok{ out }\OperatorTok{=}\NormalTok{ fs}\OperatorTok{.}\FunctionTok{openSync}\NormalTok{(}\StringTok{\textquotesingle{}./out.log\textquotesingle{}}\OperatorTok{,} \StringTok{\textquotesingle{}a\textquotesingle{}}\NormalTok{)}\OperatorTok{;}
\KeywordTok{const}\NormalTok{ err }\OperatorTok{=}\NormalTok{ fs}\OperatorTok{.}\FunctionTok{openSync}\NormalTok{(}\StringTok{\textquotesingle{}./out.log\textquotesingle{}}\OperatorTok{,} \StringTok{\textquotesingle{}a\textquotesingle{}}\NormalTok{)}\OperatorTok{;}

\KeywordTok{const}\NormalTok{ subprocess }\OperatorTok{=} \FunctionTok{spawn}\NormalTok{(}\StringTok{\textquotesingle{}prg\textquotesingle{}}\OperatorTok{,}\NormalTok{ []}\OperatorTok{,}\NormalTok{ \{}
  \DataTypeTok{detached}\OperatorTok{:} \KeywordTok{true}\OperatorTok{,}
  \DataTypeTok{stdio}\OperatorTok{:}\NormalTok{ [ }\StringTok{\textquotesingle{}ignore\textquotesingle{}}\OperatorTok{,}\NormalTok{ out}\OperatorTok{,}\NormalTok{ err ]}\OperatorTok{,}
\NormalTok{\})}\OperatorTok{;}

\NormalTok{subprocess}\OperatorTok{.}\FunctionTok{unref}\NormalTok{()}\OperatorTok{;}
\end{Highlighting}
\end{Shaded}

\paragraph{\texorpdfstring{\texttt{options.stdio}}{options.stdio}}\label{options.stdio}

The \texttt{options.stdio} option is used to configure the pipes that
are established between the parent and child process. By default, the
child's stdin, stdout, and stderr are redirected to corresponding
\hyperref[subprocessstdin]{\texttt{subprocess.stdin}},
\hyperref[subprocessstdout]{\texttt{subprocess.stdout}}, and
\hyperref[subprocessstderr]{\texttt{subprocess.stderr}} streams on the
\hyperref[class-childprocess]{\texttt{ChildProcess}} object. This is
equivalent to setting the \texttt{options.stdio} equal to
\texttt{{[}\textquotesingle{}pipe\textquotesingle{},\ \textquotesingle{}pipe\textquotesingle{},\ \textquotesingle{}pipe\textquotesingle{}{]}}.

For convenience, \texttt{options.stdio} may be one of the following
strings:

\begin{itemize}
\tightlist
\item
  \texttt{\textquotesingle{}pipe\textquotesingle{}}: equivalent to
  \texttt{{[}\textquotesingle{}pipe\textquotesingle{},\ \textquotesingle{}pipe\textquotesingle{},\ \textquotesingle{}pipe\textquotesingle{}{]}}
  (the default)
\item
  \texttt{\textquotesingle{}overlapped\textquotesingle{}}: equivalent to
  \texttt{{[}\textquotesingle{}overlapped\textquotesingle{},\ \textquotesingle{}overlapped\textquotesingle{},\ \textquotesingle{}overlapped\textquotesingle{}{]}}
\item
  \texttt{\textquotesingle{}ignore\textquotesingle{}}: equivalent to
  \texttt{{[}\textquotesingle{}ignore\textquotesingle{},\ \textquotesingle{}ignore\textquotesingle{},\ \textquotesingle{}ignore\textquotesingle{}{]}}
\item
  \texttt{\textquotesingle{}inherit\textquotesingle{}}: equivalent to
  \texttt{{[}\textquotesingle{}inherit\textquotesingle{},\ \textquotesingle{}inherit\textquotesingle{},\ \textquotesingle{}inherit\textquotesingle{}{]}}
  or \texttt{{[}0,\ 1,\ 2{]}}
\end{itemize}

Otherwise, the value of \texttt{options.stdio} is an array where each
index corresponds to an fd in the child. The fds 0, 1, and 2 correspond
to stdin, stdout, and stderr, respectively. Additional fds can be
specified to create additional pipes between the parent and child. The
value is one of the following:

\begin{enumerate}
\def\labelenumi{\arabic{enumi}.}
\item
  \texttt{\textquotesingle{}pipe\textquotesingle{}}: Create a pipe
  between the child process and the parent process. The parent end of
  the pipe is exposed to the parent as a property on the
  \texttt{child\_process} object as
  \hyperref[subprocessstdio]{\texttt{subprocess.stdio{[}fd{]}}}. Pipes
  created for fds 0, 1, and 2 are also available as
  \hyperref[subprocessstdin]{\texttt{subprocess.stdin}},
  \hyperref[subprocessstdout]{\texttt{subprocess.stdout}} and
  \hyperref[subprocessstderr]{\texttt{subprocess.stderr}}, respectively.
  These are not actual Unix pipes and therefore the child process can
  not use them by their descriptor files, e.g.~\texttt{/dev/fd/2} or
  \texttt{/dev/stdout}.
\item
  \texttt{\textquotesingle{}overlapped\textquotesingle{}}: Same as
  \texttt{\textquotesingle{}pipe\textquotesingle{}} except that the
  \texttt{FILE\_FLAG\_OVERLAPPED} flag is set on the handle. This is
  necessary for overlapped I/O on the child process's stdio handles. See
  the
  \href{https://docs.microsoft.com/en-us/windows/win32/fileio/synchronous-and-asynchronous-i-o}{docs}
  for more details. This is exactly the same as
  \texttt{\textquotesingle{}pipe\textquotesingle{}} on non-Windows
  systems.
\item
  \texttt{\textquotesingle{}ipc\textquotesingle{}}: Create an IPC
  channel for passing messages/file descriptors between parent and
  child. A \hyperref[class-childprocess]{\texttt{ChildProcess}} may have
  at most one IPC stdio file descriptor. Setting this option enables the
  \hyperref[subprocesssendmessage-sendhandle-options-callback]{\texttt{subprocess.send()}}
  method. If the child is a Node.js process, the presence of an IPC
  channel will enable
  \href{process.md\#processsendmessage-sendhandle-options-callback}{\texttt{process.send()}}
  and
  \href{process.md\#processdisconnect}{\texttt{process.disconnect()}}
  methods, as well as
  \href{process.md\#event-disconnect}{\texttt{\textquotesingle{}disconnect\textquotesingle{}}}
  and
  \href{process.md\#event-message}{\texttt{\textquotesingle{}message\textquotesingle{}}}
  events within the child.

  Accessing the IPC channel fd in any way other than
  \href{process.md\#processsendmessage-sendhandle-options-callback}{\texttt{process.send()}}
  or using the IPC channel with a child process that is not a Node.js
  instance is not supported.
\item
  \texttt{\textquotesingle{}ignore\textquotesingle{}}: Instructs Node.js
  to ignore the fd in the child. While Node.js will always open fds 0,
  1, and 2 for the processes it spawns, setting the fd to
  \texttt{\textquotesingle{}ignore\textquotesingle{}} will cause Node.js
  to open \texttt{/dev/null} and attach it to the child's fd.
\item
  \texttt{\textquotesingle{}inherit\textquotesingle{}}: Pass through the
  corresponding stdio stream to/from the parent process. In the first
  three positions, this is equivalent to \texttt{process.stdin},
  \texttt{process.stdout}, and \texttt{process.stderr}, respectively. In
  any other position, equivalent to
  \texttt{\textquotesingle{}ignore\textquotesingle{}}.
\item
  \{Stream\} object: Share a readable or writable stream that refers to
  a tty, file, socket, or a pipe with the child process. The stream's
  underlying file descriptor is duplicated in the child process to the
  fd that corresponds to the index in the \texttt{stdio} array. The
  stream must have an underlying descriptor (file streams do not until
  the \texttt{\textquotesingle{}open\textquotesingle{}} event has
  occurred).
\item
  Positive integer: The integer value is interpreted as a file
  descriptor that is open in the parent process. It is shared with the
  child process, similar to how \{Stream\} objects can be shared.
  Passing sockets is not supported on Windows.
\item
  \texttt{null}, \texttt{undefined}: Use default value. For stdio fds 0,
  1, and 2 (in other words, stdin, stdout, and stderr) a pipe is
  created. For fd 3 and up, the default is
  \texttt{\textquotesingle{}ignore\textquotesingle{}}.
\end{enumerate}

\begin{Shaded}
\begin{Highlighting}[]
\KeywordTok{const}\NormalTok{ \{ spawn \} }\OperatorTok{=} \PreprocessorTok{require}\NormalTok{(}\StringTok{\textquotesingle{}node:child\_process\textquotesingle{}}\NormalTok{)}\OperatorTok{;}

\CommentTok{// Child will use parent\textquotesingle{}s stdios.}
\FunctionTok{spawn}\NormalTok{(}\StringTok{\textquotesingle{}prg\textquotesingle{}}\OperatorTok{,}\NormalTok{ []}\OperatorTok{,}\NormalTok{ \{ }\DataTypeTok{stdio}\OperatorTok{:} \StringTok{\textquotesingle{}inherit\textquotesingle{}}\NormalTok{ \})}\OperatorTok{;}

\CommentTok{// Spawn child sharing only stderr.}
\FunctionTok{spawn}\NormalTok{(}\StringTok{\textquotesingle{}prg\textquotesingle{}}\OperatorTok{,}\NormalTok{ []}\OperatorTok{,}\NormalTok{ \{ }\DataTypeTok{stdio}\OperatorTok{:}\NormalTok{ [}\StringTok{\textquotesingle{}pipe\textquotesingle{}}\OperatorTok{,} \StringTok{\textquotesingle{}pipe\textquotesingle{}}\OperatorTok{,} \BuiltInTok{process}\OperatorTok{.}\AttributeTok{stderr}\NormalTok{] \})}\OperatorTok{;}

\CommentTok{// Open an extra fd=4, to interact with programs presenting a}
\CommentTok{// startd{-}style interface.}
\FunctionTok{spawn}\NormalTok{(}\StringTok{\textquotesingle{}prg\textquotesingle{}}\OperatorTok{,}\NormalTok{ []}\OperatorTok{,}\NormalTok{ \{ }\DataTypeTok{stdio}\OperatorTok{:}\NormalTok{ [}\StringTok{\textquotesingle{}pipe\textquotesingle{}}\OperatorTok{,} \KeywordTok{null}\OperatorTok{,} \KeywordTok{null}\OperatorTok{,} \KeywordTok{null}\OperatorTok{,} \StringTok{\textquotesingle{}pipe\textquotesingle{}}\NormalTok{] \})}\OperatorTok{;}
\end{Highlighting}
\end{Shaded}

\emph{It is worth noting that when an IPC channel is established between
the parent and child processes, and the child is a Node.js process, the
child is launched with the IPC channel unreferenced (using
\texttt{unref()}) until the child registers an event handler for the
\href{process.md\#event-disconnect}{\texttt{\textquotesingle{}disconnect\textquotesingle{}}}
event or the
\href{process.md\#event-message}{\texttt{\textquotesingle{}message\textquotesingle{}}}
event. This allows the child to exit normally without the process being
held open by the open IPC channel.}

On Unix-like operating systems, the
\hyperref[child_processspawncommand-args-options]{\texttt{child\_process.spawn()}}
method performs memory operations synchronously before decoupling the
event loop from the child. Applications with a large memory footprint
may find frequent
\hyperref[child_processspawncommand-args-options]{\texttt{child\_process.spawn()}}
calls to be a bottleneck. For more information, see
\href{https://bugs.chromium.org/p/v8/issues/detail?id=7381}{V8 issue
7381}.

See also:
\hyperref[child_processexeccommand-options-callback]{\texttt{child\_process.exec()}}
and
\hyperref[child_processforkmodulepath-args-options]{\texttt{child\_process.fork()}}.

\subsection{Synchronous process
creation}\label{synchronous-process-creation}

The
\hyperref[child_processspawnsynccommand-args-options]{\texttt{child\_process.spawnSync()}},
\hyperref[child_processexecsynccommand-options]{\texttt{child\_process.execSync()}},
and
\hyperref[child_processexecfilesyncfile-args-options]{\texttt{child\_process.execFileSync()}}
methods are synchronous and will block the Node.js event loop, pausing
execution of any additional code until the spawned process exits.

Blocking calls like these are mostly useful for simplifying
general-purpose scripting tasks and for simplifying the
loading/processing of application configuration at startup.

\subsubsection{\texorpdfstring{\texttt{child\_process.execFileSync(file{[},\ args{]}{[},\ options{]})}}{child\_process.execFileSync(file{[}, args{]}{[}, options{]})}}\label{child_process.execfilesyncfile-args-options}

\begin{itemize}
\tightlist
\item
  \texttt{file} \{string\} The name or path of the executable file to
  run.
\item
  \texttt{args} \{string{[}{]}\} List of string arguments.
\item
  \texttt{options} \{Object\}

  \begin{itemize}
  \tightlist
  \item
    \texttt{cwd} \{string\textbar URL\} Current working directory of the
    child process.
  \item
    \texttt{input}
    \{string\textbar Buffer\textbar TypedArray\textbar DataView\} The
    value which will be passed as stdin to the spawned process. If
    \texttt{stdio{[}0{]}} is set to
    \texttt{\textquotesingle{}pipe\textquotesingle{}}, Supplying this
    value will override \texttt{stdio{[}0{]}}.
  \item
    \texttt{stdio} \{string\textbar Array\} Child's stdio configuration.
    \texttt{stderr} by default will be output to the parent process'
    stderr unless \texttt{stdio} is specified. \textbf{Default:}
    \texttt{\textquotesingle{}pipe\textquotesingle{}}.
  \item
    \texttt{env} \{Object\} Environment key-value pairs.
    \textbf{Default:} \texttt{process.env}.
  \item
    \texttt{uid} \{number\} Sets the user identity of the process (see
    setuid(2)).
  \item
    \texttt{gid} \{number\} Sets the group identity of the process (see
    setgid(2)).
  \item
    \texttt{timeout} \{number\} In milliseconds the maximum amount of
    time the process is allowed to run. \textbf{Default:}
    \texttt{undefined}.
  \item
    \texttt{killSignal} \{string\textbar integer\} The signal value to
    be used when the spawned process will be killed. \textbf{Default:}
    \texttt{\textquotesingle{}SIGTERM\textquotesingle{}}.
  \item
    \texttt{maxBuffer} \{number\} Largest amount of data in bytes
    allowed on stdout or stderr. If exceeded, the child process is
    terminated. See caveat at
    \hyperref[maxbuffer-and-unicode]{\texttt{maxBuffer} and Unicode}.
    \textbf{Default:} \texttt{1024\ *\ 1024}.
  \item
    \texttt{encoding} \{string\} The encoding used for all stdio inputs
    and outputs. \textbf{Default:}
    \texttt{\textquotesingle{}buffer\textquotesingle{}}.
  \item
    \texttt{windowsHide} \{boolean\} Hide the subprocess console window
    that would normally be created on Windows systems. \textbf{Default:}
    \texttt{false}.
  \item
    \texttt{shell} \{boolean\textbar string\} If \texttt{true}, runs
    \texttt{command} inside of a shell. Uses
    \texttt{\textquotesingle{}/bin/sh\textquotesingle{}} on Unix, and
    \texttt{process.env.ComSpec} on Windows. A different shell can be
    specified as a string. See \hyperref[shell-requirements]{Shell
    requirements} and \hyperref[default-windows-shell]{Default Windows
    shell}. \textbf{Default:} \texttt{false} (no shell).
  \end{itemize}
\item
  Returns: \{Buffer\textbar string\} The stdout from the command.
\end{itemize}

The \texttt{child\_process.execFileSync()} method is generally identical
to
\hyperref[child_processexecfilefile-args-options-callback]{\texttt{child\_process.execFile()}}
with the exception that the method will not return until the child
process has fully closed. When a timeout has been encountered and
\texttt{killSignal} is sent, the method won't return until the process
has completely exited.

If the child process intercepts and handles the \texttt{SIGTERM} signal
and does not exit, the parent process will still wait until the child
process has exited.

If the process times out or has a non-zero exit code, this method will
throw an \href{errors.md\#class-error}{\texttt{Error}} that will include
the full result of the underlying
\hyperref[child_processspawnsynccommand-args-options]{\texttt{child\_process.spawnSync()}}.

\textbf{If the \texttt{shell} option is enabled, do not pass unsanitized
user input to this function. Any input containing shell metacharacters
may be used to trigger arbitrary command execution.}

\subsubsection{\texorpdfstring{\texttt{child\_process.execSync(command{[},\ options{]})}}{child\_process.execSync(command{[}, options{]})}}\label{child_process.execsynccommand-options}

\begin{itemize}
\tightlist
\item
  \texttt{command} \{string\} The command to run.
\item
  \texttt{options} \{Object\}

  \begin{itemize}
  \tightlist
  \item
    \texttt{cwd} \{string\textbar URL\} Current working directory of the
    child process.
  \item
    \texttt{input}
    \{string\textbar Buffer\textbar TypedArray\textbar DataView\} The
    value which will be passed as stdin to the spawned process. If
    \texttt{stdio{[}0{]}} is set to
    \texttt{\textquotesingle{}pipe\textquotesingle{}}, Supplying this
    value will override \texttt{stdio{[}0{]}}.
  \item
    \texttt{stdio} \{string\textbar Array\} Child's stdio configuration.
    \texttt{stderr} by default will be output to the parent process'
    stderr unless \texttt{stdio} is specified. \textbf{Default:}
    \texttt{\textquotesingle{}pipe\textquotesingle{}}.
  \item
    \texttt{env} \{Object\} Environment key-value pairs.
    \textbf{Default:} \texttt{process.env}.
  \item
    \texttt{shell} \{string\} Shell to execute the command with. See
    \hyperref[shell-requirements]{Shell requirements} and
    \hyperref[default-windows-shell]{Default Windows shell}.
    \textbf{Default:}
    \texttt{\textquotesingle{}/bin/sh\textquotesingle{}} on Unix,
    \texttt{process.env.ComSpec} on Windows.
  \item
    \texttt{uid} \{number\} Sets the user identity of the process. (See
    setuid(2)).
  \item
    \texttt{gid} \{number\} Sets the group identity of the process. (See
    setgid(2)).
  \item
    \texttt{timeout} \{number\} In milliseconds the maximum amount of
    time the process is allowed to run. \textbf{Default:}
    \texttt{undefined}.
  \item
    \texttt{killSignal} \{string\textbar integer\} The signal value to
    be used when the spawned process will be killed. \textbf{Default:}
    \texttt{\textquotesingle{}SIGTERM\textquotesingle{}}.
  \item
    \texttt{maxBuffer} \{number\} Largest amount of data in bytes
    allowed on stdout or stderr. If exceeded, the child process is
    terminated and any output is truncated. See caveat at
    \hyperref[maxbuffer-and-unicode]{\texttt{maxBuffer} and Unicode}.
    \textbf{Default:} \texttt{1024\ *\ 1024}.
  \item
    \texttt{encoding} \{string\} The encoding used for all stdio inputs
    and outputs. \textbf{Default:}
    \texttt{\textquotesingle{}buffer\textquotesingle{}}.
  \item
    \texttt{windowsHide} \{boolean\} Hide the subprocess console window
    that would normally be created on Windows systems. \textbf{Default:}
    \texttt{false}.
  \end{itemize}
\item
  Returns: \{Buffer\textbar string\} The stdout from the command.
\end{itemize}

The \texttt{child\_process.execSync()} method is generally identical to
\hyperref[child_processexeccommand-options-callback]{\texttt{child\_process.exec()}}
with the exception that the method will not return until the child
process has fully closed. When a timeout has been encountered and
\texttt{killSignal} is sent, the method won't return until the process
has completely exited. If the child process intercepts and handles the
\texttt{SIGTERM} signal and doesn't exit, the parent process will wait
until the child process has exited.

If the process times out or has a non-zero exit code, this method will
throw. The \href{errors.md\#class-error}{\texttt{Error}} object will
contain the entire result from
\hyperref[child_processspawnsynccommand-args-options]{\texttt{child\_process.spawnSync()}}.

\textbf{Never pass unsanitized user input to this function. Any input
containing shell metacharacters may be used to trigger arbitrary command
execution.}

\subsubsection{\texorpdfstring{\texttt{child\_process.spawnSync(command{[},\ args{]}{[},\ options{]})}}{child\_process.spawnSync(command{[}, args{]}{[}, options{]})}}\label{child_process.spawnsynccommand-args-options}

\begin{itemize}
\tightlist
\item
  \texttt{command} \{string\} The command to run.
\item
  \texttt{args} \{string{[}{]}\} List of string arguments.
\item
  \texttt{options} \{Object\}

  \begin{itemize}
  \tightlist
  \item
    \texttt{cwd} \{string\textbar URL\} Current working directory of the
    child process.
  \item
    \texttt{input}
    \{string\textbar Buffer\textbar TypedArray\textbar DataView\} The
    value which will be passed as stdin to the spawned process. If
    \texttt{stdio{[}0{]}} is set to
    \texttt{\textquotesingle{}pipe\textquotesingle{}}, Supplying this
    value will override \texttt{stdio{[}0{]}}.
  \item
    \texttt{argv0} \{string\} Explicitly set the value of
    \texttt{argv{[}0{]}} sent to the child process. This will be set to
    \texttt{command} if not specified.
  \item
    \texttt{stdio} \{string\textbar Array\} Child's stdio configuration.
    \textbf{Default:} \texttt{\textquotesingle{}pipe\textquotesingle{}}.
  \item
    \texttt{env} \{Object\} Environment key-value pairs.
    \textbf{Default:} \texttt{process.env}.
  \item
    \texttt{uid} \{number\} Sets the user identity of the process (see
    setuid(2)).
  \item
    \texttt{gid} \{number\} Sets the group identity of the process (see
    setgid(2)).
  \item
    \texttt{timeout} \{number\} In milliseconds the maximum amount of
    time the process is allowed to run. \textbf{Default:}
    \texttt{undefined}.
  \item
    \texttt{killSignal} \{string\textbar integer\} The signal value to
    be used when the spawned process will be killed. \textbf{Default:}
    \texttt{\textquotesingle{}SIGTERM\textquotesingle{}}.
  \item
    \texttt{maxBuffer} \{number\} Largest amount of data in bytes
    allowed on stdout or stderr. If exceeded, the child process is
    terminated and any output is truncated. See caveat at
    \hyperref[maxbuffer-and-unicode]{\texttt{maxBuffer} and Unicode}.
    \textbf{Default:} \texttt{1024\ *\ 1024}.
  \item
    \texttt{encoding} \{string\} The encoding used for all stdio inputs
    and outputs. \textbf{Default:}
    \texttt{\textquotesingle{}buffer\textquotesingle{}}.
  \item
    \texttt{shell} \{boolean\textbar string\} If \texttt{true}, runs
    \texttt{command} inside of a shell. Uses
    \texttt{\textquotesingle{}/bin/sh\textquotesingle{}} on Unix, and
    \texttt{process.env.ComSpec} on Windows. A different shell can be
    specified as a string. See \hyperref[shell-requirements]{Shell
    requirements} and \hyperref[default-windows-shell]{Default Windows
    shell}. \textbf{Default:} \texttt{false} (no shell).
  \item
    \texttt{windowsVerbatimArguments} \{boolean\} No quoting or escaping
    of arguments is done on Windows. Ignored on Unix. This is set to
    \texttt{true} automatically when \texttt{shell} is specified and is
    CMD. \textbf{Default:} \texttt{false}.
  \item
    \texttt{windowsHide} \{boolean\} Hide the subprocess console window
    that would normally be created on Windows systems. \textbf{Default:}
    \texttt{false}.
  \end{itemize}
\item
  Returns: \{Object\}

  \begin{itemize}
  \tightlist
  \item
    \texttt{pid} \{number\} Pid of the child process.
  \item
    \texttt{output} \{Array\} Array of results from stdio output.
  \item
    \texttt{stdout} \{Buffer\textbar string\} The contents of
    \texttt{output{[}1{]}}.
  \item
    \texttt{stderr} \{Buffer\textbar string\} The contents of
    \texttt{output{[}2{]}}.
  \item
    \texttt{status} \{number\textbar null\} The exit code of the
    subprocess, or \texttt{null} if the subprocess terminated due to a
    signal.
  \item
    \texttt{signal} \{string\textbar null\} The signal used to kill the
    subprocess, or \texttt{null} if the subprocess did not terminate due
    to a signal.
  \item
    \texttt{error} \{Error\} The error object if the child process
    failed or timed out.
  \end{itemize}
\end{itemize}

The \texttt{child\_process.spawnSync()} method is generally identical to
\hyperref[child_processspawncommand-args-options]{\texttt{child\_process.spawn()}}
with the exception that the function will not return until the child
process has fully closed. When a timeout has been encountered and
\texttt{killSignal} is sent, the method won't return until the process
has completely exited. If the process intercepts and handles the
\texttt{SIGTERM} signal and doesn't exit, the parent process will wait
until the child process has exited.

\textbf{If the \texttt{shell} option is enabled, do not pass unsanitized
user input to this function. Any input containing shell metacharacters
may be used to trigger arbitrary command execution.}

\subsection{\texorpdfstring{Class:
\texttt{ChildProcess}}{Class: ChildProcess}}\label{class-childprocess}

\begin{itemize}
\tightlist
\item
  Extends: \{EventEmitter\}
\end{itemize}

Instances of the \texttt{ChildProcess} represent spawned child
processes.

Instances of \texttt{ChildProcess} are not intended to be created
directly. Rather, use the
\hyperref[child_processspawncommand-args-options]{\texttt{child\_process.spawn()}},
\hyperref[child_processexeccommand-options-callback]{\texttt{child\_process.exec()}},
\hyperref[child_processexecfilefile-args-options-callback]{\texttt{child\_process.execFile()}},
or
\hyperref[child_processforkmodulepath-args-options]{\texttt{child\_process.fork()}}
methods to create instances of \texttt{ChildProcess}.

\subsubsection{\texorpdfstring{Event:
\texttt{\textquotesingle{}close\textquotesingle{}}}{Event: \textquotesingle close\textquotesingle{}}}\label{event-close}

\begin{itemize}
\tightlist
\item
  \texttt{code} \{number\} The exit code if the child exited on its own.
\item
  \texttt{signal} \{string\} The signal by which the child process was
  terminated.
\end{itemize}

The \texttt{\textquotesingle{}close\textquotesingle{}} event is emitted
after a process has ended \emph{and} the stdio streams of a child
process have been closed. This is distinct from the
\hyperref[event-exit]{\texttt{\textquotesingle{}exit\textquotesingle{}}}
event, since multiple processes might share the same stdio streams. The
\texttt{\textquotesingle{}close\textquotesingle{}} event will always
emit after
\hyperref[event-exit]{\texttt{\textquotesingle{}exit\textquotesingle{}}}
was already emitted, or
\hyperref[event-error]{\texttt{\textquotesingle{}error\textquotesingle{}}}
if the child failed to spawn.

\begin{Shaded}
\begin{Highlighting}[]
\KeywordTok{const}\NormalTok{ \{ spawn \} }\OperatorTok{=} \PreprocessorTok{require}\NormalTok{(}\StringTok{\textquotesingle{}node:child\_process\textquotesingle{}}\NormalTok{)}\OperatorTok{;}
\KeywordTok{const}\NormalTok{ ls }\OperatorTok{=} \FunctionTok{spawn}\NormalTok{(}\StringTok{\textquotesingle{}ls\textquotesingle{}}\OperatorTok{,}\NormalTok{ [}\StringTok{\textquotesingle{}{-}lh\textquotesingle{}}\OperatorTok{,} \StringTok{\textquotesingle{}/usr\textquotesingle{}}\NormalTok{])}\OperatorTok{;}

\NormalTok{ls}\OperatorTok{.}\AttributeTok{stdout}\OperatorTok{.}\FunctionTok{on}\NormalTok{(}\StringTok{\textquotesingle{}data\textquotesingle{}}\OperatorTok{,}\NormalTok{ (data) }\KeywordTok{=\textgreater{}}\NormalTok{ \{}
  \BuiltInTok{console}\OperatorTok{.}\FunctionTok{log}\NormalTok{(}\VerbatimStringTok{\textasciigrave{}stdout: }\SpecialCharTok{$\{}\NormalTok{data}\SpecialCharTok{\}}\VerbatimStringTok{\textasciigrave{}}\NormalTok{)}\OperatorTok{;}
\NormalTok{\})}\OperatorTok{;}

\NormalTok{ls}\OperatorTok{.}\FunctionTok{on}\NormalTok{(}\StringTok{\textquotesingle{}close\textquotesingle{}}\OperatorTok{,}\NormalTok{ (code) }\KeywordTok{=\textgreater{}}\NormalTok{ \{}
  \BuiltInTok{console}\OperatorTok{.}\FunctionTok{log}\NormalTok{(}\VerbatimStringTok{\textasciigrave{}child process close all stdio with code }\SpecialCharTok{$\{}\NormalTok{code}\SpecialCharTok{\}}\VerbatimStringTok{\textasciigrave{}}\NormalTok{)}\OperatorTok{;}
\NormalTok{\})}\OperatorTok{;}

\NormalTok{ls}\OperatorTok{.}\FunctionTok{on}\NormalTok{(}\StringTok{\textquotesingle{}exit\textquotesingle{}}\OperatorTok{,}\NormalTok{ (code) }\KeywordTok{=\textgreater{}}\NormalTok{ \{}
  \BuiltInTok{console}\OperatorTok{.}\FunctionTok{log}\NormalTok{(}\VerbatimStringTok{\textasciigrave{}child process exited with code }\SpecialCharTok{$\{}\NormalTok{code}\SpecialCharTok{\}}\VerbatimStringTok{\textasciigrave{}}\NormalTok{)}\OperatorTok{;}
\NormalTok{\})}\OperatorTok{;}
\end{Highlighting}
\end{Shaded}

\subsubsection{\texorpdfstring{Event:
\texttt{\textquotesingle{}disconnect\textquotesingle{}}}{Event: \textquotesingle disconnect\textquotesingle{}}}\label{event-disconnect}

The \texttt{\textquotesingle{}disconnect\textquotesingle{}} event is
emitted after calling the
\hyperref[subprocessdisconnect]{\texttt{subprocess.disconnect()}} method
in parent process or
\href{process.md\#processdisconnect}{\texttt{process.disconnect()}} in
child process. After disconnecting it is no longer possible to send or
receive messages, and the
\hyperref[subprocessconnected]{\texttt{subprocess.connected}} property
is \texttt{false}.

\subsubsection{\texorpdfstring{Event:
\texttt{\textquotesingle{}error\textquotesingle{}}}{Event: \textquotesingle error\textquotesingle{}}}\label{event-error}

\begin{itemize}
\tightlist
\item
  \texttt{err} \{Error\} The error.
\end{itemize}

The \texttt{\textquotesingle{}error\textquotesingle{}} event is emitted
whenever:

\begin{itemize}
\tightlist
\item
  The process could not be spawned.
\item
  The process could not be killed.
\item
  Sending a message to the child process failed.
\item
  The child process was aborted via the \texttt{signal} option.
\end{itemize}

The \texttt{\textquotesingle{}exit\textquotesingle{}} event may or may
not fire after an error has occurred. When listening to both the
\texttt{\textquotesingle{}exit\textquotesingle{}} and
\texttt{\textquotesingle{}error\textquotesingle{}} events, guard against
accidentally invoking handler functions multiple times.

See also \hyperref[subprocesskillsignal]{\texttt{subprocess.kill()}} and
\hyperref[subprocesssendmessage-sendhandle-options-callback]{\texttt{subprocess.send()}}.

\subsubsection{\texorpdfstring{Event:
\texttt{\textquotesingle{}exit\textquotesingle{}}}{Event: \textquotesingle exit\textquotesingle{}}}\label{event-exit}

\begin{itemize}
\tightlist
\item
  \texttt{code} \{number\} The exit code if the child exited on its own.
\item
  \texttt{signal} \{string\} The signal by which the child process was
  terminated.
\end{itemize}

The \texttt{\textquotesingle{}exit\textquotesingle{}} event is emitted
after the child process ends. If the process exited, \texttt{code} is
the final exit code of the process, otherwise \texttt{null}. If the
process terminated due to receipt of a signal, \texttt{signal} is the
string name of the signal, otherwise \texttt{null}. One of the two will
always be non-\texttt{null}.

When the \texttt{\textquotesingle{}exit\textquotesingle{}} event is
triggered, child process stdio streams might still be open.

Node.js establishes signal handlers for \texttt{SIGINT} and
\texttt{SIGTERM} and Node.js processes will not terminate immediately
due to receipt of those signals. Rather, Node.js will perform a sequence
of cleanup actions and then will re-raise the handled signal.

See waitpid(2).

\subsubsection{\texorpdfstring{Event:
\texttt{\textquotesingle{}message\textquotesingle{}}}{Event: \textquotesingle message\textquotesingle{}}}\label{event-message}

\begin{itemize}
\tightlist
\item
  \texttt{message} \{Object\} A parsed JSON object or primitive value.
\item
  \texttt{sendHandle} \{Handle\} A
  \href{net.md\#class-netsocket}{\texttt{net.Socket}} or
  \href{net.md\#class-netserver}{\texttt{net.Server}} object, or
  undefined.
\end{itemize}

The \texttt{\textquotesingle{}message\textquotesingle{}} event is
triggered when a child process uses
\href{process.md\#processsendmessage-sendhandle-options-callback}{\texttt{process.send()}}
to send messages.

The message goes through serialization and parsing. The resulting
message might not be the same as what is originally sent.

If the \texttt{serialization} option was set to
\texttt{\textquotesingle{}advanced\textquotesingle{}} used when spawning
the child process, the \texttt{message} argument can contain data that
JSON is not able to represent. See
\hyperref[advanced-serialization]{Advanced serialization} for more
details.

\subsubsection{\texorpdfstring{Event:
\texttt{\textquotesingle{}spawn\textquotesingle{}}}{Event: \textquotesingle spawn\textquotesingle{}}}\label{event-spawn}

The \texttt{\textquotesingle{}spawn\textquotesingle{}} event is emitted
once the child process has spawned successfully. If the child process
does not spawn successfully, the
\texttt{\textquotesingle{}spawn\textquotesingle{}} event is not emitted
and the \texttt{\textquotesingle{}error\textquotesingle{}} event is
emitted instead.

If emitted, the \texttt{\textquotesingle{}spawn\textquotesingle{}} event
comes before all other events and before any data is received via
\texttt{stdout} or \texttt{stderr}.

The \texttt{\textquotesingle{}spawn\textquotesingle{}} event will fire
regardless of whether an error occurs \textbf{within} the spawned
process. For example, if \texttt{bash\ some-command} spawns
successfully, the \texttt{\textquotesingle{}spawn\textquotesingle{}}
event will fire, though \texttt{bash} may fail to spawn
\texttt{some-command}. This caveat also applies when using
\texttt{\{\ shell:\ true\ \}}.

\subsubsection{\texorpdfstring{\texttt{subprocess.channel}}{subprocess.channel}}\label{subprocess.channel}

\begin{itemize}
\tightlist
\item
  \{Object\} A pipe representing the IPC channel to the child process.
\end{itemize}

The \texttt{subprocess.channel} property is a reference to the child's
IPC channel. If no IPC channel exists, this property is
\texttt{undefined}.

\paragraph{\texorpdfstring{\texttt{subprocess.channel.ref()}}{subprocess.channel.ref()}}\label{subprocess.channel.ref}

This method makes the IPC channel keep the event loop of the parent
process running if \texttt{.unref()} has been called before.

\paragraph{\texorpdfstring{\texttt{subprocess.channel.unref()}}{subprocess.channel.unref()}}\label{subprocess.channel.unref}

This method makes the IPC channel not keep the event loop of the parent
process running, and lets it finish even while the channel is open.

\subsubsection{\texorpdfstring{\texttt{subprocess.connected}}{subprocess.connected}}\label{subprocess.connected}

\begin{itemize}
\tightlist
\item
  \{boolean\} Set to \texttt{false} after
  \texttt{subprocess.disconnect()} is called.
\end{itemize}

The \texttt{subprocess.connected} property indicates whether it is still
possible to send and receive messages from a child process. When
\texttt{subprocess.connected} is \texttt{false}, it is no longer
possible to send or receive messages.

\subsubsection{\texorpdfstring{\texttt{subprocess.disconnect()}}{subprocess.disconnect()}}\label{subprocess.disconnect}

Closes the IPC channel between parent and child, allowing the child to
exit gracefully once there are no other connections keeping it alive.
After calling this method the \texttt{subprocess.connected} and
\texttt{process.connected} properties in both the parent and child
(respectively) will be set to \texttt{false}, and it will be no longer
possible to pass messages between the processes.

The \texttt{\textquotesingle{}disconnect\textquotesingle{}} event will
be emitted when there are no messages in the process of being received.
This will most often be triggered immediately after calling
\texttt{subprocess.disconnect()}.

When the child process is a Node.js instance (e.g.~spawned using
\hyperref[child_processforkmodulepath-args-options]{\texttt{child\_process.fork()}}),
the \texttt{process.disconnect()} method can be invoked within the child
process to close the IPC channel as well.

\subsubsection{\texorpdfstring{\texttt{subprocess.exitCode}}{subprocess.exitCode}}\label{subprocess.exitcode}

\begin{itemize}
\tightlist
\item
  \{integer\}
\end{itemize}

The \texttt{subprocess.exitCode} property indicates the exit code of the
child process. If the child process is still running, the field will be
\texttt{null}.

\subsubsection{\texorpdfstring{\texttt{subprocess.kill({[}signal{]})}}{subprocess.kill({[}signal{]})}}\label{subprocess.killsignal}

\begin{itemize}
\tightlist
\item
  \texttt{signal} \{number\textbar string\}
\item
  Returns: \{boolean\}
\end{itemize}

The \texttt{subprocess.kill()} method sends a signal to the child
process. If no argument is given, the process will be sent the
\texttt{\textquotesingle{}SIGTERM\textquotesingle{}} signal. See
signal(7) for a list of available signals. This function returns
\texttt{true} if kill(2) succeeds, and \texttt{false} otherwise.

\begin{Shaded}
\begin{Highlighting}[]
\KeywordTok{const}\NormalTok{ \{ spawn \} }\OperatorTok{=} \PreprocessorTok{require}\NormalTok{(}\StringTok{\textquotesingle{}node:child\_process\textquotesingle{}}\NormalTok{)}\OperatorTok{;}
\KeywordTok{const}\NormalTok{ grep }\OperatorTok{=} \FunctionTok{spawn}\NormalTok{(}\StringTok{\textquotesingle{}grep\textquotesingle{}}\OperatorTok{,}\NormalTok{ [}\StringTok{\textquotesingle{}ssh\textquotesingle{}}\NormalTok{])}\OperatorTok{;}

\NormalTok{grep}\OperatorTok{.}\FunctionTok{on}\NormalTok{(}\StringTok{\textquotesingle{}close\textquotesingle{}}\OperatorTok{,}\NormalTok{ (code}\OperatorTok{,}\NormalTok{ signal) }\KeywordTok{=\textgreater{}}\NormalTok{ \{}
  \BuiltInTok{console}\OperatorTok{.}\FunctionTok{log}\NormalTok{(}
    \VerbatimStringTok{\textasciigrave{}child process terminated due to receipt of signal }\SpecialCharTok{$\{}\NormalTok{signal}\SpecialCharTok{\}}\VerbatimStringTok{\textasciigrave{}}\NormalTok{)}\OperatorTok{;}
\NormalTok{\})}\OperatorTok{;}

\CommentTok{// Send SIGHUP to process.}
\NormalTok{grep}\OperatorTok{.}\FunctionTok{kill}\NormalTok{(}\StringTok{\textquotesingle{}SIGHUP\textquotesingle{}}\NormalTok{)}\OperatorTok{;}
\end{Highlighting}
\end{Shaded}

The \hyperref[class-childprocess]{\texttt{ChildProcess}} object may emit
an
\hyperref[event-error]{\texttt{\textquotesingle{}error\textquotesingle{}}}
event if the signal cannot be delivered. Sending a signal to a child
process that has already exited is not an error but may have unforeseen
consequences. Specifically, if the process identifier (PID) has been
reassigned to another process, the signal will be delivered to that
process instead which can have unexpected results.

While the function is called \texttt{kill}, the signal delivered to the
child process may not actually terminate the process.

See kill(2) for reference.

On Windows, where POSIX signals do not exist, the \texttt{signal}
argument will be ignored, and the process will be killed forcefully and
abruptly (similar to
\texttt{\textquotesingle{}SIGKILL\textquotesingle{}}). See
\href{process.md\#signal-events}{Signal Events} for more details.

On Linux, child processes of child processes will not be terminated when
attempting to kill their parent. This is likely to happen when running a
new process in a shell or with the use of the \texttt{shell} option of
\texttt{ChildProcess}:

\begin{Shaded}
\begin{Highlighting}[]
\StringTok{\textquotesingle{}use strict\textquotesingle{}}\OperatorTok{;}
\KeywordTok{const}\NormalTok{ \{ spawn \} }\OperatorTok{=} \PreprocessorTok{require}\NormalTok{(}\StringTok{\textquotesingle{}node:child\_process\textquotesingle{}}\NormalTok{)}\OperatorTok{;}

\KeywordTok{const}\NormalTok{ subprocess }\OperatorTok{=} \FunctionTok{spawn}\NormalTok{(}
  \StringTok{\textquotesingle{}sh\textquotesingle{}}\OperatorTok{,}
\NormalTok{  [}
    \StringTok{\textquotesingle{}{-}c\textquotesingle{}}\OperatorTok{,}
    \VerbatimStringTok{\textasciigrave{}node {-}e "setInterval(() =\textgreater{} \{}
\VerbatimStringTok{      console.log(process.pid, \textquotesingle{}is alive\textquotesingle{})}
\VerbatimStringTok{    \}, 500);"\textasciigrave{}}\OperatorTok{,}
\NormalTok{  ]}\OperatorTok{,}\NormalTok{ \{}
    \DataTypeTok{stdio}\OperatorTok{:}\NormalTok{ [}\StringTok{\textquotesingle{}inherit\textquotesingle{}}\OperatorTok{,} \StringTok{\textquotesingle{}inherit\textquotesingle{}}\OperatorTok{,} \StringTok{\textquotesingle{}inherit\textquotesingle{}}\NormalTok{]}\OperatorTok{,}
\NormalTok{  \}}\OperatorTok{,}
\NormalTok{)}\OperatorTok{;}

\PreprocessorTok{setTimeout}\NormalTok{(() }\KeywordTok{=\textgreater{}}\NormalTok{ \{}
\NormalTok{  subprocess}\OperatorTok{.}\FunctionTok{kill}\NormalTok{()}\OperatorTok{;} \CommentTok{// Does not terminate the Node.js process in the shell.}
\NormalTok{\}}\OperatorTok{,} \DecValTok{2000}\NormalTok{)}\OperatorTok{;}
\end{Highlighting}
\end{Shaded}

\subsubsection{\texorpdfstring{\texttt{subprocess{[}Symbol.dispose{]}()}}{subprocess{[}Symbol.dispose{]}()}}\label{subprocesssymbol.dispose}

\begin{quote}
Stability: 1 - Experimental
\end{quote}

Calls \hyperref[subprocesskillsignal]{\texttt{subprocess.kill()}} with
\texttt{\textquotesingle{}SIGTERM\textquotesingle{}}.

\subsubsection{\texorpdfstring{\texttt{subprocess.killed}}{subprocess.killed}}\label{subprocess.killed}

\begin{itemize}
\tightlist
\item
  \{boolean\} Set to \texttt{true} after \texttt{subprocess.kill()} is
  used to successfully send a signal to the child process.
\end{itemize}

The \texttt{subprocess.killed} property indicates whether the child
process successfully received a signal from \texttt{subprocess.kill()}.
The \texttt{killed} property does not indicate that the child process
has been terminated.

\subsubsection{\texorpdfstring{\texttt{subprocess.pid}}{subprocess.pid}}\label{subprocess.pid}

\begin{itemize}
\tightlist
\item
  \{integer\textbar undefined\}
\end{itemize}

Returns the process identifier (PID) of the child process. If the child
process fails to spawn due to errors, then the value is
\texttt{undefined} and \texttt{error} is emitted.

\begin{Shaded}
\begin{Highlighting}[]
\KeywordTok{const}\NormalTok{ \{ spawn \} }\OperatorTok{=} \PreprocessorTok{require}\NormalTok{(}\StringTok{\textquotesingle{}node:child\_process\textquotesingle{}}\NormalTok{)}\OperatorTok{;}
\KeywordTok{const}\NormalTok{ grep }\OperatorTok{=} \FunctionTok{spawn}\NormalTok{(}\StringTok{\textquotesingle{}grep\textquotesingle{}}\OperatorTok{,}\NormalTok{ [}\StringTok{\textquotesingle{}ssh\textquotesingle{}}\NormalTok{])}\OperatorTok{;}

\BuiltInTok{console}\OperatorTok{.}\FunctionTok{log}\NormalTok{(}\VerbatimStringTok{\textasciigrave{}Spawned child pid: }\SpecialCharTok{$\{}\NormalTok{grep}\OperatorTok{.}\AttributeTok{pid}\SpecialCharTok{\}}\VerbatimStringTok{\textasciigrave{}}\NormalTok{)}\OperatorTok{;}
\NormalTok{grep}\OperatorTok{.}\AttributeTok{stdin}\OperatorTok{.}\FunctionTok{end}\NormalTok{()}\OperatorTok{;}
\end{Highlighting}
\end{Shaded}

\subsubsection{\texorpdfstring{\texttt{subprocess.ref()}}{subprocess.ref()}}\label{subprocess.ref}

Calling \texttt{subprocess.ref()} after making a call to
\texttt{subprocess.unref()} will restore the removed reference count for
the child process, forcing the parent to wait for the child to exit
before exiting itself.

\begin{Shaded}
\begin{Highlighting}[]
\KeywordTok{const}\NormalTok{ \{ spawn \} }\OperatorTok{=} \PreprocessorTok{require}\NormalTok{(}\StringTok{\textquotesingle{}node:child\_process\textquotesingle{}}\NormalTok{)}\OperatorTok{;}

\KeywordTok{const}\NormalTok{ subprocess }\OperatorTok{=} \FunctionTok{spawn}\NormalTok{(}\BuiltInTok{process}\OperatorTok{.}\AttributeTok{argv}\NormalTok{[}\DecValTok{0}\NormalTok{]}\OperatorTok{,}\NormalTok{ [}\StringTok{\textquotesingle{}child\_program.js\textquotesingle{}}\NormalTok{]}\OperatorTok{,}\NormalTok{ \{}
  \DataTypeTok{detached}\OperatorTok{:} \KeywordTok{true}\OperatorTok{,}
  \DataTypeTok{stdio}\OperatorTok{:} \StringTok{\textquotesingle{}ignore\textquotesingle{}}\OperatorTok{,}
\NormalTok{\})}\OperatorTok{;}

\NormalTok{subprocess}\OperatorTok{.}\FunctionTok{unref}\NormalTok{()}\OperatorTok{;}
\NormalTok{subprocess}\OperatorTok{.}\FunctionTok{ref}\NormalTok{()}\OperatorTok{;}
\end{Highlighting}
\end{Shaded}

\subsubsection{\texorpdfstring{\texttt{subprocess.send(message{[},\ sendHandle{[},\ options{]}{]}{[},\ callback{]})}}{subprocess.send(message{[}, sendHandle{[}, options{]}{]}{[}, callback{]})}}\label{subprocess.sendmessage-sendhandle-options-callback}

\begin{itemize}
\tightlist
\item
  \texttt{message} \{Object\}
\item
  \texttt{sendHandle} \{Handle\}
\item
  \texttt{options} \{Object\} The \texttt{options} argument, if present,
  is an object used to parameterize the sending of certain types of
  handles. \texttt{options} supports the following properties:

  \begin{itemize}
  \tightlist
  \item
    \texttt{keepOpen} \{boolean\} A value that can be used when passing
    instances of \texttt{net.Socket}. When \texttt{true}, the socket is
    kept open in the sending process. \textbf{Default:} \texttt{false}.
  \end{itemize}
\item
  \texttt{callback} \{Function\}
\item
  Returns: \{boolean\}
\end{itemize}

When an IPC channel has been established between the parent and child (
i.e.~when using
\hyperref[child_processforkmodulepath-args-options]{\texttt{child\_process.fork()}}),
the \texttt{subprocess.send()} method can be used to send messages to
the child process. When the child process is a Node.js instance, these
messages can be received via the
\href{process.md\#event-message}{\texttt{\textquotesingle{}message\textquotesingle{}}}
event.

The message goes through serialization and parsing. The resulting
message might not be the same as what is originally sent.

For example, in the parent script:

\begin{Shaded}
\begin{Highlighting}[]
\KeywordTok{const}\NormalTok{ cp }\OperatorTok{=} \PreprocessorTok{require}\NormalTok{(}\StringTok{\textquotesingle{}node:child\_process\textquotesingle{}}\NormalTok{)}\OperatorTok{;}
\KeywordTok{const}\NormalTok{ n }\OperatorTok{=}\NormalTok{ cp}\OperatorTok{.}\FunctionTok{fork}\NormalTok{(}\VerbatimStringTok{\textasciigrave{}}\SpecialCharTok{$\{}\BuiltInTok{\_\_dirname}\SpecialCharTok{\}}\VerbatimStringTok{/sub.js\textasciigrave{}}\NormalTok{)}\OperatorTok{;}

\NormalTok{n}\OperatorTok{.}\FunctionTok{on}\NormalTok{(}\StringTok{\textquotesingle{}message\textquotesingle{}}\OperatorTok{,}\NormalTok{ (m) }\KeywordTok{=\textgreater{}}\NormalTok{ \{}
  \BuiltInTok{console}\OperatorTok{.}\FunctionTok{log}\NormalTok{(}\StringTok{\textquotesingle{}PARENT got message:\textquotesingle{}}\OperatorTok{,}\NormalTok{ m)}\OperatorTok{;}
\NormalTok{\})}\OperatorTok{;}

\CommentTok{// Causes the child to print: CHILD got message: \{ hello: \textquotesingle{}world\textquotesingle{} \}}
\NormalTok{n}\OperatorTok{.}\FunctionTok{send}\NormalTok{(\{ }\DataTypeTok{hello}\OperatorTok{:} \StringTok{\textquotesingle{}world\textquotesingle{}}\NormalTok{ \})}\OperatorTok{;}
\end{Highlighting}
\end{Shaded}

And then the child script,
\texttt{\textquotesingle{}sub.js\textquotesingle{}} might look like
this:

\begin{Shaded}
\begin{Highlighting}[]
\BuiltInTok{process}\OperatorTok{.}\FunctionTok{on}\NormalTok{(}\StringTok{\textquotesingle{}message\textquotesingle{}}\OperatorTok{,}\NormalTok{ (m) }\KeywordTok{=\textgreater{}}\NormalTok{ \{}
  \BuiltInTok{console}\OperatorTok{.}\FunctionTok{log}\NormalTok{(}\StringTok{\textquotesingle{}CHILD got message:\textquotesingle{}}\OperatorTok{,}\NormalTok{ m)}\OperatorTok{;}
\NormalTok{\})}\OperatorTok{;}

\CommentTok{// Causes the parent to print: PARENT got message: \{ foo: \textquotesingle{}bar\textquotesingle{}, baz: null \}}
\BuiltInTok{process}\OperatorTok{.}\FunctionTok{send}\NormalTok{(\{ }\DataTypeTok{foo}\OperatorTok{:} \StringTok{\textquotesingle{}bar\textquotesingle{}}\OperatorTok{,} \DataTypeTok{baz}\OperatorTok{:} \KeywordTok{NaN}\NormalTok{ \})}\OperatorTok{;}
\end{Highlighting}
\end{Shaded}

Child Node.js processes will have a
\href{process.md\#processsendmessage-sendhandle-options-callback}{\texttt{process.send()}}
method of their own that allows the child to send messages back to the
parent.

There is a special case when sending a
\texttt{\{cmd:\ \textquotesingle{}NODE\_foo\textquotesingle{}\}}
message. Messages containing a \texttt{NODE\_} prefix in the
\texttt{cmd} property are reserved for use within Node.js core and will
not be emitted in the child's
\href{process.md\#event-message}{\texttt{\textquotesingle{}message\textquotesingle{}}}
event. Rather, such messages are emitted using the
\texttt{\textquotesingle{}internalMessage\textquotesingle{}} event and
are consumed internally by Node.js. Applications should avoid using such
messages or listening for
\texttt{\textquotesingle{}internalMessage\textquotesingle{}} events as
it is subject to change without notice.

The optional \texttt{sendHandle} argument that may be passed to
\texttt{subprocess.send()} is for passing a TCP server or socket object
to the child process. The child will receive the object as the second
argument passed to the callback function registered on the
\href{process.md\#event-message}{\texttt{\textquotesingle{}message\textquotesingle{}}}
event. Any data that is received and buffered in the socket will not be
sent to the child.

The optional \texttt{callback} is a function that is invoked after the
message is sent but before the child may have received it. The function
is called with a single argument: \texttt{null} on success, or an
\href{errors.md\#class-error}{\texttt{Error}} object on failure.

If no \texttt{callback} function is provided and the message cannot be
sent, an \texttt{\textquotesingle{}error\textquotesingle{}} event will
be emitted by the \hyperref[class-childprocess]{\texttt{ChildProcess}}
object. This can happen, for instance, when the child process has
already exited.

\texttt{subprocess.send()} will return \texttt{false} if the channel has
closed or when the backlog of unsent messages exceeds a threshold that
makes it unwise to send more. Otherwise, the method returns
\texttt{true}. The \texttt{callback} function can be used to implement
flow control.

\paragraph{Example: sending a server
object}\label{example-sending-a-server-object}

The \texttt{sendHandle} argument can be used, for instance, to pass the
handle of a TCP server object to the child process as illustrated in the
example below:

\begin{Shaded}
\begin{Highlighting}[]
\KeywordTok{const}\NormalTok{ subprocess }\OperatorTok{=} \PreprocessorTok{require}\NormalTok{(}\StringTok{\textquotesingle{}node:child\_process\textquotesingle{}}\NormalTok{)}\OperatorTok{.}\FunctionTok{fork}\NormalTok{(}\StringTok{\textquotesingle{}subprocess.js\textquotesingle{}}\NormalTok{)}\OperatorTok{;}

\CommentTok{// Open up the server object and send the handle.}
\KeywordTok{const}\NormalTok{ server }\OperatorTok{=} \PreprocessorTok{require}\NormalTok{(}\StringTok{\textquotesingle{}node:net\textquotesingle{}}\NormalTok{)}\OperatorTok{.}\FunctionTok{createServer}\NormalTok{()}\OperatorTok{;}
\NormalTok{server}\OperatorTok{.}\FunctionTok{on}\NormalTok{(}\StringTok{\textquotesingle{}connection\textquotesingle{}}\OperatorTok{,}\NormalTok{ (socket) }\KeywordTok{=\textgreater{}}\NormalTok{ \{}
\NormalTok{  socket}\OperatorTok{.}\FunctionTok{end}\NormalTok{(}\StringTok{\textquotesingle{}handled by parent\textquotesingle{}}\NormalTok{)}\OperatorTok{;}
\NormalTok{\})}\OperatorTok{;}
\NormalTok{server}\OperatorTok{.}\FunctionTok{listen}\NormalTok{(}\DecValTok{1337}\OperatorTok{,}\NormalTok{ () }\KeywordTok{=\textgreater{}}\NormalTok{ \{}
\NormalTok{  subprocess}\OperatorTok{.}\FunctionTok{send}\NormalTok{(}\StringTok{\textquotesingle{}server\textquotesingle{}}\OperatorTok{,}\NormalTok{ server)}\OperatorTok{;}
\NormalTok{\})}\OperatorTok{;}
\end{Highlighting}
\end{Shaded}

The child would then receive the server object as:

\begin{Shaded}
\begin{Highlighting}[]
\BuiltInTok{process}\OperatorTok{.}\FunctionTok{on}\NormalTok{(}\StringTok{\textquotesingle{}message\textquotesingle{}}\OperatorTok{,}\NormalTok{ (m}\OperatorTok{,}\NormalTok{ server) }\KeywordTok{=\textgreater{}}\NormalTok{ \{}
  \ControlFlowTok{if}\NormalTok{ (m }\OperatorTok{===} \StringTok{\textquotesingle{}server\textquotesingle{}}\NormalTok{) \{}
\NormalTok{    server}\OperatorTok{.}\FunctionTok{on}\NormalTok{(}\StringTok{\textquotesingle{}connection\textquotesingle{}}\OperatorTok{,}\NormalTok{ (socket) }\KeywordTok{=\textgreater{}}\NormalTok{ \{}
\NormalTok{      socket}\OperatorTok{.}\FunctionTok{end}\NormalTok{(}\StringTok{\textquotesingle{}handled by child\textquotesingle{}}\NormalTok{)}\OperatorTok{;}
\NormalTok{    \})}\OperatorTok{;}
\NormalTok{  \}}
\NormalTok{\})}\OperatorTok{;}
\end{Highlighting}
\end{Shaded}

Once the server is now shared between the parent and child, some
connections can be handled by the parent and some by the child.

While the example above uses a server created using the
\texttt{node:net} module, \texttt{node:dgram} module servers use exactly
the same workflow with the exceptions of listening on a
\texttt{\textquotesingle{}message\textquotesingle{}} event instead of
\texttt{\textquotesingle{}connection\textquotesingle{}} and using
\texttt{server.bind()} instead of \texttt{server.listen()}. This is,
however, only supported on Unix platforms.

\paragraph{Example: sending a socket
object}\label{example-sending-a-socket-object}

Similarly, the \texttt{sendHandler} argument can be used to pass the
handle of a socket to the child process. The example below spawns two
children that each handle connections with ``normal'' or ``special''
priority:

\begin{Shaded}
\begin{Highlighting}[]
\KeywordTok{const}\NormalTok{ \{ fork \} }\OperatorTok{=} \PreprocessorTok{require}\NormalTok{(}\StringTok{\textquotesingle{}node:child\_process\textquotesingle{}}\NormalTok{)}\OperatorTok{;}
\KeywordTok{const}\NormalTok{ normal }\OperatorTok{=} \FunctionTok{fork}\NormalTok{(}\StringTok{\textquotesingle{}subprocess.js\textquotesingle{}}\OperatorTok{,}\NormalTok{ [}\StringTok{\textquotesingle{}normal\textquotesingle{}}\NormalTok{])}\OperatorTok{;}
\KeywordTok{const}\NormalTok{ special }\OperatorTok{=} \FunctionTok{fork}\NormalTok{(}\StringTok{\textquotesingle{}subprocess.js\textquotesingle{}}\OperatorTok{,}\NormalTok{ [}\StringTok{\textquotesingle{}special\textquotesingle{}}\NormalTok{])}\OperatorTok{;}

\CommentTok{// Open up the server and send sockets to child. Use pauseOnConnect to prevent}
\CommentTok{// the sockets from being read before they are sent to the child process.}
\KeywordTok{const}\NormalTok{ server }\OperatorTok{=} \PreprocessorTok{require}\NormalTok{(}\StringTok{\textquotesingle{}node:net\textquotesingle{}}\NormalTok{)}\OperatorTok{.}\FunctionTok{createServer}\NormalTok{(\{ }\DataTypeTok{pauseOnConnect}\OperatorTok{:} \KeywordTok{true}\NormalTok{ \})}\OperatorTok{;}
\NormalTok{server}\OperatorTok{.}\FunctionTok{on}\NormalTok{(}\StringTok{\textquotesingle{}connection\textquotesingle{}}\OperatorTok{,}\NormalTok{ (socket) }\KeywordTok{=\textgreater{}}\NormalTok{ \{}

  \CommentTok{// If this is special priority...}
  \ControlFlowTok{if}\NormalTok{ (socket}\OperatorTok{.}\AttributeTok{remoteAddress} \OperatorTok{===} \StringTok{\textquotesingle{}74.125.127.100\textquotesingle{}}\NormalTok{) \{}
\NormalTok{    special}\OperatorTok{.}\FunctionTok{send}\NormalTok{(}\StringTok{\textquotesingle{}socket\textquotesingle{}}\OperatorTok{,}\NormalTok{ socket)}\OperatorTok{;}
    \ControlFlowTok{return}\OperatorTok{;}
\NormalTok{  \}}
  \CommentTok{// This is normal priority.}
\NormalTok{  normal}\OperatorTok{.}\FunctionTok{send}\NormalTok{(}\StringTok{\textquotesingle{}socket\textquotesingle{}}\OperatorTok{,}\NormalTok{ socket)}\OperatorTok{;}
\NormalTok{\})}\OperatorTok{;}
\NormalTok{server}\OperatorTok{.}\FunctionTok{listen}\NormalTok{(}\DecValTok{1337}\NormalTok{)}\OperatorTok{;}
\end{Highlighting}
\end{Shaded}

The \texttt{subprocess.js} would receive the socket handle as the second
argument passed to the event callback function:

\begin{Shaded}
\begin{Highlighting}[]
\BuiltInTok{process}\OperatorTok{.}\FunctionTok{on}\NormalTok{(}\StringTok{\textquotesingle{}message\textquotesingle{}}\OperatorTok{,}\NormalTok{ (m}\OperatorTok{,}\NormalTok{ socket) }\KeywordTok{=\textgreater{}}\NormalTok{ \{}
  \ControlFlowTok{if}\NormalTok{ (m }\OperatorTok{===} \StringTok{\textquotesingle{}socket\textquotesingle{}}\NormalTok{) \{}
    \ControlFlowTok{if}\NormalTok{ (socket) \{}
      \CommentTok{// Check that the client socket exists.}
      \CommentTok{// It is possible for the socket to be closed between the time it is}
      \CommentTok{// sent and the time it is received in the child process.}
\NormalTok{      socket}\OperatorTok{.}\FunctionTok{end}\NormalTok{(}\VerbatimStringTok{\textasciigrave{}Request handled with }\SpecialCharTok{$\{}\BuiltInTok{process}\OperatorTok{.}\AttributeTok{argv}\NormalTok{[}\DecValTok{2}\NormalTok{]}\SpecialCharTok{\}}\VerbatimStringTok{ priority\textasciigrave{}}\NormalTok{)}\OperatorTok{;}
\NormalTok{    \}}
\NormalTok{  \}}
\NormalTok{\})}\OperatorTok{;}
\end{Highlighting}
\end{Shaded}

Do not use \texttt{.maxConnections} on a socket that has been passed to
a subprocess. The parent cannot track when the socket is destroyed.

Any \texttt{\textquotesingle{}message\textquotesingle{}} handlers in the
subprocess should verify that \texttt{socket} exists, as the connection
may have been closed during the time it takes to send the connection to
the child.

\subsubsection{\texorpdfstring{\texttt{subprocess.signalCode}}{subprocess.signalCode}}\label{subprocess.signalcode}

\begin{itemize}
\tightlist
\item
  \{string\textbar null\}
\end{itemize}

The \texttt{subprocess.signalCode} property indicates the signal
received by the child process if any, else \texttt{null}.

\subsubsection{\texorpdfstring{\texttt{subprocess.spawnargs}}{subprocess.spawnargs}}\label{subprocess.spawnargs}

\begin{itemize}
\tightlist
\item
  \{Array\}
\end{itemize}

The \texttt{subprocess.spawnargs} property represents the full list of
command-line arguments the child process was launched with.

\subsubsection{\texorpdfstring{\texttt{subprocess.spawnfile}}{subprocess.spawnfile}}\label{subprocess.spawnfile}

\begin{itemize}
\tightlist
\item
  \{string\}
\end{itemize}

The \texttt{subprocess.spawnfile} property indicates the executable file
name of the child process that is launched.

For
\hyperref[child_processforkmodulepath-args-options]{\texttt{child\_process.fork()}},
its value will be equal to
\href{process.md\#processexecpath}{\texttt{process.execPath}}. For
\hyperref[child_processspawncommand-args-options]{\texttt{child\_process.spawn()}},
its value will be the name of the executable file. For
\hyperref[child_processexeccommand-options-callback]{\texttt{child\_process.exec()}},
its value will be the name of the shell in which the child process is
launched.

\subsubsection{\texorpdfstring{\texttt{subprocess.stderr}}{subprocess.stderr}}\label{subprocess.stderr}

\begin{itemize}
\tightlist
\item
  \{stream.Readable\textbar null\textbar undefined\}
\end{itemize}

A \texttt{Readable\ Stream} that represents the child process's
\texttt{stderr}.

If the child was spawned with \texttt{stdio{[}2{]}} set to anything
other than \texttt{\textquotesingle{}pipe\textquotesingle{}}, then this
will be \texttt{null}.

\texttt{subprocess.stderr} is an alias for
\texttt{subprocess.stdio{[}2{]}}. Both properties will refer to the same
value.

The \texttt{subprocess.stderr} property can be \texttt{null} or
\texttt{undefined} if the child process could not be successfully
spawned.

\subsubsection{\texorpdfstring{\texttt{subprocess.stdin}}{subprocess.stdin}}\label{subprocess.stdin}

\begin{itemize}
\tightlist
\item
  \{stream.Writable\textbar null\textbar undefined\}
\end{itemize}

A \texttt{Writable\ Stream} that represents the child process's
\texttt{stdin}.

If a child process waits to read all of its input, the child will not
continue until this stream has been closed via \texttt{end()}.

If the child was spawned with \texttt{stdio{[}0{]}} set to anything
other than \texttt{\textquotesingle{}pipe\textquotesingle{}}, then this
will be \texttt{null}.

\texttt{subprocess.stdin} is an alias for
\texttt{subprocess.stdio{[}0{]}}. Both properties will refer to the same
value.

The \texttt{subprocess.stdin} property can be \texttt{null} or
\texttt{undefined} if the child process could not be successfully
spawned.

\subsubsection{\texorpdfstring{\texttt{subprocess.stdio}}{subprocess.stdio}}\label{subprocess.stdio}

\begin{itemize}
\tightlist
\item
  \{Array\}
\end{itemize}

A sparse array of pipes to the child process, corresponding with
positions in the \hyperref[optionsstdio]{\texttt{stdio}} option passed
to
\hyperref[child_processspawncommand-args-options]{\texttt{child\_process.spawn()}}
that have been set to the value
\texttt{\textquotesingle{}pipe\textquotesingle{}}.
\texttt{subprocess.stdio{[}0{]}}, \texttt{subprocess.stdio{[}1{]}}, and
\texttt{subprocess.stdio{[}2{]}} are also available as
\texttt{subprocess.stdin}, \texttt{subprocess.stdout}, and
\texttt{subprocess.stderr}, respectively.

In the following example, only the child's fd \texttt{1} (stdout) is
configured as a pipe, so only the parent's
\texttt{subprocess.stdio{[}1{]}} is a stream, all other values in the
array are \texttt{null}.

\begin{Shaded}
\begin{Highlighting}[]
\KeywordTok{const}\NormalTok{ assert }\OperatorTok{=} \PreprocessorTok{require}\NormalTok{(}\StringTok{\textquotesingle{}node:assert\textquotesingle{}}\NormalTok{)}\OperatorTok{;}
\KeywordTok{const}\NormalTok{ fs }\OperatorTok{=} \PreprocessorTok{require}\NormalTok{(}\StringTok{\textquotesingle{}node:fs\textquotesingle{}}\NormalTok{)}\OperatorTok{;}
\KeywordTok{const}\NormalTok{ child\_process }\OperatorTok{=} \PreprocessorTok{require}\NormalTok{(}\StringTok{\textquotesingle{}node:child\_process\textquotesingle{}}\NormalTok{)}\OperatorTok{;}

\KeywordTok{const}\NormalTok{ subprocess }\OperatorTok{=}\NormalTok{ child\_process}\OperatorTok{.}\FunctionTok{spawn}\NormalTok{(}\StringTok{\textquotesingle{}ls\textquotesingle{}}\OperatorTok{,}\NormalTok{ \{}
  \DataTypeTok{stdio}\OperatorTok{:}\NormalTok{ [}
    \DecValTok{0}\OperatorTok{,} \CommentTok{// Use parent\textquotesingle{}s stdin for child.}
    \StringTok{\textquotesingle{}pipe\textquotesingle{}}\OperatorTok{,} \CommentTok{// Pipe child\textquotesingle{}s stdout to parent.}
\NormalTok{    fs}\OperatorTok{.}\FunctionTok{openSync}\NormalTok{(}\StringTok{\textquotesingle{}err.out\textquotesingle{}}\OperatorTok{,} \StringTok{\textquotesingle{}w\textquotesingle{}}\NormalTok{)}\OperatorTok{,} \CommentTok{// Direct child\textquotesingle{}s stderr to a file.}
\NormalTok{  ]}\OperatorTok{,}
\NormalTok{\})}\OperatorTok{;}

\NormalTok{assert}\OperatorTok{.}\FunctionTok{strictEqual}\NormalTok{(subprocess}\OperatorTok{.}\AttributeTok{stdio}\NormalTok{[}\DecValTok{0}\NormalTok{]}\OperatorTok{,} \KeywordTok{null}\NormalTok{)}\OperatorTok{;}
\NormalTok{assert}\OperatorTok{.}\FunctionTok{strictEqual}\NormalTok{(subprocess}\OperatorTok{.}\AttributeTok{stdio}\NormalTok{[}\DecValTok{0}\NormalTok{]}\OperatorTok{,}\NormalTok{ subprocess}\OperatorTok{.}\AttributeTok{stdin}\NormalTok{)}\OperatorTok{;}

\FunctionTok{assert}\NormalTok{(subprocess}\OperatorTok{.}\AttributeTok{stdout}\NormalTok{)}\OperatorTok{;}
\NormalTok{assert}\OperatorTok{.}\FunctionTok{strictEqual}\NormalTok{(subprocess}\OperatorTok{.}\AttributeTok{stdio}\NormalTok{[}\DecValTok{1}\NormalTok{]}\OperatorTok{,}\NormalTok{ subprocess}\OperatorTok{.}\AttributeTok{stdout}\NormalTok{)}\OperatorTok{;}

\NormalTok{assert}\OperatorTok{.}\FunctionTok{strictEqual}\NormalTok{(subprocess}\OperatorTok{.}\AttributeTok{stdio}\NormalTok{[}\DecValTok{2}\NormalTok{]}\OperatorTok{,} \KeywordTok{null}\NormalTok{)}\OperatorTok{;}
\NormalTok{assert}\OperatorTok{.}\FunctionTok{strictEqual}\NormalTok{(subprocess}\OperatorTok{.}\AttributeTok{stdio}\NormalTok{[}\DecValTok{2}\NormalTok{]}\OperatorTok{,}\NormalTok{ subprocess}\OperatorTok{.}\AttributeTok{stderr}\NormalTok{)}\OperatorTok{;}
\end{Highlighting}
\end{Shaded}

The \texttt{subprocess.stdio} property can be \texttt{undefined} if the
child process could not be successfully spawned.

\subsubsection{\texorpdfstring{\texttt{subprocess.stdout}}{subprocess.stdout}}\label{subprocess.stdout}

\begin{itemize}
\tightlist
\item
  \{stream.Readable\textbar null\textbar undefined\}
\end{itemize}

A \texttt{Readable\ Stream} that represents the child process's
\texttt{stdout}.

If the child was spawned with \texttt{stdio{[}1{]}} set to anything
other than \texttt{\textquotesingle{}pipe\textquotesingle{}}, then this
will be \texttt{null}.

\texttt{subprocess.stdout} is an alias for
\texttt{subprocess.stdio{[}1{]}}. Both properties will refer to the same
value.

\begin{Shaded}
\begin{Highlighting}[]
\KeywordTok{const}\NormalTok{ \{ spawn \} }\OperatorTok{=} \PreprocessorTok{require}\NormalTok{(}\StringTok{\textquotesingle{}node:child\_process\textquotesingle{}}\NormalTok{)}\OperatorTok{;}

\KeywordTok{const}\NormalTok{ subprocess }\OperatorTok{=} \FunctionTok{spawn}\NormalTok{(}\StringTok{\textquotesingle{}ls\textquotesingle{}}\NormalTok{)}\OperatorTok{;}

\NormalTok{subprocess}\OperatorTok{.}\AttributeTok{stdout}\OperatorTok{.}\FunctionTok{on}\NormalTok{(}\StringTok{\textquotesingle{}data\textquotesingle{}}\OperatorTok{,}\NormalTok{ (data) }\KeywordTok{=\textgreater{}}\NormalTok{ \{}
  \BuiltInTok{console}\OperatorTok{.}\FunctionTok{log}\NormalTok{(}\VerbatimStringTok{\textasciigrave{}Received chunk }\SpecialCharTok{$\{}\NormalTok{data}\SpecialCharTok{\}}\VerbatimStringTok{\textasciigrave{}}\NormalTok{)}\OperatorTok{;}
\NormalTok{\})}\OperatorTok{;}
\end{Highlighting}
\end{Shaded}

The \texttt{subprocess.stdout} property can be \texttt{null} or
\texttt{undefined} if the child process could not be successfully
spawned.

\subsubsection{\texorpdfstring{\texttt{subprocess.unref()}}{subprocess.unref()}}\label{subprocess.unref}

By default, the parent will wait for the detached child to exit. To
prevent the parent from waiting for a given \texttt{subprocess} to exit,
use the \texttt{subprocess.unref()} method. Doing so will cause the
parent's event loop to not include the child in its reference count,
allowing the parent to exit independently of the child, unless there is
an established IPC channel between the child and the parent.

\begin{Shaded}
\begin{Highlighting}[]
\KeywordTok{const}\NormalTok{ \{ spawn \} }\OperatorTok{=} \PreprocessorTok{require}\NormalTok{(}\StringTok{\textquotesingle{}node:child\_process\textquotesingle{}}\NormalTok{)}\OperatorTok{;}

\KeywordTok{const}\NormalTok{ subprocess }\OperatorTok{=} \FunctionTok{spawn}\NormalTok{(}\BuiltInTok{process}\OperatorTok{.}\AttributeTok{argv}\NormalTok{[}\DecValTok{0}\NormalTok{]}\OperatorTok{,}\NormalTok{ [}\StringTok{\textquotesingle{}child\_program.js\textquotesingle{}}\NormalTok{]}\OperatorTok{,}\NormalTok{ \{}
  \DataTypeTok{detached}\OperatorTok{:} \KeywordTok{true}\OperatorTok{,}
  \DataTypeTok{stdio}\OperatorTok{:} \StringTok{\textquotesingle{}ignore\textquotesingle{}}\OperatorTok{,}
\NormalTok{\})}\OperatorTok{;}

\NormalTok{subprocess}\OperatorTok{.}\FunctionTok{unref}\NormalTok{()}\OperatorTok{;}
\end{Highlighting}
\end{Shaded}

\subsection{\texorpdfstring{\texttt{maxBuffer} and
Unicode}{maxBuffer and Unicode}}\label{maxbuffer-and-unicode}

The \texttt{maxBuffer} option specifies the largest number of bytes
allowed on \texttt{stdout} or \texttt{stderr}. If this value is
exceeded, then the child process is terminated. This impacts output that
includes multibyte character encodings such as UTF-8 or UTF-16. For
instance,
\texttt{console.log(\textquotesingle{}中文测试\textquotesingle{})} will
send 13 UTF-8 encoded bytes to \texttt{stdout} although there are only 4
characters.

\subsection{Shell requirements}\label{shell-requirements}

The shell should understand the \texttt{-c} switch. If the shell is
\texttt{\textquotesingle{}cmd.exe\textquotesingle{}}, it should
understand the \texttt{/d\ /s\ /c} switches and command-line parsing
should be compatible.

\subsection{Default Windows shell}\label{default-windows-shell}

Although Microsoft specifies \texttt{\%COMSPEC\%} must contain the path
to \texttt{\textquotesingle{}cmd.exe\textquotesingle{}} in the root
environment, child processes are not always subject to the same
requirement. Thus, in \texttt{child\_process} functions where a shell
can be spawned, \texttt{\textquotesingle{}cmd.exe\textquotesingle{}} is
used as a fallback if \texttt{process.env.ComSpec} is unavailable.

\subsection{Advanced serialization}\label{advanced-serialization}

Child processes support a serialization mechanism for IPC that is based
on the \href{v8.md\#serialization-api}{serialization API of the
\texttt{node:v8} module}, based on the
\href{https://developer.mozilla.org/en-US/docs/Web/API/Web_Workers_API/Structured_clone_algorithm}{HTML
structured clone algorithm}. This is generally more powerful and
supports more built-in JavaScript object types, such as \texttt{BigInt},
\texttt{Map} and \texttt{Set}, \texttt{ArrayBuffer} and
\texttt{TypedArray}, \texttt{Buffer}, \texttt{Error}, \texttt{RegExp}
etc.

However, this format is not a full superset of JSON, and e.g.~properties
set on objects of such built-in types will not be passed on through the
serialization step. Additionally, performance may not be equivalent to
that of JSON, depending on the structure of the passed data. Therefore,
this feature requires opting in by setting the \texttt{serialization}
option to \texttt{\textquotesingle{}advanced\textquotesingle{}} when
calling
\hyperref[child_processspawncommand-args-options]{\texttt{child\_process.spawn()}}
or
\hyperref[child_processforkmodulepath-args-options]{\texttt{child\_process.fork()}}.
