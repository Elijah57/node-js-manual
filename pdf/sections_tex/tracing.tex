\section{Trace events}\label{trace-events}

\begin{quote}
Stability: 1 - Experimental
\end{quote}

The \texttt{node:trace\_events} module provides a mechanism to
centralize tracing information generated by V8, Node.js core, and
userspace code.

Tracing can be enabled with the \texttt{-\/-trace-event-categories}
command-line flag or by using the \texttt{node:trace\_events} module.
The \texttt{-\/-trace-event-categories} flag accepts a list of
comma-separated category names.

The available categories are:

\begin{itemize}
\tightlist
\item
  \texttt{node}: An empty placeholder.
\item
  \texttt{node.async\_hooks}: Enables capture of detailed
  \href{async_hooks.md}{\texttt{async\_hooks}} trace data. The
  \href{async_hooks.md}{\texttt{async\_hooks}} events have a unique
  \texttt{asyncId} and a special \texttt{triggerId}
  \texttt{triggerAsyncId} property.
\item
  \texttt{node.bootstrap}: Enables capture of Node.js bootstrap
  milestones.
\item
  \texttt{node.console}: Enables capture of \texttt{console.time()} and
  \texttt{console.count()} output.
\item
  \texttt{node.threadpoolwork.sync}: Enables capture of trace data for
  threadpool synchronous operations, such as \texttt{blob},
  \texttt{zlib}, \texttt{crypto} and \texttt{node\_api}.
\item
  \texttt{node.threadpoolwork.async}: Enables capture of trace data for
  threadpool asynchronous operations, such as \texttt{blob},
  \texttt{zlib}, \texttt{crypto} and \texttt{node\_api}.
\item
  \texttt{node.dns.native}: Enables capture of trace data for DNS
  queries.
\item
  \texttt{node.net.native}: Enables capture of trace data for network.
\item
  \texttt{node.environment}: Enables capture of Node.js Environment
  milestones.
\item
  \texttt{node.fs.sync}: Enables capture of trace data for file system
  sync methods.
\item
  \texttt{node.fs\_dir.sync}: Enables capture of trace data for file
  system sync directory methods.
\item
  \texttt{node.fs.async}: Enables capture of trace data for file system
  async methods.
\item
  \texttt{node.fs\_dir.async}: Enables capture of trace data for file
  system async directory methods.
\item
  \texttt{node.perf}: Enables capture of
  \href{perf_hooks.md}{Performance API} measurements.

  \begin{itemize}
  \tightlist
  \item
    \texttt{node.perf.usertiming}: Enables capture of only Performance
    API User Timing measures and marks.
  \item
    \texttt{node.perf.timerify}: Enables capture of only Performance API
    timerify measurements.
  \end{itemize}
\item
  \texttt{node.promises.rejections}: Enables capture of trace data
  tracking the number of unhandled Promise rejections and
  handled-after-rejections.
\item
  \texttt{node.vm.script}: Enables capture of trace data for the
  \texttt{node:vm} module's \texttt{runInNewContext()},
  \texttt{runInContext()}, and \texttt{runInThisContext()} methods.
\item
  \texttt{v8}: The \href{v8.md}{V8} events are GC, compiling, and
  execution related.
\item
  \texttt{node.http}: Enables capture of trace data for http request /
  response.
\end{itemize}

By default the \texttt{node}, \texttt{node.async\_hooks}, and
\texttt{v8} categories are enabled.

\begin{Shaded}
\begin{Highlighting}[]
\ExtensionTok{node} \AttributeTok{{-}{-}trace{-}event{-}categories}\NormalTok{ v8,node,node.async\_hooks server.js}
\end{Highlighting}
\end{Shaded}

Prior versions of Node.js required the use of the
\texttt{-\/-trace-events-enabled} flag to enable trace events. This
requirement has been removed. However, the
\texttt{-\/-trace-events-enabled} flag \emph{may} still be used and will
enable the \texttt{node}, \texttt{node.async\_hooks}, and \texttt{v8}
trace event categories by default.

\begin{Shaded}
\begin{Highlighting}[]
\ExtensionTok{node} \AttributeTok{{-}{-}trace{-}events{-}enabled}

\CommentTok{\# is equivalent to}

\ExtensionTok{node} \AttributeTok{{-}{-}trace{-}event{-}categories}\NormalTok{ v8,node,node.async\_hooks}
\end{Highlighting}
\end{Shaded}

Alternatively, trace events may be enabled using the
\texttt{node:trace\_events} module:

\begin{Shaded}
\begin{Highlighting}[]
\KeywordTok{const}\NormalTok{ trace\_events }\OperatorTok{=} \PreprocessorTok{require}\NormalTok{(}\StringTok{\textquotesingle{}node:trace\_events\textquotesingle{}}\NormalTok{)}\OperatorTok{;}
\KeywordTok{const}\NormalTok{ tracing }\OperatorTok{=}\NormalTok{ trace\_events}\OperatorTok{.}\FunctionTok{createTracing}\NormalTok{(\{ }\DataTypeTok{categories}\OperatorTok{:}\NormalTok{ [}\StringTok{\textquotesingle{}node.perf\textquotesingle{}}\NormalTok{] \})}\OperatorTok{;}
\NormalTok{tracing}\OperatorTok{.}\FunctionTok{enable}\NormalTok{()}\OperatorTok{;}  \CommentTok{// Enable trace event capture for the \textquotesingle{}node.perf\textquotesingle{} category}

\CommentTok{// do work}

\NormalTok{tracing}\OperatorTok{.}\FunctionTok{disable}\NormalTok{()}\OperatorTok{;}  \CommentTok{// Disable trace event capture for the \textquotesingle{}node.perf\textquotesingle{} category}
\end{Highlighting}
\end{Shaded}

Running Node.js with tracing enabled will produce log files that can be
opened in the
\href{https://www.chromium.org/developers/how-tos/trace-event-profiling-tool}{\texttt{chrome://tracing}}
tab of Chrome.

The logging file is by default called
\texttt{node\_trace.\$\{rotation\}.log}, where \texttt{\$\{rotation\}}
is an incrementing log-rotation id. The filepath pattern can be
specified with \texttt{-\/-trace-event-file-pattern} that accepts a
template string that supports \texttt{\$\{rotation\}} and
\texttt{\$\{pid\}}:

\begin{Shaded}
\begin{Highlighting}[]
\ExtensionTok{node} \AttributeTok{{-}{-}trace{-}event{-}categories}\NormalTok{ v8 }\AttributeTok{{-}{-}trace{-}event{-}file{-}pattern} \StringTok{\textquotesingle{}$\{pid\}{-}$\{rotation\}.log\textquotesingle{}}\NormalTok{ server.js}
\end{Highlighting}
\end{Shaded}

To guarantee that the log file is properly generated after signal events
like \texttt{SIGINT}, \texttt{SIGTERM}, or \texttt{SIGBREAK}, make sure
to have the appropriate handlers in your code, such as:

\begin{Shaded}
\begin{Highlighting}[]
\BuiltInTok{process}\OperatorTok{.}\FunctionTok{on}\NormalTok{(}\StringTok{\textquotesingle{}SIGINT\textquotesingle{}}\OperatorTok{,} \KeywordTok{function} \FunctionTok{onSigint}\NormalTok{() \{}
  \BuiltInTok{console}\OperatorTok{.}\FunctionTok{info}\NormalTok{(}\StringTok{\textquotesingle{}Received SIGINT.\textquotesingle{}}\NormalTok{)}\OperatorTok{;}
  \BuiltInTok{process}\OperatorTok{.}\FunctionTok{exit}\NormalTok{(}\DecValTok{130}\NormalTok{)}\OperatorTok{;}  \CommentTok{// Or applicable exit code depending on OS and signal}
\NormalTok{\})}\OperatorTok{;}
\end{Highlighting}
\end{Shaded}

The tracing system uses the same time source as the one used by
\texttt{process.hrtime()}. However the trace-event timestamps are
expressed in microseconds, unlike \texttt{process.hrtime()} which
returns nanoseconds.

The features from this module are not available in
\href{worker_threads.md\#class-worker}{\texttt{Worker}} threads.

\subsection{\texorpdfstring{The \texttt{node:trace\_events}
module}{The node:trace\_events module}}\label{the-nodetrace_events-module}

\subsubsection{\texorpdfstring{\texttt{Tracing}
object}{Tracing object}}\label{tracing-object}

The \texttt{Tracing} object is used to enable or disable tracing for
sets of categories. Instances are created using the
\texttt{trace\_events.createTracing()} method.

When created, the \texttt{Tracing} object is disabled. Calling the
\texttt{tracing.enable()} method adds the categories to the set of
enabled trace event categories. Calling \texttt{tracing.disable()} will
remove the categories from the set of enabled trace event categories.

\paragraph{\texorpdfstring{\texttt{tracing.categories}}{tracing.categories}}\label{tracing.categories}

\begin{itemize}
\tightlist
\item
  \{string\}
\end{itemize}

A comma-separated list of the trace event categories covered by this
\texttt{Tracing} object.

\paragraph{\texorpdfstring{\texttt{tracing.disable()}}{tracing.disable()}}\label{tracing.disable}

Disables this \texttt{Tracing} object.

Only trace event categories \emph{not} covered by other enabled
\texttt{Tracing} objects and \emph{not} specified by the
\texttt{-\/-trace-event-categories} flag will be disabled.

\begin{Shaded}
\begin{Highlighting}[]
\KeywordTok{const}\NormalTok{ trace\_events }\OperatorTok{=} \PreprocessorTok{require}\NormalTok{(}\StringTok{\textquotesingle{}node:trace\_events\textquotesingle{}}\NormalTok{)}\OperatorTok{;}
\KeywordTok{const}\NormalTok{ t1 }\OperatorTok{=}\NormalTok{ trace\_events}\OperatorTok{.}\FunctionTok{createTracing}\NormalTok{(\{ }\DataTypeTok{categories}\OperatorTok{:}\NormalTok{ [}\StringTok{\textquotesingle{}node\textquotesingle{}}\OperatorTok{,} \StringTok{\textquotesingle{}v8\textquotesingle{}}\NormalTok{] \})}\OperatorTok{;}
\KeywordTok{const}\NormalTok{ t2 }\OperatorTok{=}\NormalTok{ trace\_events}\OperatorTok{.}\FunctionTok{createTracing}\NormalTok{(\{ }\DataTypeTok{categories}\OperatorTok{:}\NormalTok{ [}\StringTok{\textquotesingle{}node.perf\textquotesingle{}}\OperatorTok{,} \StringTok{\textquotesingle{}node\textquotesingle{}}\NormalTok{] \})}\OperatorTok{;}
\NormalTok{t1}\OperatorTok{.}\FunctionTok{enable}\NormalTok{()}\OperatorTok{;}
\NormalTok{t2}\OperatorTok{.}\FunctionTok{enable}\NormalTok{()}\OperatorTok{;}

\CommentTok{// Prints \textquotesingle{}node,node.perf,v8\textquotesingle{}}
\BuiltInTok{console}\OperatorTok{.}\FunctionTok{log}\NormalTok{(trace\_events}\OperatorTok{.}\FunctionTok{getEnabledCategories}\NormalTok{())}\OperatorTok{;}

\NormalTok{t2}\OperatorTok{.}\FunctionTok{disable}\NormalTok{()}\OperatorTok{;} \CommentTok{// Will only disable emission of the \textquotesingle{}node.perf\textquotesingle{} category}

\CommentTok{// Prints \textquotesingle{}node,v8\textquotesingle{}}
\BuiltInTok{console}\OperatorTok{.}\FunctionTok{log}\NormalTok{(trace\_events}\OperatorTok{.}\FunctionTok{getEnabledCategories}\NormalTok{())}\OperatorTok{;}
\end{Highlighting}
\end{Shaded}

\paragraph{\texorpdfstring{\texttt{tracing.enable()}}{tracing.enable()}}\label{tracing.enable}

Enables this \texttt{Tracing} object for the set of categories covered
by the \texttt{Tracing} object.

\paragraph{\texorpdfstring{\texttt{tracing.enabled}}{tracing.enabled}}\label{tracing.enabled}

\begin{itemize}
\tightlist
\item
  \{boolean\} \texttt{true} only if the \texttt{Tracing} object has been
  enabled.
\end{itemize}

\subsubsection{\texorpdfstring{\texttt{trace\_events.createTracing(options)}}{trace\_events.createTracing(options)}}\label{trace_events.createtracingoptions}

\begin{itemize}
\tightlist
\item
  \texttt{options} \{Object\}

  \begin{itemize}
  \tightlist
  \item
    \texttt{categories} \{string{[}{]}\} An array of trace category
    names. Values included in the array are coerced to a string when
    possible. An error will be thrown if the value cannot be coerced.
  \end{itemize}
\item
  Returns: \{Tracing\}.
\end{itemize}

Creates and returns a \texttt{Tracing} object for the given set of
\texttt{categories}.

\begin{Shaded}
\begin{Highlighting}[]
\KeywordTok{const}\NormalTok{ trace\_events }\OperatorTok{=} \PreprocessorTok{require}\NormalTok{(}\StringTok{\textquotesingle{}node:trace\_events\textquotesingle{}}\NormalTok{)}\OperatorTok{;}
\KeywordTok{const}\NormalTok{ categories }\OperatorTok{=}\NormalTok{ [}\StringTok{\textquotesingle{}node.perf\textquotesingle{}}\OperatorTok{,} \StringTok{\textquotesingle{}node.async\_hooks\textquotesingle{}}\NormalTok{]}\OperatorTok{;}
\KeywordTok{const}\NormalTok{ tracing }\OperatorTok{=}\NormalTok{ trace\_events}\OperatorTok{.}\FunctionTok{createTracing}\NormalTok{(\{ categories \})}\OperatorTok{;}
\NormalTok{tracing}\OperatorTok{.}\FunctionTok{enable}\NormalTok{()}\OperatorTok{;}
\CommentTok{// do stuff}
\NormalTok{tracing}\OperatorTok{.}\FunctionTok{disable}\NormalTok{()}\OperatorTok{;}
\end{Highlighting}
\end{Shaded}

\subsubsection{\texorpdfstring{\texttt{trace\_events.getEnabledCategories()}}{trace\_events.getEnabledCategories()}}\label{trace_events.getenabledcategories}

\begin{itemize}
\tightlist
\item
  Returns: \{string\}
\end{itemize}

Returns a comma-separated list of all currently-enabled trace event
categories. The current set of enabled trace event categories is
determined by the \emph{union} of all currently-enabled \texttt{Tracing}
objects and any categories enabled using the
\texttt{-\/-trace-event-categories} flag.

Given the file \texttt{test.js} below, the command
\texttt{node\ -\/-trace-event-categories\ node.perf\ test.js} will print
\texttt{\textquotesingle{}node.async\_hooks,node.perf\textquotesingle{}}
to the console.

\begin{Shaded}
\begin{Highlighting}[]
\KeywordTok{const}\NormalTok{ trace\_events }\OperatorTok{=} \PreprocessorTok{require}\NormalTok{(}\StringTok{\textquotesingle{}node:trace\_events\textquotesingle{}}\NormalTok{)}\OperatorTok{;}
\KeywordTok{const}\NormalTok{ t1 }\OperatorTok{=}\NormalTok{ trace\_events}\OperatorTok{.}\FunctionTok{createTracing}\NormalTok{(\{ }\DataTypeTok{categories}\OperatorTok{:}\NormalTok{ [}\StringTok{\textquotesingle{}node.async\_hooks\textquotesingle{}}\NormalTok{] \})}\OperatorTok{;}
\KeywordTok{const}\NormalTok{ t2 }\OperatorTok{=}\NormalTok{ trace\_events}\OperatorTok{.}\FunctionTok{createTracing}\NormalTok{(\{ }\DataTypeTok{categories}\OperatorTok{:}\NormalTok{ [}\StringTok{\textquotesingle{}node.perf\textquotesingle{}}\NormalTok{] \})}\OperatorTok{;}
\KeywordTok{const}\NormalTok{ t3 }\OperatorTok{=}\NormalTok{ trace\_events}\OperatorTok{.}\FunctionTok{createTracing}\NormalTok{(\{ }\DataTypeTok{categories}\OperatorTok{:}\NormalTok{ [}\StringTok{\textquotesingle{}v8\textquotesingle{}}\NormalTok{] \})}\OperatorTok{;}

\NormalTok{t1}\OperatorTok{.}\FunctionTok{enable}\NormalTok{()}\OperatorTok{;}
\NormalTok{t2}\OperatorTok{.}\FunctionTok{enable}\NormalTok{()}\OperatorTok{;}

\BuiltInTok{console}\OperatorTok{.}\FunctionTok{log}\NormalTok{(trace\_events}\OperatorTok{.}\FunctionTok{getEnabledCategories}\NormalTok{())}\OperatorTok{;}
\end{Highlighting}
\end{Shaded}

\subsection{Examples}\label{examples}

\subsubsection{Collect trace events data by
inspector}\label{collect-trace-events-data-by-inspector}

\begin{Shaded}
\begin{Highlighting}[]
\StringTok{\textquotesingle{}use strict\textquotesingle{}}\OperatorTok{;}

\KeywordTok{const}\NormalTok{ \{ Session \} }\OperatorTok{=} \PreprocessorTok{require}\NormalTok{(}\StringTok{\textquotesingle{}inspector\textquotesingle{}}\NormalTok{)}\OperatorTok{;}
\KeywordTok{const}\NormalTok{ session }\OperatorTok{=} \KeywordTok{new} \FunctionTok{Session}\NormalTok{()}\OperatorTok{;}
\NormalTok{session}\OperatorTok{.}\FunctionTok{connect}\NormalTok{()}\OperatorTok{;}

\KeywordTok{function} \FunctionTok{post}\NormalTok{(message}\OperatorTok{,}\NormalTok{ data) \{}
  \ControlFlowTok{return} \KeywordTok{new} \BuiltInTok{Promise}\NormalTok{((resolve}\OperatorTok{,}\NormalTok{ reject) }\KeywordTok{=\textgreater{}}\NormalTok{ \{}
\NormalTok{    session}\OperatorTok{.}\FunctionTok{post}\NormalTok{(message}\OperatorTok{,}\NormalTok{ data}\OperatorTok{,}\NormalTok{ (err}\OperatorTok{,}\NormalTok{ result) }\KeywordTok{=\textgreater{}}\NormalTok{ \{}
      \ControlFlowTok{if}\NormalTok{ (err)}
        \FunctionTok{reject}\NormalTok{(}\KeywordTok{new} \BuiltInTok{Error}\NormalTok{(}\BuiltInTok{JSON}\OperatorTok{.}\FunctionTok{stringify}\NormalTok{(err)))}\OperatorTok{;}
      \ControlFlowTok{else}
        \FunctionTok{resolve}\NormalTok{(result)}\OperatorTok{;}
\NormalTok{    \})}\OperatorTok{;}
\NormalTok{  \})}\OperatorTok{;}
\NormalTok{\}}

\KeywordTok{async} \KeywordTok{function} \FunctionTok{collect}\NormalTok{() \{}
  \KeywordTok{const}\NormalTok{ data }\OperatorTok{=}\NormalTok{ []}\OperatorTok{;}
\NormalTok{  session}\OperatorTok{.}\FunctionTok{on}\NormalTok{(}\StringTok{\textquotesingle{}NodeTracing.dataCollected\textquotesingle{}}\OperatorTok{,}\NormalTok{ (chunk) }\KeywordTok{=\textgreater{}}\NormalTok{ data}\OperatorTok{.}\FunctionTok{push}\NormalTok{(chunk))}\OperatorTok{;}
\NormalTok{  session}\OperatorTok{.}\FunctionTok{on}\NormalTok{(}\StringTok{\textquotesingle{}NodeTracing.tracingComplete\textquotesingle{}}\OperatorTok{,}\NormalTok{ () }\KeywordTok{=\textgreater{}}\NormalTok{ \{}
    \CommentTok{// done}
\NormalTok{  \})}\OperatorTok{;}
  \KeywordTok{const}\NormalTok{ traceConfig }\OperatorTok{=}\NormalTok{ \{ }\DataTypeTok{includedCategories}\OperatorTok{:}\NormalTok{ [}\StringTok{\textquotesingle{}v8\textquotesingle{}}\NormalTok{] \}}\OperatorTok{;}
  \ControlFlowTok{await} \FunctionTok{post}\NormalTok{(}\StringTok{\textquotesingle{}NodeTracing.start\textquotesingle{}}\OperatorTok{,}\NormalTok{ \{ traceConfig \})}\OperatorTok{;}
  \CommentTok{// do something}
  \PreprocessorTok{setTimeout}\NormalTok{(() }\KeywordTok{=\textgreater{}}\NormalTok{ \{}
    \FunctionTok{post}\NormalTok{(}\StringTok{\textquotesingle{}NodeTracing.stop\textquotesingle{}}\NormalTok{)}\OperatorTok{.}\FunctionTok{then}\NormalTok{(() }\KeywordTok{=\textgreater{}}\NormalTok{ \{}
\NormalTok{      session}\OperatorTok{.}\FunctionTok{disconnect}\NormalTok{()}\OperatorTok{;}
      \BuiltInTok{console}\OperatorTok{.}\FunctionTok{log}\NormalTok{(data)}\OperatorTok{;}
\NormalTok{    \})}\OperatorTok{;}
\NormalTok{  \}}\OperatorTok{,} \DecValTok{1000}\NormalTok{)}\OperatorTok{;}
\NormalTok{\}}

\FunctionTok{collect}\NormalTok{()}\OperatorTok{;}
\end{Highlighting}
\end{Shaded}
