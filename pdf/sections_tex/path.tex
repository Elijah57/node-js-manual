\section{Path}\label{path}

\begin{quote}
Stability: 2 - Stable
\end{quote}

The \texttt{node:path} module provides utilities for working with file
and directory paths. It can be accessed using:

\begin{Shaded}
\begin{Highlighting}[]
\KeywordTok{const}\NormalTok{ path }\OperatorTok{=} \PreprocessorTok{require}\NormalTok{(}\StringTok{\textquotesingle{}node:path\textquotesingle{}}\NormalTok{)}\OperatorTok{;}
\end{Highlighting}
\end{Shaded}

\subsection{Windows vs.~POSIX}\label{windows-vs.-posix}

The default operation of the \texttt{node:path} module varies based on
the operating system on which a Node.js application is running.
Specifically, when running on a Windows operating system, the
\texttt{node:path} module will assume that Windows-style paths are being
used.

So using \texttt{path.basename()} might yield different results on POSIX
and Windows:

On POSIX:

\begin{Shaded}
\begin{Highlighting}[]
\NormalTok{path}\OperatorTok{.}\FunctionTok{basename}\NormalTok{(}\StringTok{\textquotesingle{}C:}\SpecialCharTok{\textbackslash{}\textbackslash{}}\StringTok{temp}\SpecialCharTok{\textbackslash{}\textbackslash{}}\StringTok{myfile.html\textquotesingle{}}\NormalTok{)}\OperatorTok{;}
\CommentTok{// Returns: \textquotesingle{}C:\textbackslash{}\textbackslash{}temp\textbackslash{}\textbackslash{}myfile.html\textquotesingle{}}
\end{Highlighting}
\end{Shaded}

On Windows:

\begin{Shaded}
\begin{Highlighting}[]
\NormalTok{path}\OperatorTok{.}\FunctionTok{basename}\NormalTok{(}\StringTok{\textquotesingle{}C:}\SpecialCharTok{\textbackslash{}\textbackslash{}}\StringTok{temp}\SpecialCharTok{\textbackslash{}\textbackslash{}}\StringTok{myfile.html\textquotesingle{}}\NormalTok{)}\OperatorTok{;}
\CommentTok{// Returns: \textquotesingle{}myfile.html\textquotesingle{}}
\end{Highlighting}
\end{Shaded}

To achieve consistent results when working with Windows file paths on
any operating system, use \hyperref[pathwin32]{\texttt{path.win32}}:

On POSIX and Windows:

\begin{Shaded}
\begin{Highlighting}[]
\NormalTok{path}\OperatorTok{.}\AttributeTok{win32}\OperatorTok{.}\FunctionTok{basename}\NormalTok{(}\StringTok{\textquotesingle{}C:}\SpecialCharTok{\textbackslash{}\textbackslash{}}\StringTok{temp}\SpecialCharTok{\textbackslash{}\textbackslash{}}\StringTok{myfile.html\textquotesingle{}}\NormalTok{)}\OperatorTok{;}
\CommentTok{// Returns: \textquotesingle{}myfile.html\textquotesingle{}}
\end{Highlighting}
\end{Shaded}

To achieve consistent results when working with POSIX file paths on any
operating system, use \hyperref[pathposix]{\texttt{path.posix}}:

On POSIX and Windows:

\begin{Shaded}
\begin{Highlighting}[]
\NormalTok{path}\OperatorTok{.}\AttributeTok{posix}\OperatorTok{.}\FunctionTok{basename}\NormalTok{(}\StringTok{\textquotesingle{}/tmp/myfile.html\textquotesingle{}}\NormalTok{)}\OperatorTok{;}
\CommentTok{// Returns: \textquotesingle{}myfile.html\textquotesingle{}}
\end{Highlighting}
\end{Shaded}

On Windows Node.js follows the concept of per-drive working directory.
This behavior can be observed when using a drive path without a
backslash. For example,
\texttt{path.resolve(\textquotesingle{}C:\textbackslash{}\textbackslash{}\textquotesingle{})}
can potentially return a different result than
\texttt{path.resolve(\textquotesingle{}C:\textquotesingle{})}. For more
information, see
\href{https://docs.microsoft.com/en-us/windows/desktop/FileIO/naming-a-file\#fully-qualified-vs-relative-paths}{this
MSDN page}.

\subsection{\texorpdfstring{\texttt{path.basename(path{[},\ suffix{]})}}{path.basename(path{[}, suffix{]})}}\label{path.basenamepath-suffix}

\begin{itemize}
\tightlist
\item
  \texttt{path} \{string\}
\item
  \texttt{suffix} \{string\} An optional suffix to remove
\item
  Returns: \{string\}
\end{itemize}

The \texttt{path.basename()} method returns the last portion of a
\texttt{path}, similar to the Unix \texttt{basename} command. Trailing
\hyperref[pathsep]{directory separators} are ignored.

\begin{Shaded}
\begin{Highlighting}[]
\NormalTok{path}\OperatorTok{.}\FunctionTok{basename}\NormalTok{(}\StringTok{\textquotesingle{}/foo/bar/baz/asdf/quux.html\textquotesingle{}}\NormalTok{)}\OperatorTok{;}
\CommentTok{// Returns: \textquotesingle{}quux.html\textquotesingle{}}

\NormalTok{path}\OperatorTok{.}\FunctionTok{basename}\NormalTok{(}\StringTok{\textquotesingle{}/foo/bar/baz/asdf/quux.html\textquotesingle{}}\OperatorTok{,} \StringTok{\textquotesingle{}.html\textquotesingle{}}\NormalTok{)}\OperatorTok{;}
\CommentTok{// Returns: \textquotesingle{}quux\textquotesingle{}}
\end{Highlighting}
\end{Shaded}

Although Windows usually treats file names, including file extensions,
in a case-insensitive manner, this function does not. For example,
\texttt{C:\textbackslash{}\textbackslash{}foo.html} and
\texttt{C:\textbackslash{}\textbackslash{}foo.HTML} refer to the same
file, but \texttt{basename} treats the extension as a case-sensitive
string:

\begin{Shaded}
\begin{Highlighting}[]
\NormalTok{path}\OperatorTok{.}\AttributeTok{win32}\OperatorTok{.}\FunctionTok{basename}\NormalTok{(}\StringTok{\textquotesingle{}C:}\SpecialCharTok{\textbackslash{}\textbackslash{}}\StringTok{foo.html\textquotesingle{}}\OperatorTok{,} \StringTok{\textquotesingle{}.html\textquotesingle{}}\NormalTok{)}\OperatorTok{;}
\CommentTok{// Returns: \textquotesingle{}foo\textquotesingle{}}

\NormalTok{path}\OperatorTok{.}\AttributeTok{win32}\OperatorTok{.}\FunctionTok{basename}\NormalTok{(}\StringTok{\textquotesingle{}C:}\SpecialCharTok{\textbackslash{}\textbackslash{}}\StringTok{foo.HTML\textquotesingle{}}\OperatorTok{,} \StringTok{\textquotesingle{}.html\textquotesingle{}}\NormalTok{)}\OperatorTok{;}
\CommentTok{// Returns: \textquotesingle{}foo.HTML\textquotesingle{}}
\end{Highlighting}
\end{Shaded}

A \href{errors.md\#class-typeerror}{\texttt{TypeError}} is thrown if
\texttt{path} is not a string or if \texttt{suffix} is given and is not
a string.

\subsection{\texorpdfstring{\texttt{path.delimiter}}{path.delimiter}}\label{path.delimiter}

\begin{itemize}
\tightlist
\item
  \{string\}
\end{itemize}

Provides the platform-specific path delimiter:

\begin{itemize}
\tightlist
\item
  \texttt{;} for Windows
\item
  \texttt{:} for POSIX
\end{itemize}

For example, on POSIX:

\begin{Shaded}
\begin{Highlighting}[]
\BuiltInTok{console}\OperatorTok{.}\FunctionTok{log}\NormalTok{(}\BuiltInTok{process}\OperatorTok{.}\AttributeTok{env}\OperatorTok{.}\AttributeTok{PATH}\NormalTok{)}\OperatorTok{;}
\CommentTok{// Prints: \textquotesingle{}/usr/bin:/bin:/usr/sbin:/sbin:/usr/local/bin\textquotesingle{}}

\BuiltInTok{process}\OperatorTok{.}\AttributeTok{env}\OperatorTok{.}\AttributeTok{PATH}\OperatorTok{.}\FunctionTok{split}\NormalTok{(path}\OperatorTok{.}\AttributeTok{delimiter}\NormalTok{)}\OperatorTok{;}
\CommentTok{// Returns: [\textquotesingle{}/usr/bin\textquotesingle{}, \textquotesingle{}/bin\textquotesingle{}, \textquotesingle{}/usr/sbin\textquotesingle{}, \textquotesingle{}/sbin\textquotesingle{}, \textquotesingle{}/usr/local/bin\textquotesingle{}]}
\end{Highlighting}
\end{Shaded}

On Windows:

\begin{Shaded}
\begin{Highlighting}[]
\BuiltInTok{console}\OperatorTok{.}\FunctionTok{log}\NormalTok{(}\BuiltInTok{process}\OperatorTok{.}\AttributeTok{env}\OperatorTok{.}\AttributeTok{PATH}\NormalTok{)}\OperatorTok{;}
\CommentTok{// Prints: \textquotesingle{}C:\textbackslash{}Windows\textbackslash{}system32;C:\textbackslash{}Windows;C:\textbackslash{}Program Files\textbackslash{}node\textbackslash{}\textquotesingle{}}

\BuiltInTok{process}\OperatorTok{.}\AttributeTok{env}\OperatorTok{.}\AttributeTok{PATH}\OperatorTok{.}\FunctionTok{split}\NormalTok{(path}\OperatorTok{.}\AttributeTok{delimiter}\NormalTok{)}\OperatorTok{;}
\CommentTok{// Returns [\textquotesingle{}C:\textbackslash{}\textbackslash{}Windows\textbackslash{}\textbackslash{}system32\textquotesingle{}, \textquotesingle{}C:\textbackslash{}\textbackslash{}Windows\textquotesingle{}, \textquotesingle{}C:\textbackslash{}\textbackslash{}Program Files\textbackslash{}\textbackslash{}node\textbackslash{}\textbackslash{}\textquotesingle{}]}
\end{Highlighting}
\end{Shaded}

\subsection{\texorpdfstring{\texttt{path.dirname(path)}}{path.dirname(path)}}\label{path.dirnamepath}

\begin{itemize}
\tightlist
\item
  \texttt{path} \{string\}
\item
  Returns: \{string\}
\end{itemize}

The \texttt{path.dirname()} method returns the directory name of a
\texttt{path}, similar to the Unix \texttt{dirname} command. Trailing
directory separators are ignored, see
\hyperref[pathsep]{\texttt{path.sep}}.

\begin{Shaded}
\begin{Highlighting}[]
\NormalTok{path}\OperatorTok{.}\FunctionTok{dirname}\NormalTok{(}\StringTok{\textquotesingle{}/foo/bar/baz/asdf/quux\textquotesingle{}}\NormalTok{)}\OperatorTok{;}
\CommentTok{// Returns: \textquotesingle{}/foo/bar/baz/asdf\textquotesingle{}}
\end{Highlighting}
\end{Shaded}

A \href{errors.md\#class-typeerror}{\texttt{TypeError}} is thrown if
\texttt{path} is not a string.

\subsection{\texorpdfstring{\texttt{path.extname(path)}}{path.extname(path)}}\label{path.extnamepath}

\begin{itemize}
\tightlist
\item
  \texttt{path} \{string\}
\item
  Returns: \{string\}
\end{itemize}

The \texttt{path.extname()} method returns the extension of the
\texttt{path}, from the last occurrence of the \texttt{.} (period)
character to end of string in the last portion of the \texttt{path}. If
there is no \texttt{.} in the last portion of the \texttt{path}, or if
there are no \texttt{.} characters other than the first character of the
basename of \texttt{path} (see \texttt{path.basename()}) , an empty
string is returned.

\begin{Shaded}
\begin{Highlighting}[]
\NormalTok{path}\OperatorTok{.}\FunctionTok{extname}\NormalTok{(}\StringTok{\textquotesingle{}index.html\textquotesingle{}}\NormalTok{)}\OperatorTok{;}
\CommentTok{// Returns: \textquotesingle{}.html\textquotesingle{}}

\NormalTok{path}\OperatorTok{.}\FunctionTok{extname}\NormalTok{(}\StringTok{\textquotesingle{}index.coffee.md\textquotesingle{}}\NormalTok{)}\OperatorTok{;}
\CommentTok{// Returns: \textquotesingle{}.md\textquotesingle{}}

\NormalTok{path}\OperatorTok{.}\FunctionTok{extname}\NormalTok{(}\StringTok{\textquotesingle{}index.\textquotesingle{}}\NormalTok{)}\OperatorTok{;}
\CommentTok{// Returns: \textquotesingle{}.\textquotesingle{}}

\NormalTok{path}\OperatorTok{.}\FunctionTok{extname}\NormalTok{(}\StringTok{\textquotesingle{}index\textquotesingle{}}\NormalTok{)}\OperatorTok{;}
\CommentTok{// Returns: \textquotesingle{}\textquotesingle{}}

\NormalTok{path}\OperatorTok{.}\FunctionTok{extname}\NormalTok{(}\StringTok{\textquotesingle{}.index\textquotesingle{}}\NormalTok{)}\OperatorTok{;}
\CommentTok{// Returns: \textquotesingle{}\textquotesingle{}}

\NormalTok{path}\OperatorTok{.}\FunctionTok{extname}\NormalTok{(}\StringTok{\textquotesingle{}.index.md\textquotesingle{}}\NormalTok{)}\OperatorTok{;}
\CommentTok{// Returns: \textquotesingle{}.md\textquotesingle{}}
\end{Highlighting}
\end{Shaded}

A \href{errors.md\#class-typeerror}{\texttt{TypeError}} is thrown if
\texttt{path} is not a string.

\subsection{\texorpdfstring{\texttt{path.format(pathObject)}}{path.format(pathObject)}}\label{path.formatpathobject}

\begin{itemize}
\tightlist
\item
  \texttt{pathObject} \{Object\} Any JavaScript object having the
  following properties:

  \begin{itemize}
  \tightlist
  \item
    \texttt{dir} \{string\}
  \item
    \texttt{root} \{string\}
  \item
    \texttt{base} \{string\}
  \item
    \texttt{name} \{string\}
  \item
    \texttt{ext} \{string\}
  \end{itemize}
\item
  Returns: \{string\}
\end{itemize}

The \texttt{path.format()} method returns a path string from an object.
This is the opposite of \hyperref[pathparsepath]{\texttt{path.parse()}}.

When providing properties to the \texttt{pathObject} remember that there
are combinations where one property has priority over another:

\begin{itemize}
\tightlist
\item
  \texttt{pathObject.root} is ignored if \texttt{pathObject.dir} is
  provided
\item
  \texttt{pathObject.ext} and \texttt{pathObject.name} are ignored if
  \texttt{pathObject.base} exists
\end{itemize}

For example, on POSIX:

\begin{Shaded}
\begin{Highlighting}[]
\CommentTok{// If \textasciigrave{}dir\textasciigrave{}, \textasciigrave{}root\textasciigrave{} and \textasciigrave{}base\textasciigrave{} are provided,}
\CommentTok{// \textasciigrave{}$\{dir\}$\{path.sep\}$\{base\}\textasciigrave{}}
\CommentTok{// will be returned. \textasciigrave{}root\textasciigrave{} is ignored.}
\NormalTok{path}\OperatorTok{.}\FunctionTok{format}\NormalTok{(\{}
  \DataTypeTok{root}\OperatorTok{:} \StringTok{\textquotesingle{}/ignored\textquotesingle{}}\OperatorTok{,}
  \DataTypeTok{dir}\OperatorTok{:} \StringTok{\textquotesingle{}/home/user/dir\textquotesingle{}}\OperatorTok{,}
  \DataTypeTok{base}\OperatorTok{:} \StringTok{\textquotesingle{}file.txt\textquotesingle{}}\OperatorTok{,}
\NormalTok{\})}\OperatorTok{;}
\CommentTok{// Returns: \textquotesingle{}/home/user/dir/file.txt\textquotesingle{}}

\CommentTok{// \textasciigrave{}root\textasciigrave{} will be used if \textasciigrave{}dir\textasciigrave{} is not specified.}
\CommentTok{// If only \textasciigrave{}root\textasciigrave{} is provided or \textasciigrave{}dir\textasciigrave{} is equal to \textasciigrave{}root\textasciigrave{} then the}
\CommentTok{// platform separator will not be included. \textasciigrave{}ext\textasciigrave{} will be ignored.}
\NormalTok{path}\OperatorTok{.}\FunctionTok{format}\NormalTok{(\{}
  \DataTypeTok{root}\OperatorTok{:} \StringTok{\textquotesingle{}/\textquotesingle{}}\OperatorTok{,}
  \DataTypeTok{base}\OperatorTok{:} \StringTok{\textquotesingle{}file.txt\textquotesingle{}}\OperatorTok{,}
  \DataTypeTok{ext}\OperatorTok{:} \StringTok{\textquotesingle{}ignored\textquotesingle{}}\OperatorTok{,}
\NormalTok{\})}\OperatorTok{;}
\CommentTok{// Returns: \textquotesingle{}/file.txt\textquotesingle{}}

\CommentTok{// \textasciigrave{}name\textasciigrave{} + \textasciigrave{}ext\textasciigrave{} will be used if \textasciigrave{}base\textasciigrave{} is not specified.}
\NormalTok{path}\OperatorTok{.}\FunctionTok{format}\NormalTok{(\{}
  \DataTypeTok{root}\OperatorTok{:} \StringTok{\textquotesingle{}/\textquotesingle{}}\OperatorTok{,}
  \DataTypeTok{name}\OperatorTok{:} \StringTok{\textquotesingle{}file\textquotesingle{}}\OperatorTok{,}
  \DataTypeTok{ext}\OperatorTok{:} \StringTok{\textquotesingle{}.txt\textquotesingle{}}\OperatorTok{,}
\NormalTok{\})}\OperatorTok{;}
\CommentTok{// Returns: \textquotesingle{}/file.txt\textquotesingle{}}

\CommentTok{// The dot will be added if it is not specified in \textasciigrave{}ext\textasciigrave{}.}
\NormalTok{path}\OperatorTok{.}\FunctionTok{format}\NormalTok{(\{}
  \DataTypeTok{root}\OperatorTok{:} \StringTok{\textquotesingle{}/\textquotesingle{}}\OperatorTok{,}
  \DataTypeTok{name}\OperatorTok{:} \StringTok{\textquotesingle{}file\textquotesingle{}}\OperatorTok{,}
  \DataTypeTok{ext}\OperatorTok{:} \StringTok{\textquotesingle{}txt\textquotesingle{}}\OperatorTok{,}
\NormalTok{\})}\OperatorTok{;}
\CommentTok{// Returns: \textquotesingle{}/file.txt\textquotesingle{}}
\end{Highlighting}
\end{Shaded}

On Windows:

\begin{Shaded}
\begin{Highlighting}[]
\NormalTok{path}\OperatorTok{.}\FunctionTok{format}\NormalTok{(\{}
  \DataTypeTok{dir}\OperatorTok{:} \StringTok{\textquotesingle{}C:}\SpecialCharTok{\textbackslash{}\textbackslash{}}\StringTok{path}\SpecialCharTok{\textbackslash{}\textbackslash{}}\StringTok{dir\textquotesingle{}}\OperatorTok{,}
  \DataTypeTok{base}\OperatorTok{:} \StringTok{\textquotesingle{}file.txt\textquotesingle{}}\OperatorTok{,}
\NormalTok{\})}\OperatorTok{;}
\CommentTok{// Returns: \textquotesingle{}C:\textbackslash{}\textbackslash{}path\textbackslash{}\textbackslash{}dir\textbackslash{}\textbackslash{}file.txt\textquotesingle{}}
\end{Highlighting}
\end{Shaded}

\subsection{\texorpdfstring{\texttt{path.isAbsolute(path)}}{path.isAbsolute(path)}}\label{path.isabsolutepath}

\begin{itemize}
\tightlist
\item
  \texttt{path} \{string\}
\item
  Returns: \{boolean\}
\end{itemize}

The \texttt{path.isAbsolute()} method determines if \texttt{path} is an
absolute path.

If the given \texttt{path} is a zero-length string, \texttt{false} will
be returned.

For example, on POSIX:

\begin{Shaded}
\begin{Highlighting}[]
\NormalTok{path}\OperatorTok{.}\FunctionTok{isAbsolute}\NormalTok{(}\StringTok{\textquotesingle{}/foo/bar\textquotesingle{}}\NormalTok{)}\OperatorTok{;} \CommentTok{// true}
\NormalTok{path}\OperatorTok{.}\FunctionTok{isAbsolute}\NormalTok{(}\StringTok{\textquotesingle{}/baz/..\textquotesingle{}}\NormalTok{)}\OperatorTok{;}  \CommentTok{// true}
\NormalTok{path}\OperatorTok{.}\FunctionTok{isAbsolute}\NormalTok{(}\StringTok{\textquotesingle{}qux/\textquotesingle{}}\NormalTok{)}\OperatorTok{;}     \CommentTok{// false}
\NormalTok{path}\OperatorTok{.}\FunctionTok{isAbsolute}\NormalTok{(}\StringTok{\textquotesingle{}.\textquotesingle{}}\NormalTok{)}\OperatorTok{;}        \CommentTok{// false}
\end{Highlighting}
\end{Shaded}

On Windows:

\begin{Shaded}
\begin{Highlighting}[]
\NormalTok{path}\OperatorTok{.}\FunctionTok{isAbsolute}\NormalTok{(}\StringTok{\textquotesingle{}//server\textquotesingle{}}\NormalTok{)}\OperatorTok{;}    \CommentTok{// true}
\NormalTok{path}\OperatorTok{.}\FunctionTok{isAbsolute}\NormalTok{(}\StringTok{\textquotesingle{}}\SpecialCharTok{\textbackslash{}\textbackslash{}\textbackslash{}\textbackslash{}}\StringTok{server\textquotesingle{}}\NormalTok{)}\OperatorTok{;}  \CommentTok{// true}
\NormalTok{path}\OperatorTok{.}\FunctionTok{isAbsolute}\NormalTok{(}\StringTok{\textquotesingle{}C:/foo/..\textquotesingle{}}\NormalTok{)}\OperatorTok{;}   \CommentTok{// true}
\NormalTok{path}\OperatorTok{.}\FunctionTok{isAbsolute}\NormalTok{(}\StringTok{\textquotesingle{}C:}\SpecialCharTok{\textbackslash{}\textbackslash{}}\StringTok{foo}\SpecialCharTok{\textbackslash{}\textbackslash{}}\StringTok{..\textquotesingle{}}\NormalTok{)}\OperatorTok{;} \CommentTok{// true}
\NormalTok{path}\OperatorTok{.}\FunctionTok{isAbsolute}\NormalTok{(}\StringTok{\textquotesingle{}bar}\SpecialCharTok{\textbackslash{}\textbackslash{}}\StringTok{baz\textquotesingle{}}\NormalTok{)}\OperatorTok{;}    \CommentTok{// false}
\NormalTok{path}\OperatorTok{.}\FunctionTok{isAbsolute}\NormalTok{(}\StringTok{\textquotesingle{}bar/baz\textquotesingle{}}\NormalTok{)}\OperatorTok{;}     \CommentTok{// false}
\NormalTok{path}\OperatorTok{.}\FunctionTok{isAbsolute}\NormalTok{(}\StringTok{\textquotesingle{}.\textquotesingle{}}\NormalTok{)}\OperatorTok{;}           \CommentTok{// false}
\end{Highlighting}
\end{Shaded}

A \href{errors.md\#class-typeerror}{\texttt{TypeError}} is thrown if
\texttt{path} is not a string.

\subsection{\texorpdfstring{\texttt{path.join({[}...paths{]})}}{path.join({[}...paths{]})}}\label{path.join...paths}

\begin{itemize}
\tightlist
\item
  \texttt{...paths} \{string\} A sequence of path segments
\item
  Returns: \{string\}
\end{itemize}

The \texttt{path.join()} method joins all given \texttt{path} segments
together using the platform-specific separator as a delimiter, then
normalizes the resulting path.

Zero-length \texttt{path} segments are ignored. If the joined path
string is a zero-length string then
\texttt{\textquotesingle{}.\textquotesingle{}} will be returned,
representing the current working directory.

\begin{Shaded}
\begin{Highlighting}[]
\NormalTok{path}\OperatorTok{.}\FunctionTok{join}\NormalTok{(}\StringTok{\textquotesingle{}/foo\textquotesingle{}}\OperatorTok{,} \StringTok{\textquotesingle{}bar\textquotesingle{}}\OperatorTok{,} \StringTok{\textquotesingle{}baz/asdf\textquotesingle{}}\OperatorTok{,} \StringTok{\textquotesingle{}quux\textquotesingle{}}\OperatorTok{,} \StringTok{\textquotesingle{}..\textquotesingle{}}\NormalTok{)}\OperatorTok{;}
\CommentTok{// Returns: \textquotesingle{}/foo/bar/baz/asdf\textquotesingle{}}

\NormalTok{path}\OperatorTok{.}\FunctionTok{join}\NormalTok{(}\StringTok{\textquotesingle{}foo\textquotesingle{}}\OperatorTok{,}\NormalTok{ \{\}}\OperatorTok{,} \StringTok{\textquotesingle{}bar\textquotesingle{}}\NormalTok{)}\OperatorTok{;}
\CommentTok{// Throws \textquotesingle{}TypeError: Path must be a string. Received \{\}\textquotesingle{}}
\end{Highlighting}
\end{Shaded}

A \href{errors.md\#class-typeerror}{\texttt{TypeError}} is thrown if any
of the path segments is not a string.

\subsection{\texorpdfstring{\texttt{path.normalize(path)}}{path.normalize(path)}}\label{path.normalizepath}

\begin{itemize}
\tightlist
\item
  \texttt{path} \{string\}
\item
  Returns: \{string\}
\end{itemize}

The \texttt{path.normalize()} method normalizes the given \texttt{path},
resolving \texttt{\textquotesingle{}..\textquotesingle{}} and
\texttt{\textquotesingle{}.\textquotesingle{}} segments.

When multiple, sequential path segment separation characters are found
(e.g. \texttt{/} on POSIX and either \texttt{\textbackslash{}} or
\texttt{/} on Windows), they are replaced by a single instance of the
platform-specific path segment separator (\texttt{/} on POSIX and
\texttt{\textbackslash{}} on Windows). Trailing separators are
preserved.

If the \texttt{path} is a zero-length string,
\texttt{\textquotesingle{}.\textquotesingle{}} is returned, representing
the current working directory.

For example, on POSIX:

\begin{Shaded}
\begin{Highlighting}[]
\NormalTok{path}\OperatorTok{.}\FunctionTok{normalize}\NormalTok{(}\StringTok{\textquotesingle{}/foo/bar//baz/asdf/quux/..\textquotesingle{}}\NormalTok{)}\OperatorTok{;}
\CommentTok{// Returns: \textquotesingle{}/foo/bar/baz/asdf\textquotesingle{}}
\end{Highlighting}
\end{Shaded}

On Windows:

\begin{Shaded}
\begin{Highlighting}[]
\NormalTok{path}\OperatorTok{.}\FunctionTok{normalize}\NormalTok{(}\StringTok{\textquotesingle{}C:}\SpecialCharTok{\textbackslash{}\textbackslash{}}\StringTok{temp}\SpecialCharTok{\textbackslash{}\textbackslash{}\textbackslash{}\textbackslash{}}\StringTok{foo}\SpecialCharTok{\textbackslash{}\textbackslash{}}\StringTok{bar}\SpecialCharTok{\textbackslash{}\textbackslash{}}\StringTok{..}\SpecialCharTok{\textbackslash{}\textbackslash{}}\StringTok{\textquotesingle{}}\NormalTok{)}\OperatorTok{;}
\CommentTok{// Returns: \textquotesingle{}C:\textbackslash{}\textbackslash{}temp\textbackslash{}\textbackslash{}foo\textbackslash{}\textbackslash{}\textquotesingle{}}
\end{Highlighting}
\end{Shaded}

Since Windows recognizes multiple path separators, both separators will
be replaced by instances of the Windows preferred separator
(\texttt{\textbackslash{}}):

\begin{Shaded}
\begin{Highlighting}[]
\NormalTok{path}\OperatorTok{.}\AttributeTok{win32}\OperatorTok{.}\FunctionTok{normalize}\NormalTok{(}\StringTok{\textquotesingle{}C:////temp}\SpecialCharTok{\textbackslash{}\textbackslash{}\textbackslash{}\textbackslash{}}\StringTok{/}\SpecialCharTok{\textbackslash{}\textbackslash{}}\StringTok{/}\SpecialCharTok{\textbackslash{}\textbackslash{}}\StringTok{/foo/bar\textquotesingle{}}\NormalTok{)}\OperatorTok{;}
\CommentTok{// Returns: \textquotesingle{}C:\textbackslash{}\textbackslash{}temp\textbackslash{}\textbackslash{}foo\textbackslash{}\textbackslash{}bar\textquotesingle{}}
\end{Highlighting}
\end{Shaded}

A \href{errors.md\#class-typeerror}{\texttt{TypeError}} is thrown if
\texttt{path} is not a string.

\subsection{\texorpdfstring{\texttt{path.parse(path)}}{path.parse(path)}}\label{path.parsepath}

\begin{itemize}
\tightlist
\item
  \texttt{path} \{string\}
\item
  Returns: \{Object\}
\end{itemize}

The \texttt{path.parse()} method returns an object whose properties
represent significant elements of the \texttt{path}. Trailing directory
separators are ignored, see \hyperref[pathsep]{\texttt{path.sep}}.

The returned object will have the following properties:

\begin{itemize}
\tightlist
\item
  \texttt{dir} \{string\}
\item
  \texttt{root} \{string\}
\item
  \texttt{base} \{string\}
\item
  \texttt{name} \{string\}
\item
  \texttt{ext} \{string\}
\end{itemize}

For example, on POSIX:

\begin{Shaded}
\begin{Highlighting}[]
\NormalTok{path}\OperatorTok{.}\FunctionTok{parse}\NormalTok{(}\StringTok{\textquotesingle{}/home/user/dir/file.txt\textquotesingle{}}\NormalTok{)}\OperatorTok{;}
\CommentTok{// Returns:}
\CommentTok{// \{ root: \textquotesingle{}/\textquotesingle{},}
\CommentTok{//   dir: \textquotesingle{}/home/user/dir\textquotesingle{},}
\CommentTok{//   base: \textquotesingle{}file.txt\textquotesingle{},}
\CommentTok{//   ext: \textquotesingle{}.txt\textquotesingle{},}
\CommentTok{//   name: \textquotesingle{}file\textquotesingle{} \}}
\end{Highlighting}
\end{Shaded}

\begin{Shaded}
\begin{Highlighting}[]
\NormalTok{┌─────────────────────┬────────────┐}
\NormalTok{│          dir        │    base    │}
\NormalTok{├──────┬              ├──────┬─────┤}
\NormalTok{│ root │              │ name │ ext │}
\NormalTok{"  /    home/user/dir / file  .txt "}
\NormalTok{└──────┴──────────────┴──────┴─────┘}
\NormalTok{(All spaces in the "" line should be ignored. They are purely for formatting.)}
\end{Highlighting}
\end{Shaded}

On Windows:

\begin{Shaded}
\begin{Highlighting}[]
\NormalTok{path}\OperatorTok{.}\FunctionTok{parse}\NormalTok{(}\StringTok{\textquotesingle{}C:}\SpecialCharTok{\textbackslash{}\textbackslash{}}\StringTok{path}\SpecialCharTok{\textbackslash{}\textbackslash{}}\StringTok{dir}\SpecialCharTok{\textbackslash{}\textbackslash{}}\StringTok{file.txt\textquotesingle{}}\NormalTok{)}\OperatorTok{;}
\CommentTok{// Returns:}
\CommentTok{// \{ root: \textquotesingle{}C:\textbackslash{}\textbackslash{}\textquotesingle{},}
\CommentTok{//   dir: \textquotesingle{}C:\textbackslash{}\textbackslash{}path\textbackslash{}\textbackslash{}dir\textquotesingle{},}
\CommentTok{//   base: \textquotesingle{}file.txt\textquotesingle{},}
\CommentTok{//   ext: \textquotesingle{}.txt\textquotesingle{},}
\CommentTok{//   name: \textquotesingle{}file\textquotesingle{} \}}
\end{Highlighting}
\end{Shaded}

\begin{Shaded}
\begin{Highlighting}[]
\NormalTok{┌─────────────────────┬────────────┐}
\NormalTok{│          dir        │    base    │}
\NormalTok{├──────┬              ├──────┬─────┤}
\NormalTok{│ root │              │ name │ ext │}
\NormalTok{" C:\textbackslash{}      path\textbackslash{}dir   \textbackslash{} file  .txt "}
\NormalTok{└──────┴──────────────┴──────┴─────┘}
\NormalTok{(All spaces in the "" line should be ignored. They are purely for formatting.)}
\end{Highlighting}
\end{Shaded}

A \href{errors.md\#class-typeerror}{\texttt{TypeError}} is thrown if
\texttt{path} is not a string.

\subsection{\texorpdfstring{\texttt{path.posix}}{path.posix}}\label{path.posix}

\begin{itemize}
\tightlist
\item
  \{Object\}
\end{itemize}

The \texttt{path.posix} property provides access to POSIX specific
implementations of the \texttt{path} methods.

The API is accessible via
\texttt{require(\textquotesingle{}node:path\textquotesingle{}).posix} or
\texttt{require(\textquotesingle{}node:path/posix\textquotesingle{})}.

\subsection{\texorpdfstring{\texttt{path.relative(from,\ to)}}{path.relative(from, to)}}\label{path.relativefrom-to}

\begin{itemize}
\tightlist
\item
  \texttt{from} \{string\}
\item
  \texttt{to} \{string\}
\item
  Returns: \{string\}
\end{itemize}

The \texttt{path.relative()} method returns the relative path from
\texttt{from} to \texttt{to} based on the current working directory. If
\texttt{from} and \texttt{to} each resolve to the same path (after
calling \texttt{path.resolve()} on each), a zero-length string is
returned.

If a zero-length string is passed as \texttt{from} or \texttt{to}, the
current working directory will be used instead of the zero-length
strings.

For example, on POSIX:

\begin{Shaded}
\begin{Highlighting}[]
\NormalTok{path}\OperatorTok{.}\FunctionTok{relative}\NormalTok{(}\StringTok{\textquotesingle{}/data/orandea/test/aaa\textquotesingle{}}\OperatorTok{,} \StringTok{\textquotesingle{}/data/orandea/impl/bbb\textquotesingle{}}\NormalTok{)}\OperatorTok{;}
\CommentTok{// Returns: \textquotesingle{}../../impl/bbb\textquotesingle{}}
\end{Highlighting}
\end{Shaded}

On Windows:

\begin{Shaded}
\begin{Highlighting}[]
\NormalTok{path}\OperatorTok{.}\FunctionTok{relative}\NormalTok{(}\StringTok{\textquotesingle{}C:}\SpecialCharTok{\textbackslash{}\textbackslash{}}\StringTok{orandea}\SpecialCharTok{\textbackslash{}\textbackslash{}}\StringTok{test}\SpecialCharTok{\textbackslash{}\textbackslash{}}\StringTok{aaa\textquotesingle{}}\OperatorTok{,} \StringTok{\textquotesingle{}C:}\SpecialCharTok{\textbackslash{}\textbackslash{}}\StringTok{orandea}\SpecialCharTok{\textbackslash{}\textbackslash{}}\StringTok{impl}\SpecialCharTok{\textbackslash{}\textbackslash{}}\StringTok{bbb\textquotesingle{}}\NormalTok{)}\OperatorTok{;}
\CommentTok{// Returns: \textquotesingle{}..\textbackslash{}\textbackslash{}..\textbackslash{}\textbackslash{}impl\textbackslash{}\textbackslash{}bbb\textquotesingle{}}
\end{Highlighting}
\end{Shaded}

A \href{errors.md\#class-typeerror}{\texttt{TypeError}} is thrown if
either \texttt{from} or \texttt{to} is not a string.

\subsection{\texorpdfstring{\texttt{path.resolve({[}...paths{]})}}{path.resolve({[}...paths{]})}}\label{path.resolve...paths}

\begin{itemize}
\tightlist
\item
  \texttt{...paths} \{string\} A sequence of paths or path segments
\item
  Returns: \{string\}
\end{itemize}

The \texttt{path.resolve()} method resolves a sequence of paths or path
segments into an absolute path.

The given sequence of paths is processed from right to left, with each
subsequent \texttt{path} prepended until an absolute path is
constructed. For instance, given the sequence of path segments:
\texttt{/foo}, \texttt{/bar}, \texttt{baz}, calling
\texttt{path.resolve(\textquotesingle{}/foo\textquotesingle{},\ \textquotesingle{}/bar\textquotesingle{},\ \textquotesingle{}baz\textquotesingle{})}
would return \texttt{/bar/baz} because
\texttt{\textquotesingle{}baz\textquotesingle{}} is not an absolute path
but
\texttt{\textquotesingle{}/bar\textquotesingle{}\ +\ \textquotesingle{}/\textquotesingle{}\ +\ \textquotesingle{}baz\textquotesingle{}}
is.

If, after processing all given \texttt{path} segments, an absolute path
has not yet been generated, the current working directory is used.

The resulting path is normalized and trailing slashes are removed unless
the path is resolved to the root directory.

Zero-length \texttt{path} segments are ignored.

If no \texttt{path} segments are passed, \texttt{path.resolve()} will
return the absolute path of the current working directory.

\begin{Shaded}
\begin{Highlighting}[]
\NormalTok{path}\OperatorTok{.}\FunctionTok{resolve}\NormalTok{(}\StringTok{\textquotesingle{}/foo/bar\textquotesingle{}}\OperatorTok{,} \StringTok{\textquotesingle{}./baz\textquotesingle{}}\NormalTok{)}\OperatorTok{;}
\CommentTok{// Returns: \textquotesingle{}/foo/bar/baz\textquotesingle{}}

\NormalTok{path}\OperatorTok{.}\FunctionTok{resolve}\NormalTok{(}\StringTok{\textquotesingle{}/foo/bar\textquotesingle{}}\OperatorTok{,} \StringTok{\textquotesingle{}/tmp/file/\textquotesingle{}}\NormalTok{)}\OperatorTok{;}
\CommentTok{// Returns: \textquotesingle{}/tmp/file\textquotesingle{}}

\NormalTok{path}\OperatorTok{.}\FunctionTok{resolve}\NormalTok{(}\StringTok{\textquotesingle{}wwwroot\textquotesingle{}}\OperatorTok{,} \StringTok{\textquotesingle{}static\_files/png/\textquotesingle{}}\OperatorTok{,} \StringTok{\textquotesingle{}../gif/image.gif\textquotesingle{}}\NormalTok{)}\OperatorTok{;}
\CommentTok{// If the current working directory is /home/myself/node,}
\CommentTok{// this returns \textquotesingle{}/home/myself/node/wwwroot/static\_files/gif/image.gif\textquotesingle{}}
\end{Highlighting}
\end{Shaded}

A \href{errors.md\#class-typeerror}{\texttt{TypeError}} is thrown if any
of the arguments is not a string.

\subsection{\texorpdfstring{\texttt{path.sep}}{path.sep}}\label{path.sep}

\begin{itemize}
\tightlist
\item
  \{string\}
\end{itemize}

Provides the platform-specific path segment separator:

\begin{itemize}
\tightlist
\item
  \texttt{\textbackslash{}} on Windows
\item
  \texttt{/} on POSIX
\end{itemize}

For example, on POSIX:

\begin{Shaded}
\begin{Highlighting}[]
\StringTok{\textquotesingle{}foo/bar/baz\textquotesingle{}}\OperatorTok{.}\FunctionTok{split}\NormalTok{(path}\OperatorTok{.}\AttributeTok{sep}\NormalTok{)}\OperatorTok{;}
\CommentTok{// Returns: [\textquotesingle{}foo\textquotesingle{}, \textquotesingle{}bar\textquotesingle{}, \textquotesingle{}baz\textquotesingle{}]}
\end{Highlighting}
\end{Shaded}

On Windows:

\begin{Shaded}
\begin{Highlighting}[]
\StringTok{\textquotesingle{}foo}\SpecialCharTok{\textbackslash{}\textbackslash{}}\StringTok{bar}\SpecialCharTok{\textbackslash{}\textbackslash{}}\StringTok{baz\textquotesingle{}}\OperatorTok{.}\FunctionTok{split}\NormalTok{(path}\OperatorTok{.}\AttributeTok{sep}\NormalTok{)}\OperatorTok{;}
\CommentTok{// Returns: [\textquotesingle{}foo\textquotesingle{}, \textquotesingle{}bar\textquotesingle{}, \textquotesingle{}baz\textquotesingle{}]}
\end{Highlighting}
\end{Shaded}

On Windows, both the forward slash (\texttt{/}) and backward slash
(\texttt{\textbackslash{}}) are accepted as path segment separators;
however, the \texttt{path} methods only add backward slashes
(\texttt{\textbackslash{}}).

\subsection{\texorpdfstring{\texttt{path.toNamespacedPath(path)}}{path.toNamespacedPath(path)}}\label{path.tonamespacedpathpath}

\begin{itemize}
\tightlist
\item
  \texttt{path} \{string\}
\item
  Returns: \{string\}
\end{itemize}

On Windows systems only, returns an equivalent
\href{https://docs.microsoft.com/en-us/windows/desktop/FileIO/naming-a-file\#namespaces}{namespace-prefixed
path} for the given \texttt{path}. If \texttt{path} is not a string,
\texttt{path} will be returned without modifications.

This method is meaningful only on Windows systems. On POSIX systems, the
method is non-operational and always returns \texttt{path} without
modifications.

\subsection{\texorpdfstring{\texttt{path.win32}}{path.win32}}\label{path.win32}

\begin{itemize}
\tightlist
\item
  \{Object\}
\end{itemize}

The \texttt{path.win32} property provides access to Windows-specific
implementations of the \texttt{path} methods.

The API is accessible via
\texttt{require(\textquotesingle{}node:path\textquotesingle{}).win32} or
\texttt{require(\textquotesingle{}node:path/win32\textquotesingle{})}.
