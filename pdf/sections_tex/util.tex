\section{Util}\label{util}

\begin{quote}
Stability: 2 - Stable
\end{quote}

The \texttt{node:util} module supports the needs of Node.js internal
APIs. Many of the utilities are useful for application and module
developers as well. To access it:

\begin{Shaded}
\begin{Highlighting}[]
\KeywordTok{const}\NormalTok{ util }\OperatorTok{=} \PreprocessorTok{require}\NormalTok{(}\StringTok{\textquotesingle{}node:util\textquotesingle{}}\NormalTok{)}\OperatorTok{;}
\end{Highlighting}
\end{Shaded}

\subsection{\texorpdfstring{\texttt{util.callbackify(original)}}{util.callbackify(original)}}\label{util.callbackifyoriginal}

\begin{itemize}
\tightlist
\item
  \texttt{original} \{Function\} An \texttt{async} function
\item
  Returns: \{Function\} a callback style function
\end{itemize}

Takes an \texttt{async} function (or a function that returns a
\texttt{Promise}) and returns a function following the error-first
callback style, i.e.~taking an
\texttt{(err,\ value)\ =\textgreater{}\ ...} callback as the last
argument. In the callback, the first argument will be the rejection
reason (or \texttt{null} if the \texttt{Promise} resolved), and the
second argument will be the resolved value.

\begin{Shaded}
\begin{Highlighting}[]
\KeywordTok{const}\NormalTok{ util }\OperatorTok{=} \PreprocessorTok{require}\NormalTok{(}\StringTok{\textquotesingle{}node:util\textquotesingle{}}\NormalTok{)}\OperatorTok{;}

\KeywordTok{async} \KeywordTok{function} \FunctionTok{fn}\NormalTok{() \{}
  \ControlFlowTok{return} \StringTok{\textquotesingle{}hello world\textquotesingle{}}\OperatorTok{;}
\NormalTok{\}}
\KeywordTok{const}\NormalTok{ callbackFunction }\OperatorTok{=}\NormalTok{ util}\OperatorTok{.}\FunctionTok{callbackify}\NormalTok{(fn)}\OperatorTok{;}

\FunctionTok{callbackFunction}\NormalTok{((err}\OperatorTok{,}\NormalTok{ ret) }\KeywordTok{=\textgreater{}}\NormalTok{ \{}
  \ControlFlowTok{if}\NormalTok{ (err) }\ControlFlowTok{throw}\NormalTok{ err}\OperatorTok{;}
  \BuiltInTok{console}\OperatorTok{.}\FunctionTok{log}\NormalTok{(ret)}\OperatorTok{;}
\NormalTok{\})}\OperatorTok{;}
\end{Highlighting}
\end{Shaded}

Will print:

\begin{Shaded}
\begin{Highlighting}[]
\NormalTok{hello world}
\end{Highlighting}
\end{Shaded}

The callback is executed asynchronously, and will have a limited stack
trace. If the callback throws, the process will emit an
\href{process.md\#event-uncaughtexception}{\texttt{\textquotesingle{}uncaughtException\textquotesingle{}}}
event, and if not handled will exit.

Since \texttt{null} has a special meaning as the first argument to a
callback, if a wrapped function rejects a \texttt{Promise} with a falsy
value as a reason, the value is wrapped in an \texttt{Error} with the
original value stored in a field named \texttt{reason}.

\begin{Shaded}
\begin{Highlighting}[]
\KeywordTok{function} \FunctionTok{fn}\NormalTok{() \{}
  \ControlFlowTok{return} \BuiltInTok{Promise}\OperatorTok{.}\FunctionTok{reject}\NormalTok{(}\KeywordTok{null}\NormalTok{)}\OperatorTok{;}
\NormalTok{\}}
\KeywordTok{const}\NormalTok{ callbackFunction }\OperatorTok{=}\NormalTok{ util}\OperatorTok{.}\FunctionTok{callbackify}\NormalTok{(fn)}\OperatorTok{;}

\FunctionTok{callbackFunction}\NormalTok{((err}\OperatorTok{,}\NormalTok{ ret) }\KeywordTok{=\textgreater{}}\NormalTok{ \{}
  \CommentTok{// When the Promise was rejected with \textasciigrave{}null\textasciigrave{} it is wrapped with an Error and}
  \CommentTok{// the original value is stored in \textasciigrave{}reason\textasciigrave{}.}
\NormalTok{  err }\OperatorTok{\&\&} \BuiltInTok{Object}\OperatorTok{.}\FunctionTok{hasOwn}\NormalTok{(err}\OperatorTok{,} \StringTok{\textquotesingle{}reason\textquotesingle{}}\NormalTok{) }\OperatorTok{\&\&}\NormalTok{ err}\OperatorTok{.}\AttributeTok{reason} \OperatorTok{===} \KeywordTok{null}\OperatorTok{;}  \CommentTok{// true}
\NormalTok{\})}\OperatorTok{;}
\end{Highlighting}
\end{Shaded}

\subsection{\texorpdfstring{\texttt{util.debuglog(section{[},\ callback{]})}}{util.debuglog(section{[}, callback{]})}}\label{util.debuglogsection-callback}

\begin{itemize}
\tightlist
\item
  \texttt{section} \{string\} A string identifying the portion of the
  application for which the \texttt{debuglog} function is being created.
\item
  \texttt{callback} \{Function\} A callback invoked the first time the
  logging function is called with a function argument that is a more
  optimized logging function.
\item
  Returns: \{Function\} The logging function
\end{itemize}

The \texttt{util.debuglog()} method is used to create a function that
conditionally writes debug messages to \texttt{stderr} based on the
existence of the \texttt{NODE\_DEBUG} environment variable. If the
\texttt{section} name appears within the value of that environment
variable, then the returned function operates similar to
\href{console.md\#consoleerrordata-args}{\texttt{console.error()}}. If
not, then the returned function is a no-op.

\begin{Shaded}
\begin{Highlighting}[]
\KeywordTok{const}\NormalTok{ util }\OperatorTok{=} \PreprocessorTok{require}\NormalTok{(}\StringTok{\textquotesingle{}node:util\textquotesingle{}}\NormalTok{)}\OperatorTok{;}
\KeywordTok{const}\NormalTok{ debuglog }\OperatorTok{=}\NormalTok{ util}\OperatorTok{.}\FunctionTok{debuglog}\NormalTok{(}\StringTok{\textquotesingle{}foo\textquotesingle{}}\NormalTok{)}\OperatorTok{;}

\FunctionTok{debuglog}\NormalTok{(}\StringTok{\textquotesingle{}hello from foo [\%d]\textquotesingle{}}\OperatorTok{,} \DecValTok{123}\NormalTok{)}\OperatorTok{;}
\end{Highlighting}
\end{Shaded}

If this program is run with \texttt{NODE\_DEBUG=foo} in the environment,
then it will output something like:

\begin{Shaded}
\begin{Highlighting}[]
\NormalTok{FOO 3245: hello from foo [123]}
\end{Highlighting}
\end{Shaded}

where \texttt{3245} is the process id. If it is not run with that
environment variable set, then it will not print anything.

The \texttt{section} supports wildcard also:

\begin{Shaded}
\begin{Highlighting}[]
\KeywordTok{const}\NormalTok{ util }\OperatorTok{=} \PreprocessorTok{require}\NormalTok{(}\StringTok{\textquotesingle{}node:util\textquotesingle{}}\NormalTok{)}\OperatorTok{;}
\KeywordTok{const}\NormalTok{ debuglog }\OperatorTok{=}\NormalTok{ util}\OperatorTok{.}\FunctionTok{debuglog}\NormalTok{(}\StringTok{\textquotesingle{}foo{-}bar\textquotesingle{}}\NormalTok{)}\OperatorTok{;}

\FunctionTok{debuglog}\NormalTok{(}\StringTok{\textquotesingle{}hi there, it}\SpecialCharTok{\textbackslash{}\textquotesingle{}}\StringTok{s foo{-}bar [\%d]\textquotesingle{}}\OperatorTok{,} \DecValTok{2333}\NormalTok{)}\OperatorTok{;}
\end{Highlighting}
\end{Shaded}

if it is run with \texttt{NODE\_DEBUG=foo*} in the environment, then it
will output something like:

\begin{Shaded}
\begin{Highlighting}[]
\NormalTok{FOO{-}BAR 3257: hi there, it\textquotesingle{}s foo{-}bar [2333]}
\end{Highlighting}
\end{Shaded}

Multiple comma-separated \texttt{section} names may be specified in the
\texttt{NODE\_DEBUG} environment variable:
\texttt{NODE\_DEBUG=fs,net,tls}.

The optional \texttt{callback} argument can be used to replace the
logging function with a different function that doesn't have any
initialization or unnecessary wrapping.

\begin{Shaded}
\begin{Highlighting}[]
\KeywordTok{const}\NormalTok{ util }\OperatorTok{=} \PreprocessorTok{require}\NormalTok{(}\StringTok{\textquotesingle{}node:util\textquotesingle{}}\NormalTok{)}\OperatorTok{;}
\KeywordTok{let}\NormalTok{ debuglog }\OperatorTok{=}\NormalTok{ util}\OperatorTok{.}\FunctionTok{debuglog}\NormalTok{(}\StringTok{\textquotesingle{}internals\textquotesingle{}}\OperatorTok{,}\NormalTok{ (debug) }\KeywordTok{=\textgreater{}}\NormalTok{ \{}
  \CommentTok{// Replace with a logging function that optimizes out}
  \CommentTok{// testing if the section is enabled}
\NormalTok{  debuglog }\OperatorTok{=}\NormalTok{ debug}\OperatorTok{;}
\NormalTok{\})}\OperatorTok{;}
\end{Highlighting}
\end{Shaded}

\subsubsection{\texorpdfstring{\texttt{debuglog().enabled}}{debuglog().enabled}}\label{debuglog.enabled}

\begin{itemize}
\tightlist
\item
  \{boolean\}
\end{itemize}

The \texttt{util.debuglog().enabled} getter is used to create a test
that can be used in conditionals based on the existence of the
\texttt{NODE\_DEBUG} environment variable. If the \texttt{section} name
appears within the value of that environment variable, then the returned
value will be \texttt{true}. If not, then the returned value will be
\texttt{false}.

\begin{Shaded}
\begin{Highlighting}[]
\KeywordTok{const}\NormalTok{ util }\OperatorTok{=} \PreprocessorTok{require}\NormalTok{(}\StringTok{\textquotesingle{}node:util\textquotesingle{}}\NormalTok{)}\OperatorTok{;}
\KeywordTok{const}\NormalTok{ enabled }\OperatorTok{=}\NormalTok{ util}\OperatorTok{.}\FunctionTok{debuglog}\NormalTok{(}\StringTok{\textquotesingle{}foo\textquotesingle{}}\NormalTok{)}\OperatorTok{.}\AttributeTok{enabled}\OperatorTok{;}
\ControlFlowTok{if}\NormalTok{ (enabled) \{}
  \BuiltInTok{console}\OperatorTok{.}\FunctionTok{log}\NormalTok{(}\StringTok{\textquotesingle{}hello from foo [\%d]\textquotesingle{}}\OperatorTok{,} \DecValTok{123}\NormalTok{)}\OperatorTok{;}
\NormalTok{\}}
\end{Highlighting}
\end{Shaded}

If this program is run with \texttt{NODE\_DEBUG=foo} in the environment,
then it will output something like:

\begin{Shaded}
\begin{Highlighting}[]
\NormalTok{hello from foo [123]}
\end{Highlighting}
\end{Shaded}

\subsection{\texorpdfstring{\texttt{util.debug(section)}}{util.debug(section)}}\label{util.debugsection}

Alias for \texttt{util.debuglog}. Usage allows for readability of that
doesn't imply logging when only using \texttt{util.debuglog().enabled}.

\subsection{\texorpdfstring{\texttt{util.deprecate(fn,\ msg{[},\ code{]})}}{util.deprecate(fn, msg{[}, code{]})}}\label{util.deprecatefn-msg-code}

\begin{itemize}
\tightlist
\item
  \texttt{fn} \{Function\} The function that is being deprecated.
\item
  \texttt{msg} \{string\} A warning message to display when the
  deprecated function is invoked.
\item
  \texttt{code} \{string\} A deprecation code. See the
  \href{deprecations.md\#list-of-deprecated-apis}{list of deprecated
  APIs} for a list of codes.
\item
  Returns: \{Function\} The deprecated function wrapped to emit a
  warning.
\end{itemize}

The \texttt{util.deprecate()} method wraps \texttt{fn} (which may be a
function or class) in such a way that it is marked as deprecated.

\begin{Shaded}
\begin{Highlighting}[]
\KeywordTok{const}\NormalTok{ util }\OperatorTok{=} \PreprocessorTok{require}\NormalTok{(}\StringTok{\textquotesingle{}node:util\textquotesingle{}}\NormalTok{)}\OperatorTok{;}

\NormalTok{exports}\OperatorTok{.}\AttributeTok{obsoleteFunction} \OperatorTok{=}\NormalTok{ util}\OperatorTok{.}\FunctionTok{deprecate}\NormalTok{(() }\KeywordTok{=\textgreater{}}\NormalTok{ \{}
  \CommentTok{// Do something here.}
\NormalTok{\}}\OperatorTok{,} \StringTok{\textquotesingle{}obsoleteFunction() is deprecated. Use newShinyFunction() instead.\textquotesingle{}}\NormalTok{)}\OperatorTok{;}
\end{Highlighting}
\end{Shaded}

When called, \texttt{util.deprecate()} will return a function that will
emit a \texttt{DeprecationWarning} using the
\href{process.md\#event-warning}{\texttt{\textquotesingle{}warning\textquotesingle{}}}
event. The warning will be emitted and printed to \texttt{stderr} the
first time the returned function is called. After the warning is
emitted, the wrapped function is called without emitting a warning.

If the same optional \texttt{code} is supplied in multiple calls to
\texttt{util.deprecate()}, the warning will be emitted only once for
that \texttt{code}.

\begin{Shaded}
\begin{Highlighting}[]
\KeywordTok{const}\NormalTok{ util }\OperatorTok{=} \PreprocessorTok{require}\NormalTok{(}\StringTok{\textquotesingle{}node:util\textquotesingle{}}\NormalTok{)}\OperatorTok{;}

\KeywordTok{const}\NormalTok{ fn1 }\OperatorTok{=}\NormalTok{ util}\OperatorTok{.}\FunctionTok{deprecate}\NormalTok{(someFunction}\OperatorTok{,}\NormalTok{ someMessage}\OperatorTok{,} \StringTok{\textquotesingle{}DEP0001\textquotesingle{}}\NormalTok{)}\OperatorTok{;}
\KeywordTok{const}\NormalTok{ fn2 }\OperatorTok{=}\NormalTok{ util}\OperatorTok{.}\FunctionTok{deprecate}\NormalTok{(someOtherFunction}\OperatorTok{,}\NormalTok{ someOtherMessage}\OperatorTok{,} \StringTok{\textquotesingle{}DEP0001\textquotesingle{}}\NormalTok{)}\OperatorTok{;}
\FunctionTok{fn1}\NormalTok{()}\OperatorTok{;} \CommentTok{// Emits a deprecation warning with code DEP0001}
\FunctionTok{fn2}\NormalTok{()}\OperatorTok{;} \CommentTok{// Does not emit a deprecation warning because it has the same code}
\end{Highlighting}
\end{Shaded}

If either the \texttt{-\/-no-deprecation} or \texttt{-\/-no-warnings}
command-line flags are used, or if the \texttt{process.noDeprecation}
property is set to \texttt{true} \emph{prior} to the first deprecation
warning, the \texttt{util.deprecate()} method does nothing.

If the \texttt{-\/-trace-deprecation} or \texttt{-\/-trace-warnings}
command-line flags are set, or the \texttt{process.traceDeprecation}
property is set to \texttt{true}, a warning and a stack trace are
printed to \texttt{stderr} the first time the deprecated function is
called.

If the \texttt{-\/-throw-deprecation} command-line flag is set, or the
\texttt{process.throwDeprecation} property is set to \texttt{true}, then
an exception will be thrown when the deprecated function is called.

The \texttt{-\/-throw-deprecation} command-line flag and
\texttt{process.throwDeprecation} property take precedence over
\texttt{-\/-trace-deprecation} and \texttt{process.traceDeprecation}.

\subsection{\texorpdfstring{\texttt{util.format(format{[},\ ...args{]})}}{util.format(format{[}, ...args{]})}}\label{util.formatformat-...args}

\begin{itemize}
\tightlist
\item
  \texttt{format} \{string\} A \texttt{printf}-like format string.
\end{itemize}

The \texttt{util.format()} method returns a formatted string using the
first argument as a \texttt{printf}-like format string which can contain
zero or more format specifiers. Each specifier is replaced with the
converted value from the corresponding argument. Supported specifiers
are:

\begin{itemize}
\tightlist
\item
  \texttt{\%s}: \texttt{String} will be used to convert all values
  except \texttt{BigInt}, \texttt{Object} and \texttt{-0}.
  \texttt{BigInt} values will be represented with an \texttt{n} and
  Objects that have no user defined \texttt{toString} function are
  inspected using \texttt{util.inspect()} with options
  \texttt{\{\ depth:\ 0,\ colors:\ false,\ compact:\ 3\ \}}.
\item
  \texttt{\%d}: \texttt{Number} will be used to convert all values
  except \texttt{BigInt} and \texttt{Symbol}.
\item
  \texttt{\%i}: \texttt{parseInt(value,\ 10)} is used for all values
  except \texttt{BigInt} and \texttt{Symbol}.
\item
  \texttt{\%f}: \texttt{parseFloat(value)} is used for all values expect
  \texttt{Symbol}.
\item
  \texttt{\%j}: JSON. Replaced with the string
  \texttt{\textquotesingle{}{[}Circular{]}\textquotesingle{}} if the
  argument contains circular references.
\item
  \texttt{\%o}: \texttt{Object}. A string representation of an object
  with generic JavaScript object formatting. Similar to
  \texttt{util.inspect()} with options
  \texttt{\{\ showHidden:\ true,\ showProxy:\ true\ \}}. This will show
  the full object including non-enumerable properties and proxies.
\item
  \texttt{\%O}: \texttt{Object}. A string representation of an object
  with generic JavaScript object formatting. Similar to
  \texttt{util.inspect()} without options. This will show the full
  object not including non-enumerable properties and proxies.
\item
  \texttt{\%c}: \texttt{CSS}. This specifier is ignored and will skip
  any CSS passed in.
\item
  \texttt{\%\%}: single percent sign
  (\texttt{\textquotesingle{}\%\textquotesingle{}}). This does not
  consume an argument.
\item
  Returns: \{string\} The formatted string
\end{itemize}

If a specifier does not have a corresponding argument, it is not
replaced:

\begin{Shaded}
\begin{Highlighting}[]
\NormalTok{util}\OperatorTok{.}\FunctionTok{format}\NormalTok{(}\StringTok{\textquotesingle{}\%s:\%s\textquotesingle{}}\OperatorTok{,} \StringTok{\textquotesingle{}foo\textquotesingle{}}\NormalTok{)}\OperatorTok{;}
\CommentTok{// Returns: \textquotesingle{}foo:\%s\textquotesingle{}}
\end{Highlighting}
\end{Shaded}

Values that are not part of the format string are formatted using
\texttt{util.inspect()} if their type is not \texttt{string}.

If there are more arguments passed to the \texttt{util.format()} method
than the number of specifiers, the extra arguments are concatenated to
the returned string, separated by spaces:

\begin{Shaded}
\begin{Highlighting}[]
\NormalTok{util}\OperatorTok{.}\FunctionTok{format}\NormalTok{(}\StringTok{\textquotesingle{}\%s:\%s\textquotesingle{}}\OperatorTok{,} \StringTok{\textquotesingle{}foo\textquotesingle{}}\OperatorTok{,} \StringTok{\textquotesingle{}bar\textquotesingle{}}\OperatorTok{,} \StringTok{\textquotesingle{}baz\textquotesingle{}}\NormalTok{)}\OperatorTok{;}
\CommentTok{// Returns: \textquotesingle{}foo:bar baz\textquotesingle{}}
\end{Highlighting}
\end{Shaded}

If the first argument does not contain a valid format specifier,
\texttt{util.format()} returns a string that is the concatenation of all
arguments separated by spaces:

\begin{Shaded}
\begin{Highlighting}[]
\NormalTok{util}\OperatorTok{.}\FunctionTok{format}\NormalTok{(}\DecValTok{1}\OperatorTok{,} \DecValTok{2}\OperatorTok{,} \DecValTok{3}\NormalTok{)}\OperatorTok{;}
\CommentTok{// Returns: \textquotesingle{}1 2 3\textquotesingle{}}
\end{Highlighting}
\end{Shaded}

If only one argument is passed to \texttt{util.format()}, it is returned
as it is without any formatting:

\begin{Shaded}
\begin{Highlighting}[]
\NormalTok{util}\OperatorTok{.}\FunctionTok{format}\NormalTok{(}\StringTok{\textquotesingle{}\%\% \%s\textquotesingle{}}\NormalTok{)}\OperatorTok{;}
\CommentTok{// Returns: \textquotesingle{}\%\% \%s\textquotesingle{}}
\end{Highlighting}
\end{Shaded}

\texttt{util.format()} is a synchronous method that is intended as a
debugging tool. Some input values can have a significant performance
overhead that can block the event loop. Use this function with care and
never in a hot code path.

\subsection{\texorpdfstring{\texttt{util.formatWithOptions(inspectOptions,\ format{[},\ ...args{]})}}{util.formatWithOptions(inspectOptions, format{[}, ...args{]})}}\label{util.formatwithoptionsinspectoptions-format-...args}

\begin{itemize}
\tightlist
\item
  \texttt{inspectOptions} \{Object\}
\item
  \texttt{format} \{string\}
\end{itemize}

This function is identical to
\hyperref[utilformatformat-args]{\texttt{util.format()}}, except in that
it takes an \texttt{inspectOptions} argument which specifies options
that are passed along to
\hyperref[utilinspectobject-options]{\texttt{util.inspect()}}.

\begin{Shaded}
\begin{Highlighting}[]
\NormalTok{util}\OperatorTok{.}\FunctionTok{formatWithOptions}\NormalTok{(\{ }\DataTypeTok{colors}\OperatorTok{:} \KeywordTok{true}\NormalTok{ \}}\OperatorTok{,} \StringTok{\textquotesingle{}See object \%O\textquotesingle{}}\OperatorTok{,}\NormalTok{ \{ }\DataTypeTok{foo}\OperatorTok{:} \DecValTok{42}\NormalTok{ \})}\OperatorTok{;}
\CommentTok{// Returns \textquotesingle{}See object \{ foo: 42 \}\textquotesingle{}, where \textasciigrave{}42\textasciigrave{} is colored as a number}
\CommentTok{// when printed to a terminal.}
\end{Highlighting}
\end{Shaded}

\subsection{\texorpdfstring{\texttt{util.getSystemErrorName(err)}}{util.getSystemErrorName(err)}}\label{util.getsystemerrornameerr}

\begin{itemize}
\tightlist
\item
  \texttt{err} \{number\}
\item
  Returns: \{string\}
\end{itemize}

Returns the string name for a numeric error code that comes from a
Node.js API. The mapping between error codes and error names is
platform-dependent. See \href{errors.md\#common-system-errors}{Common
System Errors} for the names of common errors.

\begin{Shaded}
\begin{Highlighting}[]
\NormalTok{fs}\OperatorTok{.}\FunctionTok{access}\NormalTok{(}\StringTok{\textquotesingle{}file/that/does/not/exist\textquotesingle{}}\OperatorTok{,}\NormalTok{ (err) }\KeywordTok{=\textgreater{}}\NormalTok{ \{}
  \KeywordTok{const}\NormalTok{ name }\OperatorTok{=}\NormalTok{ util}\OperatorTok{.}\FunctionTok{getSystemErrorName}\NormalTok{(err}\OperatorTok{.}\AttributeTok{errno}\NormalTok{)}\OperatorTok{;}
  \BuiltInTok{console}\OperatorTok{.}\FunctionTok{error}\NormalTok{(name)}\OperatorTok{;}  \CommentTok{// ENOENT}
\NormalTok{\})}\OperatorTok{;}
\end{Highlighting}
\end{Shaded}

\subsection{\texorpdfstring{\texttt{util.getSystemErrorMap()}}{util.getSystemErrorMap()}}\label{util.getsystemerrormap}

\begin{itemize}
\tightlist
\item
  Returns: \{Map\}
\end{itemize}

Returns a Map of all system error codes available from the Node.js API.
The mapping between error codes and error names is platform-dependent.
See \href{errors.md\#common-system-errors}{Common System Errors} for the
names of common errors.

\begin{Shaded}
\begin{Highlighting}[]
\NormalTok{fs}\OperatorTok{.}\FunctionTok{access}\NormalTok{(}\StringTok{\textquotesingle{}file/that/does/not/exist\textquotesingle{}}\OperatorTok{,}\NormalTok{ (err) }\KeywordTok{=\textgreater{}}\NormalTok{ \{}
  \KeywordTok{const}\NormalTok{ errorMap }\OperatorTok{=}\NormalTok{ util}\OperatorTok{.}\FunctionTok{getSystemErrorMap}\NormalTok{()}\OperatorTok{;}
  \KeywordTok{const}\NormalTok{ name }\OperatorTok{=}\NormalTok{ errorMap}\OperatorTok{.}\FunctionTok{get}\NormalTok{(err}\OperatorTok{.}\AttributeTok{errno}\NormalTok{)}\OperatorTok{;}
  \BuiltInTok{console}\OperatorTok{.}\FunctionTok{error}\NormalTok{(name)}\OperatorTok{;}  \CommentTok{// ENOENT}
\NormalTok{\})}\OperatorTok{;}
\end{Highlighting}
\end{Shaded}

\subsection{\texorpdfstring{\texttt{util.inherits(constructor,\ superConstructor)}}{util.inherits(constructor, superConstructor)}}\label{util.inheritsconstructor-superconstructor}

\begin{quote}
Stability: 3 - Legacy: Use ES2015 class syntax and \texttt{extends}
keyword instead.
\end{quote}

\begin{itemize}
\tightlist
\item
  \texttt{constructor} \{Function\}
\item
  \texttt{superConstructor} \{Function\}
\end{itemize}

Usage of \texttt{util.inherits()} is discouraged. Please use the ES6
\texttt{class} and \texttt{extends} keywords to get language level
inheritance support. Also note that the two styles are
\href{https://github.com/nodejs/node/issues/4179}{semantically
incompatible}.

Inherit the prototype methods from one
\href{https://developer.mozilla.org/en-US/docs/Web/JavaScript/Reference/Global_Objects/Object/constructor}{constructor}
into another. The prototype of \texttt{constructor} will be set to a new
object created from \texttt{superConstructor}.

This mainly adds some input validation on top of
\texttt{Object.setPrototypeOf(constructor.prototype,\ superConstructor.prototype)}.
As an additional convenience, \texttt{superConstructor} will be
accessible through the \texttt{constructor.super\_} property.

\begin{Shaded}
\begin{Highlighting}[]
\KeywordTok{const}\NormalTok{ util }\OperatorTok{=} \PreprocessorTok{require}\NormalTok{(}\StringTok{\textquotesingle{}node:util\textquotesingle{}}\NormalTok{)}\OperatorTok{;}
\KeywordTok{const} \BuiltInTok{EventEmitter} \OperatorTok{=} \PreprocessorTok{require}\NormalTok{(}\StringTok{\textquotesingle{}node:events\textquotesingle{}}\NormalTok{)}\OperatorTok{;}

\KeywordTok{function} \FunctionTok{MyStream}\NormalTok{() \{}
  \BuiltInTok{EventEmitter}\OperatorTok{.}\FunctionTok{call}\NormalTok{(}\KeywordTok{this}\NormalTok{)}\OperatorTok{;}
\NormalTok{\}}

\NormalTok{util}\OperatorTok{.}\FunctionTok{inherits}\NormalTok{(MyStream}\OperatorTok{,} \BuiltInTok{EventEmitter}\NormalTok{)}\OperatorTok{;}

\NormalTok{MyStream}\OperatorTok{.}\AttributeTok{prototype}\OperatorTok{.}\AttributeTok{write} \OperatorTok{=} \KeywordTok{function}\NormalTok{(data) \{}
  \KeywordTok{this}\OperatorTok{.}\FunctionTok{emit}\NormalTok{(}\StringTok{\textquotesingle{}data\textquotesingle{}}\OperatorTok{,}\NormalTok{ data)}\OperatorTok{;}
\NormalTok{\}}\OperatorTok{;}

\KeywordTok{const}\NormalTok{ stream }\OperatorTok{=} \KeywordTok{new} \FunctionTok{MyStream}\NormalTok{()}\OperatorTok{;}

\BuiltInTok{console}\OperatorTok{.}\FunctionTok{log}\NormalTok{(stream }\KeywordTok{instanceof} \BuiltInTok{EventEmitter}\NormalTok{)}\OperatorTok{;} \CommentTok{// true}
\BuiltInTok{console}\OperatorTok{.}\FunctionTok{log}\NormalTok{(MyStream}\OperatorTok{.}\AttributeTok{super\_} \OperatorTok{===} \BuiltInTok{EventEmitter}\NormalTok{)}\OperatorTok{;} \CommentTok{// true}

\NormalTok{stream}\OperatorTok{.}\FunctionTok{on}\NormalTok{(}\StringTok{\textquotesingle{}data\textquotesingle{}}\OperatorTok{,}\NormalTok{ (data) }\KeywordTok{=\textgreater{}}\NormalTok{ \{}
  \BuiltInTok{console}\OperatorTok{.}\FunctionTok{log}\NormalTok{(}\VerbatimStringTok{\textasciigrave{}Received data: "}\SpecialCharTok{$\{}\NormalTok{data}\SpecialCharTok{\}}\VerbatimStringTok{"\textasciigrave{}}\NormalTok{)}\OperatorTok{;}
\NormalTok{\})}\OperatorTok{;}
\NormalTok{stream}\OperatorTok{.}\FunctionTok{write}\NormalTok{(}\StringTok{\textquotesingle{}It works!\textquotesingle{}}\NormalTok{)}\OperatorTok{;} \CommentTok{// Received data: "It works!"}
\end{Highlighting}
\end{Shaded}

ES6 example using \texttt{class} and \texttt{extends}:

\begin{Shaded}
\begin{Highlighting}[]
\KeywordTok{const} \BuiltInTok{EventEmitter} \OperatorTok{=} \PreprocessorTok{require}\NormalTok{(}\StringTok{\textquotesingle{}node:events\textquotesingle{}}\NormalTok{)}\OperatorTok{;}

\KeywordTok{class}\NormalTok{ MyStream }\KeywordTok{extends} \BuiltInTok{EventEmitter}\NormalTok{ \{}
  \FunctionTok{write}\NormalTok{(data) \{}
    \KeywordTok{this}\OperatorTok{.}\FunctionTok{emit}\NormalTok{(}\StringTok{\textquotesingle{}data\textquotesingle{}}\OperatorTok{,}\NormalTok{ data)}\OperatorTok{;}
\NormalTok{  \}}
\NormalTok{\}}

\KeywordTok{const}\NormalTok{ stream }\OperatorTok{=} \KeywordTok{new} \FunctionTok{MyStream}\NormalTok{()}\OperatorTok{;}

\NormalTok{stream}\OperatorTok{.}\FunctionTok{on}\NormalTok{(}\StringTok{\textquotesingle{}data\textquotesingle{}}\OperatorTok{,}\NormalTok{ (data) }\KeywordTok{=\textgreater{}}\NormalTok{ \{}
  \BuiltInTok{console}\OperatorTok{.}\FunctionTok{log}\NormalTok{(}\VerbatimStringTok{\textasciigrave{}Received data: "}\SpecialCharTok{$\{}\NormalTok{data}\SpecialCharTok{\}}\VerbatimStringTok{"\textasciigrave{}}\NormalTok{)}\OperatorTok{;}
\NormalTok{\})}\OperatorTok{;}
\NormalTok{stream}\OperatorTok{.}\FunctionTok{write}\NormalTok{(}\StringTok{\textquotesingle{}With ES6\textquotesingle{}}\NormalTok{)}\OperatorTok{;}
\end{Highlighting}
\end{Shaded}

\subsection{\texorpdfstring{\texttt{util.inspect(object{[},\ options{]})}}{util.inspect(object{[}, options{]})}}\label{util.inspectobject-options}

\subsection{\texorpdfstring{\texttt{util.inspect(object{[},\ showHidden{[},\ depth{[},\ colors{]}{]}{]})}}{util.inspect(object{[}, showHidden{[}, depth{[}, colors{]}{]}{]})}}\label{util.inspectobject-showhidden-depth-colors}

\begin{itemize}
\tightlist
\item
  \texttt{object} \{any\} Any JavaScript primitive or \texttt{Object}.
\item
  \texttt{options} \{Object\}

  \begin{itemize}
  \tightlist
  \item
    \texttt{showHidden} \{boolean\} If \texttt{true}, \texttt{object}'s
    non-enumerable symbols and properties are included in the formatted
    result.
    \href{https://developer.mozilla.org/en-US/docs/Web/JavaScript/Reference/Global_Objects/WeakMap}{\texttt{WeakMap}}
    and
    \href{https://developer.mozilla.org/en-US/docs/Web/JavaScript/Reference/Global_Objects/WeakSet}{\texttt{WeakSet}}
    entries are also included as well as user defined prototype
    properties (excluding method properties). \textbf{Default:}
    \texttt{false}.
  \item
    \texttt{depth} \{number\} Specifies the number of times to recurse
    while formatting \texttt{object}. This is useful for inspecting
    large objects. To recurse up to the maximum call stack size pass
    \texttt{Infinity} or \texttt{null}. \textbf{Default:} \texttt{2}.
  \item
    \texttt{colors} \{boolean\} If \texttt{true}, the output is styled
    with ANSI color codes. Colors are customizable. See
    \hyperref[customizing-utilinspect-colors]{Customizing
    \texttt{util.inspect} colors}. \textbf{Default:} \texttt{false}.
  \item
    \texttt{customInspect} \{boolean\} If \texttt{false},
    \texttt{{[}util.inspect.custom{]}(depth,\ opts,\ inspect)} functions
    are not invoked. \textbf{Default:} \texttt{true}.
  \item
    \texttt{showProxy} \{boolean\} If \texttt{true}, \texttt{Proxy}
    inspection includes the
    \href{https://developer.mozilla.org/en-US/docs/Web/JavaScript/Reference/Global_Objects/Proxy\#Terminology}{\texttt{target}
    and \texttt{handler}} objects. \textbf{Default:} \texttt{false}.
  \item
    \texttt{maxArrayLength} \{integer\} Specifies the maximum number of
    \texttt{Array},
    \href{https://developer.mozilla.org/en-US/docs/Web/JavaScript/Reference/Global_Objects/TypedArray}{\texttt{TypedArray}},
    \href{https://developer.mozilla.org/en-US/docs/Web/JavaScript/Reference/Global_Objects/Map}{\texttt{Map}},
    \href{https://developer.mozilla.org/en-US/docs/Web/JavaScript/Reference/Global_Objects/Set}{\texttt{Set}},
    \href{https://developer.mozilla.org/en-US/docs/Web/JavaScript/Reference/Global_Objects/WeakMap}{\texttt{WeakMap}},
    and
    \href{https://developer.mozilla.org/en-US/docs/Web/JavaScript/Reference/Global_Objects/WeakSet}{\texttt{WeakSet}}
    elements to include when formatting. Set to \texttt{null} or
    \texttt{Infinity} to show all elements. Set to \texttt{0} or
    negative to show no elements. \textbf{Default:} \texttt{100}.
  \item
    \texttt{maxStringLength} \{integer\} Specifies the maximum number of
    characters to include when formatting. Set to \texttt{null} or
    \texttt{Infinity} to show all elements. Set to \texttt{0} or
    negative to show no characters. \textbf{Default:} \texttt{10000}.
  \item
    \texttt{breakLength} \{integer\} The length at which input values
    are split across multiple lines. Set to \texttt{Infinity} to format
    the input as a single line (in combination with \texttt{compact} set
    to \texttt{true} or any number \textgreater= \texttt{1}).
    \textbf{Default:} \texttt{80}.
  \item
    \texttt{compact} \{boolean\textbar integer\} Setting this to
    \texttt{false} causes each object key to be displayed on a new line.
    It will break on new lines in text that is longer than
    \texttt{breakLength}. If set to a number, the most \texttt{n} inner
    elements are united on a single line as long as all properties fit
    into \texttt{breakLength}. Short array elements are also grouped
    together. For more information, see the example below.
    \textbf{Default:} \texttt{3}.
  \item
    \texttt{sorted} \{boolean\textbar Function\} If set to \texttt{true}
    or a function, all properties of an object, and \texttt{Set} and
    \texttt{Map} entries are sorted in the resulting string. If set to
    \texttt{true} the
    \href{https://developer.mozilla.org/en-US/docs/Web/JavaScript/Reference/Global_Objects/Array/sort}{default
    sort} is used. If set to a function, it is used as a
    \href{https://developer.mozilla.org/en-US/docs/Web/JavaScript/Reference/Global_Objects/Array/sort\#Parameters}{compare
    function}.
  \item
    \texttt{getters} \{boolean\textbar string\} If set to \texttt{true},
    getters are inspected. If set to
    \texttt{\textquotesingle{}get\textquotesingle{}}, only getters
    without a corresponding setter are inspected. If set to
    \texttt{\textquotesingle{}set\textquotesingle{}}, only getters with
    a corresponding setter are inspected. This might cause side effects
    depending on the getter function. \textbf{Default:} \texttt{false}.
  \item
    \texttt{numericSeparator} \{boolean\} If set to \texttt{true}, an
    underscore is used to separate every three digits in all bigints and
    numbers. \textbf{Default:} \texttt{false}.
  \end{itemize}
\item
  Returns: \{string\} The representation of \texttt{object}.
\end{itemize}

The \texttt{util.inspect()} method returns a string representation of
\texttt{object} that is intended for debugging. The output of
\texttt{util.inspect} may change at any time and should not be depended
upon programmatically. Additional \texttt{options} may be passed that
alter the result. \texttt{util.inspect()} will use the constructor's
name and/or \texttt{@@toStringTag} to make an identifiable tag for an
inspected value.

\begin{Shaded}
\begin{Highlighting}[]
\KeywordTok{class}\NormalTok{ Foo \{}
  \KeywordTok{get}\NormalTok{ [}\BuiltInTok{Symbol}\OperatorTok{.}\AttributeTok{toStringTag}\NormalTok{]() \{}
    \ControlFlowTok{return} \StringTok{\textquotesingle{}bar\textquotesingle{}}\OperatorTok{;}
\NormalTok{  \}}
\NormalTok{\}}

\KeywordTok{class}\NormalTok{ Bar \{\}}

\KeywordTok{const}\NormalTok{ baz }\OperatorTok{=} \BuiltInTok{Object}\OperatorTok{.}\FunctionTok{create}\NormalTok{(}\KeywordTok{null}\OperatorTok{,}\NormalTok{ \{ [}\BuiltInTok{Symbol}\OperatorTok{.}\AttributeTok{toStringTag}\NormalTok{]}\OperatorTok{:}\NormalTok{ \{ }\DataTypeTok{value}\OperatorTok{:} \StringTok{\textquotesingle{}foo\textquotesingle{}}\NormalTok{ \} \})}\OperatorTok{;}

\NormalTok{util}\OperatorTok{.}\FunctionTok{inspect}\NormalTok{(}\KeywordTok{new} \FunctionTok{Foo}\NormalTok{())}\OperatorTok{;} \CommentTok{// \textquotesingle{}Foo [bar] \{\}\textquotesingle{}}
\NormalTok{util}\OperatorTok{.}\FunctionTok{inspect}\NormalTok{(}\KeywordTok{new} \FunctionTok{Bar}\NormalTok{())}\OperatorTok{;} \CommentTok{// \textquotesingle{}Bar \{\}\textquotesingle{}}
\NormalTok{util}\OperatorTok{.}\FunctionTok{inspect}\NormalTok{(baz)}\OperatorTok{;}       \CommentTok{// \textquotesingle{}[foo] \{\}\textquotesingle{}}
\end{Highlighting}
\end{Shaded}

Circular references point to their anchor by using a reference index:

\begin{Shaded}
\begin{Highlighting}[]
\KeywordTok{const}\NormalTok{ \{ inspect \} }\OperatorTok{=} \PreprocessorTok{require}\NormalTok{(}\StringTok{\textquotesingle{}node:util\textquotesingle{}}\NormalTok{)}\OperatorTok{;}

\KeywordTok{const}\NormalTok{ obj }\OperatorTok{=}\NormalTok{ \{\}}\OperatorTok{;}
\NormalTok{obj}\OperatorTok{.}\AttributeTok{a} \OperatorTok{=}\NormalTok{ [obj]}\OperatorTok{;}
\NormalTok{obj}\OperatorTok{.}\AttributeTok{b} \OperatorTok{=}\NormalTok{ \{\}}\OperatorTok{;}
\NormalTok{obj}\OperatorTok{.}\AttributeTok{b}\OperatorTok{.}\AttributeTok{inner} \OperatorTok{=}\NormalTok{ obj}\OperatorTok{.}\AttributeTok{b}\OperatorTok{;}
\NormalTok{obj}\OperatorTok{.}\AttributeTok{b}\OperatorTok{.}\AttributeTok{obj} \OperatorTok{=}\NormalTok{ obj}\OperatorTok{;}

\BuiltInTok{console}\OperatorTok{.}\FunctionTok{log}\NormalTok{(}\FunctionTok{inspect}\NormalTok{(obj))}\OperatorTok{;}
\CommentTok{// \textless{}ref *1\textgreater{} \{}
\CommentTok{//   a: [ [Circular *1] ],}
\CommentTok{//   b: \textless{}ref *2\textgreater{} \{ inner: [Circular *2], obj: [Circular *1] \}}
\CommentTok{// \}}
\end{Highlighting}
\end{Shaded}

The following example inspects all properties of the \texttt{util}
object:

\begin{Shaded}
\begin{Highlighting}[]
\KeywordTok{const}\NormalTok{ util }\OperatorTok{=} \PreprocessorTok{require}\NormalTok{(}\StringTok{\textquotesingle{}node:util\textquotesingle{}}\NormalTok{)}\OperatorTok{;}

\BuiltInTok{console}\OperatorTok{.}\FunctionTok{log}\NormalTok{(util}\OperatorTok{.}\FunctionTok{inspect}\NormalTok{(util}\OperatorTok{,}\NormalTok{ \{ }\DataTypeTok{showHidden}\OperatorTok{:} \KeywordTok{true}\OperatorTok{,} \DataTypeTok{depth}\OperatorTok{:} \KeywordTok{null}\NormalTok{ \}))}\OperatorTok{;}
\end{Highlighting}
\end{Shaded}

The following example highlights the effect of the \texttt{compact}
option:

\begin{Shaded}
\begin{Highlighting}[]
\KeywordTok{const}\NormalTok{ util }\OperatorTok{=} \PreprocessorTok{require}\NormalTok{(}\StringTok{\textquotesingle{}node:util\textquotesingle{}}\NormalTok{)}\OperatorTok{;}

\KeywordTok{const}\NormalTok{ o }\OperatorTok{=}\NormalTok{ \{}
  \DataTypeTok{a}\OperatorTok{:}\NormalTok{ [}\DecValTok{1}\OperatorTok{,} \DecValTok{2}\OperatorTok{,}\NormalTok{ [[}
    \StringTok{\textquotesingle{}Lorem ipsum dolor sit amet,}\SpecialCharTok{\textbackslash{}n}\StringTok{consectetur adipiscing elit, sed do \textquotesingle{}} \OperatorTok{+}
      \StringTok{\textquotesingle{}eiusmod }\SpecialCharTok{\textbackslash{}n}\StringTok{tempor incididunt ut labore et dolore magna aliqua.\textquotesingle{}}\OperatorTok{,}
    \StringTok{\textquotesingle{}test\textquotesingle{}}\OperatorTok{,}
    \StringTok{\textquotesingle{}foo\textquotesingle{}}\NormalTok{]]}\OperatorTok{,} \DecValTok{4}\NormalTok{]}\OperatorTok{,}
  \DataTypeTok{b}\OperatorTok{:} \KeywordTok{new} \BuiltInTok{Map}\NormalTok{([[}\StringTok{\textquotesingle{}za\textquotesingle{}}\OperatorTok{,} \DecValTok{1}\NormalTok{]}\OperatorTok{,}\NormalTok{ [}\StringTok{\textquotesingle{}zb\textquotesingle{}}\OperatorTok{,} \StringTok{\textquotesingle{}test\textquotesingle{}}\NormalTok{]])}\OperatorTok{,}
\NormalTok{\}}\OperatorTok{;}
\BuiltInTok{console}\OperatorTok{.}\FunctionTok{log}\NormalTok{(util}\OperatorTok{.}\FunctionTok{inspect}\NormalTok{(o}\OperatorTok{,}\NormalTok{ \{ }\DataTypeTok{compact}\OperatorTok{:} \KeywordTok{true}\OperatorTok{,} \DataTypeTok{depth}\OperatorTok{:} \DecValTok{5}\OperatorTok{,} \DataTypeTok{breakLength}\OperatorTok{:} \DecValTok{80}\NormalTok{ \}))}\OperatorTok{;}

\CommentTok{// \{ a:}
\CommentTok{//   [ 1,}
\CommentTok{//     2,}
\CommentTok{//     [ [ \textquotesingle{}Lorem ipsum dolor sit amet,\textbackslash{}nconsectetur [...]\textquotesingle{}, // A long line}
\CommentTok{//           \textquotesingle{}test\textquotesingle{},}
\CommentTok{//           \textquotesingle{}foo\textquotesingle{} ] ],}
\CommentTok{//     4 ],}
\CommentTok{//   b: Map(2) \{ \textquotesingle{}za\textquotesingle{} =\textgreater{} 1, \textquotesingle{}zb\textquotesingle{} =\textgreater{} \textquotesingle{}test\textquotesingle{} \} \}}

\CommentTok{// Setting \textasciigrave{}compact\textasciigrave{} to false or an integer creates more reader friendly output.}
\BuiltInTok{console}\OperatorTok{.}\FunctionTok{log}\NormalTok{(util}\OperatorTok{.}\FunctionTok{inspect}\NormalTok{(o}\OperatorTok{,}\NormalTok{ \{ }\DataTypeTok{compact}\OperatorTok{:} \KeywordTok{false}\OperatorTok{,} \DataTypeTok{depth}\OperatorTok{:} \DecValTok{5}\OperatorTok{,} \DataTypeTok{breakLength}\OperatorTok{:} \DecValTok{80}\NormalTok{ \}))}\OperatorTok{;}

\CommentTok{// \{}
\CommentTok{//   a: [}
\CommentTok{//     1,}
\CommentTok{//     2,}
\CommentTok{//     [}
\CommentTok{//       [}
\CommentTok{//         \textquotesingle{}Lorem ipsum dolor sit amet,\textbackslash{}n\textquotesingle{} +}
\CommentTok{//           \textquotesingle{}consectetur adipiscing elit, sed do eiusmod \textbackslash{}n\textquotesingle{} +}
\CommentTok{//           \textquotesingle{}tempor incididunt ut labore et dolore magna aliqua.\textquotesingle{},}
\CommentTok{//         \textquotesingle{}test\textquotesingle{},}
\CommentTok{//         \textquotesingle{}foo\textquotesingle{}}
\CommentTok{//       ]}
\CommentTok{//     ],}
\CommentTok{//     4}
\CommentTok{//   ],}
\CommentTok{//   b: Map(2) \{}
\CommentTok{//     \textquotesingle{}za\textquotesingle{} =\textgreater{} 1,}
\CommentTok{//     \textquotesingle{}zb\textquotesingle{} =\textgreater{} \textquotesingle{}test\textquotesingle{}}
\CommentTok{//   \}}
\CommentTok{// \}}

\CommentTok{// Setting \textasciigrave{}breakLength\textasciigrave{} to e.g. 150 will print the "Lorem ipsum" text in a}
\CommentTok{// single line.}
\end{Highlighting}
\end{Shaded}

The \texttt{showHidden} option allows
\href{https://developer.mozilla.org/en-US/docs/Web/JavaScript/Reference/Global_Objects/WeakMap}{\texttt{WeakMap}}
and
\href{https://developer.mozilla.org/en-US/docs/Web/JavaScript/Reference/Global_Objects/WeakSet}{\texttt{WeakSet}}
entries to be inspected. If there are more entries than
\texttt{maxArrayLength}, there is no guarantee which entries are
displayed. That means retrieving the same
\href{https://developer.mozilla.org/en-US/docs/Web/JavaScript/Reference/Global_Objects/WeakSet}{\texttt{WeakSet}}
entries twice may result in different output. Furthermore, entries with
no remaining strong references may be garbage collected at any time.

\begin{Shaded}
\begin{Highlighting}[]
\KeywordTok{const}\NormalTok{ \{ inspect \} }\OperatorTok{=} \PreprocessorTok{require}\NormalTok{(}\StringTok{\textquotesingle{}node:util\textquotesingle{}}\NormalTok{)}\OperatorTok{;}

\KeywordTok{const}\NormalTok{ obj }\OperatorTok{=}\NormalTok{ \{ }\DataTypeTok{a}\OperatorTok{:} \DecValTok{1}\NormalTok{ \}}\OperatorTok{;}
\KeywordTok{const}\NormalTok{ obj2 }\OperatorTok{=}\NormalTok{ \{ }\DataTypeTok{b}\OperatorTok{:} \DecValTok{2}\NormalTok{ \}}\OperatorTok{;}
\KeywordTok{const}\NormalTok{ weakSet }\OperatorTok{=} \KeywordTok{new} \BuiltInTok{WeakSet}\NormalTok{([obj}\OperatorTok{,}\NormalTok{ obj2])}\OperatorTok{;}

\BuiltInTok{console}\OperatorTok{.}\FunctionTok{log}\NormalTok{(}\FunctionTok{inspect}\NormalTok{(weakSet}\OperatorTok{,}\NormalTok{ \{ }\DataTypeTok{showHidden}\OperatorTok{:} \KeywordTok{true}\NormalTok{ \}))}\OperatorTok{;}
\CommentTok{// WeakSet \{ \{ a: 1 \}, \{ b: 2 \} \}}
\end{Highlighting}
\end{Shaded}

The \texttt{sorted} option ensures that an object's property insertion
order does not impact the result of \texttt{util.inspect()}.

\begin{Shaded}
\begin{Highlighting}[]
\KeywordTok{const}\NormalTok{ \{ inspect \} }\OperatorTok{=} \PreprocessorTok{require}\NormalTok{(}\StringTok{\textquotesingle{}node:util\textquotesingle{}}\NormalTok{)}\OperatorTok{;}
\KeywordTok{const}\NormalTok{ assert }\OperatorTok{=} \PreprocessorTok{require}\NormalTok{(}\StringTok{\textquotesingle{}node:assert\textquotesingle{}}\NormalTok{)}\OperatorTok{;}

\KeywordTok{const}\NormalTok{ o1 }\OperatorTok{=}\NormalTok{ \{}
  \DataTypeTok{b}\OperatorTok{:}\NormalTok{ [}\DecValTok{2}\OperatorTok{,} \DecValTok{3}\OperatorTok{,} \DecValTok{1}\NormalTok{]}\OperatorTok{,}
  \DataTypeTok{a}\OperatorTok{:} \StringTok{\textquotesingle{}\textasciigrave{}a\textasciigrave{} comes before \textasciigrave{}b\textasciigrave{}\textquotesingle{}}\OperatorTok{,}
  \DataTypeTok{c}\OperatorTok{:} \KeywordTok{new} \BuiltInTok{Set}\NormalTok{([}\DecValTok{2}\OperatorTok{,} \DecValTok{3}\OperatorTok{,} \DecValTok{1}\NormalTok{])}\OperatorTok{,}
\NormalTok{\}}\OperatorTok{;}
\BuiltInTok{console}\OperatorTok{.}\FunctionTok{log}\NormalTok{(}\FunctionTok{inspect}\NormalTok{(o1}\OperatorTok{,}\NormalTok{ \{ }\DataTypeTok{sorted}\OperatorTok{:} \KeywordTok{true}\NormalTok{ \}))}\OperatorTok{;}
\CommentTok{// \{ a: \textquotesingle{}\textasciigrave{}a\textasciigrave{} comes before \textasciigrave{}b\textasciigrave{}\textquotesingle{}, b: [ 2, 3, 1 ], c: Set(3) \{ 1, 2, 3 \} \}}
\BuiltInTok{console}\OperatorTok{.}\FunctionTok{log}\NormalTok{(}\FunctionTok{inspect}\NormalTok{(o1}\OperatorTok{,}\NormalTok{ \{ }\DataTypeTok{sorted}\OperatorTok{:}\NormalTok{ (a}\OperatorTok{,}\NormalTok{ b) }\KeywordTok{=\textgreater{}}\NormalTok{ b}\OperatorTok{.}\FunctionTok{localeCompare}\NormalTok{(a) \}))}\OperatorTok{;}
\CommentTok{// \{ c: Set(3) \{ 3, 2, 1 \}, b: [ 2, 3, 1 ], a: \textquotesingle{}\textasciigrave{}a\textasciigrave{} comes before \textasciigrave{}b\textasciigrave{}\textquotesingle{} \}}

\KeywordTok{const}\NormalTok{ o2 }\OperatorTok{=}\NormalTok{ \{}
  \DataTypeTok{c}\OperatorTok{:} \KeywordTok{new} \BuiltInTok{Set}\NormalTok{([}\DecValTok{2}\OperatorTok{,} \DecValTok{1}\OperatorTok{,} \DecValTok{3}\NormalTok{])}\OperatorTok{,}
  \DataTypeTok{a}\OperatorTok{:} \StringTok{\textquotesingle{}\textasciigrave{}a\textasciigrave{} comes before \textasciigrave{}b\textasciigrave{}\textquotesingle{}}\OperatorTok{,}
  \DataTypeTok{b}\OperatorTok{:}\NormalTok{ [}\DecValTok{2}\OperatorTok{,} \DecValTok{3}\OperatorTok{,} \DecValTok{1}\NormalTok{]}\OperatorTok{,}
\NormalTok{\}}\OperatorTok{;}
\NormalTok{assert}\OperatorTok{.}\AttributeTok{strict}\OperatorTok{.}\FunctionTok{equal}\NormalTok{(}
  \FunctionTok{inspect}\NormalTok{(o1}\OperatorTok{,}\NormalTok{ \{ }\DataTypeTok{sorted}\OperatorTok{:} \KeywordTok{true}\NormalTok{ \})}\OperatorTok{,}
  \FunctionTok{inspect}\NormalTok{(o2}\OperatorTok{,}\NormalTok{ \{ }\DataTypeTok{sorted}\OperatorTok{:} \KeywordTok{true}\NormalTok{ \})}\OperatorTok{,}
\NormalTok{)}\OperatorTok{;}
\end{Highlighting}
\end{Shaded}

The \texttt{numericSeparator} option adds an underscore every three
digits to all numbers.

\begin{Shaded}
\begin{Highlighting}[]
\KeywordTok{const}\NormalTok{ \{ inspect \} }\OperatorTok{=} \PreprocessorTok{require}\NormalTok{(}\StringTok{\textquotesingle{}node:util\textquotesingle{}}\NormalTok{)}\OperatorTok{;}

\KeywordTok{const}\NormalTok{ thousand }\OperatorTok{=} \DecValTok{1\_000}\OperatorTok{;}
\KeywordTok{const}\NormalTok{ million }\OperatorTok{=} \DecValTok{1\_000\_000}\OperatorTok{;}
\KeywordTok{const}\NormalTok{ bigNumber }\OperatorTok{=} \DecValTok{123\_456\_789}\NormalTok{n}\OperatorTok{;}
\KeywordTok{const}\NormalTok{ bigDecimal }\OperatorTok{=} \FloatTok{1\_234.123\_45}\OperatorTok{;}

\BuiltInTok{console}\OperatorTok{.}\FunctionTok{log}\NormalTok{(}\FunctionTok{inspect}\NormalTok{(thousand}\OperatorTok{,}\NormalTok{ \{ }\DataTypeTok{numericSeparator}\OperatorTok{:} \KeywordTok{true}\NormalTok{ \}))}\OperatorTok{;}
\CommentTok{// 1\_000}
\BuiltInTok{console}\OperatorTok{.}\FunctionTok{log}\NormalTok{(}\FunctionTok{inspect}\NormalTok{(million}\OperatorTok{,}\NormalTok{ \{ }\DataTypeTok{numericSeparator}\OperatorTok{:} \KeywordTok{true}\NormalTok{ \}))}\OperatorTok{;}
\CommentTok{// 1\_000\_000}
\BuiltInTok{console}\OperatorTok{.}\FunctionTok{log}\NormalTok{(}\FunctionTok{inspect}\NormalTok{(bigNumber}\OperatorTok{,}\NormalTok{ \{ }\DataTypeTok{numericSeparator}\OperatorTok{:} \KeywordTok{true}\NormalTok{ \}))}\OperatorTok{;}
\CommentTok{// 123\_456\_789n}
\BuiltInTok{console}\OperatorTok{.}\FunctionTok{log}\NormalTok{(}\FunctionTok{inspect}\NormalTok{(bigDecimal}\OperatorTok{,}\NormalTok{ \{ }\DataTypeTok{numericSeparator}\OperatorTok{:} \KeywordTok{true}\NormalTok{ \}))}\OperatorTok{;}
\CommentTok{// 1\_234.123\_45}
\end{Highlighting}
\end{Shaded}

\texttt{util.inspect()} is a synchronous method intended for debugging.
Its maximum output length is approximately 128 MiB. Inputs that result
in longer output will be truncated.

\subsubsection{\texorpdfstring{Customizing \texttt{util.inspect}
colors}{Customizing util.inspect colors}}\label{customizing-util.inspect-colors}

Color output (if enabled) of \texttt{util.inspect} is customizable
globally via the \texttt{util.inspect.styles} and
\texttt{util.inspect.colors} properties.

\texttt{util.inspect.styles} is a map associating a style name to a
color from \texttt{util.inspect.colors}.

The default styles and associated colors are:

\begin{itemize}
\tightlist
\item
  \texttt{bigint}: \texttt{yellow}
\item
  \texttt{boolean}: \texttt{yellow}
\item
  \texttt{date}: \texttt{magenta}
\item
  \texttt{module}: \texttt{underline}
\item
  \texttt{name}: (no styling)
\item
  \texttt{null}: \texttt{bold}
\item
  \texttt{number}: \texttt{yellow}
\item
  \texttt{regexp}: \texttt{red}
\item
  \texttt{special}: \texttt{cyan} (e.g., \texttt{Proxies})
\item
  \texttt{string}: \texttt{green}
\item
  \texttt{symbol}: \texttt{green}
\item
  \texttt{undefined}: \texttt{grey}
\end{itemize}

Color styling uses ANSI control codes that may not be supported on all
terminals. To verify color support use
\href{tty.md\#writestreamhascolorscount-env}{\texttt{tty.hasColors()}}.

Predefined control codes are listed below (grouped as ``Modifiers'',
``Foreground colors'', and ``Background colors'').

\paragraph{Modifiers}\label{modifiers}

Modifier support varies throughout different terminals. They will mostly
be ignored, if not supported.

\begin{itemize}
\tightlist
\item
  \texttt{reset} - Resets all (color) modifiers to their defaults
\item
  \textbf{bold} - Make text bold
\item
  \emph{italic} - Make text italic
\item
  {underline} - Make text underlined
\item
  \st{strikethrough} - Puts a horizontal line through the center of the
  text (Alias: \texttt{strikeThrough}, \texttt{crossedout},
  \texttt{crossedOut})
\item
  \texttt{hidden} - Prints the text, but makes it invisible (Alias:
  conceal)
\item
  {dim} - Decreased color intensity (Alias: \texttt{faint})
\item
  {overlined} - Make text overlined
\item
  blink - Hides and shows the text in an interval
\item
  {inverse} - Swap foreground and background colors (Alias:
  \texttt{swapcolors}, \texttt{swapColors})
\item
  {doubleunderline} - Make text double underlined (Alias:
  \texttt{doubleUnderline})
\item
  {framed} - Draw a frame around the text
\end{itemize}

\paragraph{Foreground colors}\label{foreground-colors}

\begin{itemize}
\tightlist
\item
  \texttt{black}
\item
  \texttt{red}
\item
  \texttt{green}
\item
  \texttt{yellow}
\item
  \texttt{blue}
\item
  \texttt{magenta}
\item
  \texttt{cyan}
\item
  \texttt{white}
\item
  \texttt{gray} (alias: \texttt{grey}, \texttt{blackBright})
\item
  \texttt{redBright}
\item
  \texttt{greenBright}
\item
  \texttt{yellowBright}
\item
  \texttt{blueBright}
\item
  \texttt{magentaBright}
\item
  \texttt{cyanBright}
\item
  \texttt{whiteBright}
\end{itemize}

\paragraph{Background colors}\label{background-colors}

\begin{itemize}
\tightlist
\item
  \texttt{bgBlack}
\item
  \texttt{bgRed}
\item
  \texttt{bgGreen}
\item
  \texttt{bgYellow}
\item
  \texttt{bgBlue}
\item
  \texttt{bgMagenta}
\item
  \texttt{bgCyan}
\item
  \texttt{bgWhite}
\item
  \texttt{bgGray} (alias: \texttt{bgGrey}, \texttt{bgBlackBright})
\item
  \texttt{bgRedBright}
\item
  \texttt{bgGreenBright}
\item
  \texttt{bgYellowBright}
\item
  \texttt{bgBlueBright}
\item
  \texttt{bgMagentaBright}
\item
  \texttt{bgCyanBright}
\item
  \texttt{bgWhiteBright}
\end{itemize}

\subsubsection{Custom inspection functions on
objects}\label{custom-inspection-functions-on-objects}

Objects may also define their own
\hyperref[utilinspectcustom]{\texttt{{[}util.inspect.custom{]}(depth,\ opts,\ inspect)}}
function, which \texttt{util.inspect()} will invoke and use the result
of when inspecting the object.

\begin{Shaded}
\begin{Highlighting}[]
\KeywordTok{const}\NormalTok{ util }\OperatorTok{=} \PreprocessorTok{require}\NormalTok{(}\StringTok{\textquotesingle{}node:util\textquotesingle{}}\NormalTok{)}\OperatorTok{;}

\KeywordTok{class}\NormalTok{ Box \{}
  \FunctionTok{constructor}\NormalTok{(value) \{}
    \KeywordTok{this}\OperatorTok{.}\AttributeTok{value} \OperatorTok{=}\NormalTok{ value}\OperatorTok{;}
\NormalTok{  \}}

\NormalTok{  [util}\OperatorTok{.}\AttributeTok{inspect}\OperatorTok{.}\AttributeTok{custom}\NormalTok{](depth}\OperatorTok{,}\NormalTok{ options}\OperatorTok{,}\NormalTok{ inspect) \{}
    \ControlFlowTok{if}\NormalTok{ (depth }\OperatorTok{\textless{}} \DecValTok{0}\NormalTok{) \{}
      \ControlFlowTok{return}\NormalTok{ options}\OperatorTok{.}\FunctionTok{stylize}\NormalTok{(}\StringTok{\textquotesingle{}[Box]\textquotesingle{}}\OperatorTok{,} \StringTok{\textquotesingle{}special\textquotesingle{}}\NormalTok{)}\OperatorTok{;}
\NormalTok{    \}}

    \KeywordTok{const}\NormalTok{ newOptions }\OperatorTok{=} \BuiltInTok{Object}\OperatorTok{.}\FunctionTok{assign}\NormalTok{(\{\}}\OperatorTok{,}\NormalTok{ options}\OperatorTok{,}\NormalTok{ \{}
      \DataTypeTok{depth}\OperatorTok{:}\NormalTok{ options}\OperatorTok{.}\AttributeTok{depth} \OperatorTok{===} \KeywordTok{null} \OperatorTok{?} \KeywordTok{null} \OperatorTok{:}\NormalTok{ options}\OperatorTok{.}\AttributeTok{depth} \OperatorTok{{-}} \DecValTok{1}\OperatorTok{,}
\NormalTok{    \})}\OperatorTok{;}

    \CommentTok{// Five space padding because that\textquotesingle{}s the size of "Box\textless{} ".}
    \KeywordTok{const}\NormalTok{ padding }\OperatorTok{=} \StringTok{\textquotesingle{} \textquotesingle{}}\OperatorTok{.}\FunctionTok{repeat}\NormalTok{(}\DecValTok{5}\NormalTok{)}\OperatorTok{;}
    \KeywordTok{const}\NormalTok{ inner }\OperatorTok{=} \FunctionTok{inspect}\NormalTok{(}\KeywordTok{this}\OperatorTok{.}\AttributeTok{value}\OperatorTok{,}\NormalTok{ newOptions)}
                  \OperatorTok{.}\FunctionTok{replace}\NormalTok{(}\SpecialStringTok{/}\SpecialCharTok{\textbackslash{}n}\SpecialStringTok{/g}\OperatorTok{,} \VerbatimStringTok{\textasciigrave{}}\SpecialCharTok{\textbackslash{}n$\{}\NormalTok{padding}\SpecialCharTok{\}}\VerbatimStringTok{\textasciigrave{}}\NormalTok{)}\OperatorTok{;}
    \ControlFlowTok{return} \VerbatimStringTok{\textasciigrave{}}\SpecialCharTok{$\{}\NormalTok{options}\OperatorTok{.}\FunctionTok{stylize}\NormalTok{(}\StringTok{\textquotesingle{}Box\textquotesingle{}}\OperatorTok{,} \StringTok{\textquotesingle{}special\textquotesingle{}}\NormalTok{)}\SpecialCharTok{\}}\VerbatimStringTok{\textless{} }\SpecialCharTok{$\{}\NormalTok{inner}\SpecialCharTok{\}}\VerbatimStringTok{ \textgreater{}\textasciigrave{}}\OperatorTok{;}
\NormalTok{  \}}
\NormalTok{\}}

\KeywordTok{const}\NormalTok{ box }\OperatorTok{=} \KeywordTok{new} \FunctionTok{Box}\NormalTok{(}\KeywordTok{true}\NormalTok{)}\OperatorTok{;}

\NormalTok{util}\OperatorTok{.}\FunctionTok{inspect}\NormalTok{(box)}\OperatorTok{;}
\CommentTok{// Returns: "Box\textless{} true \textgreater{}"}
\end{Highlighting}
\end{Shaded}

Custom \texttt{{[}util.inspect.custom{]}(depth,\ opts,\ inspect)}
functions typically return a string but may return a value of any type
that will be formatted accordingly by \texttt{util.inspect()}.

\begin{Shaded}
\begin{Highlighting}[]
\KeywordTok{const}\NormalTok{ util }\OperatorTok{=} \PreprocessorTok{require}\NormalTok{(}\StringTok{\textquotesingle{}node:util\textquotesingle{}}\NormalTok{)}\OperatorTok{;}

\KeywordTok{const}\NormalTok{ obj }\OperatorTok{=}\NormalTok{ \{ }\DataTypeTok{foo}\OperatorTok{:} \StringTok{\textquotesingle{}this will not show up in the inspect() output\textquotesingle{}}\NormalTok{ \}}\OperatorTok{;}
\NormalTok{obj[util}\OperatorTok{.}\AttributeTok{inspect}\OperatorTok{.}\AttributeTok{custom}\NormalTok{] }\OperatorTok{=}\NormalTok{ (depth) }\KeywordTok{=\textgreater{}}\NormalTok{ \{}
  \ControlFlowTok{return}\NormalTok{ \{ }\DataTypeTok{bar}\OperatorTok{:} \StringTok{\textquotesingle{}baz\textquotesingle{}}\NormalTok{ \}}\OperatorTok{;}
\NormalTok{\}}\OperatorTok{;}

\NormalTok{util}\OperatorTok{.}\FunctionTok{inspect}\NormalTok{(obj)}\OperatorTok{;}
\CommentTok{// Returns: "\{ bar: \textquotesingle{}baz\textquotesingle{} \}"}
\end{Highlighting}
\end{Shaded}

\subsubsection{\texorpdfstring{\texttt{util.inspect.custom}}{util.inspect.custom}}\label{util.inspect.custom}

\begin{itemize}
\tightlist
\item
  \{symbol\} that can be used to declare custom inspect functions.
\end{itemize}

In addition to being accessible through \texttt{util.inspect.custom},
this symbol is
\href{https://developer.mozilla.org/en-US/docs/Web/JavaScript/Reference/Global_Objects/Symbol/for}{registered
globally} and can be accessed in any environment as
\texttt{Symbol.for(\textquotesingle{}nodejs.util.inspect.custom\textquotesingle{})}.

Using this allows code to be written in a portable fashion, so that the
custom inspect function is used in an Node.js environment and ignored in
the browser. The \texttt{util.inspect()} function itself is passed as
third argument to the custom inspect function to allow further
portability.

\begin{Shaded}
\begin{Highlighting}[]
\KeywordTok{const}\NormalTok{ customInspectSymbol }\OperatorTok{=} \BuiltInTok{Symbol}\OperatorTok{.}\FunctionTok{for}\NormalTok{(}\StringTok{\textquotesingle{}nodejs.util.inspect.custom\textquotesingle{}}\NormalTok{)}\OperatorTok{;}

\KeywordTok{class}\NormalTok{ Password \{}
  \FunctionTok{constructor}\NormalTok{(value) \{}
    \KeywordTok{this}\OperatorTok{.}\AttributeTok{value} \OperatorTok{=}\NormalTok{ value}\OperatorTok{;}
\NormalTok{  \}}

  \FunctionTok{toString}\NormalTok{() \{}
    \ControlFlowTok{return} \StringTok{\textquotesingle{}xxxxxxxx\textquotesingle{}}\OperatorTok{;}
\NormalTok{  \}}

\NormalTok{  [customInspectSymbol](depth}\OperatorTok{,}\NormalTok{ inspectOptions}\OperatorTok{,}\NormalTok{ inspect) \{}
    \ControlFlowTok{return} \VerbatimStringTok{\textasciigrave{}Password \textless{}}\SpecialCharTok{$\{}\KeywordTok{this}\OperatorTok{.}\FunctionTok{toString}\NormalTok{()}\SpecialCharTok{\}}\VerbatimStringTok{\textgreater{}\textasciigrave{}}\OperatorTok{;}
\NormalTok{  \}}
\NormalTok{\}}

\KeywordTok{const}\NormalTok{ password }\OperatorTok{=} \KeywordTok{new} \FunctionTok{Password}\NormalTok{(}\StringTok{\textquotesingle{}r0sebud\textquotesingle{}}\NormalTok{)}\OperatorTok{;}
\BuiltInTok{console}\OperatorTok{.}\FunctionTok{log}\NormalTok{(password)}\OperatorTok{;}
\CommentTok{// Prints Password \textless{}xxxxxxxx\textgreater{}}
\end{Highlighting}
\end{Shaded}

See \hyperref[custom-inspection-functions-on-objects]{Custom inspection
functions on Objects} for more details.

\subsubsection{\texorpdfstring{\texttt{util.inspect.defaultOptions}}{util.inspect.defaultOptions}}\label{util.inspect.defaultoptions}

The \texttt{defaultOptions} value allows customization of the default
options used by \texttt{util.inspect}. This is useful for functions like
\texttt{console.log} or \texttt{util.format} which implicitly call into
\texttt{util.inspect}. It shall be set to an object containing one or
more valid \hyperref[utilinspectobject-options]{\texttt{util.inspect()}}
options. Setting option properties directly is also supported.

\begin{Shaded}
\begin{Highlighting}[]
\KeywordTok{const}\NormalTok{ util }\OperatorTok{=} \PreprocessorTok{require}\NormalTok{(}\StringTok{\textquotesingle{}node:util\textquotesingle{}}\NormalTok{)}\OperatorTok{;}
\KeywordTok{const}\NormalTok{ arr }\OperatorTok{=} \BuiltInTok{Array}\NormalTok{(}\DecValTok{101}\NormalTok{)}\OperatorTok{.}\FunctionTok{fill}\NormalTok{(}\DecValTok{0}\NormalTok{)}\OperatorTok{;}

\BuiltInTok{console}\OperatorTok{.}\FunctionTok{log}\NormalTok{(arr)}\OperatorTok{;} \CommentTok{// Logs the truncated array}
\NormalTok{util}\OperatorTok{.}\AttributeTok{inspect}\OperatorTok{.}\AttributeTok{defaultOptions}\OperatorTok{.}\AttributeTok{maxArrayLength} \OperatorTok{=} \KeywordTok{null}\OperatorTok{;}
\BuiltInTok{console}\OperatorTok{.}\FunctionTok{log}\NormalTok{(arr)}\OperatorTok{;} \CommentTok{// logs the full array}
\end{Highlighting}
\end{Shaded}

\subsection{\texorpdfstring{\texttt{util.isDeepStrictEqual(val1,\ val2)}}{util.isDeepStrictEqual(val1, val2)}}\label{util.isdeepstrictequalval1-val2}

\begin{itemize}
\tightlist
\item
  \texttt{val1} \{any\}
\item
  \texttt{val2} \{any\}
\item
  Returns: \{boolean\}
\end{itemize}

Returns \texttt{true} if there is deep strict equality between
\texttt{val1} and \texttt{val2}. Otherwise, returns \texttt{false}.

See
\href{assert.md\#assertdeepstrictequalactual-expected-message}{\texttt{assert.deepStrictEqual()}}
for more information about deep strict equality.

\subsection{\texorpdfstring{Class:
\texttt{util.MIMEType}}{Class: util.MIMEType}}\label{class-util.mimetype}

\begin{quote}
Stability: 1 - Experimental
\end{quote}

An implementation of
\href{https://bmeck.github.io/node-proposal-mime-api/}{the MIMEType
class}.

In accordance with browser conventions, all properties of
\texttt{MIMEType} objects are implemented as getters and setters on the
class prototype, rather than as data properties on the object itself.

A MIME string is a structured string containing multiple meaningful
components. When parsed, a \texttt{MIMEType} object is returned
containing properties for each of these components.

\subsubsection{\texorpdfstring{Constructor:
\texttt{new\ MIMEType(input)}}{Constructor: new MIMEType(input)}}\label{constructor-new-mimetypeinput}

\begin{itemize}
\tightlist
\item
  \texttt{input} \{string\} The input MIME to parse
\end{itemize}

Creates a new \texttt{MIMEType} object by parsing the \texttt{input}.

\begin{Shaded}
\begin{Highlighting}[]
\ImportTok{import}\NormalTok{ \{ MIMEType \} }\ImportTok{from} \StringTok{\textquotesingle{}node:util\textquotesingle{}}\OperatorTok{;}

\KeywordTok{const}\NormalTok{ myMIME }\OperatorTok{=} \KeywordTok{new} \FunctionTok{MIMEType}\NormalTok{(}\StringTok{\textquotesingle{}text/plain\textquotesingle{}}\NormalTok{)}\OperatorTok{;}
\end{Highlighting}
\end{Shaded}

\begin{Shaded}
\begin{Highlighting}[]
\KeywordTok{const}\NormalTok{ \{ MIMEType \} }\OperatorTok{=} \PreprocessorTok{require}\NormalTok{(}\StringTok{\textquotesingle{}node:util\textquotesingle{}}\NormalTok{)}\OperatorTok{;}

\KeywordTok{const}\NormalTok{ myMIME }\OperatorTok{=} \KeywordTok{new} \FunctionTok{MIMEType}\NormalTok{(}\StringTok{\textquotesingle{}text/plain\textquotesingle{}}\NormalTok{)}\OperatorTok{;}
\end{Highlighting}
\end{Shaded}

A \texttt{TypeError} will be thrown if the \texttt{input} is not a valid
MIME. Note that an effort will be made to coerce the given values into
strings. For instance:

\begin{Shaded}
\begin{Highlighting}[]
\ImportTok{import}\NormalTok{ \{ MIMEType \} }\ImportTok{from} \StringTok{\textquotesingle{}node:util\textquotesingle{}}\OperatorTok{;}
\KeywordTok{const}\NormalTok{ myMIME }\OperatorTok{=} \KeywordTok{new} \FunctionTok{MIMEType}\NormalTok{(\{ }\DataTypeTok{toString}\OperatorTok{:}\NormalTok{ () }\KeywordTok{=\textgreater{}} \StringTok{\textquotesingle{}text/plain\textquotesingle{}}\NormalTok{ \})}\OperatorTok{;}
\BuiltInTok{console}\OperatorTok{.}\FunctionTok{log}\NormalTok{(}\BuiltInTok{String}\NormalTok{(myMIME))}\OperatorTok{;}
\CommentTok{// Prints: text/plain}
\end{Highlighting}
\end{Shaded}

\begin{Shaded}
\begin{Highlighting}[]
\KeywordTok{const}\NormalTok{ \{ MIMEType \} }\OperatorTok{=} \PreprocessorTok{require}\NormalTok{(}\StringTok{\textquotesingle{}node:util\textquotesingle{}}\NormalTok{)}\OperatorTok{;}
\KeywordTok{const}\NormalTok{ myMIME }\OperatorTok{=} \KeywordTok{new} \FunctionTok{MIMEType}\NormalTok{(\{ }\DataTypeTok{toString}\OperatorTok{:}\NormalTok{ () }\KeywordTok{=\textgreater{}} \StringTok{\textquotesingle{}text/plain\textquotesingle{}}\NormalTok{ \})}\OperatorTok{;}
\BuiltInTok{console}\OperatorTok{.}\FunctionTok{log}\NormalTok{(}\BuiltInTok{String}\NormalTok{(myMIME))}\OperatorTok{;}
\CommentTok{// Prints: text/plain}
\end{Highlighting}
\end{Shaded}

\subsubsection{\texorpdfstring{\texttt{mime.type}}{mime.type}}\label{mime.type}

\begin{itemize}
\tightlist
\item
  \{string\}
\end{itemize}

Gets and sets the type portion of the MIME.

\begin{Shaded}
\begin{Highlighting}[]
\ImportTok{import}\NormalTok{ \{ MIMEType \} }\ImportTok{from} \StringTok{\textquotesingle{}node:util\textquotesingle{}}\OperatorTok{;}

\KeywordTok{const}\NormalTok{ myMIME }\OperatorTok{=} \KeywordTok{new} \FunctionTok{MIMEType}\NormalTok{(}\StringTok{\textquotesingle{}text/javascript\textquotesingle{}}\NormalTok{)}\OperatorTok{;}
\BuiltInTok{console}\OperatorTok{.}\FunctionTok{log}\NormalTok{(myMIME}\OperatorTok{.}\AttributeTok{type}\NormalTok{)}\OperatorTok{;}
\CommentTok{// Prints: text}
\NormalTok{myMIME}\OperatorTok{.}\AttributeTok{type} \OperatorTok{=} \StringTok{\textquotesingle{}application\textquotesingle{}}\OperatorTok{;}
\BuiltInTok{console}\OperatorTok{.}\FunctionTok{log}\NormalTok{(myMIME}\OperatorTok{.}\AttributeTok{type}\NormalTok{)}\OperatorTok{;}
\CommentTok{// Prints: application}
\BuiltInTok{console}\OperatorTok{.}\FunctionTok{log}\NormalTok{(}\BuiltInTok{String}\NormalTok{(myMIME))}\OperatorTok{;}
\CommentTok{// Prints: application/javascript}
\end{Highlighting}
\end{Shaded}

\begin{Shaded}
\begin{Highlighting}[]
\KeywordTok{const}\NormalTok{ \{ MIMEType \} }\OperatorTok{=} \PreprocessorTok{require}\NormalTok{(}\StringTok{\textquotesingle{}node:util\textquotesingle{}}\NormalTok{)}\OperatorTok{;}

\KeywordTok{const}\NormalTok{ myMIME }\OperatorTok{=} \KeywordTok{new} \FunctionTok{MIMEType}\NormalTok{(}\StringTok{\textquotesingle{}text/javascript\textquotesingle{}}\NormalTok{)}\OperatorTok{;}
\BuiltInTok{console}\OperatorTok{.}\FunctionTok{log}\NormalTok{(myMIME}\OperatorTok{.}\AttributeTok{type}\NormalTok{)}\OperatorTok{;}
\CommentTok{// Prints: text}
\NormalTok{myMIME}\OperatorTok{.}\AttributeTok{type} \OperatorTok{=} \StringTok{\textquotesingle{}application\textquotesingle{}}\OperatorTok{;}
\BuiltInTok{console}\OperatorTok{.}\FunctionTok{log}\NormalTok{(myMIME}\OperatorTok{.}\AttributeTok{type}\NormalTok{)}\OperatorTok{;}
\CommentTok{// Prints: application}
\BuiltInTok{console}\OperatorTok{.}\FunctionTok{log}\NormalTok{(}\BuiltInTok{String}\NormalTok{(myMIME))}\OperatorTok{;}
\CommentTok{// Prints: application/javascript}
\end{Highlighting}
\end{Shaded}

\subsubsection{\texorpdfstring{\texttt{mime.subtype}}{mime.subtype}}\label{mime.subtype}

\begin{itemize}
\tightlist
\item
  \{string\}
\end{itemize}

Gets and sets the subtype portion of the MIME.

\begin{Shaded}
\begin{Highlighting}[]
\ImportTok{import}\NormalTok{ \{ MIMEType \} }\ImportTok{from} \StringTok{\textquotesingle{}node:util\textquotesingle{}}\OperatorTok{;}

\KeywordTok{const}\NormalTok{ myMIME }\OperatorTok{=} \KeywordTok{new} \FunctionTok{MIMEType}\NormalTok{(}\StringTok{\textquotesingle{}text/ecmascript\textquotesingle{}}\NormalTok{)}\OperatorTok{;}
\BuiltInTok{console}\OperatorTok{.}\FunctionTok{log}\NormalTok{(myMIME}\OperatorTok{.}\AttributeTok{subtype}\NormalTok{)}\OperatorTok{;}
\CommentTok{// Prints: ecmascript}
\NormalTok{myMIME}\OperatorTok{.}\AttributeTok{subtype} \OperatorTok{=} \StringTok{\textquotesingle{}javascript\textquotesingle{}}\OperatorTok{;}
\BuiltInTok{console}\OperatorTok{.}\FunctionTok{log}\NormalTok{(myMIME}\OperatorTok{.}\AttributeTok{subtype}\NormalTok{)}\OperatorTok{;}
\CommentTok{// Prints: javascript}
\BuiltInTok{console}\OperatorTok{.}\FunctionTok{log}\NormalTok{(}\BuiltInTok{String}\NormalTok{(myMIME))}\OperatorTok{;}
\CommentTok{// Prints: text/javascript}
\end{Highlighting}
\end{Shaded}

\begin{Shaded}
\begin{Highlighting}[]
\KeywordTok{const}\NormalTok{ \{ MIMEType \} }\OperatorTok{=} \PreprocessorTok{require}\NormalTok{(}\StringTok{\textquotesingle{}node:util\textquotesingle{}}\NormalTok{)}\OperatorTok{;}

\KeywordTok{const}\NormalTok{ myMIME }\OperatorTok{=} \KeywordTok{new} \FunctionTok{MIMEType}\NormalTok{(}\StringTok{\textquotesingle{}text/ecmascript\textquotesingle{}}\NormalTok{)}\OperatorTok{;}
\BuiltInTok{console}\OperatorTok{.}\FunctionTok{log}\NormalTok{(myMIME}\OperatorTok{.}\AttributeTok{subtype}\NormalTok{)}\OperatorTok{;}
\CommentTok{// Prints: ecmascript}
\NormalTok{myMIME}\OperatorTok{.}\AttributeTok{subtype} \OperatorTok{=} \StringTok{\textquotesingle{}javascript\textquotesingle{}}\OperatorTok{;}
\BuiltInTok{console}\OperatorTok{.}\FunctionTok{log}\NormalTok{(myMIME}\OperatorTok{.}\AttributeTok{subtype}\NormalTok{)}\OperatorTok{;}
\CommentTok{// Prints: javascript}
\BuiltInTok{console}\OperatorTok{.}\FunctionTok{log}\NormalTok{(}\BuiltInTok{String}\NormalTok{(myMIME))}\OperatorTok{;}
\CommentTok{// Prints: text/javascript}
\end{Highlighting}
\end{Shaded}

\subsubsection{\texorpdfstring{\texttt{mime.essence}}{mime.essence}}\label{mime.essence}

\begin{itemize}
\tightlist
\item
  \{string\}
\end{itemize}

Gets the essence of the MIME. This property is read only. Use
\texttt{mime.type} or \texttt{mime.subtype} to alter the MIME.

\begin{Shaded}
\begin{Highlighting}[]
\ImportTok{import}\NormalTok{ \{ MIMEType \} }\ImportTok{from} \StringTok{\textquotesingle{}node:util\textquotesingle{}}\OperatorTok{;}

\KeywordTok{const}\NormalTok{ myMIME }\OperatorTok{=} \KeywordTok{new} \FunctionTok{MIMEType}\NormalTok{(}\StringTok{\textquotesingle{}text/javascript;key=value\textquotesingle{}}\NormalTok{)}\OperatorTok{;}
\BuiltInTok{console}\OperatorTok{.}\FunctionTok{log}\NormalTok{(myMIME}\OperatorTok{.}\AttributeTok{essence}\NormalTok{)}\OperatorTok{;}
\CommentTok{// Prints: text/javascript}
\NormalTok{myMIME}\OperatorTok{.}\AttributeTok{type} \OperatorTok{=} \StringTok{\textquotesingle{}application\textquotesingle{}}\OperatorTok{;}
\BuiltInTok{console}\OperatorTok{.}\FunctionTok{log}\NormalTok{(myMIME}\OperatorTok{.}\AttributeTok{essence}\NormalTok{)}\OperatorTok{;}
\CommentTok{// Prints: application/javascript}
\BuiltInTok{console}\OperatorTok{.}\FunctionTok{log}\NormalTok{(}\BuiltInTok{String}\NormalTok{(myMIME))}\OperatorTok{;}
\CommentTok{// Prints: application/javascript;key=value}
\end{Highlighting}
\end{Shaded}

\begin{Shaded}
\begin{Highlighting}[]
\KeywordTok{const}\NormalTok{ \{ MIMEType \} }\OperatorTok{=} \PreprocessorTok{require}\NormalTok{(}\StringTok{\textquotesingle{}node:util\textquotesingle{}}\NormalTok{)}\OperatorTok{;}

\KeywordTok{const}\NormalTok{ myMIME }\OperatorTok{=} \KeywordTok{new} \FunctionTok{MIMEType}\NormalTok{(}\StringTok{\textquotesingle{}text/javascript;key=value\textquotesingle{}}\NormalTok{)}\OperatorTok{;}
\BuiltInTok{console}\OperatorTok{.}\FunctionTok{log}\NormalTok{(myMIME}\OperatorTok{.}\AttributeTok{essence}\NormalTok{)}\OperatorTok{;}
\CommentTok{// Prints: text/javascript}
\NormalTok{myMIME}\OperatorTok{.}\AttributeTok{type} \OperatorTok{=} \StringTok{\textquotesingle{}application\textquotesingle{}}\OperatorTok{;}
\BuiltInTok{console}\OperatorTok{.}\FunctionTok{log}\NormalTok{(myMIME}\OperatorTok{.}\AttributeTok{essence}\NormalTok{)}\OperatorTok{;}
\CommentTok{// Prints: application/javascript}
\BuiltInTok{console}\OperatorTok{.}\FunctionTok{log}\NormalTok{(}\BuiltInTok{String}\NormalTok{(myMIME))}\OperatorTok{;}
\CommentTok{// Prints: application/javascript;key=value}
\end{Highlighting}
\end{Shaded}

\subsubsection{\texorpdfstring{\texttt{mime.params}}{mime.params}}\label{mime.params}

\begin{itemize}
\tightlist
\item
  \{MIMEParams\}
\end{itemize}

Gets the \hyperref[class-utilmimeparams]{\texttt{MIMEParams}} object
representing the parameters of the MIME. This property is read-only. See
\hyperref[class-utilmimeparams]{\texttt{MIMEParams}} documentation for
details.

\subsubsection{\texorpdfstring{\texttt{mime.toString()}}{mime.toString()}}\label{mime.tostring}

\begin{itemize}
\tightlist
\item
  Returns: \{string\}
\end{itemize}

The \texttt{toString()} method on the \texttt{MIMEType} object returns
the serialized MIME.

Because of the need for standard compliance, this method does not allow
users to customize the serialization process of the MIME.

\subsubsection{\texorpdfstring{\texttt{mime.toJSON()}}{mime.toJSON()}}\label{mime.tojson}

\begin{itemize}
\tightlist
\item
  Returns: \{string\}
\end{itemize}

Alias for \hyperref[mimetostring]{\texttt{mime.toString()}}.

This method is automatically called when an \texttt{MIMEType} object is
serialized with
\href{https://developer.mozilla.org/en-US/docs/Web/JavaScript/Reference/Global_Objects/JSON/stringify}{\texttt{JSON.stringify()}}.

\begin{Shaded}
\begin{Highlighting}[]
\ImportTok{import}\NormalTok{ \{ MIMEType \} }\ImportTok{from} \StringTok{\textquotesingle{}node:util\textquotesingle{}}\OperatorTok{;}

\KeywordTok{const}\NormalTok{ myMIMES }\OperatorTok{=}\NormalTok{ [}
  \KeywordTok{new} \FunctionTok{MIMEType}\NormalTok{(}\StringTok{\textquotesingle{}image/png\textquotesingle{}}\NormalTok{)}\OperatorTok{,}
  \KeywordTok{new} \FunctionTok{MIMEType}\NormalTok{(}\StringTok{\textquotesingle{}image/gif\textquotesingle{}}\NormalTok{)}\OperatorTok{,}
\NormalTok{]}\OperatorTok{;}
\BuiltInTok{console}\OperatorTok{.}\FunctionTok{log}\NormalTok{(}\BuiltInTok{JSON}\OperatorTok{.}\FunctionTok{stringify}\NormalTok{(myMIMES))}\OperatorTok{;}
\CommentTok{// Prints: ["image/png", "image/gif"]}
\end{Highlighting}
\end{Shaded}

\begin{Shaded}
\begin{Highlighting}[]
\KeywordTok{const}\NormalTok{ \{ MIMEType \} }\OperatorTok{=} \PreprocessorTok{require}\NormalTok{(}\StringTok{\textquotesingle{}node:util\textquotesingle{}}\NormalTok{)}\OperatorTok{;}

\KeywordTok{const}\NormalTok{ myMIMES }\OperatorTok{=}\NormalTok{ [}
  \KeywordTok{new} \FunctionTok{MIMEType}\NormalTok{(}\StringTok{\textquotesingle{}image/png\textquotesingle{}}\NormalTok{)}\OperatorTok{,}
  \KeywordTok{new} \FunctionTok{MIMEType}\NormalTok{(}\StringTok{\textquotesingle{}image/gif\textquotesingle{}}\NormalTok{)}\OperatorTok{,}
\NormalTok{]}\OperatorTok{;}
\BuiltInTok{console}\OperatorTok{.}\FunctionTok{log}\NormalTok{(}\BuiltInTok{JSON}\OperatorTok{.}\FunctionTok{stringify}\NormalTok{(myMIMES))}\OperatorTok{;}
\CommentTok{// Prints: ["image/png", "image/gif"]}
\end{Highlighting}
\end{Shaded}

\subsection{\texorpdfstring{Class:
\texttt{util.MIMEParams}}{Class: util.MIMEParams}}\label{class-util.mimeparams}

The \texttt{MIMEParams} API provides read and write access to the
parameters of a \texttt{MIMEType}.

\subsubsection{\texorpdfstring{Constructor:
\texttt{new\ MIMEParams()}}{Constructor: new MIMEParams()}}\label{constructor-new-mimeparams}

Creates a new \texttt{MIMEParams} object by with empty parameters

\begin{Shaded}
\begin{Highlighting}[]
\ImportTok{import}\NormalTok{ \{ MIMEParams \} }\ImportTok{from} \StringTok{\textquotesingle{}node:util\textquotesingle{}}\OperatorTok{;}

\KeywordTok{const}\NormalTok{ myParams }\OperatorTok{=} \KeywordTok{new} \FunctionTok{MIMEParams}\NormalTok{()}\OperatorTok{;}
\end{Highlighting}
\end{Shaded}

\begin{Shaded}
\begin{Highlighting}[]
\KeywordTok{const}\NormalTok{ \{ MIMEParams \} }\OperatorTok{=} \PreprocessorTok{require}\NormalTok{(}\StringTok{\textquotesingle{}node:util\textquotesingle{}}\NormalTok{)}\OperatorTok{;}

\KeywordTok{const}\NormalTok{ myParams }\OperatorTok{=} \KeywordTok{new} \FunctionTok{MIMEParams}\NormalTok{()}\OperatorTok{;}
\end{Highlighting}
\end{Shaded}

\subsubsection{\texorpdfstring{\texttt{mimeParams.delete(name)}}{mimeParams.delete(name)}}\label{mimeparams.deletename}

\begin{itemize}
\tightlist
\item
  \texttt{name} \{string\}
\end{itemize}

Remove all name-value pairs whose name is \texttt{name}.

\subsubsection{\texorpdfstring{\texttt{mimeParams.entries()}}{mimeParams.entries()}}\label{mimeparams.entries}

\begin{itemize}
\tightlist
\item
  Returns: \{Iterator\}
\end{itemize}

Returns an iterator over each of the name-value pairs in the parameters.
Each item of the iterator is a JavaScript \texttt{Array}. The first item
of the array is the \texttt{name}, the second item of the array is the
\texttt{value}.

\subsubsection{\texorpdfstring{\texttt{mimeParams.get(name)}}{mimeParams.get(name)}}\label{mimeparams.getname}

\begin{itemize}
\tightlist
\item
  \texttt{name} \{string\}
\item
  Returns: \{string\} or \texttt{null} if there is no name-value pair
  with the given \texttt{name}.
\end{itemize}

Returns the value of the first name-value pair whose name is
\texttt{name}. If there are no such pairs, \texttt{null} is returned.

\subsubsection{\texorpdfstring{\texttt{mimeParams.has(name)}}{mimeParams.has(name)}}\label{mimeparams.hasname}

\begin{itemize}
\tightlist
\item
  \texttt{name} \{string\}
\item
  Returns: \{boolean\}
\end{itemize}

Returns \texttt{true} if there is at least one name-value pair whose
name is \texttt{name}.

\subsubsection{\texorpdfstring{\texttt{mimeParams.keys()}}{mimeParams.keys()}}\label{mimeparams.keys}

\begin{itemize}
\tightlist
\item
  Returns: \{Iterator\}
\end{itemize}

Returns an iterator over the names of each name-value pair.

\begin{Shaded}
\begin{Highlighting}[]
\ImportTok{import}\NormalTok{ \{ MIMEType \} }\ImportTok{from} \StringTok{\textquotesingle{}node:util\textquotesingle{}}\OperatorTok{;}

\KeywordTok{const}\NormalTok{ \{ params \} }\OperatorTok{=} \KeywordTok{new} \FunctionTok{MIMEType}\NormalTok{(}\StringTok{\textquotesingle{}text/plain;foo=0;bar=1\textquotesingle{}}\NormalTok{)}\OperatorTok{;}
\ControlFlowTok{for}\NormalTok{ (}\KeywordTok{const}\NormalTok{ name }\KeywordTok{of}\NormalTok{ params}\OperatorTok{.}\FunctionTok{keys}\NormalTok{()) \{}
  \BuiltInTok{console}\OperatorTok{.}\FunctionTok{log}\NormalTok{(name)}\OperatorTok{;}
\NormalTok{\}}
\CommentTok{// Prints:}
\CommentTok{//   foo}
\CommentTok{//   bar}
\end{Highlighting}
\end{Shaded}

\begin{Shaded}
\begin{Highlighting}[]
\KeywordTok{const}\NormalTok{ \{ MIMEType \} }\OperatorTok{=} \PreprocessorTok{require}\NormalTok{(}\StringTok{\textquotesingle{}node:util\textquotesingle{}}\NormalTok{)}\OperatorTok{;}

\KeywordTok{const}\NormalTok{ \{ params \} }\OperatorTok{=} \KeywordTok{new} \FunctionTok{MIMEType}\NormalTok{(}\StringTok{\textquotesingle{}text/plain;foo=0;bar=1\textquotesingle{}}\NormalTok{)}\OperatorTok{;}
\ControlFlowTok{for}\NormalTok{ (}\KeywordTok{const}\NormalTok{ name }\KeywordTok{of}\NormalTok{ params}\OperatorTok{.}\FunctionTok{keys}\NormalTok{()) \{}
  \BuiltInTok{console}\OperatorTok{.}\FunctionTok{log}\NormalTok{(name)}\OperatorTok{;}
\NormalTok{\}}
\CommentTok{// Prints:}
\CommentTok{//   foo}
\CommentTok{//   bar}
\end{Highlighting}
\end{Shaded}

\subsubsection{\texorpdfstring{\texttt{mimeParams.set(name,\ value)}}{mimeParams.set(name, value)}}\label{mimeparams.setname-value}

\begin{itemize}
\tightlist
\item
  \texttt{name} \{string\}
\item
  \texttt{value} \{string\}
\end{itemize}

Sets the value in the \texttt{MIMEParams} object associated with
\texttt{name} to \texttt{value}. If there are any pre-existing
name-value pairs whose names are \texttt{name}, set the first such
pair's value to \texttt{value}.

\begin{Shaded}
\begin{Highlighting}[]
\ImportTok{import}\NormalTok{ \{ MIMEType \} }\ImportTok{from} \StringTok{\textquotesingle{}node:util\textquotesingle{}}\OperatorTok{;}

\KeywordTok{const}\NormalTok{ \{ params \} }\OperatorTok{=} \KeywordTok{new} \FunctionTok{MIMEType}\NormalTok{(}\StringTok{\textquotesingle{}text/plain;foo=0;bar=1\textquotesingle{}}\NormalTok{)}\OperatorTok{;}
\NormalTok{params}\OperatorTok{.}\FunctionTok{set}\NormalTok{(}\StringTok{\textquotesingle{}foo\textquotesingle{}}\OperatorTok{,} \StringTok{\textquotesingle{}def\textquotesingle{}}\NormalTok{)}\OperatorTok{;}
\NormalTok{params}\OperatorTok{.}\FunctionTok{set}\NormalTok{(}\StringTok{\textquotesingle{}baz\textquotesingle{}}\OperatorTok{,} \StringTok{\textquotesingle{}xyz\textquotesingle{}}\NormalTok{)}\OperatorTok{;}
\BuiltInTok{console}\OperatorTok{.}\FunctionTok{log}\NormalTok{(params}\OperatorTok{.}\FunctionTok{toString}\NormalTok{())}\OperatorTok{;}
\CommentTok{// Prints: foo=def;bar=1;baz=xyz}
\end{Highlighting}
\end{Shaded}

\begin{Shaded}
\begin{Highlighting}[]
\KeywordTok{const}\NormalTok{ \{ MIMEType \} }\OperatorTok{=} \PreprocessorTok{require}\NormalTok{(}\StringTok{\textquotesingle{}node:util\textquotesingle{}}\NormalTok{)}\OperatorTok{;}

\KeywordTok{const}\NormalTok{ \{ params \} }\OperatorTok{=} \KeywordTok{new} \FunctionTok{MIMEType}\NormalTok{(}\StringTok{\textquotesingle{}text/plain;foo=0;bar=1\textquotesingle{}}\NormalTok{)}\OperatorTok{;}
\NormalTok{params}\OperatorTok{.}\FunctionTok{set}\NormalTok{(}\StringTok{\textquotesingle{}foo\textquotesingle{}}\OperatorTok{,} \StringTok{\textquotesingle{}def\textquotesingle{}}\NormalTok{)}\OperatorTok{;}
\NormalTok{params}\OperatorTok{.}\FunctionTok{set}\NormalTok{(}\StringTok{\textquotesingle{}baz\textquotesingle{}}\OperatorTok{,} \StringTok{\textquotesingle{}xyz\textquotesingle{}}\NormalTok{)}\OperatorTok{;}
\BuiltInTok{console}\OperatorTok{.}\FunctionTok{log}\NormalTok{(params}\OperatorTok{.}\FunctionTok{toString}\NormalTok{())}\OperatorTok{;}
\CommentTok{// Prints: foo=def;bar=1;baz=xyz}
\end{Highlighting}
\end{Shaded}

\subsubsection{\texorpdfstring{\texttt{mimeParams.values()}}{mimeParams.values()}}\label{mimeparams.values}

\begin{itemize}
\tightlist
\item
  Returns: \{Iterator\}
\end{itemize}

Returns an iterator over the values of each name-value pair.

\subsubsection{\texorpdfstring{\texttt{mimeParams{[}@@iterator{]}()}}{mimeParams{[}@@iterator{]}()}}\label{mimeparamsiterator}

\begin{itemize}
\tightlist
\item
  Returns: \{Iterator\}
\end{itemize}

Alias for \hyperref[mimeparamsentries]{\texttt{mimeParams.entries()}}.

\begin{Shaded}
\begin{Highlighting}[]
\ImportTok{import}\NormalTok{ \{ MIMEType \} }\ImportTok{from} \StringTok{\textquotesingle{}node:util\textquotesingle{}}\OperatorTok{;}

\KeywordTok{const}\NormalTok{ \{ params \} }\OperatorTok{=} \KeywordTok{new} \FunctionTok{MIMEType}\NormalTok{(}\StringTok{\textquotesingle{}text/plain;foo=bar;xyz=baz\textquotesingle{}}\NormalTok{)}\OperatorTok{;}
\ControlFlowTok{for}\NormalTok{ (}\KeywordTok{const}\NormalTok{ [name}\OperatorTok{,}\NormalTok{ value] }\KeywordTok{of}\NormalTok{ params) \{}
  \BuiltInTok{console}\OperatorTok{.}\FunctionTok{log}\NormalTok{(name}\OperatorTok{,}\NormalTok{ value)}\OperatorTok{;}
\NormalTok{\}}
\CommentTok{// Prints:}
\CommentTok{//   foo bar}
\CommentTok{//   xyz baz}
\end{Highlighting}
\end{Shaded}

\begin{Shaded}
\begin{Highlighting}[]
\KeywordTok{const}\NormalTok{ \{ MIMEType \} }\OperatorTok{=} \PreprocessorTok{require}\NormalTok{(}\StringTok{\textquotesingle{}node:util\textquotesingle{}}\NormalTok{)}\OperatorTok{;}

\KeywordTok{const}\NormalTok{ \{ params \} }\OperatorTok{=} \KeywordTok{new} \FunctionTok{MIMEType}\NormalTok{(}\StringTok{\textquotesingle{}text/plain;foo=bar;xyz=baz\textquotesingle{}}\NormalTok{)}\OperatorTok{;}
\ControlFlowTok{for}\NormalTok{ (}\KeywordTok{const}\NormalTok{ [name}\OperatorTok{,}\NormalTok{ value] }\KeywordTok{of}\NormalTok{ params) \{}
  \BuiltInTok{console}\OperatorTok{.}\FunctionTok{log}\NormalTok{(name}\OperatorTok{,}\NormalTok{ value)}\OperatorTok{;}
\NormalTok{\}}
\CommentTok{// Prints:}
\CommentTok{//   foo bar}
\CommentTok{//   xyz baz}
\end{Highlighting}
\end{Shaded}

\subsection{\texorpdfstring{\texttt{util.parseArgs({[}config{]})}}{util.parseArgs({[}config{]})}}\label{util.parseargsconfig}

\begin{itemize}
\tightlist
\item
  \texttt{config} \{Object\} Used to provide arguments for parsing and
  to configure the parser. \texttt{config} supports the following
  properties:

  \begin{itemize}
  \tightlist
  \item
    \texttt{args} \{string{[}{]}\} array of argument strings.
    \textbf{Default:} \texttt{process.argv} with \texttt{execPath} and
    \texttt{filename} removed.
  \item
    \texttt{options} \{Object\} Used to describe arguments known to the
    parser. Keys of \texttt{options} are the long names of options and
    values are an \{Object\} accepting the following properties:

    \begin{itemize}
    \tightlist
    \item
      \texttt{type} \{string\} Type of argument, which must be either
      \texttt{boolean} or \texttt{string}.
    \item
      \texttt{multiple} \{boolean\} Whether this option can be provided
      multiple times. If \texttt{true}, all values will be collected in
      an array. If \texttt{false}, values for the option are last-wins.
      \textbf{Default:} \texttt{false}.
    \item
      \texttt{short} \{string\} A single character alias for the option.
    \item
      \texttt{default} \{string \textbar{} boolean \textbar{}
      string{[}{]} \textbar{} boolean{[}{]}\} The default option value
      when it is not set by args. It must be of the same type as the
      \texttt{type} property. When \texttt{multiple} is \texttt{true},
      it must be an array.
    \end{itemize}
  \item
    \texttt{strict} \{boolean\} Should an error be thrown when unknown
    arguments are encountered, or when arguments are passed that do not
    match the \texttt{type} configured in \texttt{options}.
    \textbf{Default:} \texttt{true}.
  \item
    \texttt{allowPositionals} \{boolean\} Whether this command accepts
    positional arguments. \textbf{Default:} \texttt{false} if
    \texttt{strict} is \texttt{true}, otherwise \texttt{true}.
  \item
    \texttt{tokens} \{boolean\} Return the parsed tokens. This is useful
    for extending the built-in behavior, from adding additional checks
    through to reprocessing the tokens in different ways.
    \textbf{Default:} \texttt{false}.
  \end{itemize}
\item
  Returns: \{Object\} The parsed command line arguments:

  \begin{itemize}
  \tightlist
  \item
    \texttt{values} \{Object\} A mapping of parsed option names with
    their \{string\} or \{boolean\} values.
  \item
    \texttt{positionals} \{string{[}{]}\} Positional arguments.
  \item
    \texttt{tokens} \{Object{[}{]} \textbar{} undefined\} See
    \hyperref[parseargs-tokens]{parseArgs tokens} section. Only returned
    if \texttt{config} includes \texttt{tokens:\ true}.
  \end{itemize}
\end{itemize}

Provides a higher level API for command-line argument parsing than
interacting with \texttt{process.argv} directly. Takes a specification
for the expected arguments and returns a structured object with the
parsed options and positionals.

\begin{Shaded}
\begin{Highlighting}[]
\ImportTok{import}\NormalTok{ \{ parseArgs \} }\ImportTok{from} \StringTok{\textquotesingle{}node:util\textquotesingle{}}\OperatorTok{;}
\KeywordTok{const}\NormalTok{ args }\OperatorTok{=}\NormalTok{ [}\StringTok{\textquotesingle{}{-}f\textquotesingle{}}\OperatorTok{,} \StringTok{\textquotesingle{}{-}{-}bar\textquotesingle{}}\OperatorTok{,} \StringTok{\textquotesingle{}b\textquotesingle{}}\NormalTok{]}\OperatorTok{;}
\KeywordTok{const}\NormalTok{ options }\OperatorTok{=}\NormalTok{ \{}
  \DataTypeTok{foo}\OperatorTok{:}\NormalTok{ \{}
    \DataTypeTok{type}\OperatorTok{:} \StringTok{\textquotesingle{}boolean\textquotesingle{}}\OperatorTok{,}
    \DataTypeTok{short}\OperatorTok{:} \StringTok{\textquotesingle{}f\textquotesingle{}}\OperatorTok{,}
\NormalTok{  \}}\OperatorTok{,}
  \DataTypeTok{bar}\OperatorTok{:}\NormalTok{ \{}
    \DataTypeTok{type}\OperatorTok{:} \StringTok{\textquotesingle{}string\textquotesingle{}}\OperatorTok{,}
\NormalTok{  \}}\OperatorTok{,}
\NormalTok{\}}\OperatorTok{;}
\KeywordTok{const}\NormalTok{ \{}
\NormalTok{  values}\OperatorTok{,}
\NormalTok{  positionals}\OperatorTok{,}
\NormalTok{\} }\OperatorTok{=} \FunctionTok{parseArgs}\NormalTok{(\{ args}\OperatorTok{,}\NormalTok{ options \})}\OperatorTok{;}
\BuiltInTok{console}\OperatorTok{.}\FunctionTok{log}\NormalTok{(values}\OperatorTok{,}\NormalTok{ positionals)}\OperatorTok{;}
\CommentTok{// Prints: [Object: null prototype] \{ foo: true, bar: \textquotesingle{}b\textquotesingle{} \} []}
\end{Highlighting}
\end{Shaded}

\begin{Shaded}
\begin{Highlighting}[]
\KeywordTok{const}\NormalTok{ \{ parseArgs \} }\OperatorTok{=} \PreprocessorTok{require}\NormalTok{(}\StringTok{\textquotesingle{}node:util\textquotesingle{}}\NormalTok{)}\OperatorTok{;}
\KeywordTok{const}\NormalTok{ args }\OperatorTok{=}\NormalTok{ [}\StringTok{\textquotesingle{}{-}f\textquotesingle{}}\OperatorTok{,} \StringTok{\textquotesingle{}{-}{-}bar\textquotesingle{}}\OperatorTok{,} \StringTok{\textquotesingle{}b\textquotesingle{}}\NormalTok{]}\OperatorTok{;}
\KeywordTok{const}\NormalTok{ options }\OperatorTok{=}\NormalTok{ \{}
  \DataTypeTok{foo}\OperatorTok{:}\NormalTok{ \{}
    \DataTypeTok{type}\OperatorTok{:} \StringTok{\textquotesingle{}boolean\textquotesingle{}}\OperatorTok{,}
    \DataTypeTok{short}\OperatorTok{:} \StringTok{\textquotesingle{}f\textquotesingle{}}\OperatorTok{,}
\NormalTok{  \}}\OperatorTok{,}
  \DataTypeTok{bar}\OperatorTok{:}\NormalTok{ \{}
    \DataTypeTok{type}\OperatorTok{:} \StringTok{\textquotesingle{}string\textquotesingle{}}\OperatorTok{,}
\NormalTok{  \}}\OperatorTok{,}
\NormalTok{\}}\OperatorTok{;}
\KeywordTok{const}\NormalTok{ \{}
\NormalTok{  values}\OperatorTok{,}
\NormalTok{  positionals}\OperatorTok{,}
\NormalTok{\} }\OperatorTok{=} \FunctionTok{parseArgs}\NormalTok{(\{ args}\OperatorTok{,}\NormalTok{ options \})}\OperatorTok{;}
\BuiltInTok{console}\OperatorTok{.}\FunctionTok{log}\NormalTok{(values}\OperatorTok{,}\NormalTok{ positionals)}\OperatorTok{;}
\CommentTok{// Prints: [Object: null prototype] \{ foo: true, bar: \textquotesingle{}b\textquotesingle{} \} []}
\end{Highlighting}
\end{Shaded}

\subsubsection{\texorpdfstring{\texttt{parseArgs}
\texttt{tokens}}{parseArgs tokens}}\label{parseargs-tokens}

Detailed parse information is available for adding custom behaviors by
specifying \texttt{tokens:\ true} in the configuration. The returned
tokens have properties describing:

\begin{itemize}
\tightlist
\item
  all tokens

  \begin{itemize}
  \tightlist
  \item
    \texttt{kind} \{string\} One of `option', `positional', or
    `option-terminator'.
  \item
    \texttt{index} \{number\} Index of element in \texttt{args}
    containing token. So the source argument for a token is
    \texttt{args{[}token.index{]}}.
  \end{itemize}
\item
  option tokens

  \begin{itemize}
  \tightlist
  \item
    \texttt{name} \{string\} Long name of option.
  \item
    \texttt{rawName} \{string\} How option used in args, like
    \texttt{-f} of \texttt{-\/-foo}.
  \item
    \texttt{value} \{string \textbar{} undefined\} Option value
    specified in args. Undefined for boolean options.
  \item
    \texttt{inlineValue} \{boolean \textbar{} undefined\} Whether option
    value specified inline, like \texttt{-\/-foo=bar}.
  \end{itemize}
\item
  positional tokens

  \begin{itemize}
  \tightlist
  \item
    \texttt{value} \{string\} The value of the positional argument in
    args (i.e.~\texttt{args{[}index{]}}).
  \end{itemize}
\item
  option-terminator token
\end{itemize}

The returned tokens are in the order encountered in the input args.
Options that appear more than once in args produce a token for each use.
Short option groups like \texttt{-xy} expand to a token for each option.
So \texttt{-xxx} produces three tokens.

For example to use the returned tokens to add support for a negated
option like \texttt{-\/-no-color}, the tokens can be reprocessed to
change the value stored for the negated option.

\begin{Shaded}
\begin{Highlighting}[]
\ImportTok{import}\NormalTok{ \{ parseArgs \} }\ImportTok{from} \StringTok{\textquotesingle{}node:util\textquotesingle{}}\OperatorTok{;}

\KeywordTok{const}\NormalTok{ options }\OperatorTok{=}\NormalTok{ \{}
  \StringTok{\textquotesingle{}color\textquotesingle{}}\OperatorTok{:}\NormalTok{ \{ }\DataTypeTok{type}\OperatorTok{:} \StringTok{\textquotesingle{}boolean\textquotesingle{}}\NormalTok{ \}}\OperatorTok{,}
  \StringTok{\textquotesingle{}no{-}color\textquotesingle{}}\OperatorTok{:}\NormalTok{ \{ }\DataTypeTok{type}\OperatorTok{:} \StringTok{\textquotesingle{}boolean\textquotesingle{}}\NormalTok{ \}}\OperatorTok{,}
  \StringTok{\textquotesingle{}logfile\textquotesingle{}}\OperatorTok{:}\NormalTok{ \{ }\DataTypeTok{type}\OperatorTok{:} \StringTok{\textquotesingle{}string\textquotesingle{}}\NormalTok{ \}}\OperatorTok{,}
  \StringTok{\textquotesingle{}no{-}logfile\textquotesingle{}}\OperatorTok{:}\NormalTok{ \{ }\DataTypeTok{type}\OperatorTok{:} \StringTok{\textquotesingle{}boolean\textquotesingle{}}\NormalTok{ \}}\OperatorTok{,}
\NormalTok{\}}\OperatorTok{;}
\KeywordTok{const}\NormalTok{ \{ values}\OperatorTok{,}\NormalTok{ tokens \} }\OperatorTok{=} \FunctionTok{parseArgs}\NormalTok{(\{ options}\OperatorTok{,} \DataTypeTok{tokens}\OperatorTok{:} \KeywordTok{true}\NormalTok{ \})}\OperatorTok{;}

\CommentTok{// Reprocess the option tokens and overwrite the returned values.}
\NormalTok{tokens}
  \OperatorTok{.}\FunctionTok{filter}\NormalTok{((token) }\KeywordTok{=\textgreater{}}\NormalTok{ token}\OperatorTok{.}\AttributeTok{kind} \OperatorTok{===} \StringTok{\textquotesingle{}option\textquotesingle{}}\NormalTok{)}
  \OperatorTok{.}\FunctionTok{forEach}\NormalTok{((token) }\KeywordTok{=\textgreater{}}\NormalTok{ \{}
    \ControlFlowTok{if}\NormalTok{ (token}\OperatorTok{.}\AttributeTok{name}\OperatorTok{.}\FunctionTok{startsWith}\NormalTok{(}\StringTok{\textquotesingle{}no{-}\textquotesingle{}}\NormalTok{)) \{}
      \CommentTok{// Store foo:false for {-}{-}no{-}foo}
      \KeywordTok{const}\NormalTok{ positiveName }\OperatorTok{=}\NormalTok{ token}\OperatorTok{.}\AttributeTok{name}\OperatorTok{.}\FunctionTok{slice}\NormalTok{(}\DecValTok{3}\NormalTok{)}\OperatorTok{;}
\NormalTok{      values[positiveName] }\OperatorTok{=} \KeywordTok{false}\OperatorTok{;}
      \KeywordTok{delete}\NormalTok{ values[token}\OperatorTok{.}\AttributeTok{name}\NormalTok{]}\OperatorTok{;}
\NormalTok{    \} }\ControlFlowTok{else}\NormalTok{ \{}
      \CommentTok{// Resave value so last one wins if both {-}{-}foo and {-}{-}no{-}foo.}
\NormalTok{      values[token}\OperatorTok{.}\AttributeTok{name}\NormalTok{] }\OperatorTok{=}\NormalTok{ token}\OperatorTok{.}\AttributeTok{value} \OperatorTok{??} \KeywordTok{true}\OperatorTok{;}
\NormalTok{    \}}
\NormalTok{  \})}\OperatorTok{;}

\KeywordTok{const}\NormalTok{ color }\OperatorTok{=}\NormalTok{ values}\OperatorTok{.}\AttributeTok{color}\OperatorTok{;}
\KeywordTok{const}\NormalTok{ logfile }\OperatorTok{=}\NormalTok{ values}\OperatorTok{.}\AttributeTok{logfile} \OperatorTok{??} \StringTok{\textquotesingle{}default.log\textquotesingle{}}\OperatorTok{;}

\BuiltInTok{console}\OperatorTok{.}\FunctionTok{log}\NormalTok{(\{ logfile}\OperatorTok{,}\NormalTok{ color \})}\OperatorTok{;}
\end{Highlighting}
\end{Shaded}

\begin{Shaded}
\begin{Highlighting}[]
\KeywordTok{const}\NormalTok{ \{ parseArgs \} }\OperatorTok{=} \PreprocessorTok{require}\NormalTok{(}\StringTok{\textquotesingle{}node:util\textquotesingle{}}\NormalTok{)}\OperatorTok{;}

\KeywordTok{const}\NormalTok{ options }\OperatorTok{=}\NormalTok{ \{}
  \StringTok{\textquotesingle{}color\textquotesingle{}}\OperatorTok{:}\NormalTok{ \{ }\DataTypeTok{type}\OperatorTok{:} \StringTok{\textquotesingle{}boolean\textquotesingle{}}\NormalTok{ \}}\OperatorTok{,}
  \StringTok{\textquotesingle{}no{-}color\textquotesingle{}}\OperatorTok{:}\NormalTok{ \{ }\DataTypeTok{type}\OperatorTok{:} \StringTok{\textquotesingle{}boolean\textquotesingle{}}\NormalTok{ \}}\OperatorTok{,}
  \StringTok{\textquotesingle{}logfile\textquotesingle{}}\OperatorTok{:}\NormalTok{ \{ }\DataTypeTok{type}\OperatorTok{:} \StringTok{\textquotesingle{}string\textquotesingle{}}\NormalTok{ \}}\OperatorTok{,}
  \StringTok{\textquotesingle{}no{-}logfile\textquotesingle{}}\OperatorTok{:}\NormalTok{ \{ }\DataTypeTok{type}\OperatorTok{:} \StringTok{\textquotesingle{}boolean\textquotesingle{}}\NormalTok{ \}}\OperatorTok{,}
\NormalTok{\}}\OperatorTok{;}
\KeywordTok{const}\NormalTok{ \{ values}\OperatorTok{,}\NormalTok{ tokens \} }\OperatorTok{=} \FunctionTok{parseArgs}\NormalTok{(\{ options}\OperatorTok{,} \DataTypeTok{tokens}\OperatorTok{:} \KeywordTok{true}\NormalTok{ \})}\OperatorTok{;}

\CommentTok{// Reprocess the option tokens and overwrite the returned values.}
\NormalTok{tokens}
  \OperatorTok{.}\FunctionTok{filter}\NormalTok{((token) }\KeywordTok{=\textgreater{}}\NormalTok{ token}\OperatorTok{.}\AttributeTok{kind} \OperatorTok{===} \StringTok{\textquotesingle{}option\textquotesingle{}}\NormalTok{)}
  \OperatorTok{.}\FunctionTok{forEach}\NormalTok{((token) }\KeywordTok{=\textgreater{}}\NormalTok{ \{}
    \ControlFlowTok{if}\NormalTok{ (token}\OperatorTok{.}\AttributeTok{name}\OperatorTok{.}\FunctionTok{startsWith}\NormalTok{(}\StringTok{\textquotesingle{}no{-}\textquotesingle{}}\NormalTok{)) \{}
      \CommentTok{// Store foo:false for {-}{-}no{-}foo}
      \KeywordTok{const}\NormalTok{ positiveName }\OperatorTok{=}\NormalTok{ token}\OperatorTok{.}\AttributeTok{name}\OperatorTok{.}\FunctionTok{slice}\NormalTok{(}\DecValTok{3}\NormalTok{)}\OperatorTok{;}
\NormalTok{      values[positiveName] }\OperatorTok{=} \KeywordTok{false}\OperatorTok{;}
      \KeywordTok{delete}\NormalTok{ values[token}\OperatorTok{.}\AttributeTok{name}\NormalTok{]}\OperatorTok{;}
\NormalTok{    \} }\ControlFlowTok{else}\NormalTok{ \{}
      \CommentTok{// Resave value so last one wins if both {-}{-}foo and {-}{-}no{-}foo.}
\NormalTok{      values[token}\OperatorTok{.}\AttributeTok{name}\NormalTok{] }\OperatorTok{=}\NormalTok{ token}\OperatorTok{.}\AttributeTok{value} \OperatorTok{??} \KeywordTok{true}\OperatorTok{;}
\NormalTok{    \}}
\NormalTok{  \})}\OperatorTok{;}

\KeywordTok{const}\NormalTok{ color }\OperatorTok{=}\NormalTok{ values}\OperatorTok{.}\AttributeTok{color}\OperatorTok{;}
\KeywordTok{const}\NormalTok{ logfile }\OperatorTok{=}\NormalTok{ values}\OperatorTok{.}\AttributeTok{logfile} \OperatorTok{??} \StringTok{\textquotesingle{}default.log\textquotesingle{}}\OperatorTok{;}

\BuiltInTok{console}\OperatorTok{.}\FunctionTok{log}\NormalTok{(\{ logfile}\OperatorTok{,}\NormalTok{ color \})}\OperatorTok{;}
\end{Highlighting}
\end{Shaded}

Example usage showing negated options, and when an option is used
multiple ways then last one wins.

\begin{Shaded}
\begin{Highlighting}[]
\NormalTok{$ node negate.js}
\NormalTok{\{ logfile: \textquotesingle{}default.log\textquotesingle{}, color: undefined \}}
\NormalTok{$ node negate.js {-}{-}no{-}logfile {-}{-}no{-}color}
\NormalTok{\{ logfile: false, color: false \}}
\NormalTok{$ node negate.js {-}{-}logfile=test.log {-}{-}color}
\NormalTok{\{ logfile: \textquotesingle{}test.log\textquotesingle{}, color: true \}}
\NormalTok{$ node negate.js {-}{-}no{-}logfile {-}{-}logfile=test.log {-}{-}color {-}{-}no{-}color}
\NormalTok{\{ logfile: \textquotesingle{}test.log\textquotesingle{}, color: false \}}
\end{Highlighting}
\end{Shaded}

\subsection{\texorpdfstring{\texttt{util.promisify(original)}}{util.promisify(original)}}\label{util.promisifyoriginal}

\begin{itemize}
\tightlist
\item
  \texttt{original} \{Function\}
\item
  Returns: \{Function\}
\end{itemize}

Takes a function following the common error-first callback style,
i.e.~taking an \texttt{(err,\ value)\ =\textgreater{}\ ...} callback as
the last argument, and returns a version that returns promises.

\begin{Shaded}
\begin{Highlighting}[]
\KeywordTok{const}\NormalTok{ util }\OperatorTok{=} \PreprocessorTok{require}\NormalTok{(}\StringTok{\textquotesingle{}node:util\textquotesingle{}}\NormalTok{)}\OperatorTok{;}
\KeywordTok{const}\NormalTok{ fs }\OperatorTok{=} \PreprocessorTok{require}\NormalTok{(}\StringTok{\textquotesingle{}node:fs\textquotesingle{}}\NormalTok{)}\OperatorTok{;}

\KeywordTok{const}\NormalTok{ stat }\OperatorTok{=}\NormalTok{ util}\OperatorTok{.}\FunctionTok{promisify}\NormalTok{(fs}\OperatorTok{.}\AttributeTok{stat}\NormalTok{)}\OperatorTok{;}
\FunctionTok{stat}\NormalTok{(}\StringTok{\textquotesingle{}.\textquotesingle{}}\NormalTok{)}\OperatorTok{.}\FunctionTok{then}\NormalTok{((stats) }\KeywordTok{=\textgreater{}}\NormalTok{ \{}
  \CommentTok{// Do something with \textasciigrave{}stats\textasciigrave{}}
\NormalTok{\})}\OperatorTok{.}\FunctionTok{catch}\NormalTok{((error) }\KeywordTok{=\textgreater{}}\NormalTok{ \{}
  \CommentTok{// Handle the error.}
\NormalTok{\})}\OperatorTok{;}
\end{Highlighting}
\end{Shaded}

Or, equivalently using \texttt{async\ function}s:

\begin{Shaded}
\begin{Highlighting}[]
\KeywordTok{const}\NormalTok{ util }\OperatorTok{=} \PreprocessorTok{require}\NormalTok{(}\StringTok{\textquotesingle{}node:util\textquotesingle{}}\NormalTok{)}\OperatorTok{;}
\KeywordTok{const}\NormalTok{ fs }\OperatorTok{=} \PreprocessorTok{require}\NormalTok{(}\StringTok{\textquotesingle{}node:fs\textquotesingle{}}\NormalTok{)}\OperatorTok{;}

\KeywordTok{const}\NormalTok{ stat }\OperatorTok{=}\NormalTok{ util}\OperatorTok{.}\FunctionTok{promisify}\NormalTok{(fs}\OperatorTok{.}\AttributeTok{stat}\NormalTok{)}\OperatorTok{;}

\KeywordTok{async} \KeywordTok{function} \FunctionTok{callStat}\NormalTok{() \{}
  \KeywordTok{const}\NormalTok{ stats }\OperatorTok{=} \ControlFlowTok{await} \FunctionTok{stat}\NormalTok{(}\StringTok{\textquotesingle{}.\textquotesingle{}}\NormalTok{)}\OperatorTok{;}
  \BuiltInTok{console}\OperatorTok{.}\FunctionTok{log}\NormalTok{(}\VerbatimStringTok{\textasciigrave{}This directory is owned by }\SpecialCharTok{$\{}\NormalTok{stats}\OperatorTok{.}\AttributeTok{uid}\SpecialCharTok{\}}\VerbatimStringTok{\textasciigrave{}}\NormalTok{)}\OperatorTok{;}
\NormalTok{\}}

\FunctionTok{callStat}\NormalTok{()}\OperatorTok{;}
\end{Highlighting}
\end{Shaded}

If there is an \texttt{original{[}util.promisify.custom{]}} property
present, \texttt{promisify} will return its value, see
\hyperref[custom-promisified-functions]{Custom promisified functions}.

\texttt{promisify()} assumes that \texttt{original} is a function taking
a callback as its final argument in all cases. If \texttt{original} is
not a function, \texttt{promisify()} will throw an error. If
\texttt{original} is a function but its last argument is not an
error-first callback, it will still be passed an error-first callback as
its last argument.

Using \texttt{promisify()} on class methods or other methods that use
\texttt{this} may not work as expected unless handled specially:

\begin{Shaded}
\begin{Highlighting}[]
\KeywordTok{const}\NormalTok{ util }\OperatorTok{=} \PreprocessorTok{require}\NormalTok{(}\StringTok{\textquotesingle{}node:util\textquotesingle{}}\NormalTok{)}\OperatorTok{;}

\KeywordTok{class}\NormalTok{ Foo \{}
  \FunctionTok{constructor}\NormalTok{() \{}
    \KeywordTok{this}\OperatorTok{.}\AttributeTok{a} \OperatorTok{=} \DecValTok{42}\OperatorTok{;}
\NormalTok{  \}}

  \FunctionTok{bar}\NormalTok{(callback) \{}
    \FunctionTok{callback}\NormalTok{(}\KeywordTok{null}\OperatorTok{,} \KeywordTok{this}\OperatorTok{.}\AttributeTok{a}\NormalTok{)}\OperatorTok{;}
\NormalTok{  \}}
\NormalTok{\}}

\KeywordTok{const}\NormalTok{ foo }\OperatorTok{=} \KeywordTok{new} \FunctionTok{Foo}\NormalTok{()}\OperatorTok{;}

\KeywordTok{const}\NormalTok{ naiveBar }\OperatorTok{=}\NormalTok{ util}\OperatorTok{.}\FunctionTok{promisify}\NormalTok{(foo}\OperatorTok{.}\AttributeTok{bar}\NormalTok{)}\OperatorTok{;}
\CommentTok{// TypeError: Cannot read property \textquotesingle{}a\textquotesingle{} of undefined}
\CommentTok{// naiveBar().then(a =\textgreater{} console.log(a));}

\NormalTok{naiveBar}\OperatorTok{.}\FunctionTok{call}\NormalTok{(foo)}\OperatorTok{.}\FunctionTok{then}\NormalTok{((a) }\KeywordTok{=\textgreater{}} \BuiltInTok{console}\OperatorTok{.}\FunctionTok{log}\NormalTok{(a))}\OperatorTok{;} \CommentTok{// \textquotesingle{}42\textquotesingle{}}

\KeywordTok{const}\NormalTok{ bindBar }\OperatorTok{=}\NormalTok{ naiveBar}\OperatorTok{.}\FunctionTok{bind}\NormalTok{(foo)}\OperatorTok{;}
\FunctionTok{bindBar}\NormalTok{()}\OperatorTok{.}\FunctionTok{then}\NormalTok{((a) }\KeywordTok{=\textgreater{}} \BuiltInTok{console}\OperatorTok{.}\FunctionTok{log}\NormalTok{(a))}\OperatorTok{;} \CommentTok{// \textquotesingle{}42\textquotesingle{}}
\end{Highlighting}
\end{Shaded}

\subsubsection{Custom promisified
functions}\label{custom-promisified-functions}

Using the \texttt{util.promisify.custom} symbol one can override the
return value of
\hyperref[utilpromisifyoriginal]{\texttt{util.promisify()}}:

\begin{Shaded}
\begin{Highlighting}[]
\KeywordTok{const}\NormalTok{ util }\OperatorTok{=} \PreprocessorTok{require}\NormalTok{(}\StringTok{\textquotesingle{}node:util\textquotesingle{}}\NormalTok{)}\OperatorTok{;}

\KeywordTok{function} \FunctionTok{doSomething}\NormalTok{(foo}\OperatorTok{,}\NormalTok{ callback) \{}
  \CommentTok{// ...}
\NormalTok{\}}

\NormalTok{doSomething[util}\OperatorTok{.}\AttributeTok{promisify}\OperatorTok{.}\AttributeTok{custom}\NormalTok{] }\OperatorTok{=}\NormalTok{ (foo) }\KeywordTok{=\textgreater{}}\NormalTok{ \{}
  \ControlFlowTok{return} \FunctionTok{getPromiseSomehow}\NormalTok{()}\OperatorTok{;}
\NormalTok{\}}\OperatorTok{;}

\KeywordTok{const}\NormalTok{ promisified }\OperatorTok{=}\NormalTok{ util}\OperatorTok{.}\FunctionTok{promisify}\NormalTok{(doSomething)}\OperatorTok{;}
\BuiltInTok{console}\OperatorTok{.}\FunctionTok{log}\NormalTok{(promisified }\OperatorTok{===}\NormalTok{ doSomething[util}\OperatorTok{.}\AttributeTok{promisify}\OperatorTok{.}\AttributeTok{custom}\NormalTok{])}\OperatorTok{;}
\CommentTok{// prints \textquotesingle{}true\textquotesingle{}}
\end{Highlighting}
\end{Shaded}

This can be useful for cases where the original function does not follow
the standard format of taking an error-first callback as the last
argument.

For example, with a function that takes in
\texttt{(foo,\ onSuccessCallback,\ onErrorCallback)}:

\begin{Shaded}
\begin{Highlighting}[]
\NormalTok{doSomething[util}\OperatorTok{.}\AttributeTok{promisify}\OperatorTok{.}\AttributeTok{custom}\NormalTok{] }\OperatorTok{=}\NormalTok{ (foo) }\KeywordTok{=\textgreater{}}\NormalTok{ \{}
  \ControlFlowTok{return} \KeywordTok{new} \BuiltInTok{Promise}\NormalTok{((resolve}\OperatorTok{,}\NormalTok{ reject) }\KeywordTok{=\textgreater{}}\NormalTok{ \{}
    \FunctionTok{doSomething}\NormalTok{(foo}\OperatorTok{,}\NormalTok{ resolve}\OperatorTok{,}\NormalTok{ reject)}\OperatorTok{;}
\NormalTok{  \})}\OperatorTok{;}
\NormalTok{\}}\OperatorTok{;}
\end{Highlighting}
\end{Shaded}

If \texttt{promisify.custom} is defined but is not a function,
\texttt{promisify()} will throw an error.

\subsubsection{\texorpdfstring{\texttt{util.promisify.custom}}{util.promisify.custom}}\label{util.promisify.custom}

\begin{itemize}
\tightlist
\item
  \{symbol\} that can be used to declare custom promisified variants of
  functions, see \hyperref[custom-promisified-functions]{Custom
  promisified functions}.
\end{itemize}

In addition to being accessible through \texttt{util.promisify.custom},
this symbol is
\href{https://developer.mozilla.org/en-US/docs/Web/JavaScript/Reference/Global_Objects/Symbol/for}{registered
globally} and can be accessed in any environment as
\texttt{Symbol.for(\textquotesingle{}nodejs.util.promisify.custom\textquotesingle{})}.

For example, with a function that takes in
\texttt{(foo,\ onSuccessCallback,\ onErrorCallback)}:

\begin{Shaded}
\begin{Highlighting}[]
\KeywordTok{const}\NormalTok{ kCustomPromisifiedSymbol }\OperatorTok{=} \BuiltInTok{Symbol}\OperatorTok{.}\FunctionTok{for}\NormalTok{(}\StringTok{\textquotesingle{}nodejs.util.promisify.custom\textquotesingle{}}\NormalTok{)}\OperatorTok{;}

\NormalTok{doSomething[kCustomPromisifiedSymbol] }\OperatorTok{=}\NormalTok{ (foo) }\KeywordTok{=\textgreater{}}\NormalTok{ \{}
  \ControlFlowTok{return} \KeywordTok{new} \BuiltInTok{Promise}\NormalTok{((resolve}\OperatorTok{,}\NormalTok{ reject) }\KeywordTok{=\textgreater{}}\NormalTok{ \{}
    \FunctionTok{doSomething}\NormalTok{(foo}\OperatorTok{,}\NormalTok{ resolve}\OperatorTok{,}\NormalTok{ reject)}\OperatorTok{;}
\NormalTok{  \})}\OperatorTok{;}
\NormalTok{\}}\OperatorTok{;}
\end{Highlighting}
\end{Shaded}

\subsection{\texorpdfstring{\texttt{util.stripVTControlCharacters(str)}}{util.stripVTControlCharacters(str)}}\label{util.stripvtcontrolcharactersstr}

\begin{itemize}
\tightlist
\item
  \texttt{str} \{string\}
\item
  Returns: \{string\}
\end{itemize}

Returns \texttt{str} with any ANSI escape codes removed.

\begin{Shaded}
\begin{Highlighting}[]
\BuiltInTok{console}\OperatorTok{.}\FunctionTok{log}\NormalTok{(util}\OperatorTok{.}\FunctionTok{stripVTControlCharacters}\NormalTok{(}\StringTok{\textquotesingle{}}\SpecialCharTok{\textbackslash{}u001B}\StringTok{[4mvalue}\SpecialCharTok{\textbackslash{}u001B}\StringTok{[0m\textquotesingle{}}\NormalTok{))}\OperatorTok{;}
\CommentTok{// Prints "value"}
\end{Highlighting}
\end{Shaded}

\subsection{\texorpdfstring{Class:
\texttt{util.TextDecoder}}{Class: util.TextDecoder}}\label{class-util.textdecoder}

An implementation of the \href{https://encoding.spec.whatwg.org/}{WHATWG
Encoding Standard} \texttt{TextDecoder} API.

\begin{Shaded}
\begin{Highlighting}[]
\KeywordTok{const}\NormalTok{ decoder }\OperatorTok{=} \KeywordTok{new} \FunctionTok{TextDecoder}\NormalTok{()}\OperatorTok{;}
\KeywordTok{const}\NormalTok{ u8arr }\OperatorTok{=} \KeywordTok{new} \BuiltInTok{Uint8Array}\NormalTok{([}\DecValTok{72}\OperatorTok{,} \DecValTok{101}\OperatorTok{,} \DecValTok{108}\OperatorTok{,} \DecValTok{108}\OperatorTok{,} \DecValTok{111}\NormalTok{])}\OperatorTok{;}
\BuiltInTok{console}\OperatorTok{.}\FunctionTok{log}\NormalTok{(decoder}\OperatorTok{.}\FunctionTok{decode}\NormalTok{(u8arr))}\OperatorTok{;} \CommentTok{// Hello}
\end{Highlighting}
\end{Shaded}

\subsubsection{WHATWG supported
encodings}\label{whatwg-supported-encodings}

Per the \href{https://encoding.spec.whatwg.org/}{WHATWG Encoding
Standard}, the encodings supported by the \texttt{TextDecoder} API are
outlined in the tables below. For each encoding, one or more aliases may
be used.

Different Node.js build configurations support different sets of
encodings. (see \href{intl.md}{Internationalization})

\paragraph{Encodings supported by default (with full ICU
data)}\label{encodings-supported-by-default-with-full-icu-data}

\begin{longtable}[]{@{}
  >{\raggedright\arraybackslash}p{(\columnwidth - 2\tabcolsep) * \real{0.0735}}
  >{\raggedright\arraybackslash}p{(\columnwidth - 2\tabcolsep) * \real{0.9265}}@{}}
\toprule\noalign{}
\begin{minipage}[b]{\linewidth}\raggedright
Encoding
\end{minipage} & \begin{minipage}[b]{\linewidth}\raggedright
Aliases
\end{minipage} \\
\midrule\noalign{}
\endhead
\bottomrule\noalign{}
\endlastfoot
\texttt{\textquotesingle{}ibm866\textquotesingle{}} &
\texttt{\textquotesingle{}866\textquotesingle{}},
\texttt{\textquotesingle{}cp866\textquotesingle{}},
\texttt{\textquotesingle{}csibm866\textquotesingle{}} \\
\texttt{\textquotesingle{}iso-8859-2\textquotesingle{}} &
\texttt{\textquotesingle{}csisolatin2\textquotesingle{}},
\texttt{\textquotesingle{}iso-ir-101\textquotesingle{}},
\texttt{\textquotesingle{}iso8859-2\textquotesingle{}},
\texttt{\textquotesingle{}iso88592\textquotesingle{}},
\texttt{\textquotesingle{}iso\_8859-2\textquotesingle{}},
\texttt{\textquotesingle{}iso\_8859-2:1987\textquotesingle{}},
\texttt{\textquotesingle{}l2\textquotesingle{}},
\texttt{\textquotesingle{}latin2\textquotesingle{}} \\
\texttt{\textquotesingle{}iso-8859-3\textquotesingle{}} &
\texttt{\textquotesingle{}csisolatin3\textquotesingle{}},
\texttt{\textquotesingle{}iso-ir-109\textquotesingle{}},
\texttt{\textquotesingle{}iso8859-3\textquotesingle{}},
\texttt{\textquotesingle{}iso88593\textquotesingle{}},
\texttt{\textquotesingle{}iso\_8859-3\textquotesingle{}},
\texttt{\textquotesingle{}iso\_8859-3:1988\textquotesingle{}},
\texttt{\textquotesingle{}l3\textquotesingle{}},
\texttt{\textquotesingle{}latin3\textquotesingle{}} \\
\texttt{\textquotesingle{}iso-8859-4\textquotesingle{}} &
\texttt{\textquotesingle{}csisolatin4\textquotesingle{}},
\texttt{\textquotesingle{}iso-ir-110\textquotesingle{}},
\texttt{\textquotesingle{}iso8859-4\textquotesingle{}},
\texttt{\textquotesingle{}iso88594\textquotesingle{}},
\texttt{\textquotesingle{}iso\_8859-4\textquotesingle{}},
\texttt{\textquotesingle{}iso\_8859-4:1988\textquotesingle{}},
\texttt{\textquotesingle{}l4\textquotesingle{}},
\texttt{\textquotesingle{}latin4\textquotesingle{}} \\
\texttt{\textquotesingle{}iso-8859-5\textquotesingle{}} &
\texttt{\textquotesingle{}csisolatincyrillic\textquotesingle{}},
\texttt{\textquotesingle{}cyrillic\textquotesingle{}},
\texttt{\textquotesingle{}iso-ir-144\textquotesingle{}},
\texttt{\textquotesingle{}iso8859-5\textquotesingle{}},
\texttt{\textquotesingle{}iso88595\textquotesingle{}},
\texttt{\textquotesingle{}iso\_8859-5\textquotesingle{}},
\texttt{\textquotesingle{}iso\_8859-5:1988\textquotesingle{}} \\
\texttt{\textquotesingle{}iso-8859-6\textquotesingle{}} &
\texttt{\textquotesingle{}arabic\textquotesingle{}},
\texttt{\textquotesingle{}asmo-708\textquotesingle{}},
\texttt{\textquotesingle{}csiso88596e\textquotesingle{}},
\texttt{\textquotesingle{}csiso88596i\textquotesingle{}},
\texttt{\textquotesingle{}csisolatinarabic\textquotesingle{}},
\texttt{\textquotesingle{}ecma-114\textquotesingle{}},
\texttt{\textquotesingle{}iso-8859-6-e\textquotesingle{}},
\texttt{\textquotesingle{}iso-8859-6-i\textquotesingle{}},
\texttt{\textquotesingle{}iso-ir-127\textquotesingle{}},
\texttt{\textquotesingle{}iso8859-6\textquotesingle{}},
\texttt{\textquotesingle{}iso88596\textquotesingle{}},
\texttt{\textquotesingle{}iso\_8859-6\textquotesingle{}},
\texttt{\textquotesingle{}iso\_8859-6:1987\textquotesingle{}} \\
\texttt{\textquotesingle{}iso-8859-7\textquotesingle{}} &
\texttt{\textquotesingle{}csisolatingreek\textquotesingle{}},
\texttt{\textquotesingle{}ecma-118\textquotesingle{}},
\texttt{\textquotesingle{}elot\_928\textquotesingle{}},
\texttt{\textquotesingle{}greek\textquotesingle{}},
\texttt{\textquotesingle{}greek8\textquotesingle{}},
\texttt{\textquotesingle{}iso-ir-126\textquotesingle{}},
\texttt{\textquotesingle{}iso8859-7\textquotesingle{}},
\texttt{\textquotesingle{}iso88597\textquotesingle{}},
\texttt{\textquotesingle{}iso\_8859-7\textquotesingle{}},
\texttt{\textquotesingle{}iso\_8859-7:1987\textquotesingle{}},
\texttt{\textquotesingle{}sun\_eu\_greek\textquotesingle{}} \\
\texttt{\textquotesingle{}iso-8859-8\textquotesingle{}} &
\texttt{\textquotesingle{}csiso88598e\textquotesingle{}},
\texttt{\textquotesingle{}csisolatinhebrew\textquotesingle{}},
\texttt{\textquotesingle{}hebrew\textquotesingle{}},
\texttt{\textquotesingle{}iso-8859-8-e\textquotesingle{}},
\texttt{\textquotesingle{}iso-ir-138\textquotesingle{}},
\texttt{\textquotesingle{}iso8859-8\textquotesingle{}},
\texttt{\textquotesingle{}iso88598\textquotesingle{}},
\texttt{\textquotesingle{}iso\_8859-8\textquotesingle{}},
\texttt{\textquotesingle{}iso\_8859-8:1988\textquotesingle{}},
\texttt{\textquotesingle{}visual\textquotesingle{}} \\
\texttt{\textquotesingle{}iso-8859-8-i\textquotesingle{}} &
\texttt{\textquotesingle{}csiso88598i\textquotesingle{}},
\texttt{\textquotesingle{}logical\textquotesingle{}} \\
\texttt{\textquotesingle{}iso-8859-10\textquotesingle{}} &
\texttt{\textquotesingle{}csisolatin6\textquotesingle{}},
\texttt{\textquotesingle{}iso-ir-157\textquotesingle{}},
\texttt{\textquotesingle{}iso8859-10\textquotesingle{}},
\texttt{\textquotesingle{}iso885910\textquotesingle{}},
\texttt{\textquotesingle{}l6\textquotesingle{}},
\texttt{\textquotesingle{}latin6\textquotesingle{}} \\
\texttt{\textquotesingle{}iso-8859-13\textquotesingle{}} &
\texttt{\textquotesingle{}iso8859-13\textquotesingle{}},
\texttt{\textquotesingle{}iso885913\textquotesingle{}} \\
\texttt{\textquotesingle{}iso-8859-14\textquotesingle{}} &
\texttt{\textquotesingle{}iso8859-14\textquotesingle{}},
\texttt{\textquotesingle{}iso885914\textquotesingle{}} \\
\texttt{\textquotesingle{}iso-8859-15\textquotesingle{}} &
\texttt{\textquotesingle{}csisolatin9\textquotesingle{}},
\texttt{\textquotesingle{}iso8859-15\textquotesingle{}},
\texttt{\textquotesingle{}iso885915\textquotesingle{}},
\texttt{\textquotesingle{}iso\_8859-15\textquotesingle{}},
\texttt{\textquotesingle{}l9\textquotesingle{}} \\
\texttt{\textquotesingle{}koi8-r\textquotesingle{}} &
\texttt{\textquotesingle{}cskoi8r\textquotesingle{}},
\texttt{\textquotesingle{}koi\textquotesingle{}},
\texttt{\textquotesingle{}koi8\textquotesingle{}},
\texttt{\textquotesingle{}koi8\_r\textquotesingle{}} \\
\texttt{\textquotesingle{}koi8-u\textquotesingle{}} &
\texttt{\textquotesingle{}koi8-ru\textquotesingle{}} \\
\texttt{\textquotesingle{}macintosh\textquotesingle{}} &
\texttt{\textquotesingle{}csmacintosh\textquotesingle{}},
\texttt{\textquotesingle{}mac\textquotesingle{}},
\texttt{\textquotesingle{}x-mac-roman\textquotesingle{}} \\
\texttt{\textquotesingle{}windows-874\textquotesingle{}} &
\texttt{\textquotesingle{}dos-874\textquotesingle{}},
\texttt{\textquotesingle{}iso-8859-11\textquotesingle{}},
\texttt{\textquotesingle{}iso8859-11\textquotesingle{}},
\texttt{\textquotesingle{}iso885911\textquotesingle{}},
\texttt{\textquotesingle{}tis-620\textquotesingle{}} \\
\texttt{\textquotesingle{}windows-1250\textquotesingle{}} &
\texttt{\textquotesingle{}cp1250\textquotesingle{}},
\texttt{\textquotesingle{}x-cp1250\textquotesingle{}} \\
\texttt{\textquotesingle{}windows-1251\textquotesingle{}} &
\texttt{\textquotesingle{}cp1251\textquotesingle{}},
\texttt{\textquotesingle{}x-cp1251\textquotesingle{}} \\
\texttt{\textquotesingle{}windows-1252\textquotesingle{}} &
\texttt{\textquotesingle{}ansi\_x3.4-1968\textquotesingle{}},
\texttt{\textquotesingle{}ascii\textquotesingle{}},
\texttt{\textquotesingle{}cp1252\textquotesingle{}},
\texttt{\textquotesingle{}cp819\textquotesingle{}},
\texttt{\textquotesingle{}csisolatin1\textquotesingle{}},
\texttt{\textquotesingle{}ibm819\textquotesingle{}},
\texttt{\textquotesingle{}iso-8859-1\textquotesingle{}},
\texttt{\textquotesingle{}iso-ir-100\textquotesingle{}},
\texttt{\textquotesingle{}iso8859-1\textquotesingle{}},
\texttt{\textquotesingle{}iso88591\textquotesingle{}},
\texttt{\textquotesingle{}iso\_8859-1\textquotesingle{}},
\texttt{\textquotesingle{}iso\_8859-1:1987\textquotesingle{}},
\texttt{\textquotesingle{}l1\textquotesingle{}},
\texttt{\textquotesingle{}latin1\textquotesingle{}},
\texttt{\textquotesingle{}us-ascii\textquotesingle{}},
\texttt{\textquotesingle{}x-cp1252\textquotesingle{}} \\
\texttt{\textquotesingle{}windows-1253\textquotesingle{}} &
\texttt{\textquotesingle{}cp1253\textquotesingle{}},
\texttt{\textquotesingle{}x-cp1253\textquotesingle{}} \\
\texttt{\textquotesingle{}windows-1254\textquotesingle{}} &
\texttt{\textquotesingle{}cp1254\textquotesingle{}},
\texttt{\textquotesingle{}csisolatin5\textquotesingle{}},
\texttt{\textquotesingle{}iso-8859-9\textquotesingle{}},
\texttt{\textquotesingle{}iso-ir-148\textquotesingle{}},
\texttt{\textquotesingle{}iso8859-9\textquotesingle{}},
\texttt{\textquotesingle{}iso88599\textquotesingle{}},
\texttt{\textquotesingle{}iso\_8859-9\textquotesingle{}},
\texttt{\textquotesingle{}iso\_8859-9:1989\textquotesingle{}},
\texttt{\textquotesingle{}l5\textquotesingle{}},
\texttt{\textquotesingle{}latin5\textquotesingle{}},
\texttt{\textquotesingle{}x-cp1254\textquotesingle{}} \\
\texttt{\textquotesingle{}windows-1255\textquotesingle{}} &
\texttt{\textquotesingle{}cp1255\textquotesingle{}},
\texttt{\textquotesingle{}x-cp1255\textquotesingle{}} \\
\texttt{\textquotesingle{}windows-1256\textquotesingle{}} &
\texttt{\textquotesingle{}cp1256\textquotesingle{}},
\texttt{\textquotesingle{}x-cp1256\textquotesingle{}} \\
\texttt{\textquotesingle{}windows-1257\textquotesingle{}} &
\texttt{\textquotesingle{}cp1257\textquotesingle{}},
\texttt{\textquotesingle{}x-cp1257\textquotesingle{}} \\
\texttt{\textquotesingle{}windows-1258\textquotesingle{}} &
\texttt{\textquotesingle{}cp1258\textquotesingle{}},
\texttt{\textquotesingle{}x-cp1258\textquotesingle{}} \\
\texttt{\textquotesingle{}x-mac-cyrillic\textquotesingle{}} &
\texttt{\textquotesingle{}x-mac-ukrainian\textquotesingle{}} \\
\texttt{\textquotesingle{}gbk\textquotesingle{}} &
\texttt{\textquotesingle{}chinese\textquotesingle{}},
\texttt{\textquotesingle{}csgb2312\textquotesingle{}},
\texttt{\textquotesingle{}csiso58gb231280\textquotesingle{}},
\texttt{\textquotesingle{}gb2312\textquotesingle{}},
\texttt{\textquotesingle{}gb\_2312\textquotesingle{}},
\texttt{\textquotesingle{}gb\_2312-80\textquotesingle{}},
\texttt{\textquotesingle{}iso-ir-58\textquotesingle{}},
\texttt{\textquotesingle{}x-gbk\textquotesingle{}} \\
\texttt{\textquotesingle{}gb18030\textquotesingle{}} & \\
\texttt{\textquotesingle{}big5\textquotesingle{}} &
\texttt{\textquotesingle{}big5-hkscs\textquotesingle{}},
\texttt{\textquotesingle{}cn-big5\textquotesingle{}},
\texttt{\textquotesingle{}csbig5\textquotesingle{}},
\texttt{\textquotesingle{}x-x-big5\textquotesingle{}} \\
\texttt{\textquotesingle{}euc-jp\textquotesingle{}} &
\texttt{\textquotesingle{}cseucpkdfmtjapanese\textquotesingle{}},
\texttt{\textquotesingle{}x-euc-jp\textquotesingle{}} \\
\texttt{\textquotesingle{}iso-2022-jp\textquotesingle{}} &
\texttt{\textquotesingle{}csiso2022jp\textquotesingle{}} \\
\texttt{\textquotesingle{}shift\_jis\textquotesingle{}} &
\texttt{\textquotesingle{}csshiftjis\textquotesingle{}},
\texttt{\textquotesingle{}ms932\textquotesingle{}},
\texttt{\textquotesingle{}ms\_kanji\textquotesingle{}},
\texttt{\textquotesingle{}shift-jis\textquotesingle{}},
\texttt{\textquotesingle{}sjis\textquotesingle{}},
\texttt{\textquotesingle{}windows-31j\textquotesingle{}},
\texttt{\textquotesingle{}x-sjis\textquotesingle{}} \\
\texttt{\textquotesingle{}euc-kr\textquotesingle{}} &
\texttt{\textquotesingle{}cseuckr\textquotesingle{}},
\texttt{\textquotesingle{}csksc56011987\textquotesingle{}},
\texttt{\textquotesingle{}iso-ir-149\textquotesingle{}},
\texttt{\textquotesingle{}korean\textquotesingle{}},
\texttt{\textquotesingle{}ks\_c\_5601-1987\textquotesingle{}},
\texttt{\textquotesingle{}ks\_c\_5601-1989\textquotesingle{}},
\texttt{\textquotesingle{}ksc5601\textquotesingle{}},
\texttt{\textquotesingle{}ksc\_5601\textquotesingle{}},
\texttt{\textquotesingle{}windows-949\textquotesingle{}} \\
\end{longtable}

\paragraph{\texorpdfstring{Encodings supported when Node.js is built
with the \texttt{small-icu}
option}{Encodings supported when Node.js is built with the small-icu option}}\label{encodings-supported-when-node.js-is-built-with-the-small-icu-option}

\begin{longtable}[]{@{}ll@{}}
\toprule\noalign{}
Encoding & Aliases \\
\midrule\noalign{}
\endhead
\bottomrule\noalign{}
\endlastfoot
\texttt{\textquotesingle{}utf-8\textquotesingle{}} &
\texttt{\textquotesingle{}unicode-1-1-utf-8\textquotesingle{}},
\texttt{\textquotesingle{}utf8\textquotesingle{}} \\
\texttt{\textquotesingle{}utf-16le\textquotesingle{}} &
\texttt{\textquotesingle{}utf-16\textquotesingle{}} \\
\texttt{\textquotesingle{}utf-16be\textquotesingle{}} & \\
\end{longtable}

\paragraph{Encodings supported when ICU is
disabled}\label{encodings-supported-when-icu-is-disabled}

\begin{longtable}[]{@{}ll@{}}
\toprule\noalign{}
Encoding & Aliases \\
\midrule\noalign{}
\endhead
\bottomrule\noalign{}
\endlastfoot
\texttt{\textquotesingle{}utf-8\textquotesingle{}} &
\texttt{\textquotesingle{}unicode-1-1-utf-8\textquotesingle{}},
\texttt{\textquotesingle{}utf8\textquotesingle{}} \\
\texttt{\textquotesingle{}utf-16le\textquotesingle{}} &
\texttt{\textquotesingle{}utf-16\textquotesingle{}} \\
\end{longtable}

The \texttt{\textquotesingle{}iso-8859-16\textquotesingle{}} encoding
listed in the \href{https://encoding.spec.whatwg.org/}{WHATWG Encoding
Standard} is not supported.

\subsubsection{\texorpdfstring{\texttt{new\ TextDecoder({[}encoding{[},\ options{]}{]})}}{new TextDecoder({[}encoding{[}, options{]}{]})}}\label{new-textdecoderencoding-options}

\begin{itemize}
\tightlist
\item
  \texttt{encoding} \{string\} Identifies the \texttt{encoding} that
  this \texttt{TextDecoder} instance supports. \textbf{Default:}
  \texttt{\textquotesingle{}utf-8\textquotesingle{}}.
\item
  \texttt{options} \{Object\}

  \begin{itemize}
  \tightlist
  \item
    \texttt{fatal} \{boolean\} \texttt{true} if decoding failures are
    fatal. This option is not supported when ICU is disabled (see
    \href{intl.md}{Internationalization}). \textbf{Default:}
    \texttt{false}.
  \item
    \texttt{ignoreBOM} \{boolean\} When \texttt{true}, the
    \texttt{TextDecoder} will include the byte order mark in the decoded
    result. When \texttt{false}, the byte order mark will be removed
    from the output. This option is only used when \texttt{encoding} is
    \texttt{\textquotesingle{}utf-8\textquotesingle{}},
    \texttt{\textquotesingle{}utf-16be\textquotesingle{}}, or
    \texttt{\textquotesingle{}utf-16le\textquotesingle{}}.
    \textbf{Default:} \texttt{false}.
  \end{itemize}
\end{itemize}

Creates a new \texttt{TextDecoder} instance. The \texttt{encoding} may
specify one of the supported encodings or an alias.

The \texttt{TextDecoder} class is also available on the global object.

\subsubsection{\texorpdfstring{\texttt{textDecoder.decode({[}input{[},\ options{]}{]})}}{textDecoder.decode({[}input{[}, options{]}{]})}}\label{textdecoder.decodeinput-options}

\begin{itemize}
\tightlist
\item
  \texttt{input} \{ArrayBuffer\textbar DataView\textbar TypedArray\} An
  \texttt{ArrayBuffer}, \texttt{DataView}, or \texttt{TypedArray}
  instance containing the encoded data.
\item
  \texttt{options} \{Object\}

  \begin{itemize}
  \tightlist
  \item
    \texttt{stream} \{boolean\} \texttt{true} if additional chunks of
    data are expected. \textbf{Default:} \texttt{false}.
  \end{itemize}
\item
  Returns: \{string\}
\end{itemize}

Decodes the \texttt{input} and returns a string. If
\texttt{options.stream} is \texttt{true}, any incomplete byte sequences
occurring at the end of the \texttt{input} are buffered internally and
emitted after the next call to \texttt{textDecoder.decode()}.

If \texttt{textDecoder.fatal} is \texttt{true}, decoding errors that
occur will result in a \texttt{TypeError} being thrown.

\subsubsection{\texorpdfstring{\texttt{textDecoder.encoding}}{textDecoder.encoding}}\label{textdecoder.encoding}

\begin{itemize}
\tightlist
\item
  \{string\}
\end{itemize}

The encoding supported by the \texttt{TextDecoder} instance.

\subsubsection{\texorpdfstring{\texttt{textDecoder.fatal}}{textDecoder.fatal}}\label{textdecoder.fatal}

\begin{itemize}
\tightlist
\item
  \{boolean\}
\end{itemize}

The value will be \texttt{true} if decoding errors result in a
\texttt{TypeError} being thrown.

\subsubsection{\texorpdfstring{\texttt{textDecoder.ignoreBOM}}{textDecoder.ignoreBOM}}\label{textdecoder.ignorebom}

\begin{itemize}
\tightlist
\item
  \{boolean\}
\end{itemize}

The value will be \texttt{true} if the decoding result will include the
byte order mark.

\subsection{\texorpdfstring{Class:
\texttt{util.TextEncoder}}{Class: util.TextEncoder}}\label{class-util.textencoder}

An implementation of the \href{https://encoding.spec.whatwg.org/}{WHATWG
Encoding Standard} \texttt{TextEncoder} API. All instances of
\texttt{TextEncoder} only support UTF-8 encoding.

\begin{Shaded}
\begin{Highlighting}[]
\KeywordTok{const}\NormalTok{ encoder }\OperatorTok{=} \KeywordTok{new} \FunctionTok{TextEncoder}\NormalTok{()}\OperatorTok{;}
\KeywordTok{const}\NormalTok{ uint8array }\OperatorTok{=}\NormalTok{ encoder}\OperatorTok{.}\FunctionTok{encode}\NormalTok{(}\StringTok{\textquotesingle{}this is some data\textquotesingle{}}\NormalTok{)}\OperatorTok{;}
\end{Highlighting}
\end{Shaded}

The \texttt{TextEncoder} class is also available on the global object.

\subsubsection{\texorpdfstring{\texttt{textEncoder.encode({[}input{]})}}{textEncoder.encode({[}input{]})}}\label{textencoder.encodeinput}

\begin{itemize}
\tightlist
\item
  \texttt{input} \{string\} The text to encode. \textbf{Default:} an
  empty string.
\item
  Returns: \{Uint8Array\}
\end{itemize}

UTF-8 encodes the \texttt{input} string and returns a
\texttt{Uint8Array} containing the encoded bytes.

\subsubsection{\texorpdfstring{\texttt{textEncoder.encodeInto(src,\ dest)}}{textEncoder.encodeInto(src, dest)}}\label{textencoder.encodeintosrc-dest}

\begin{itemize}
\tightlist
\item
  \texttt{src} \{string\} The text to encode.
\item
  \texttt{dest} \{Uint8Array\} The array to hold the encode result.
\item
  Returns: \{Object\}

  \begin{itemize}
  \tightlist
  \item
    \texttt{read} \{number\} The read Unicode code units of src.
  \item
    \texttt{written} \{number\} The written UTF-8 bytes of dest.
  \end{itemize}
\end{itemize}

UTF-8 encodes the \texttt{src} string to the \texttt{dest} Uint8Array
and returns an object containing the read Unicode code units and written
UTF-8 bytes.

\begin{Shaded}
\begin{Highlighting}[]
\KeywordTok{const}\NormalTok{ encoder }\OperatorTok{=} \KeywordTok{new} \FunctionTok{TextEncoder}\NormalTok{()}\OperatorTok{;}
\KeywordTok{const}\NormalTok{ src }\OperatorTok{=} \StringTok{\textquotesingle{}this is some data\textquotesingle{}}\OperatorTok{;}
\KeywordTok{const}\NormalTok{ dest }\OperatorTok{=} \KeywordTok{new} \BuiltInTok{Uint8Array}\NormalTok{(}\DecValTok{10}\NormalTok{)}\OperatorTok{;}
\KeywordTok{const}\NormalTok{ \{ read}\OperatorTok{,}\NormalTok{ written \} }\OperatorTok{=}\NormalTok{ encoder}\OperatorTok{.}\FunctionTok{encodeInto}\NormalTok{(src}\OperatorTok{,}\NormalTok{ dest)}\OperatorTok{;}
\end{Highlighting}
\end{Shaded}

\subsubsection{\texorpdfstring{\texttt{textEncoder.encoding}}{textEncoder.encoding}}\label{textencoder.encoding}

\begin{itemize}
\tightlist
\item
  \{string\}
\end{itemize}

The encoding supported by the \texttt{TextEncoder} instance. Always set
to \texttt{\textquotesingle{}utf-8\textquotesingle{}}.

\subsection{\texorpdfstring{\texttt{util.toUSVString(string)}}{util.toUSVString(string)}}\label{util.tousvstringstring}

\begin{itemize}
\tightlist
\item
  \texttt{string} \{string\}
\end{itemize}

Returns the \texttt{string} after replacing any surrogate code points
(or equivalently, any unpaired surrogate code units) with the Unicode
``replacement character'' U+FFFD.

\subsection{\texorpdfstring{\texttt{util.transferableAbortController()}}{util.transferableAbortController()}}\label{util.transferableabortcontroller}

\begin{quote}
Stability: 1 - Experimental
\end{quote}

Creates and returns an \{AbortController\} instance whose
\{AbortSignal\} is marked as transferable and can be used with
\texttt{structuredClone()} or \texttt{postMessage()}.

\subsection{\texorpdfstring{\texttt{util.transferableAbortSignal(signal)}}{util.transferableAbortSignal(signal)}}\label{util.transferableabortsignalsignal}

\begin{quote}
Stability: 1 - Experimental
\end{quote}

\begin{itemize}
\tightlist
\item
  \texttt{signal} \{AbortSignal\}
\item
  Returns: \{AbortSignal\}
\end{itemize}

Marks the given \{AbortSignal\} as transferable so that it can be used
with \texttt{structuredClone()} and \texttt{postMessage()}.

\begin{Shaded}
\begin{Highlighting}[]
\KeywordTok{const}\NormalTok{ signal }\OperatorTok{=} \FunctionTok{transferableAbortSignal}\NormalTok{(AbortSignal}\OperatorTok{.}\FunctionTok{timeout}\NormalTok{(}\DecValTok{100}\NormalTok{))}\OperatorTok{;}
\KeywordTok{const}\NormalTok{ channel }\OperatorTok{=} \KeywordTok{new} \BuiltInTok{MessageChannel}\NormalTok{()}\OperatorTok{;}
\NormalTok{channel}\OperatorTok{.}\AttributeTok{port2}\OperatorTok{.}\FunctionTok{postMessage}\NormalTok{(signal}\OperatorTok{,}\NormalTok{ [signal])}\OperatorTok{;}
\end{Highlighting}
\end{Shaded}

\subsection{\texorpdfstring{\texttt{util.aborted(signal,\ resource)}}{util.aborted(signal, resource)}}\label{util.abortedsignal-resource}

\begin{quote}
Stability: 1 - Experimental
\end{quote}

\begin{itemize}
\tightlist
\item
  \texttt{signal} \{AbortSignal\}
\item
  \texttt{resource} \{Object\} Any non-null entity, reference to which
  is held weakly.
\item
  Returns: \{Promise\}
\end{itemize}

Listens to abort event on the provided \texttt{signal} and returns a
promise that is fulfilled when the \texttt{signal} is aborted. If the
passed \texttt{resource} is garbage collected before the \texttt{signal}
is aborted, the returned promise shall remain pending indefinitely.

\begin{Shaded}
\begin{Highlighting}[]
\KeywordTok{const}\NormalTok{ \{ aborted \} }\OperatorTok{=} \PreprocessorTok{require}\NormalTok{(}\StringTok{\textquotesingle{}node:util\textquotesingle{}}\NormalTok{)}\OperatorTok{;}

\KeywordTok{const}\NormalTok{ dependent }\OperatorTok{=} \FunctionTok{obtainSomethingAbortable}\NormalTok{()}\OperatorTok{;}

\FunctionTok{aborted}\NormalTok{(dependent}\OperatorTok{.}\AttributeTok{signal}\OperatorTok{,}\NormalTok{ dependent)}\OperatorTok{.}\FunctionTok{then}\NormalTok{(() }\KeywordTok{=\textgreater{}}\NormalTok{ \{}
  \CommentTok{// Do something when dependent is aborted.}
\NormalTok{\})}\OperatorTok{;}

\NormalTok{dependent}\OperatorTok{.}\FunctionTok{on}\NormalTok{(}\StringTok{\textquotesingle{}event\textquotesingle{}}\OperatorTok{,}\NormalTok{ () }\KeywordTok{=\textgreater{}}\NormalTok{ \{}
\NormalTok{  dependent}\OperatorTok{.}\FunctionTok{abort}\NormalTok{()}\OperatorTok{;}
\NormalTok{\})}\OperatorTok{;}
\end{Highlighting}
\end{Shaded}

\begin{Shaded}
\begin{Highlighting}[]
\ImportTok{import}\NormalTok{ \{ aborted \} }\ImportTok{from} \StringTok{\textquotesingle{}node:util\textquotesingle{}}\OperatorTok{;}

\KeywordTok{const}\NormalTok{ dependent }\OperatorTok{=} \FunctionTok{obtainSomethingAbortable}\NormalTok{()}\OperatorTok{;}

\FunctionTok{aborted}\NormalTok{(dependent}\OperatorTok{.}\AttributeTok{signal}\OperatorTok{,}\NormalTok{ dependent)}\OperatorTok{.}\FunctionTok{then}\NormalTok{(() }\KeywordTok{=\textgreater{}}\NormalTok{ \{}
  \CommentTok{// Do something when dependent is aborted.}
\NormalTok{\})}\OperatorTok{;}

\NormalTok{dependent}\OperatorTok{.}\FunctionTok{on}\NormalTok{(}\StringTok{\textquotesingle{}event\textquotesingle{}}\OperatorTok{,}\NormalTok{ () }\KeywordTok{=\textgreater{}}\NormalTok{ \{}
\NormalTok{  dependent}\OperatorTok{.}\FunctionTok{abort}\NormalTok{()}\OperatorTok{;}
\NormalTok{\})}\OperatorTok{;}
\end{Highlighting}
\end{Shaded}

\subsection{\texorpdfstring{\texttt{util.types}}{util.types}}\label{util.types}

\texttt{util.types} provides type checks for different kinds of built-in
objects. Unlike \texttt{instanceof} or
\texttt{Object.prototype.toString.call(value)}, these checks do not
inspect properties of the object that are accessible from JavaScript
(like their prototype), and usually have the overhead of calling into
C++.

The result generally does not make any guarantees about what kinds of
properties or behavior a value exposes in JavaScript. They are primarily
useful for addon developers who prefer to do type checking in
JavaScript.

The API is accessible via
\texttt{require(\textquotesingle{}node:util\textquotesingle{}).types} or
\texttt{require(\textquotesingle{}node:util/types\textquotesingle{})}.

\subsubsection{\texorpdfstring{\texttt{util.types.isAnyArrayBuffer(value)}}{util.types.isAnyArrayBuffer(value)}}\label{util.types.isanyarraybuffervalue}

\begin{itemize}
\tightlist
\item
  \texttt{value} \{any\}
\item
  Returns: \{boolean\}
\end{itemize}

Returns \texttt{true} if the value is a built-in
\href{https://developer.mozilla.org/en-US/docs/Web/JavaScript/Reference/Global_Objects/ArrayBuffer}{\texttt{ArrayBuffer}}
or
\href{https://developer.mozilla.org/en-US/docs/Web/JavaScript/Reference/Global_Objects/SharedArrayBuffer}{\texttt{SharedArrayBuffer}}
instance.

See also
\hyperref[utiltypesisarraybuffervalue]{\texttt{util.types.isArrayBuffer()}}
and
\hyperref[utiltypesissharedarraybuffervalue]{\texttt{util.types.isSharedArrayBuffer()}}.

\begin{Shaded}
\begin{Highlighting}[]
\NormalTok{util}\OperatorTok{.}\AttributeTok{types}\OperatorTok{.}\FunctionTok{isAnyArrayBuffer}\NormalTok{(}\KeywordTok{new} \BuiltInTok{ArrayBuffer}\NormalTok{())}\OperatorTok{;}  \CommentTok{// Returns true}
\NormalTok{util}\OperatorTok{.}\AttributeTok{types}\OperatorTok{.}\FunctionTok{isAnyArrayBuffer}\NormalTok{(}\KeywordTok{new} \BuiltInTok{SharedArrayBuffer}\NormalTok{())}\OperatorTok{;}  \CommentTok{// Returns true}
\end{Highlighting}
\end{Shaded}

\subsubsection{\texorpdfstring{\texttt{util.types.isArrayBufferView(value)}}{util.types.isArrayBufferView(value)}}\label{util.types.isarraybufferviewvalue}

\begin{itemize}
\tightlist
\item
  \texttt{value} \{any\}
\item
  Returns: \{boolean\}
\end{itemize}

Returns \texttt{true} if the value is an instance of one of the
\href{https://developer.mozilla.org/en-US/docs/Web/JavaScript/Reference/Global_Objects/ArrayBuffer}{\texttt{ArrayBuffer}}
views, such as typed array objects or
\href{https://developer.mozilla.org/en-US/docs/Web/JavaScript/Reference/Global_Objects/DataView}{\texttt{DataView}}.
Equivalent to
\href{https://developer.mozilla.org/en-US/docs/Web/JavaScript/Reference/Global_Objects/ArrayBuffer/isView}{\texttt{ArrayBuffer.isView()}}.

\begin{Shaded}
\begin{Highlighting}[]
\NormalTok{util}\OperatorTok{.}\AttributeTok{types}\OperatorTok{.}\FunctionTok{isArrayBufferView}\NormalTok{(}\KeywordTok{new} \BuiltInTok{Int8Array}\NormalTok{())}\OperatorTok{;}  \CommentTok{// true}
\NormalTok{util}\OperatorTok{.}\AttributeTok{types}\OperatorTok{.}\FunctionTok{isArrayBufferView}\NormalTok{(}\BuiltInTok{Buffer}\OperatorTok{.}\FunctionTok{from}\NormalTok{(}\StringTok{\textquotesingle{}hello world\textquotesingle{}}\NormalTok{))}\OperatorTok{;} \CommentTok{// true}
\NormalTok{util}\OperatorTok{.}\AttributeTok{types}\OperatorTok{.}\FunctionTok{isArrayBufferView}\NormalTok{(}\KeywordTok{new} \BuiltInTok{DataView}\NormalTok{(}\KeywordTok{new} \BuiltInTok{ArrayBuffer}\NormalTok{(}\DecValTok{16}\NormalTok{)))}\OperatorTok{;}  \CommentTok{// true}
\NormalTok{util}\OperatorTok{.}\AttributeTok{types}\OperatorTok{.}\FunctionTok{isArrayBufferView}\NormalTok{(}\KeywordTok{new} \BuiltInTok{ArrayBuffer}\NormalTok{())}\OperatorTok{;}  \CommentTok{// false}
\end{Highlighting}
\end{Shaded}

\subsubsection{\texorpdfstring{\texttt{util.types.isArgumentsObject(value)}}{util.types.isArgumentsObject(value)}}\label{util.types.isargumentsobjectvalue}

\begin{itemize}
\tightlist
\item
  \texttt{value} \{any\}
\item
  Returns: \{boolean\}
\end{itemize}

Returns \texttt{true} if the value is an \texttt{arguments} object.

\begin{Shaded}
\begin{Highlighting}[]
\KeywordTok{function} \FunctionTok{foo}\NormalTok{() \{}
\NormalTok{  util}\OperatorTok{.}\AttributeTok{types}\OperatorTok{.}\FunctionTok{isArgumentsObject}\NormalTok{(}\KeywordTok{arguments}\NormalTok{)}\OperatorTok{;}  \CommentTok{// Returns true}
\NormalTok{\}}
\end{Highlighting}
\end{Shaded}

\subsubsection{\texorpdfstring{\texttt{util.types.isArrayBuffer(value)}}{util.types.isArrayBuffer(value)}}\label{util.types.isarraybuffervalue}

\begin{itemize}
\tightlist
\item
  \texttt{value} \{any\}
\item
  Returns: \{boolean\}
\end{itemize}

Returns \texttt{true} if the value is a built-in
\href{https://developer.mozilla.org/en-US/docs/Web/JavaScript/Reference/Global_Objects/ArrayBuffer}{\texttt{ArrayBuffer}}
instance. This does \emph{not} include
\href{https://developer.mozilla.org/en-US/docs/Web/JavaScript/Reference/Global_Objects/SharedArrayBuffer}{\texttt{SharedArrayBuffer}}
instances. Usually, it is desirable to test for both; See
\hyperref[utiltypesisanyarraybuffervalue]{\texttt{util.types.isAnyArrayBuffer()}}
for that.

\begin{Shaded}
\begin{Highlighting}[]
\NormalTok{util}\OperatorTok{.}\AttributeTok{types}\OperatorTok{.}\FunctionTok{isArrayBuffer}\NormalTok{(}\KeywordTok{new} \BuiltInTok{ArrayBuffer}\NormalTok{())}\OperatorTok{;}  \CommentTok{// Returns true}
\NormalTok{util}\OperatorTok{.}\AttributeTok{types}\OperatorTok{.}\FunctionTok{isArrayBuffer}\NormalTok{(}\KeywordTok{new} \BuiltInTok{SharedArrayBuffer}\NormalTok{())}\OperatorTok{;}  \CommentTok{// Returns false}
\end{Highlighting}
\end{Shaded}

\subsubsection{\texorpdfstring{\texttt{util.types.isAsyncFunction(value)}}{util.types.isAsyncFunction(value)}}\label{util.types.isasyncfunctionvalue}

\begin{itemize}
\tightlist
\item
  \texttt{value} \{any\}
\item
  Returns: \{boolean\}
\end{itemize}

Returns \texttt{true} if the value is an
\href{https://developer.mozilla.org/en-US/docs/Web/JavaScript/Reference/Statements/async_function}{async
function}. This only reports back what the JavaScript engine is seeing;
in particular, the return value may not match the original source code
if a transpilation tool was used.

\begin{Shaded}
\begin{Highlighting}[]
\NormalTok{util}\OperatorTok{.}\AttributeTok{types}\OperatorTok{.}\FunctionTok{isAsyncFunction}\NormalTok{(}\KeywordTok{function} \FunctionTok{foo}\NormalTok{() \{\})}\OperatorTok{;}  \CommentTok{// Returns false}
\NormalTok{util}\OperatorTok{.}\AttributeTok{types}\OperatorTok{.}\FunctionTok{isAsyncFunction}\NormalTok{(}\KeywordTok{async} \KeywordTok{function} \FunctionTok{foo}\NormalTok{() \{\})}\OperatorTok{;}  \CommentTok{// Returns true}
\end{Highlighting}
\end{Shaded}

\subsubsection{\texorpdfstring{\texttt{util.types.isBigInt64Array(value)}}{util.types.isBigInt64Array(value)}}\label{util.types.isbigint64arrayvalue}

\begin{itemize}
\tightlist
\item
  \texttt{value} \{any\}
\item
  Returns: \{boolean\}
\end{itemize}

Returns \texttt{true} if the value is a \texttt{BigInt64Array} instance.

\begin{Shaded}
\begin{Highlighting}[]
\NormalTok{util}\OperatorTok{.}\AttributeTok{types}\OperatorTok{.}\FunctionTok{isBigInt64Array}\NormalTok{(}\KeywordTok{new} \BuiltInTok{BigInt64Array}\NormalTok{())}\OperatorTok{;}   \CommentTok{// Returns true}
\NormalTok{util}\OperatorTok{.}\AttributeTok{types}\OperatorTok{.}\FunctionTok{isBigInt64Array}\NormalTok{(}\KeywordTok{new} \BuiltInTok{BigUint64Array}\NormalTok{())}\OperatorTok{;}  \CommentTok{// Returns false}
\end{Highlighting}
\end{Shaded}

\subsubsection{\texorpdfstring{\texttt{util.types.isBigUint64Array(value)}}{util.types.isBigUint64Array(value)}}\label{util.types.isbiguint64arrayvalue}

\begin{itemize}
\tightlist
\item
  \texttt{value} \{any\}
\item
  Returns: \{boolean\}
\end{itemize}

Returns \texttt{true} if the value is a \texttt{BigUint64Array}
instance.

\begin{Shaded}
\begin{Highlighting}[]
\NormalTok{util}\OperatorTok{.}\AttributeTok{types}\OperatorTok{.}\FunctionTok{isBigUint64Array}\NormalTok{(}\KeywordTok{new} \BuiltInTok{BigInt64Array}\NormalTok{())}\OperatorTok{;}   \CommentTok{// Returns false}
\NormalTok{util}\OperatorTok{.}\AttributeTok{types}\OperatorTok{.}\FunctionTok{isBigUint64Array}\NormalTok{(}\KeywordTok{new} \BuiltInTok{BigUint64Array}\NormalTok{())}\OperatorTok{;}  \CommentTok{// Returns true}
\end{Highlighting}
\end{Shaded}

\subsubsection{\texorpdfstring{\texttt{util.types.isBooleanObject(value)}}{util.types.isBooleanObject(value)}}\label{util.types.isbooleanobjectvalue}

\begin{itemize}
\tightlist
\item
  \texttt{value} \{any\}
\item
  Returns: \{boolean\}
\end{itemize}

Returns \texttt{true} if the value is a boolean object, e.g.~created by
\texttt{new\ Boolean()}.

\begin{Shaded}
\begin{Highlighting}[]
\NormalTok{util}\OperatorTok{.}\AttributeTok{types}\OperatorTok{.}\FunctionTok{isBooleanObject}\NormalTok{(}\KeywordTok{false}\NormalTok{)}\OperatorTok{;}  \CommentTok{// Returns false}
\NormalTok{util}\OperatorTok{.}\AttributeTok{types}\OperatorTok{.}\FunctionTok{isBooleanObject}\NormalTok{(}\KeywordTok{true}\NormalTok{)}\OperatorTok{;}   \CommentTok{// Returns false}
\NormalTok{util}\OperatorTok{.}\AttributeTok{types}\OperatorTok{.}\FunctionTok{isBooleanObject}\NormalTok{(}\KeywordTok{new} \BuiltInTok{Boolean}\NormalTok{(}\KeywordTok{false}\NormalTok{))}\OperatorTok{;} \CommentTok{// Returns true}
\NormalTok{util}\OperatorTok{.}\AttributeTok{types}\OperatorTok{.}\FunctionTok{isBooleanObject}\NormalTok{(}\KeywordTok{new} \BuiltInTok{Boolean}\NormalTok{(}\KeywordTok{true}\NormalTok{))}\OperatorTok{;}  \CommentTok{// Returns true}
\NormalTok{util}\OperatorTok{.}\AttributeTok{types}\OperatorTok{.}\FunctionTok{isBooleanObject}\NormalTok{(}\BuiltInTok{Boolean}\NormalTok{(}\KeywordTok{false}\NormalTok{))}\OperatorTok{;} \CommentTok{// Returns false}
\NormalTok{util}\OperatorTok{.}\AttributeTok{types}\OperatorTok{.}\FunctionTok{isBooleanObject}\NormalTok{(}\BuiltInTok{Boolean}\NormalTok{(}\KeywordTok{true}\NormalTok{))}\OperatorTok{;}  \CommentTok{// Returns false}
\end{Highlighting}
\end{Shaded}

\subsubsection{\texorpdfstring{\texttt{util.types.isBoxedPrimitive(value)}}{util.types.isBoxedPrimitive(value)}}\label{util.types.isboxedprimitivevalue}

\begin{itemize}
\tightlist
\item
  \texttt{value} \{any\}
\item
  Returns: \{boolean\}
\end{itemize}

Returns \texttt{true} if the value is any boxed primitive object,
e.g.~created by \texttt{new\ Boolean()}, \texttt{new\ String()} or
\texttt{Object(Symbol())}.

For example:

\begin{Shaded}
\begin{Highlighting}[]
\NormalTok{util}\OperatorTok{.}\AttributeTok{types}\OperatorTok{.}\FunctionTok{isBoxedPrimitive}\NormalTok{(}\KeywordTok{false}\NormalTok{)}\OperatorTok{;} \CommentTok{// Returns false}
\NormalTok{util}\OperatorTok{.}\AttributeTok{types}\OperatorTok{.}\FunctionTok{isBoxedPrimitive}\NormalTok{(}\KeywordTok{new} \BuiltInTok{Boolean}\NormalTok{(}\KeywordTok{false}\NormalTok{))}\OperatorTok{;} \CommentTok{// Returns true}
\NormalTok{util}\OperatorTok{.}\AttributeTok{types}\OperatorTok{.}\FunctionTok{isBoxedPrimitive}\NormalTok{(}\BuiltInTok{Symbol}\NormalTok{(}\StringTok{\textquotesingle{}foo\textquotesingle{}}\NormalTok{))}\OperatorTok{;} \CommentTok{// Returns false}
\NormalTok{util}\OperatorTok{.}\AttributeTok{types}\OperatorTok{.}\FunctionTok{isBoxedPrimitive}\NormalTok{(}\BuiltInTok{Object}\NormalTok{(}\BuiltInTok{Symbol}\NormalTok{(}\StringTok{\textquotesingle{}foo\textquotesingle{}}\NormalTok{)))}\OperatorTok{;} \CommentTok{// Returns true}
\NormalTok{util}\OperatorTok{.}\AttributeTok{types}\OperatorTok{.}\FunctionTok{isBoxedPrimitive}\NormalTok{(}\BuiltInTok{Object}\NormalTok{(}\BuiltInTok{BigInt}\NormalTok{(}\DecValTok{5}\NormalTok{)))}\OperatorTok{;} \CommentTok{// Returns true}
\end{Highlighting}
\end{Shaded}

\subsubsection{\texorpdfstring{\texttt{util.types.isCryptoKey(value)}}{util.types.isCryptoKey(value)}}\label{util.types.iscryptokeyvalue}

\begin{itemize}
\tightlist
\item
  \texttt{value} \{Object\}
\item
  Returns: \{boolean\}
\end{itemize}

Returns \texttt{true} if \texttt{value} is a \{CryptoKey\},
\texttt{false} otherwise.

\subsubsection{\texorpdfstring{\texttt{util.types.isDataView(value)}}{util.types.isDataView(value)}}\label{util.types.isdataviewvalue}

\begin{itemize}
\tightlist
\item
  \texttt{value} \{any\}
\item
  Returns: \{boolean\}
\end{itemize}

Returns \texttt{true} if the value is a built-in
\href{https://developer.mozilla.org/en-US/docs/Web/JavaScript/Reference/Global_Objects/DataView}{\texttt{DataView}}
instance.

\begin{Shaded}
\begin{Highlighting}[]
\KeywordTok{const}\NormalTok{ ab }\OperatorTok{=} \KeywordTok{new} \BuiltInTok{ArrayBuffer}\NormalTok{(}\DecValTok{20}\NormalTok{)}\OperatorTok{;}
\NormalTok{util}\OperatorTok{.}\AttributeTok{types}\OperatorTok{.}\FunctionTok{isDataView}\NormalTok{(}\KeywordTok{new} \BuiltInTok{DataView}\NormalTok{(ab))}\OperatorTok{;}  \CommentTok{// Returns true}
\NormalTok{util}\OperatorTok{.}\AttributeTok{types}\OperatorTok{.}\FunctionTok{isDataView}\NormalTok{(}\KeywordTok{new} \BuiltInTok{Float64Array}\NormalTok{())}\OperatorTok{;}  \CommentTok{// Returns false}
\end{Highlighting}
\end{Shaded}

See also
\href{https://developer.mozilla.org/en-US/docs/Web/JavaScript/Reference/Global_Objects/ArrayBuffer/isView}{\texttt{ArrayBuffer.isView()}}.

\subsubsection{\texorpdfstring{\texttt{util.types.isDate(value)}}{util.types.isDate(value)}}\label{util.types.isdatevalue}

\begin{itemize}
\tightlist
\item
  \texttt{value} \{any\}
\item
  Returns: \{boolean\}
\end{itemize}

Returns \texttt{true} if the value is a built-in
\href{https://developer.mozilla.org/en-US/docs/Web/JavaScript/Reference/Global_Objects/Date}{\texttt{Date}}
instance.

\begin{Shaded}
\begin{Highlighting}[]
\NormalTok{util}\OperatorTok{.}\AttributeTok{types}\OperatorTok{.}\FunctionTok{isDate}\NormalTok{(}\KeywordTok{new} \BuiltInTok{Date}\NormalTok{())}\OperatorTok{;}  \CommentTok{// Returns true}
\end{Highlighting}
\end{Shaded}

\subsubsection{\texorpdfstring{\texttt{util.types.isExternal(value)}}{util.types.isExternal(value)}}\label{util.types.isexternalvalue}

\begin{itemize}
\tightlist
\item
  \texttt{value} \{any\}
\item
  Returns: \{boolean\}
\end{itemize}

Returns \texttt{true} if the value is a native \texttt{External} value.

A native \texttt{External} value is a special type of object that
contains a raw C++ pointer (\texttt{void*}) for access from native code,
and has no other properties. Such objects are created either by Node.js
internals or native addons. In JavaScript, they are
\href{https://developer.mozilla.org/en-US/docs/Web/JavaScript/Reference/Global_Objects/Object/freeze}{frozen}
objects with a \texttt{null} prototype.

\begin{Shaded}
\begin{Highlighting}[]
\PreprocessorTok{\#include }\ImportTok{\textless{}js\_native\_api.h\textgreater{}}
\PreprocessorTok{\#include }\ImportTok{\textless{}stdlib.h\textgreater{}}
\NormalTok{napi\_value result}\OperatorTok{;}
\DataTypeTok{static}\NormalTok{ napi\_value MyNapi}\OperatorTok{(}\NormalTok{napi\_env env}\OperatorTok{,}\NormalTok{ napi\_callback\_info info}\OperatorTok{)} \OperatorTok{\{}
  \DataTypeTok{int}\OperatorTok{*}\NormalTok{ raw }\OperatorTok{=} \OperatorTok{(}\DataTypeTok{int}\OperatorTok{*)}\NormalTok{ malloc}\OperatorTok{(}\DecValTok{1024}\OperatorTok{);}
\NormalTok{  napi\_status status }\OperatorTok{=}\NormalTok{ napi\_create\_external}\OperatorTok{(}\NormalTok{env}\OperatorTok{,} \OperatorTok{(}\DataTypeTok{void}\OperatorTok{*)}\NormalTok{ raw}\OperatorTok{,}\NormalTok{ NULL}\OperatorTok{,}\NormalTok{ NULL}\OperatorTok{,} \OperatorTok{\&}\NormalTok{result}\OperatorTok{);}
  \ControlFlowTok{if} \OperatorTok{(}\NormalTok{status }\OperatorTok{!=}\NormalTok{ napi\_ok}\OperatorTok{)} \OperatorTok{\{}
\NormalTok{    napi\_throw\_error}\OperatorTok{(}\NormalTok{env}\OperatorTok{,}\NormalTok{ NULL}\OperatorTok{,} \StringTok{"napi\_create\_external failed"}\OperatorTok{);}
    \ControlFlowTok{return}\NormalTok{ NULL}\OperatorTok{;}
  \OperatorTok{\}}
  \ControlFlowTok{return}\NormalTok{ result}\OperatorTok{;}
\OperatorTok{\}}
\OperatorTok{...}
\NormalTok{DECLARE\_NAPI\_PROPERTY}\OperatorTok{(}\StringTok{"myNapi"}\OperatorTok{,}\NormalTok{ MyNapi}\OperatorTok{)}
\OperatorTok{...}
\end{Highlighting}
\end{Shaded}

\begin{Shaded}
\begin{Highlighting}[]
\KeywordTok{const}\NormalTok{ native }\OperatorTok{=} \PreprocessorTok{require}\NormalTok{(}\StringTok{\textquotesingle{}napi\_addon.node\textquotesingle{}}\NormalTok{)}\OperatorTok{;}
\KeywordTok{const}\NormalTok{ data }\OperatorTok{=}\NormalTok{ native}\OperatorTok{.}\FunctionTok{myNapi}\NormalTok{()}\OperatorTok{;}
\NormalTok{util}\OperatorTok{.}\AttributeTok{types}\OperatorTok{.}\FunctionTok{isExternal}\NormalTok{(data)}\OperatorTok{;} \CommentTok{// returns true}
\NormalTok{util}\OperatorTok{.}\AttributeTok{types}\OperatorTok{.}\FunctionTok{isExternal}\NormalTok{(}\DecValTok{0}\NormalTok{)}\OperatorTok{;} \CommentTok{// returns false}
\NormalTok{util}\OperatorTok{.}\AttributeTok{types}\OperatorTok{.}\FunctionTok{isExternal}\NormalTok{(}\KeywordTok{new} \BuiltInTok{String}\NormalTok{(}\StringTok{\textquotesingle{}foo\textquotesingle{}}\NormalTok{))}\OperatorTok{;} \CommentTok{// returns false}
\end{Highlighting}
\end{Shaded}

For further information on \texttt{napi\_create\_external}, refer to
\href{n-api.md\#napi_create_external}{\texttt{napi\_create\_external()}}.

\subsubsection{\texorpdfstring{\texttt{util.types.isFloat32Array(value)}}{util.types.isFloat32Array(value)}}\label{util.types.isfloat32arrayvalue}

\begin{itemize}
\tightlist
\item
  \texttt{value} \{any\}
\item
  Returns: \{boolean\}
\end{itemize}

Returns \texttt{true} if the value is a built-in
\href{https://developer.mozilla.org/en-US/docs/Web/JavaScript/Reference/Global_Objects/Float32Array}{\texttt{Float32Array}}
instance.

\begin{Shaded}
\begin{Highlighting}[]
\NormalTok{util}\OperatorTok{.}\AttributeTok{types}\OperatorTok{.}\FunctionTok{isFloat32Array}\NormalTok{(}\KeywordTok{new} \BuiltInTok{ArrayBuffer}\NormalTok{())}\OperatorTok{;}  \CommentTok{// Returns false}
\NormalTok{util}\OperatorTok{.}\AttributeTok{types}\OperatorTok{.}\FunctionTok{isFloat32Array}\NormalTok{(}\KeywordTok{new} \BuiltInTok{Float32Array}\NormalTok{())}\OperatorTok{;}  \CommentTok{// Returns true}
\NormalTok{util}\OperatorTok{.}\AttributeTok{types}\OperatorTok{.}\FunctionTok{isFloat32Array}\NormalTok{(}\KeywordTok{new} \BuiltInTok{Float64Array}\NormalTok{())}\OperatorTok{;}  \CommentTok{// Returns false}
\end{Highlighting}
\end{Shaded}

\subsubsection{\texorpdfstring{\texttt{util.types.isFloat64Array(value)}}{util.types.isFloat64Array(value)}}\label{util.types.isfloat64arrayvalue}

\begin{itemize}
\tightlist
\item
  \texttt{value} \{any\}
\item
  Returns: \{boolean\}
\end{itemize}

Returns \texttt{true} if the value is a built-in
\href{https://developer.mozilla.org/en-US/docs/Web/JavaScript/Reference/Global_Objects/Float64Array}{\texttt{Float64Array}}
instance.

\begin{Shaded}
\begin{Highlighting}[]
\NormalTok{util}\OperatorTok{.}\AttributeTok{types}\OperatorTok{.}\FunctionTok{isFloat64Array}\NormalTok{(}\KeywordTok{new} \BuiltInTok{ArrayBuffer}\NormalTok{())}\OperatorTok{;}  \CommentTok{// Returns false}
\NormalTok{util}\OperatorTok{.}\AttributeTok{types}\OperatorTok{.}\FunctionTok{isFloat64Array}\NormalTok{(}\KeywordTok{new} \BuiltInTok{Uint8Array}\NormalTok{())}\OperatorTok{;}  \CommentTok{// Returns false}
\NormalTok{util}\OperatorTok{.}\AttributeTok{types}\OperatorTok{.}\FunctionTok{isFloat64Array}\NormalTok{(}\KeywordTok{new} \BuiltInTok{Float64Array}\NormalTok{())}\OperatorTok{;}  \CommentTok{// Returns true}
\end{Highlighting}
\end{Shaded}

\subsubsection{\texorpdfstring{\texttt{util.types.isGeneratorFunction(value)}}{util.types.isGeneratorFunction(value)}}\label{util.types.isgeneratorfunctionvalue}

\begin{itemize}
\tightlist
\item
  \texttt{value} \{any\}
\item
  Returns: \{boolean\}
\end{itemize}

Returns \texttt{true} if the value is a generator function. This only
reports back what the JavaScript engine is seeing; in particular, the
return value may not match the original source code if a transpilation
tool was used.

\begin{Shaded}
\begin{Highlighting}[]
\NormalTok{util}\OperatorTok{.}\AttributeTok{types}\OperatorTok{.}\FunctionTok{isGeneratorFunction}\NormalTok{(}\KeywordTok{function} \FunctionTok{foo}\NormalTok{() \{\})}\OperatorTok{;}  \CommentTok{// Returns false}
\NormalTok{util}\OperatorTok{.}\AttributeTok{types}\OperatorTok{.}\FunctionTok{isGeneratorFunction}\NormalTok{(}\KeywordTok{function}\OperatorTok{*} \FunctionTok{foo}\NormalTok{() \{\})}\OperatorTok{;}  \CommentTok{// Returns true}
\end{Highlighting}
\end{Shaded}

\subsubsection{\texorpdfstring{\texttt{util.types.isGeneratorObject(value)}}{util.types.isGeneratorObject(value)}}\label{util.types.isgeneratorobjectvalue}

\begin{itemize}
\tightlist
\item
  \texttt{value} \{any\}
\item
  Returns: \{boolean\}
\end{itemize}

Returns \texttt{true} if the value is a generator object as returned
from a built-in generator function. This only reports back what the
JavaScript engine is seeing; in particular, the return value may not
match the original source code if a transpilation tool was used.

\begin{Shaded}
\begin{Highlighting}[]
\KeywordTok{function}\OperatorTok{*} \FunctionTok{foo}\NormalTok{() \{\}}
\KeywordTok{const}\NormalTok{ generator }\OperatorTok{=} \FunctionTok{foo}\NormalTok{()}\OperatorTok{;}
\NormalTok{util}\OperatorTok{.}\AttributeTok{types}\OperatorTok{.}\FunctionTok{isGeneratorObject}\NormalTok{(generator)}\OperatorTok{;}  \CommentTok{// Returns true}
\end{Highlighting}
\end{Shaded}

\subsubsection{\texorpdfstring{\texttt{util.types.isInt8Array(value)}}{util.types.isInt8Array(value)}}\label{util.types.isint8arrayvalue}

\begin{itemize}
\tightlist
\item
  \texttt{value} \{any\}
\item
  Returns: \{boolean\}
\end{itemize}

Returns \texttt{true} if the value is a built-in
\href{https://developer.mozilla.org/en-US/docs/Web/JavaScript/Reference/Global_Objects/Int8Array}{\texttt{Int8Array}}
instance.

\begin{Shaded}
\begin{Highlighting}[]
\NormalTok{util}\OperatorTok{.}\AttributeTok{types}\OperatorTok{.}\FunctionTok{isInt8Array}\NormalTok{(}\KeywordTok{new} \BuiltInTok{ArrayBuffer}\NormalTok{())}\OperatorTok{;}  \CommentTok{// Returns false}
\NormalTok{util}\OperatorTok{.}\AttributeTok{types}\OperatorTok{.}\FunctionTok{isInt8Array}\NormalTok{(}\KeywordTok{new} \BuiltInTok{Int8Array}\NormalTok{())}\OperatorTok{;}  \CommentTok{// Returns true}
\NormalTok{util}\OperatorTok{.}\AttributeTok{types}\OperatorTok{.}\FunctionTok{isInt8Array}\NormalTok{(}\KeywordTok{new} \BuiltInTok{Float64Array}\NormalTok{())}\OperatorTok{;}  \CommentTok{// Returns false}
\end{Highlighting}
\end{Shaded}

\subsubsection{\texorpdfstring{\texttt{util.types.isInt16Array(value)}}{util.types.isInt16Array(value)}}\label{util.types.isint16arrayvalue}

\begin{itemize}
\tightlist
\item
  \texttt{value} \{any\}
\item
  Returns: \{boolean\}
\end{itemize}

Returns \texttt{true} if the value is a built-in
\href{https://developer.mozilla.org/en-US/docs/Web/JavaScript/Reference/Global_Objects/Int16Array}{\texttt{Int16Array}}
instance.

\begin{Shaded}
\begin{Highlighting}[]
\NormalTok{util}\OperatorTok{.}\AttributeTok{types}\OperatorTok{.}\FunctionTok{isInt16Array}\NormalTok{(}\KeywordTok{new} \BuiltInTok{ArrayBuffer}\NormalTok{())}\OperatorTok{;}  \CommentTok{// Returns false}
\NormalTok{util}\OperatorTok{.}\AttributeTok{types}\OperatorTok{.}\FunctionTok{isInt16Array}\NormalTok{(}\KeywordTok{new} \BuiltInTok{Int16Array}\NormalTok{())}\OperatorTok{;}  \CommentTok{// Returns true}
\NormalTok{util}\OperatorTok{.}\AttributeTok{types}\OperatorTok{.}\FunctionTok{isInt16Array}\NormalTok{(}\KeywordTok{new} \BuiltInTok{Float64Array}\NormalTok{())}\OperatorTok{;}  \CommentTok{// Returns false}
\end{Highlighting}
\end{Shaded}

\subsubsection{\texorpdfstring{\texttt{util.types.isInt32Array(value)}}{util.types.isInt32Array(value)}}\label{util.types.isint32arrayvalue}

\begin{itemize}
\tightlist
\item
  \texttt{value} \{any\}
\item
  Returns: \{boolean\}
\end{itemize}

Returns \texttt{true} if the value is a built-in
\href{https://developer.mozilla.org/en-US/docs/Web/JavaScript/Reference/Global_Objects/Int32Array}{\texttt{Int32Array}}
instance.

\begin{Shaded}
\begin{Highlighting}[]
\NormalTok{util}\OperatorTok{.}\AttributeTok{types}\OperatorTok{.}\FunctionTok{isInt32Array}\NormalTok{(}\KeywordTok{new} \BuiltInTok{ArrayBuffer}\NormalTok{())}\OperatorTok{;}  \CommentTok{// Returns false}
\NormalTok{util}\OperatorTok{.}\AttributeTok{types}\OperatorTok{.}\FunctionTok{isInt32Array}\NormalTok{(}\KeywordTok{new} \BuiltInTok{Int32Array}\NormalTok{())}\OperatorTok{;}  \CommentTok{// Returns true}
\NormalTok{util}\OperatorTok{.}\AttributeTok{types}\OperatorTok{.}\FunctionTok{isInt32Array}\NormalTok{(}\KeywordTok{new} \BuiltInTok{Float64Array}\NormalTok{())}\OperatorTok{;}  \CommentTok{// Returns false}
\end{Highlighting}
\end{Shaded}

\subsubsection{\texorpdfstring{\texttt{util.types.isKeyObject(value)}}{util.types.isKeyObject(value)}}\label{util.types.iskeyobjectvalue}

\begin{itemize}
\tightlist
\item
  \texttt{value} \{Object\}
\item
  Returns: \{boolean\}
\end{itemize}

Returns \texttt{true} if \texttt{value} is a \{KeyObject\},
\texttt{false} otherwise.

\subsubsection{\texorpdfstring{\texttt{util.types.isMap(value)}}{util.types.isMap(value)}}\label{util.types.ismapvalue}

\begin{itemize}
\tightlist
\item
  \texttt{value} \{any\}
\item
  Returns: \{boolean\}
\end{itemize}

Returns \texttt{true} if the value is a built-in
\href{https://developer.mozilla.org/en-US/docs/Web/JavaScript/Reference/Global_Objects/Map}{\texttt{Map}}
instance.

\begin{Shaded}
\begin{Highlighting}[]
\NormalTok{util}\OperatorTok{.}\AttributeTok{types}\OperatorTok{.}\FunctionTok{isMap}\NormalTok{(}\KeywordTok{new} \BuiltInTok{Map}\NormalTok{())}\OperatorTok{;}  \CommentTok{// Returns true}
\end{Highlighting}
\end{Shaded}

\subsubsection{\texorpdfstring{\texttt{util.types.isMapIterator(value)}}{util.types.isMapIterator(value)}}\label{util.types.ismapiteratorvalue}

\begin{itemize}
\tightlist
\item
  \texttt{value} \{any\}
\item
  Returns: \{boolean\}
\end{itemize}

Returns \texttt{true} if the value is an iterator returned for a
built-in
\href{https://developer.mozilla.org/en-US/docs/Web/JavaScript/Reference/Global_Objects/Map}{\texttt{Map}}
instance.

\begin{Shaded}
\begin{Highlighting}[]
\KeywordTok{const}\NormalTok{ map }\OperatorTok{=} \KeywordTok{new} \BuiltInTok{Map}\NormalTok{()}\OperatorTok{;}
\NormalTok{util}\OperatorTok{.}\AttributeTok{types}\OperatorTok{.}\FunctionTok{isMapIterator}\NormalTok{(map}\OperatorTok{.}\FunctionTok{keys}\NormalTok{())}\OperatorTok{;}  \CommentTok{// Returns true}
\NormalTok{util}\OperatorTok{.}\AttributeTok{types}\OperatorTok{.}\FunctionTok{isMapIterator}\NormalTok{(map}\OperatorTok{.}\FunctionTok{values}\NormalTok{())}\OperatorTok{;}  \CommentTok{// Returns true}
\NormalTok{util}\OperatorTok{.}\AttributeTok{types}\OperatorTok{.}\FunctionTok{isMapIterator}\NormalTok{(map}\OperatorTok{.}\FunctionTok{entries}\NormalTok{())}\OperatorTok{;}  \CommentTok{// Returns true}
\NormalTok{util}\OperatorTok{.}\AttributeTok{types}\OperatorTok{.}\FunctionTok{isMapIterator}\NormalTok{(map[}\BuiltInTok{Symbol}\OperatorTok{.}\AttributeTok{iterator}\NormalTok{]())}\OperatorTok{;}  \CommentTok{// Returns true}
\end{Highlighting}
\end{Shaded}

\subsubsection{\texorpdfstring{\texttt{util.types.isModuleNamespaceObject(value)}}{util.types.isModuleNamespaceObject(value)}}\label{util.types.ismodulenamespaceobjectvalue}

\begin{itemize}
\tightlist
\item
  \texttt{value} \{any\}
\item
  Returns: \{boolean\}
\end{itemize}

Returns \texttt{true} if the value is an instance of a
\href{https://tc39.github.io/ecma262/\#sec-module-namespace-exotic-objects}{Module
Namespace Object}.

\begin{Shaded}
\begin{Highlighting}[]
\ImportTok{import} \OperatorTok{*} \ImportTok{as}\NormalTok{ ns }\ImportTok{from} \StringTok{\textquotesingle{}./a.js\textquotesingle{}}\OperatorTok{;}

\NormalTok{util}\OperatorTok{.}\AttributeTok{types}\OperatorTok{.}\FunctionTok{isModuleNamespaceObject}\NormalTok{(ns)}\OperatorTok{;}  \CommentTok{// Returns true}
\end{Highlighting}
\end{Shaded}

\subsubsection{\texorpdfstring{\texttt{util.types.isNativeError(value)}}{util.types.isNativeError(value)}}\label{util.types.isnativeerrorvalue}

\begin{itemize}
\tightlist
\item
  \texttt{value} \{any\}
\item
  Returns: \{boolean\}
\end{itemize}

Returns \texttt{true} if the value was returned by the constructor of a
\href{https://tc39.es/ecma262/\#sec-error-objects}{built-in
\texttt{Error} type}.

\begin{Shaded}
\begin{Highlighting}[]
\BuiltInTok{console}\OperatorTok{.}\FunctionTok{log}\NormalTok{(util}\OperatorTok{.}\AttributeTok{types}\OperatorTok{.}\FunctionTok{isNativeError}\NormalTok{(}\KeywordTok{new} \BuiltInTok{Error}\NormalTok{()))}\OperatorTok{;}  \CommentTok{// true}
\BuiltInTok{console}\OperatorTok{.}\FunctionTok{log}\NormalTok{(util}\OperatorTok{.}\AttributeTok{types}\OperatorTok{.}\FunctionTok{isNativeError}\NormalTok{(}\KeywordTok{new} \BuiltInTok{TypeError}\NormalTok{()))}\OperatorTok{;}  \CommentTok{// true}
\BuiltInTok{console}\OperatorTok{.}\FunctionTok{log}\NormalTok{(util}\OperatorTok{.}\AttributeTok{types}\OperatorTok{.}\FunctionTok{isNativeError}\NormalTok{(}\KeywordTok{new} \BuiltInTok{RangeError}\NormalTok{()))}\OperatorTok{;}  \CommentTok{// true}
\end{Highlighting}
\end{Shaded}

Subclasses of the native error types are also native errors:

\begin{Shaded}
\begin{Highlighting}[]
\KeywordTok{class}\NormalTok{ MyError }\KeywordTok{extends} \BuiltInTok{Error}\NormalTok{ \{\}}
\BuiltInTok{console}\OperatorTok{.}\FunctionTok{log}\NormalTok{(util}\OperatorTok{.}\AttributeTok{types}\OperatorTok{.}\FunctionTok{isNativeError}\NormalTok{(}\KeywordTok{new} \FunctionTok{MyError}\NormalTok{()))}\OperatorTok{;}  \CommentTok{// true}
\end{Highlighting}
\end{Shaded}

A value being \texttt{instanceof} a native error class is not equivalent
to \texttt{isNativeError()} returning \texttt{true} for that value.
\texttt{isNativeError()} returns \texttt{true} for errors which come
from a different \href{https://tc39.es/ecma262/\#realm}{realm} while
\texttt{instanceof\ Error} returns \texttt{false} for these errors:

\begin{Shaded}
\begin{Highlighting}[]
\KeywordTok{const}\NormalTok{ vm }\OperatorTok{=} \PreprocessorTok{require}\NormalTok{(}\StringTok{\textquotesingle{}node:vm\textquotesingle{}}\NormalTok{)}\OperatorTok{;}
\KeywordTok{const}\NormalTok{ context }\OperatorTok{=}\NormalTok{ vm}\OperatorTok{.}\FunctionTok{createContext}\NormalTok{(\{\})}\OperatorTok{;}
\KeywordTok{const}\NormalTok{ myError }\OperatorTok{=}\NormalTok{ vm}\OperatorTok{.}\FunctionTok{runInContext}\NormalTok{(}\StringTok{\textquotesingle{}new Error()\textquotesingle{}}\OperatorTok{,}\NormalTok{ context)}\OperatorTok{;}
\BuiltInTok{console}\OperatorTok{.}\FunctionTok{log}\NormalTok{(util}\OperatorTok{.}\AttributeTok{types}\OperatorTok{.}\FunctionTok{isNativeError}\NormalTok{(myError))}\OperatorTok{;} \CommentTok{// true}
\BuiltInTok{console}\OperatorTok{.}\FunctionTok{log}\NormalTok{(myError }\KeywordTok{instanceof} \BuiltInTok{Error}\NormalTok{)}\OperatorTok{;} \CommentTok{// false}
\end{Highlighting}
\end{Shaded}

Conversely, \texttt{isNativeError()} returns \texttt{false} for all
objects which were not returned by the constructor of a native error.
That includes values which are \texttt{instanceof} native errors:

\begin{Shaded}
\begin{Highlighting}[]
\KeywordTok{const}\NormalTok{ myError }\OperatorTok{=}\NormalTok{ \{ }\DataTypeTok{\_\_proto\_\_}\OperatorTok{:} \BuiltInTok{Error}\OperatorTok{.}\AttributeTok{prototype}\NormalTok{ \}}\OperatorTok{;}
\BuiltInTok{console}\OperatorTok{.}\FunctionTok{log}\NormalTok{(util}\OperatorTok{.}\AttributeTok{types}\OperatorTok{.}\FunctionTok{isNativeError}\NormalTok{(myError))}\OperatorTok{;} \CommentTok{// false}
\BuiltInTok{console}\OperatorTok{.}\FunctionTok{log}\NormalTok{(myError }\KeywordTok{instanceof} \BuiltInTok{Error}\NormalTok{)}\OperatorTok{;} \CommentTok{// true}
\end{Highlighting}
\end{Shaded}

\subsubsection{\texorpdfstring{\texttt{util.types.isNumberObject(value)}}{util.types.isNumberObject(value)}}\label{util.types.isnumberobjectvalue}

\begin{itemize}
\tightlist
\item
  \texttt{value} \{any\}
\item
  Returns: \{boolean\}
\end{itemize}

Returns \texttt{true} if the value is a number object, e.g.~created by
\texttt{new\ Number()}.

\begin{Shaded}
\begin{Highlighting}[]
\NormalTok{util}\OperatorTok{.}\AttributeTok{types}\OperatorTok{.}\FunctionTok{isNumberObject}\NormalTok{(}\DecValTok{0}\NormalTok{)}\OperatorTok{;}  \CommentTok{// Returns false}
\NormalTok{util}\OperatorTok{.}\AttributeTok{types}\OperatorTok{.}\FunctionTok{isNumberObject}\NormalTok{(}\KeywordTok{new} \BuiltInTok{Number}\NormalTok{(}\DecValTok{0}\NormalTok{))}\OperatorTok{;}   \CommentTok{// Returns true}
\end{Highlighting}
\end{Shaded}

\subsubsection{\texorpdfstring{\texttt{util.types.isPromise(value)}}{util.types.isPromise(value)}}\label{util.types.ispromisevalue}

\begin{itemize}
\tightlist
\item
  \texttt{value} \{any\}
\item
  Returns: \{boolean\}
\end{itemize}

Returns \texttt{true} if the value is a built-in
\href{https://developer.mozilla.org/en-US/docs/Web/JavaScript/Reference/Global_Objects/Promise}{\texttt{Promise}}.

\begin{Shaded}
\begin{Highlighting}[]
\NormalTok{util}\OperatorTok{.}\AttributeTok{types}\OperatorTok{.}\FunctionTok{isPromise}\NormalTok{(}\BuiltInTok{Promise}\OperatorTok{.}\FunctionTok{resolve}\NormalTok{(}\DecValTok{42}\NormalTok{))}\OperatorTok{;}  \CommentTok{// Returns true}
\end{Highlighting}
\end{Shaded}

\subsubsection{\texorpdfstring{\texttt{util.types.isProxy(value)}}{util.types.isProxy(value)}}\label{util.types.isproxyvalue}

\begin{itemize}
\tightlist
\item
  \texttt{value} \{any\}
\item
  Returns: \{boolean\}
\end{itemize}

Returns \texttt{true} if the value is a
\href{https://developer.mozilla.org/en-US/docs/Web/JavaScript/Reference/Global_Objects/Proxy}{\texttt{Proxy}}
instance.

\begin{Shaded}
\begin{Highlighting}[]
\KeywordTok{const}\NormalTok{ target }\OperatorTok{=}\NormalTok{ \{\}}\OperatorTok{;}
\KeywordTok{const}\NormalTok{ proxy }\OperatorTok{=} \KeywordTok{new} \BuiltInTok{Proxy}\NormalTok{(target}\OperatorTok{,}\NormalTok{ \{\})}\OperatorTok{;}
\NormalTok{util}\OperatorTok{.}\AttributeTok{types}\OperatorTok{.}\FunctionTok{isProxy}\NormalTok{(target)}\OperatorTok{;}  \CommentTok{// Returns false}
\NormalTok{util}\OperatorTok{.}\AttributeTok{types}\OperatorTok{.}\FunctionTok{isProxy}\NormalTok{(proxy)}\OperatorTok{;}  \CommentTok{// Returns true}
\end{Highlighting}
\end{Shaded}

\subsubsection{\texorpdfstring{\texttt{util.types.isRegExp(value)}}{util.types.isRegExp(value)}}\label{util.types.isregexpvalue}

\begin{itemize}
\tightlist
\item
  \texttt{value} \{any\}
\item
  Returns: \{boolean\}
\end{itemize}

Returns \texttt{true} if the value is a regular expression object.

\begin{Shaded}
\begin{Highlighting}[]
\NormalTok{util}\OperatorTok{.}\AttributeTok{types}\OperatorTok{.}\FunctionTok{isRegExp}\NormalTok{(}\SpecialStringTok{/abc/}\NormalTok{)}\OperatorTok{;}  \CommentTok{// Returns true}
\NormalTok{util}\OperatorTok{.}\AttributeTok{types}\OperatorTok{.}\FunctionTok{isRegExp}\NormalTok{(}\KeywordTok{new} \BuiltInTok{RegExp}\NormalTok{(}\StringTok{\textquotesingle{}abc\textquotesingle{}}\NormalTok{))}\OperatorTok{;}  \CommentTok{// Returns true}
\end{Highlighting}
\end{Shaded}

\subsubsection{\texorpdfstring{\texttt{util.types.isSet(value)}}{util.types.isSet(value)}}\label{util.types.issetvalue}

\begin{itemize}
\tightlist
\item
  \texttt{value} \{any\}
\item
  Returns: \{boolean\}
\end{itemize}

Returns \texttt{true} if the value is a built-in
\href{https://developer.mozilla.org/en-US/docs/Web/JavaScript/Reference/Global_Objects/Set}{\texttt{Set}}
instance.

\begin{Shaded}
\begin{Highlighting}[]
\NormalTok{util}\OperatorTok{.}\AttributeTok{types}\OperatorTok{.}\FunctionTok{isSet}\NormalTok{(}\KeywordTok{new} \BuiltInTok{Set}\NormalTok{())}\OperatorTok{;}  \CommentTok{// Returns true}
\end{Highlighting}
\end{Shaded}

\subsubsection{\texorpdfstring{\texttt{util.types.isSetIterator(value)}}{util.types.isSetIterator(value)}}\label{util.types.issetiteratorvalue}

\begin{itemize}
\tightlist
\item
  \texttt{value} \{any\}
\item
  Returns: \{boolean\}
\end{itemize}

Returns \texttt{true} if the value is an iterator returned for a
built-in
\href{https://developer.mozilla.org/en-US/docs/Web/JavaScript/Reference/Global_Objects/Set}{\texttt{Set}}
instance.

\begin{Shaded}
\begin{Highlighting}[]
\KeywordTok{const} \KeywordTok{set} \OperatorTok{=} \KeywordTok{new} \BuiltInTok{Set}\NormalTok{()}\OperatorTok{;}
\NormalTok{util}\OperatorTok{.}\AttributeTok{types}\OperatorTok{.}\FunctionTok{isSetIterator}\NormalTok{(}\KeywordTok{set}\OperatorTok{.}\FunctionTok{keys}\NormalTok{())}\OperatorTok{;}  \CommentTok{// Returns true}
\NormalTok{util}\OperatorTok{.}\AttributeTok{types}\OperatorTok{.}\FunctionTok{isSetIterator}\NormalTok{(}\KeywordTok{set}\OperatorTok{.}\FunctionTok{values}\NormalTok{())}\OperatorTok{;}  \CommentTok{// Returns true}
\NormalTok{util}\OperatorTok{.}\AttributeTok{types}\OperatorTok{.}\FunctionTok{isSetIterator}\NormalTok{(}\KeywordTok{set}\OperatorTok{.}\FunctionTok{entries}\NormalTok{())}\OperatorTok{;}  \CommentTok{// Returns true}
\NormalTok{util}\OperatorTok{.}\AttributeTok{types}\OperatorTok{.}\FunctionTok{isSetIterator}\NormalTok{(}\KeywordTok{set}\NormalTok{[}\BuiltInTok{Symbol}\OperatorTok{.}\AttributeTok{iterator}\NormalTok{]())}\OperatorTok{;}  \CommentTok{// Returns true}
\end{Highlighting}
\end{Shaded}

\subsubsection{\texorpdfstring{\texttt{util.types.isSharedArrayBuffer(value)}}{util.types.isSharedArrayBuffer(value)}}\label{util.types.issharedarraybuffervalue}

\begin{itemize}
\tightlist
\item
  \texttt{value} \{any\}
\item
  Returns: \{boolean\}
\end{itemize}

Returns \texttt{true} if the value is a built-in
\href{https://developer.mozilla.org/en-US/docs/Web/JavaScript/Reference/Global_Objects/SharedArrayBuffer}{\texttt{SharedArrayBuffer}}
instance. This does \emph{not} include
\href{https://developer.mozilla.org/en-US/docs/Web/JavaScript/Reference/Global_Objects/ArrayBuffer}{\texttt{ArrayBuffer}}
instances. Usually, it is desirable to test for both; See
\hyperref[utiltypesisanyarraybuffervalue]{\texttt{util.types.isAnyArrayBuffer()}}
for that.

\begin{Shaded}
\begin{Highlighting}[]
\NormalTok{util}\OperatorTok{.}\AttributeTok{types}\OperatorTok{.}\FunctionTok{isSharedArrayBuffer}\NormalTok{(}\KeywordTok{new} \BuiltInTok{ArrayBuffer}\NormalTok{())}\OperatorTok{;}  \CommentTok{// Returns false}
\NormalTok{util}\OperatorTok{.}\AttributeTok{types}\OperatorTok{.}\FunctionTok{isSharedArrayBuffer}\NormalTok{(}\KeywordTok{new} \BuiltInTok{SharedArrayBuffer}\NormalTok{())}\OperatorTok{;}  \CommentTok{// Returns true}
\end{Highlighting}
\end{Shaded}

\subsubsection{\texorpdfstring{\texttt{util.types.isStringObject(value)}}{util.types.isStringObject(value)}}\label{util.types.isstringobjectvalue}

\begin{itemize}
\tightlist
\item
  \texttt{value} \{any\}
\item
  Returns: \{boolean\}
\end{itemize}

Returns \texttt{true} if the value is a string object, e.g.~created by
\texttt{new\ String()}.

\begin{Shaded}
\begin{Highlighting}[]
\NormalTok{util}\OperatorTok{.}\AttributeTok{types}\OperatorTok{.}\FunctionTok{isStringObject}\NormalTok{(}\StringTok{\textquotesingle{}foo\textquotesingle{}}\NormalTok{)}\OperatorTok{;}  \CommentTok{// Returns false}
\NormalTok{util}\OperatorTok{.}\AttributeTok{types}\OperatorTok{.}\FunctionTok{isStringObject}\NormalTok{(}\KeywordTok{new} \BuiltInTok{String}\NormalTok{(}\StringTok{\textquotesingle{}foo\textquotesingle{}}\NormalTok{))}\OperatorTok{;}   \CommentTok{// Returns true}
\end{Highlighting}
\end{Shaded}

\subsubsection{\texorpdfstring{\texttt{util.types.isSymbolObject(value)}}{util.types.isSymbolObject(value)}}\label{util.types.issymbolobjectvalue}

\begin{itemize}
\tightlist
\item
  \texttt{value} \{any\}
\item
  Returns: \{boolean\}
\end{itemize}

Returns \texttt{true} if the value is a symbol object, created by
calling \texttt{Object()} on a \texttt{Symbol} primitive.

\begin{Shaded}
\begin{Highlighting}[]
\KeywordTok{const}\NormalTok{ symbol }\OperatorTok{=} \BuiltInTok{Symbol}\NormalTok{(}\StringTok{\textquotesingle{}foo\textquotesingle{}}\NormalTok{)}\OperatorTok{;}
\NormalTok{util}\OperatorTok{.}\AttributeTok{types}\OperatorTok{.}\FunctionTok{isSymbolObject}\NormalTok{(symbol)}\OperatorTok{;}  \CommentTok{// Returns false}
\NormalTok{util}\OperatorTok{.}\AttributeTok{types}\OperatorTok{.}\FunctionTok{isSymbolObject}\NormalTok{(}\BuiltInTok{Object}\NormalTok{(symbol))}\OperatorTok{;}   \CommentTok{// Returns true}
\end{Highlighting}
\end{Shaded}

\subsubsection{\texorpdfstring{\texttt{util.types.isTypedArray(value)}}{util.types.isTypedArray(value)}}\label{util.types.istypedarrayvalue}

\begin{itemize}
\tightlist
\item
  \texttt{value} \{any\}
\item
  Returns: \{boolean\}
\end{itemize}

Returns \texttt{true} if the value is a built-in
\href{https://developer.mozilla.org/en-US/docs/Web/JavaScript/Reference/Global_Objects/TypedArray}{\texttt{TypedArray}}
instance.

\begin{Shaded}
\begin{Highlighting}[]
\NormalTok{util}\OperatorTok{.}\AttributeTok{types}\OperatorTok{.}\FunctionTok{isTypedArray}\NormalTok{(}\KeywordTok{new} \BuiltInTok{ArrayBuffer}\NormalTok{())}\OperatorTok{;}  \CommentTok{// Returns false}
\NormalTok{util}\OperatorTok{.}\AttributeTok{types}\OperatorTok{.}\FunctionTok{isTypedArray}\NormalTok{(}\KeywordTok{new} \BuiltInTok{Uint8Array}\NormalTok{())}\OperatorTok{;}  \CommentTok{// Returns true}
\NormalTok{util}\OperatorTok{.}\AttributeTok{types}\OperatorTok{.}\FunctionTok{isTypedArray}\NormalTok{(}\KeywordTok{new} \BuiltInTok{Float64Array}\NormalTok{())}\OperatorTok{;}  \CommentTok{// Returns true}
\end{Highlighting}
\end{Shaded}

See also
\href{https://developer.mozilla.org/en-US/docs/Web/JavaScript/Reference/Global_Objects/ArrayBuffer/isView}{\texttt{ArrayBuffer.isView()}}.

\subsubsection{\texorpdfstring{\texttt{util.types.isUint8Array(value)}}{util.types.isUint8Array(value)}}\label{util.types.isuint8arrayvalue}

\begin{itemize}
\tightlist
\item
  \texttt{value} \{any\}
\item
  Returns: \{boolean\}
\end{itemize}

Returns \texttt{true} if the value is a built-in
\href{https://developer.mozilla.org/en-US/docs/Web/JavaScript/Reference/Global_Objects/Uint8Array}{\texttt{Uint8Array}}
instance.

\begin{Shaded}
\begin{Highlighting}[]
\NormalTok{util}\OperatorTok{.}\AttributeTok{types}\OperatorTok{.}\FunctionTok{isUint8Array}\NormalTok{(}\KeywordTok{new} \BuiltInTok{ArrayBuffer}\NormalTok{())}\OperatorTok{;}  \CommentTok{// Returns false}
\NormalTok{util}\OperatorTok{.}\AttributeTok{types}\OperatorTok{.}\FunctionTok{isUint8Array}\NormalTok{(}\KeywordTok{new} \BuiltInTok{Uint8Array}\NormalTok{())}\OperatorTok{;}  \CommentTok{// Returns true}
\NormalTok{util}\OperatorTok{.}\AttributeTok{types}\OperatorTok{.}\FunctionTok{isUint8Array}\NormalTok{(}\KeywordTok{new} \BuiltInTok{Float64Array}\NormalTok{())}\OperatorTok{;}  \CommentTok{// Returns false}
\end{Highlighting}
\end{Shaded}

\subsubsection{\texorpdfstring{\texttt{util.types.isUint8ClampedArray(value)}}{util.types.isUint8ClampedArray(value)}}\label{util.types.isuint8clampedarrayvalue}

\begin{itemize}
\tightlist
\item
  \texttt{value} \{any\}
\item
  Returns: \{boolean\}
\end{itemize}

Returns \texttt{true} if the value is a built-in
\href{https://developer.mozilla.org/en-US/docs/Web/JavaScript/Reference/Global_Objects/Uint8ClampedArray}{\texttt{Uint8ClampedArray}}
instance.

\begin{Shaded}
\begin{Highlighting}[]
\NormalTok{util}\OperatorTok{.}\AttributeTok{types}\OperatorTok{.}\FunctionTok{isUint8ClampedArray}\NormalTok{(}\KeywordTok{new} \BuiltInTok{ArrayBuffer}\NormalTok{())}\OperatorTok{;}  \CommentTok{// Returns false}
\NormalTok{util}\OperatorTok{.}\AttributeTok{types}\OperatorTok{.}\FunctionTok{isUint8ClampedArray}\NormalTok{(}\KeywordTok{new} \BuiltInTok{Uint8ClampedArray}\NormalTok{())}\OperatorTok{;}  \CommentTok{// Returns true}
\NormalTok{util}\OperatorTok{.}\AttributeTok{types}\OperatorTok{.}\FunctionTok{isUint8ClampedArray}\NormalTok{(}\KeywordTok{new} \BuiltInTok{Float64Array}\NormalTok{())}\OperatorTok{;}  \CommentTok{// Returns false}
\end{Highlighting}
\end{Shaded}

\subsubsection{\texorpdfstring{\texttt{util.types.isUint16Array(value)}}{util.types.isUint16Array(value)}}\label{util.types.isuint16arrayvalue}

\begin{itemize}
\tightlist
\item
  \texttt{value} \{any\}
\item
  Returns: \{boolean\}
\end{itemize}

Returns \texttt{true} if the value is a built-in
\href{https://developer.mozilla.org/en-US/docs/Web/JavaScript/Reference/Global_Objects/Uint16Array}{\texttt{Uint16Array}}
instance.

\begin{Shaded}
\begin{Highlighting}[]
\NormalTok{util}\OperatorTok{.}\AttributeTok{types}\OperatorTok{.}\FunctionTok{isUint16Array}\NormalTok{(}\KeywordTok{new} \BuiltInTok{ArrayBuffer}\NormalTok{())}\OperatorTok{;}  \CommentTok{// Returns false}
\NormalTok{util}\OperatorTok{.}\AttributeTok{types}\OperatorTok{.}\FunctionTok{isUint16Array}\NormalTok{(}\KeywordTok{new} \BuiltInTok{Uint16Array}\NormalTok{())}\OperatorTok{;}  \CommentTok{// Returns true}
\NormalTok{util}\OperatorTok{.}\AttributeTok{types}\OperatorTok{.}\FunctionTok{isUint16Array}\NormalTok{(}\KeywordTok{new} \BuiltInTok{Float64Array}\NormalTok{())}\OperatorTok{;}  \CommentTok{// Returns false}
\end{Highlighting}
\end{Shaded}

\subsubsection{\texorpdfstring{\texttt{util.types.isUint32Array(value)}}{util.types.isUint32Array(value)}}\label{util.types.isuint32arrayvalue}

\begin{itemize}
\tightlist
\item
  \texttt{value} \{any\}
\item
  Returns: \{boolean\}
\end{itemize}

Returns \texttt{true} if the value is a built-in
\href{https://developer.mozilla.org/en-US/docs/Web/JavaScript/Reference/Global_Objects/Uint32Array}{\texttt{Uint32Array}}
instance.

\begin{Shaded}
\begin{Highlighting}[]
\NormalTok{util}\OperatorTok{.}\AttributeTok{types}\OperatorTok{.}\FunctionTok{isUint32Array}\NormalTok{(}\KeywordTok{new} \BuiltInTok{ArrayBuffer}\NormalTok{())}\OperatorTok{;}  \CommentTok{// Returns false}
\NormalTok{util}\OperatorTok{.}\AttributeTok{types}\OperatorTok{.}\FunctionTok{isUint32Array}\NormalTok{(}\KeywordTok{new} \BuiltInTok{Uint32Array}\NormalTok{())}\OperatorTok{;}  \CommentTok{// Returns true}
\NormalTok{util}\OperatorTok{.}\AttributeTok{types}\OperatorTok{.}\FunctionTok{isUint32Array}\NormalTok{(}\KeywordTok{new} \BuiltInTok{Float64Array}\NormalTok{())}\OperatorTok{;}  \CommentTok{// Returns false}
\end{Highlighting}
\end{Shaded}

\subsubsection{\texorpdfstring{\texttt{util.types.isWeakMap(value)}}{util.types.isWeakMap(value)}}\label{util.types.isweakmapvalue}

\begin{itemize}
\tightlist
\item
  \texttt{value} \{any\}
\item
  Returns: \{boolean\}
\end{itemize}

Returns \texttt{true} if the value is a built-in
\href{https://developer.mozilla.org/en-US/docs/Web/JavaScript/Reference/Global_Objects/WeakMap}{\texttt{WeakMap}}
instance.

\begin{Shaded}
\begin{Highlighting}[]
\NormalTok{util}\OperatorTok{.}\AttributeTok{types}\OperatorTok{.}\FunctionTok{isWeakMap}\NormalTok{(}\KeywordTok{new} \BuiltInTok{WeakMap}\NormalTok{())}\OperatorTok{;}  \CommentTok{// Returns true}
\end{Highlighting}
\end{Shaded}

\subsubsection{\texorpdfstring{\texttt{util.types.isWeakSet(value)}}{util.types.isWeakSet(value)}}\label{util.types.isweaksetvalue}

\begin{itemize}
\tightlist
\item
  \texttt{value} \{any\}
\item
  Returns: \{boolean\}
\end{itemize}

Returns \texttt{true} if the value is a built-in
\href{https://developer.mozilla.org/en-US/docs/Web/JavaScript/Reference/Global_Objects/WeakSet}{\texttt{WeakSet}}
instance.

\begin{Shaded}
\begin{Highlighting}[]
\NormalTok{util}\OperatorTok{.}\AttributeTok{types}\OperatorTok{.}\FunctionTok{isWeakSet}\NormalTok{(}\KeywordTok{new} \BuiltInTok{WeakSet}\NormalTok{())}\OperatorTok{;}  \CommentTok{// Returns true}
\end{Highlighting}
\end{Shaded}

\subsubsection{\texorpdfstring{\texttt{util.types.isWebAssemblyCompiledModule(value)}}{util.types.isWebAssemblyCompiledModule(value)}}\label{util.types.iswebassemblycompiledmodulevalue}

\begin{quote}
Stability: 0 - Deprecated: Use
\texttt{value\ instanceof\ WebAssembly.Module} instead.
\end{quote}

\begin{itemize}
\tightlist
\item
  \texttt{value} \{any\}
\item
  Returns: \{boolean\}
\end{itemize}

Returns \texttt{true} if the value is a built-in
\href{https://developer.mozilla.org/en-US/docs/Web/JavaScript/Reference/Global_Objects/WebAssembly/Module}{\texttt{WebAssembly.Module}}
instance.

\begin{Shaded}
\begin{Highlighting}[]
\KeywordTok{const}\NormalTok{ module }\OperatorTok{=} \KeywordTok{new}\NormalTok{ WebAssembly}\OperatorTok{.}\FunctionTok{Module}\NormalTok{(wasmBuffer)}\OperatorTok{;}
\NormalTok{util}\OperatorTok{.}\AttributeTok{types}\OperatorTok{.}\FunctionTok{isWebAssemblyCompiledModule}\NormalTok{(module)}\OperatorTok{;}  \CommentTok{// Returns true}
\end{Highlighting}
\end{Shaded}

\subsection{Deprecated APIs}\label{deprecated-apis}

The following APIs are deprecated and should no longer be used. Existing
applications and modules should be updated to find alternative
approaches.

\subsubsection{\texorpdfstring{\texttt{util.\_extend(target,\ source)}}{util.\_extend(target, source)}}\label{util._extendtarget-source}

\begin{quote}
Stability: 0 - Deprecated: Use
\href{https://developer.mozilla.org/en-US/docs/Web/JavaScript/Reference/Global_Objects/Object/assign}{\texttt{Object.assign()}}
instead.
\end{quote}

\begin{itemize}
\tightlist
\item
  \texttt{target} \{Object\}
\item
  \texttt{source} \{Object\}
\end{itemize}

The \texttt{util.\_extend()} method was never intended to be used
outside of internal Node.js modules. The community found and used it
anyway.

It is deprecated and should not be used in new code. JavaScript comes
with very similar built-in functionality through
\href{https://developer.mozilla.org/en-US/docs/Web/JavaScript/Reference/Global_Objects/Object/assign}{\texttt{Object.assign()}}.

\subsubsection{\texorpdfstring{\texttt{util.isArray(object)}}{util.isArray(object)}}\label{util.isarrayobject}

\begin{quote}
Stability: 0 - Deprecated: Use
\href{https://developer.mozilla.org/en-US/docs/Web/JavaScript/Reference/Global_Objects/Array/isArray}{\texttt{Array.isArray()}}
instead.
\end{quote}

\begin{itemize}
\tightlist
\item
  \texttt{object} \{any\}
\item
  Returns: \{boolean\}
\end{itemize}

Alias for
\href{https://developer.mozilla.org/en-US/docs/Web/JavaScript/Reference/Global_Objects/Array/isArray}{\texttt{Array.isArray()}}.

Returns \texttt{true} if the given \texttt{object} is an \texttt{Array}.
Otherwise, returns \texttt{false}.

\begin{Shaded}
\begin{Highlighting}[]
\KeywordTok{const}\NormalTok{ util }\OperatorTok{=} \PreprocessorTok{require}\NormalTok{(}\StringTok{\textquotesingle{}node:util\textquotesingle{}}\NormalTok{)}\OperatorTok{;}

\NormalTok{util}\OperatorTok{.}\FunctionTok{isArray}\NormalTok{([])}\OperatorTok{;}
\CommentTok{// Returns: true}
\NormalTok{util}\OperatorTok{.}\FunctionTok{isArray}\NormalTok{(}\KeywordTok{new} \BuiltInTok{Array}\NormalTok{())}\OperatorTok{;}
\CommentTok{// Returns: true}
\NormalTok{util}\OperatorTok{.}\FunctionTok{isArray}\NormalTok{(\{\})}\OperatorTok{;}
\CommentTok{// Returns: false}
\end{Highlighting}
\end{Shaded}

\subsubsection{\texorpdfstring{\texttt{util.isBoolean(object)}}{util.isBoolean(object)}}\label{util.isbooleanobject}

\begin{quote}
Stability: 0 - Deprecated: Use
\texttt{typeof\ value\ ===\ \textquotesingle{}boolean\textquotesingle{}}
instead.
\end{quote}

\begin{itemize}
\tightlist
\item
  \texttt{object} \{any\}
\item
  Returns: \{boolean\}
\end{itemize}

Returns \texttt{true} if the given \texttt{object} is a
\texttt{Boolean}. Otherwise, returns \texttt{false}.

\begin{Shaded}
\begin{Highlighting}[]
\KeywordTok{const}\NormalTok{ util }\OperatorTok{=} \PreprocessorTok{require}\NormalTok{(}\StringTok{\textquotesingle{}node:util\textquotesingle{}}\NormalTok{)}\OperatorTok{;}

\NormalTok{util}\OperatorTok{.}\FunctionTok{isBoolean}\NormalTok{(}\DecValTok{1}\NormalTok{)}\OperatorTok{;}
\CommentTok{// Returns: false}
\NormalTok{util}\OperatorTok{.}\FunctionTok{isBoolean}\NormalTok{(}\DecValTok{0}\NormalTok{)}\OperatorTok{;}
\CommentTok{// Returns: false}
\NormalTok{util}\OperatorTok{.}\FunctionTok{isBoolean}\NormalTok{(}\KeywordTok{false}\NormalTok{)}\OperatorTok{;}
\CommentTok{// Returns: true}
\end{Highlighting}
\end{Shaded}

\subsubsection{\texorpdfstring{\texttt{util.isBuffer(object)}}{util.isBuffer(object)}}\label{util.isbufferobject}

\begin{quote}
Stability: 0 - Deprecated: Use
\href{buffer.md\#static-method-bufferisbufferobj}{\texttt{Buffer.isBuffer()}}
instead.
\end{quote}

\begin{itemize}
\tightlist
\item
  \texttt{object} \{any\}
\item
  Returns: \{boolean\}
\end{itemize}

Returns \texttt{true} if the given \texttt{object} is a \texttt{Buffer}.
Otherwise, returns \texttt{false}.

\begin{Shaded}
\begin{Highlighting}[]
\KeywordTok{const}\NormalTok{ util }\OperatorTok{=} \PreprocessorTok{require}\NormalTok{(}\StringTok{\textquotesingle{}node:util\textquotesingle{}}\NormalTok{)}\OperatorTok{;}

\NormalTok{util}\OperatorTok{.}\FunctionTok{isBuffer}\NormalTok{(\{ }\DataTypeTok{length}\OperatorTok{:} \DecValTok{0}\NormalTok{ \})}\OperatorTok{;}
\CommentTok{// Returns: false}
\NormalTok{util}\OperatorTok{.}\FunctionTok{isBuffer}\NormalTok{([])}\OperatorTok{;}
\CommentTok{// Returns: false}
\NormalTok{util}\OperatorTok{.}\FunctionTok{isBuffer}\NormalTok{(}\BuiltInTok{Buffer}\OperatorTok{.}\FunctionTok{from}\NormalTok{(}\StringTok{\textquotesingle{}hello world\textquotesingle{}}\NormalTok{))}\OperatorTok{;}
\CommentTok{// Returns: true}
\end{Highlighting}
\end{Shaded}

\subsubsection{\texorpdfstring{\texttt{util.isDate(object)}}{util.isDate(object)}}\label{util.isdateobject}

\begin{quote}
Stability: 0 - Deprecated: Use
\hyperref[utiltypesisdatevalue]{\texttt{util.types.isDate()}} instead.
\end{quote}

\begin{itemize}
\tightlist
\item
  \texttt{object} \{any\}
\item
  Returns: \{boolean\}
\end{itemize}

Returns \texttt{true} if the given \texttt{object} is a \texttt{Date}.
Otherwise, returns \texttt{false}.

\begin{Shaded}
\begin{Highlighting}[]
\KeywordTok{const}\NormalTok{ util }\OperatorTok{=} \PreprocessorTok{require}\NormalTok{(}\StringTok{\textquotesingle{}node:util\textquotesingle{}}\NormalTok{)}\OperatorTok{;}

\NormalTok{util}\OperatorTok{.}\FunctionTok{isDate}\NormalTok{(}\KeywordTok{new} \BuiltInTok{Date}\NormalTok{())}\OperatorTok{;}
\CommentTok{// Returns: true}
\NormalTok{util}\OperatorTok{.}\FunctionTok{isDate}\NormalTok{(}\BuiltInTok{Date}\NormalTok{())}\OperatorTok{;}
\CommentTok{// false (without \textquotesingle{}new\textquotesingle{} returns a String)}
\NormalTok{util}\OperatorTok{.}\FunctionTok{isDate}\NormalTok{(\{\})}\OperatorTok{;}
\CommentTok{// Returns: false}
\end{Highlighting}
\end{Shaded}

\subsubsection{\texorpdfstring{\texttt{util.isError(object)}}{util.isError(object)}}\label{util.iserrorobject}

\begin{quote}
Stability: 0 - Deprecated: Use
\hyperref[utiltypesisnativeerrorvalue]{\texttt{util.types.isNativeError()}}
instead.
\end{quote}

\begin{itemize}
\tightlist
\item
  \texttt{object} \{any\}
\item
  Returns: \{boolean\}
\end{itemize}

Returns \texttt{true} if the given \texttt{object} is an
\href{errors.md\#class-error}{\texttt{Error}}. Otherwise, returns
\texttt{false}.

\begin{Shaded}
\begin{Highlighting}[]
\KeywordTok{const}\NormalTok{ util }\OperatorTok{=} \PreprocessorTok{require}\NormalTok{(}\StringTok{\textquotesingle{}node:util\textquotesingle{}}\NormalTok{)}\OperatorTok{;}

\NormalTok{util}\OperatorTok{.}\FunctionTok{isError}\NormalTok{(}\KeywordTok{new} \BuiltInTok{Error}\NormalTok{())}\OperatorTok{;}
\CommentTok{// Returns: true}
\NormalTok{util}\OperatorTok{.}\FunctionTok{isError}\NormalTok{(}\KeywordTok{new} \BuiltInTok{TypeError}\NormalTok{())}\OperatorTok{;}
\CommentTok{// Returns: true}
\NormalTok{util}\OperatorTok{.}\FunctionTok{isError}\NormalTok{(\{ }\DataTypeTok{name}\OperatorTok{:} \StringTok{\textquotesingle{}Error\textquotesingle{}}\OperatorTok{,} \DataTypeTok{message}\OperatorTok{:} \StringTok{\textquotesingle{}an error occurred\textquotesingle{}}\NormalTok{ \})}\OperatorTok{;}
\CommentTok{// Returns: false}
\end{Highlighting}
\end{Shaded}

This method relies on \texttt{Object.prototype.toString()} behavior. It
is possible to obtain an incorrect result when the \texttt{object}
argument manipulates \texttt{@@toStringTag}.

\begin{Shaded}
\begin{Highlighting}[]
\KeywordTok{const}\NormalTok{ util }\OperatorTok{=} \PreprocessorTok{require}\NormalTok{(}\StringTok{\textquotesingle{}node:util\textquotesingle{}}\NormalTok{)}\OperatorTok{;}
\KeywordTok{const}\NormalTok{ obj }\OperatorTok{=}\NormalTok{ \{ }\DataTypeTok{name}\OperatorTok{:} \StringTok{\textquotesingle{}Error\textquotesingle{}}\OperatorTok{,} \DataTypeTok{message}\OperatorTok{:} \StringTok{\textquotesingle{}an error occurred\textquotesingle{}}\NormalTok{ \}}\OperatorTok{;}

\NormalTok{util}\OperatorTok{.}\FunctionTok{isError}\NormalTok{(obj)}\OperatorTok{;}
\CommentTok{// Returns: false}
\NormalTok{obj[}\BuiltInTok{Symbol}\OperatorTok{.}\AttributeTok{toStringTag}\NormalTok{] }\OperatorTok{=} \StringTok{\textquotesingle{}Error\textquotesingle{}}\OperatorTok{;}
\NormalTok{util}\OperatorTok{.}\FunctionTok{isError}\NormalTok{(obj)}\OperatorTok{;}
\CommentTok{// Returns: true}
\end{Highlighting}
\end{Shaded}

\subsubsection{\texorpdfstring{\texttt{util.isFunction(object)}}{util.isFunction(object)}}\label{util.isfunctionobject}

\begin{quote}
Stability: 0 - Deprecated: Use
\texttt{typeof\ value\ ===\ \textquotesingle{}function\textquotesingle{}}
instead.
\end{quote}

\begin{itemize}
\tightlist
\item
  \texttt{object} \{any\}
\item
  Returns: \{boolean\}
\end{itemize}

Returns \texttt{true} if the given \texttt{object} is a
\texttt{Function}. Otherwise, returns \texttt{false}.

\begin{Shaded}
\begin{Highlighting}[]
\KeywordTok{const}\NormalTok{ util }\OperatorTok{=} \PreprocessorTok{require}\NormalTok{(}\StringTok{\textquotesingle{}node:util\textquotesingle{}}\NormalTok{)}\OperatorTok{;}

\KeywordTok{function} \FunctionTok{Foo}\NormalTok{() \{\}}
\KeywordTok{const}\NormalTok{ Bar }\OperatorTok{=}\NormalTok{ () }\KeywordTok{=\textgreater{}}\NormalTok{ \{\}}\OperatorTok{;}

\NormalTok{util}\OperatorTok{.}\FunctionTok{isFunction}\NormalTok{(\{\})}\OperatorTok{;}
\CommentTok{// Returns: false}
\NormalTok{util}\OperatorTok{.}\FunctionTok{isFunction}\NormalTok{(Foo)}\OperatorTok{;}
\CommentTok{// Returns: true}
\NormalTok{util}\OperatorTok{.}\FunctionTok{isFunction}\NormalTok{(Bar)}\OperatorTok{;}
\CommentTok{// Returns: true}
\end{Highlighting}
\end{Shaded}

\subsubsection{\texorpdfstring{\texttt{util.isNull(object)}}{util.isNull(object)}}\label{util.isnullobject}

\begin{quote}
Stability: 0 - Deprecated: Use \texttt{value\ ===\ null} instead.
\end{quote}

\begin{itemize}
\tightlist
\item
  \texttt{object} \{any\}
\item
  Returns: \{boolean\}
\end{itemize}

Returns \texttt{true} if the given \texttt{object} is strictly
\texttt{null}. Otherwise, returns \texttt{false}.

\begin{Shaded}
\begin{Highlighting}[]
\KeywordTok{const}\NormalTok{ util }\OperatorTok{=} \PreprocessorTok{require}\NormalTok{(}\StringTok{\textquotesingle{}node:util\textquotesingle{}}\NormalTok{)}\OperatorTok{;}

\NormalTok{util}\OperatorTok{.}\FunctionTok{isNull}\NormalTok{(}\DecValTok{0}\NormalTok{)}\OperatorTok{;}
\CommentTok{// Returns: false}
\NormalTok{util}\OperatorTok{.}\FunctionTok{isNull}\NormalTok{(}\KeywordTok{undefined}\NormalTok{)}\OperatorTok{;}
\CommentTok{// Returns: false}
\NormalTok{util}\OperatorTok{.}\FunctionTok{isNull}\NormalTok{(}\KeywordTok{null}\NormalTok{)}\OperatorTok{;}
\CommentTok{// Returns: true}
\end{Highlighting}
\end{Shaded}

\subsubsection{\texorpdfstring{\texttt{util.isNullOrUndefined(object)}}{util.isNullOrUndefined(object)}}\label{util.isnullorundefinedobject}

\begin{quote}
Stability: 0 - Deprecated: Use
\texttt{value\ ===\ undefined\ \textbar{}\textbar{}\ value\ ===\ null}
instead.
\end{quote}

\begin{itemize}
\tightlist
\item
  \texttt{object} \{any\}
\item
  Returns: \{boolean\}
\end{itemize}

Returns \texttt{true} if the given \texttt{object} is \texttt{null} or
\texttt{undefined}. Otherwise, returns \texttt{false}.

\begin{Shaded}
\begin{Highlighting}[]
\KeywordTok{const}\NormalTok{ util }\OperatorTok{=} \PreprocessorTok{require}\NormalTok{(}\StringTok{\textquotesingle{}node:util\textquotesingle{}}\NormalTok{)}\OperatorTok{;}

\NormalTok{util}\OperatorTok{.}\FunctionTok{isNullOrUndefined}\NormalTok{(}\DecValTok{0}\NormalTok{)}\OperatorTok{;}
\CommentTok{// Returns: false}
\NormalTok{util}\OperatorTok{.}\FunctionTok{isNullOrUndefined}\NormalTok{(}\KeywordTok{undefined}\NormalTok{)}\OperatorTok{;}
\CommentTok{// Returns: true}
\NormalTok{util}\OperatorTok{.}\FunctionTok{isNullOrUndefined}\NormalTok{(}\KeywordTok{null}\NormalTok{)}\OperatorTok{;}
\CommentTok{// Returns: true}
\end{Highlighting}
\end{Shaded}

\subsubsection{\texorpdfstring{\texttt{util.isNumber(object)}}{util.isNumber(object)}}\label{util.isnumberobject}

\begin{quote}
Stability: 0 - Deprecated: Use
\texttt{typeof\ value\ ===\ \textquotesingle{}number\textquotesingle{}}
instead.
\end{quote}

\begin{itemize}
\tightlist
\item
  \texttt{object} \{any\}
\item
  Returns: \{boolean\}
\end{itemize}

Returns \texttt{true} if the given \texttt{object} is a \texttt{Number}.
Otherwise, returns \texttt{false}.

\begin{Shaded}
\begin{Highlighting}[]
\KeywordTok{const}\NormalTok{ util }\OperatorTok{=} \PreprocessorTok{require}\NormalTok{(}\StringTok{\textquotesingle{}node:util\textquotesingle{}}\NormalTok{)}\OperatorTok{;}

\NormalTok{util}\OperatorTok{.}\FunctionTok{isNumber}\NormalTok{(}\KeywordTok{false}\NormalTok{)}\OperatorTok{;}
\CommentTok{// Returns: false}
\NormalTok{util}\OperatorTok{.}\FunctionTok{isNumber}\NormalTok{(}\KeywordTok{Infinity}\NormalTok{)}\OperatorTok{;}
\CommentTok{// Returns: true}
\NormalTok{util}\OperatorTok{.}\FunctionTok{isNumber}\NormalTok{(}\DecValTok{0}\NormalTok{)}\OperatorTok{;}
\CommentTok{// Returns: true}
\NormalTok{util}\OperatorTok{.}\FunctionTok{isNumber}\NormalTok{(}\KeywordTok{NaN}\NormalTok{)}\OperatorTok{;}
\CommentTok{// Returns: true}
\end{Highlighting}
\end{Shaded}

\subsubsection{\texorpdfstring{\texttt{util.isObject(object)}}{util.isObject(object)}}\label{util.isobjectobject}

\begin{quote}
Stability: 0 - Deprecated: Use
\texttt{value\ !==\ null\ \&\&\ typeof\ value\ ===\ \textquotesingle{}object\textquotesingle{}}
instead.
\end{quote}

\begin{itemize}
\tightlist
\item
  \texttt{object} \{any\}
\item
  Returns: \{boolean\}
\end{itemize}

Returns \texttt{true} if the given \texttt{object} is strictly an
\texttt{Object} \textbf{and} not a \texttt{Function} (even though
functions are objects in JavaScript). Otherwise, returns \texttt{false}.

\begin{Shaded}
\begin{Highlighting}[]
\KeywordTok{const}\NormalTok{ util }\OperatorTok{=} \PreprocessorTok{require}\NormalTok{(}\StringTok{\textquotesingle{}node:util\textquotesingle{}}\NormalTok{)}\OperatorTok{;}

\NormalTok{util}\OperatorTok{.}\FunctionTok{isObject}\NormalTok{(}\DecValTok{5}\NormalTok{)}\OperatorTok{;}
\CommentTok{// Returns: false}
\NormalTok{util}\OperatorTok{.}\FunctionTok{isObject}\NormalTok{(}\KeywordTok{null}\NormalTok{)}\OperatorTok{;}
\CommentTok{// Returns: false}
\NormalTok{util}\OperatorTok{.}\FunctionTok{isObject}\NormalTok{(\{\})}\OperatorTok{;}
\CommentTok{// Returns: true}
\NormalTok{util}\OperatorTok{.}\FunctionTok{isObject}\NormalTok{(() }\KeywordTok{=\textgreater{}}\NormalTok{ \{\})}\OperatorTok{;}
\CommentTok{// Returns: false}
\end{Highlighting}
\end{Shaded}

\subsubsection{\texorpdfstring{\texttt{util.isPrimitive(object)}}{util.isPrimitive(object)}}\label{util.isprimitiveobject}

\begin{quote}
Stability: 0 - Deprecated: Use
\texttt{(typeof\ value\ !==\ \textquotesingle{}object\textquotesingle{}\ \&\&\ typeof\ value\ !==\ \textquotesingle{}function\textquotesingle{})\ \textbar{}\textbar{}\ value\ ===\ null}
instead.
\end{quote}

\begin{itemize}
\tightlist
\item
  \texttt{object} \{any\}
\item
  Returns: \{boolean\}
\end{itemize}

Returns \texttt{true} if the given \texttt{object} is a primitive type.
Otherwise, returns \texttt{false}.

\begin{Shaded}
\begin{Highlighting}[]
\KeywordTok{const}\NormalTok{ util }\OperatorTok{=} \PreprocessorTok{require}\NormalTok{(}\StringTok{\textquotesingle{}node:util\textquotesingle{}}\NormalTok{)}\OperatorTok{;}

\NormalTok{util}\OperatorTok{.}\FunctionTok{isPrimitive}\NormalTok{(}\DecValTok{5}\NormalTok{)}\OperatorTok{;}
\CommentTok{// Returns: true}
\NormalTok{util}\OperatorTok{.}\FunctionTok{isPrimitive}\NormalTok{(}\StringTok{\textquotesingle{}foo\textquotesingle{}}\NormalTok{)}\OperatorTok{;}
\CommentTok{// Returns: true}
\NormalTok{util}\OperatorTok{.}\FunctionTok{isPrimitive}\NormalTok{(}\KeywordTok{false}\NormalTok{)}\OperatorTok{;}
\CommentTok{// Returns: true}
\NormalTok{util}\OperatorTok{.}\FunctionTok{isPrimitive}\NormalTok{(}\KeywordTok{null}\NormalTok{)}\OperatorTok{;}
\CommentTok{// Returns: true}
\NormalTok{util}\OperatorTok{.}\FunctionTok{isPrimitive}\NormalTok{(}\KeywordTok{undefined}\NormalTok{)}\OperatorTok{;}
\CommentTok{// Returns: true}
\NormalTok{util}\OperatorTok{.}\FunctionTok{isPrimitive}\NormalTok{(\{\})}\OperatorTok{;}
\CommentTok{// Returns: false}
\NormalTok{util}\OperatorTok{.}\FunctionTok{isPrimitive}\NormalTok{(() }\KeywordTok{=\textgreater{}}\NormalTok{ \{\})}\OperatorTok{;}
\CommentTok{// Returns: false}
\NormalTok{util}\OperatorTok{.}\FunctionTok{isPrimitive}\NormalTok{(}\SpecialStringTok{/}\SpecialCharTok{\^{}$}\SpecialStringTok{/}\NormalTok{)}\OperatorTok{;}
\CommentTok{// Returns: false}
\NormalTok{util}\OperatorTok{.}\FunctionTok{isPrimitive}\NormalTok{(}\KeywordTok{new} \BuiltInTok{Date}\NormalTok{())}\OperatorTok{;}
\CommentTok{// Returns: false}
\end{Highlighting}
\end{Shaded}

\subsubsection{\texorpdfstring{\texttt{util.isRegExp(object)}}{util.isRegExp(object)}}\label{util.isregexpobject}

\begin{quote}
Stability: 0 - Deprecated
\end{quote}

\begin{itemize}
\tightlist
\item
  \texttt{object} \{any\}
\item
  Returns: \{boolean\}
\end{itemize}

Returns \texttt{true} if the given \texttt{object} is a \texttt{RegExp}.
Otherwise, returns \texttt{false}.

\begin{Shaded}
\begin{Highlighting}[]
\KeywordTok{const}\NormalTok{ util }\OperatorTok{=} \PreprocessorTok{require}\NormalTok{(}\StringTok{\textquotesingle{}node:util\textquotesingle{}}\NormalTok{)}\OperatorTok{;}

\NormalTok{util}\OperatorTok{.}\FunctionTok{isRegExp}\NormalTok{(}\SpecialStringTok{/some regexp/}\NormalTok{)}\OperatorTok{;}
\CommentTok{// Returns: true}
\NormalTok{util}\OperatorTok{.}\FunctionTok{isRegExp}\NormalTok{(}\KeywordTok{new} \BuiltInTok{RegExp}\NormalTok{(}\StringTok{\textquotesingle{}another regexp\textquotesingle{}}\NormalTok{))}\OperatorTok{;}
\CommentTok{// Returns: true}
\NormalTok{util}\OperatorTok{.}\FunctionTok{isRegExp}\NormalTok{(\{\})}\OperatorTok{;}
\CommentTok{// Returns: false}
\end{Highlighting}
\end{Shaded}

\subsubsection{\texorpdfstring{\texttt{util.isString(object)}}{util.isString(object)}}\label{util.isstringobject}

\begin{quote}
Stability: 0 - Deprecated: Use
\texttt{typeof\ value\ ===\ \textquotesingle{}string\textquotesingle{}}
instead.
\end{quote}

\begin{itemize}
\tightlist
\item
  \texttt{object} \{any\}
\item
  Returns: \{boolean\}
\end{itemize}

Returns \texttt{true} if the given \texttt{object} is a \texttt{string}.
Otherwise, returns \texttt{false}.

\begin{Shaded}
\begin{Highlighting}[]
\KeywordTok{const}\NormalTok{ util }\OperatorTok{=} \PreprocessorTok{require}\NormalTok{(}\StringTok{\textquotesingle{}node:util\textquotesingle{}}\NormalTok{)}\OperatorTok{;}

\NormalTok{util}\OperatorTok{.}\FunctionTok{isString}\NormalTok{(}\StringTok{\textquotesingle{}\textquotesingle{}}\NormalTok{)}\OperatorTok{;}
\CommentTok{// Returns: true}
\NormalTok{util}\OperatorTok{.}\FunctionTok{isString}\NormalTok{(}\StringTok{\textquotesingle{}foo\textquotesingle{}}\NormalTok{)}\OperatorTok{;}
\CommentTok{// Returns: true}
\NormalTok{util}\OperatorTok{.}\FunctionTok{isString}\NormalTok{(}\BuiltInTok{String}\NormalTok{(}\StringTok{\textquotesingle{}foo\textquotesingle{}}\NormalTok{))}\OperatorTok{;}
\CommentTok{// Returns: true}
\NormalTok{util}\OperatorTok{.}\FunctionTok{isString}\NormalTok{(}\DecValTok{5}\NormalTok{)}\OperatorTok{;}
\CommentTok{// Returns: false}
\end{Highlighting}
\end{Shaded}

\subsubsection{\texorpdfstring{\texttt{util.isSymbol(object)}}{util.isSymbol(object)}}\label{util.issymbolobject}

\begin{quote}
Stability: 0 - Deprecated: Use
\texttt{typeof\ value\ ===\ \textquotesingle{}symbol\textquotesingle{}}
instead.
\end{quote}

\begin{itemize}
\tightlist
\item
  \texttt{object} \{any\}
\item
  Returns: \{boolean\}
\end{itemize}

Returns \texttt{true} if the given \texttt{object} is a \texttt{Symbol}.
Otherwise, returns \texttt{false}.

\begin{Shaded}
\begin{Highlighting}[]
\KeywordTok{const}\NormalTok{ util }\OperatorTok{=} \PreprocessorTok{require}\NormalTok{(}\StringTok{\textquotesingle{}node:util\textquotesingle{}}\NormalTok{)}\OperatorTok{;}

\NormalTok{util}\OperatorTok{.}\FunctionTok{isSymbol}\NormalTok{(}\DecValTok{5}\NormalTok{)}\OperatorTok{;}
\CommentTok{// Returns: false}
\NormalTok{util}\OperatorTok{.}\FunctionTok{isSymbol}\NormalTok{(}\StringTok{\textquotesingle{}foo\textquotesingle{}}\NormalTok{)}\OperatorTok{;}
\CommentTok{// Returns: false}
\NormalTok{util}\OperatorTok{.}\FunctionTok{isSymbol}\NormalTok{(}\BuiltInTok{Symbol}\NormalTok{(}\StringTok{\textquotesingle{}foo\textquotesingle{}}\NormalTok{))}\OperatorTok{;}
\CommentTok{// Returns: true}
\end{Highlighting}
\end{Shaded}

\subsubsection{\texorpdfstring{\texttt{util.isUndefined(object)}}{util.isUndefined(object)}}\label{util.isundefinedobject}

\begin{quote}
Stability: 0 - Deprecated: Use \texttt{value\ ===\ undefined} instead.
\end{quote}

\begin{itemize}
\tightlist
\item
  \texttt{object} \{any\}
\item
  Returns: \{boolean\}
\end{itemize}

Returns \texttt{true} if the given \texttt{object} is
\texttt{undefined}. Otherwise, returns \texttt{false}.

\begin{Shaded}
\begin{Highlighting}[]
\KeywordTok{const}\NormalTok{ util }\OperatorTok{=} \PreprocessorTok{require}\NormalTok{(}\StringTok{\textquotesingle{}node:util\textquotesingle{}}\NormalTok{)}\OperatorTok{;}

\KeywordTok{const}\NormalTok{ foo }\OperatorTok{=} \KeywordTok{undefined}\OperatorTok{;}
\NormalTok{util}\OperatorTok{.}\FunctionTok{isUndefined}\NormalTok{(}\DecValTok{5}\NormalTok{)}\OperatorTok{;}
\CommentTok{// Returns: false}
\NormalTok{util}\OperatorTok{.}\FunctionTok{isUndefined}\NormalTok{(foo)}\OperatorTok{;}
\CommentTok{// Returns: true}
\NormalTok{util}\OperatorTok{.}\FunctionTok{isUndefined}\NormalTok{(}\KeywordTok{null}\NormalTok{)}\OperatorTok{;}
\CommentTok{// Returns: false}
\end{Highlighting}
\end{Shaded}

\subsubsection{\texorpdfstring{\texttt{util.log(string)}}{util.log(string)}}\label{util.logstring}

\begin{quote}
Stability: 0 - Deprecated: Use a third party module instead.
\end{quote}

\begin{itemize}
\tightlist
\item
  \texttt{string} \{string\}
\end{itemize}

The \texttt{util.log()} method prints the given \texttt{string} to
\texttt{stdout} with an included timestamp.

\begin{Shaded}
\begin{Highlighting}[]
\KeywordTok{const}\NormalTok{ util }\OperatorTok{=} \PreprocessorTok{require}\NormalTok{(}\StringTok{\textquotesingle{}node:util\textquotesingle{}}\NormalTok{)}\OperatorTok{;}

\NormalTok{util}\OperatorTok{.}\FunctionTok{log}\NormalTok{(}\StringTok{\textquotesingle{}Timestamped message.\textquotesingle{}}\NormalTok{)}\OperatorTok{;}
\end{Highlighting}
\end{Shaded}
