\section{C++ addons}\label{c-addons}

\emph{Addons} are dynamically-linked shared objects written in C++. The
\href{modules.md\#requireid}{\texttt{require()}} function can load
addons as ordinary Node.js modules. Addons provide an interface between
JavaScript and C/C++ libraries.

There are three options for implementing addons: Node-API, nan, or
direct use of internal V8, libuv, and Node.js libraries. Unless there is
a need for direct access to functionality which is not exposed by
Node-API, use Node-API. Refer to \href{n-api.md}{C/C++ addons with
Node-API} for more information on Node-API.

When not using Node-API, implementing addons is complicated, involving
knowledge of several components and APIs:

\begin{itemize}
\item
  \href{https://v8.dev/}{V8}: the C++ library Node.js uses to provide
  the JavaScript implementation. V8 provides the mechanisms for creating
  objects, calling functions, etc. V8's API is documented mostly in the
  \texttt{v8.h} header file (\texttt{deps/v8/include/v8.h} in the
  Node.js source tree), which is also available
  \href{https://v8docs.nodesource.com/}{online}.
\item
  \href{https://github.com/libuv/libuv}{libuv}: The C library that
  implements the Node.js event loop, its worker threads and all of the
  asynchronous behaviors of the platform. It also serves as a
  cross-platform abstraction library, giving easy, POSIX-like access
  across all major operating systems to many common system tasks, such
  as interacting with the file system, sockets, timers, and system
  events. libuv also provides a threading abstraction similar to POSIX
  threads for more sophisticated asynchronous addons that need to move
  beyond the standard event loop. Addon authors should avoid blocking
  the event loop with I/O or other time-intensive tasks by offloading
  work via libuv to non-blocking system operations, worker threads, or a
  custom use of libuv threads.
\item
  Internal Node.js libraries. Node.js itself exports C++ APIs that
  addons can use, the most important of which is the
  \texttt{node::ObjectWrap} class.
\item
  Node.js includes other statically linked libraries including OpenSSL.
  These other libraries are located in the \texttt{deps/} directory in
  the Node.js source tree. Only the libuv, OpenSSL, V8, and zlib symbols
  are purposefully re-exported by Node.js and may be used to various
  extents by addons. See
  \hyperref[linking-to-libraries-included-with-nodejs]{Linking to
  libraries included with Node.js} for additional information.
\end{itemize}

All of the following examples are available for
\href{https://github.com/nodejs/node-addon-examples}{download} and may
be used as the starting-point for an addon.

\subsection{Hello world}\label{hello-world}

This ``Hello world'' example is a simple addon, written in C++, that is
the equivalent of the following JavaScript code:

\begin{Shaded}
\begin{Highlighting}[]
\NormalTok{module}\OperatorTok{.}\AttributeTok{exports}\OperatorTok{.}\AttributeTok{hello} \OperatorTok{=}\NormalTok{ () }\KeywordTok{=\textgreater{}} \StringTok{\textquotesingle{}world\textquotesingle{}}\OperatorTok{;}
\end{Highlighting}
\end{Shaded}

First, create the file \texttt{hello.cc}:

\begin{Shaded}
\begin{Highlighting}[]
\CommentTok{// hello.cc}
\PreprocessorTok{\#include }\ImportTok{\textless{}node.h\textgreater{}}

\KeywordTok{namespace}\NormalTok{ demo }\OperatorTok{\{}

\KeywordTok{using}\NormalTok{ v8}\OperatorTok{::}\NormalTok{FunctionCallbackInfo}\OperatorTok{;}
\KeywordTok{using}\NormalTok{ v8}\OperatorTok{::}\NormalTok{Isolate}\OperatorTok{;}
\KeywordTok{using}\NormalTok{ v8}\OperatorTok{::}\NormalTok{Local}\OperatorTok{;}
\KeywordTok{using}\NormalTok{ v8}\OperatorTok{::}\NormalTok{Object}\OperatorTok{;}
\KeywordTok{using}\NormalTok{ v8}\OperatorTok{::}\NormalTok{String}\OperatorTok{;}
\KeywordTok{using}\NormalTok{ v8}\OperatorTok{::}\NormalTok{Value}\OperatorTok{;}

\DataTypeTok{void}\NormalTok{ Method}\OperatorTok{(}\AttributeTok{const}\NormalTok{ FunctionCallbackInfo}\OperatorTok{\textless{}}\NormalTok{Value}\OperatorTok{\textgreater{}\&}\NormalTok{ args}\OperatorTok{)} \OperatorTok{\{}
\NormalTok{  Isolate}\OperatorTok{*}\NormalTok{ isolate }\OperatorTok{=}\NormalTok{ args}\OperatorTok{.}\NormalTok{GetIsolate}\OperatorTok{();}
\NormalTok{  args}\OperatorTok{.}\NormalTok{GetReturnValue}\OperatorTok{().}\NormalTok{Set}\OperatorTok{(}\NormalTok{String}\OperatorTok{::}\NormalTok{NewFromUtf8}\OperatorTok{(}
\NormalTok{      isolate}\OperatorTok{,} \StringTok{"world"}\OperatorTok{).}\NormalTok{ToLocalChecked}\OperatorTok{());}
\OperatorTok{\}}

\DataTypeTok{void}\NormalTok{ Initialize}\OperatorTok{(}\NormalTok{Local}\OperatorTok{\textless{}}\NormalTok{Object}\OperatorTok{\textgreater{}}\NormalTok{ exports}\OperatorTok{)} \OperatorTok{\{}
\NormalTok{  NODE\_SET\_METHOD}\OperatorTok{(}\NormalTok{exports}\OperatorTok{,} \StringTok{"hello"}\OperatorTok{,}\NormalTok{ Method}\OperatorTok{);}
\OperatorTok{\}}

\NormalTok{NODE\_MODULE}\OperatorTok{(}\NormalTok{NODE\_GYP\_MODULE\_NAME}\OperatorTok{,}\NormalTok{ Initialize}\OperatorTok{)}

\OperatorTok{\}}  \CommentTok{// namespace demo}
\end{Highlighting}
\end{Shaded}

All Node.js addons must export an initialization function following the
pattern:

\begin{Shaded}
\begin{Highlighting}[]
\DataTypeTok{void}\NormalTok{ Initialize}\OperatorTok{(}\NormalTok{Local}\OperatorTok{\textless{}}\NormalTok{Object}\OperatorTok{\textgreater{}}\NormalTok{ exports}\OperatorTok{);}
\NormalTok{NODE\_MODULE}\OperatorTok{(}\NormalTok{NODE\_GYP\_MODULE\_NAME}\OperatorTok{,}\NormalTok{ Initialize}\OperatorTok{)}
\end{Highlighting}
\end{Shaded}

There is no semi-colon after \texttt{NODE\_MODULE} as it's not a
function (see \texttt{node.h}).

The \texttt{module\_name} must match the filename of the final binary
(excluding the \texttt{.node} suffix).

In the \texttt{hello.cc} example, then, the initialization function is
\texttt{Initialize} and the addon module name is \texttt{addon}.

When building addons with \texttt{node-gyp}, using the macro
\texttt{NODE\_GYP\_MODULE\_NAME} as the first parameter of
\texttt{NODE\_MODULE()} will ensure that the name of the final binary
will be passed to \texttt{NODE\_MODULE()}.

Addons defined with \texttt{NODE\_MODULE()} can not be loaded in
multiple contexts or multiple threads at the same time.

\subsubsection{Context-aware addons}\label{context-aware-addons}

There are environments in which Node.js addons may need to be loaded
multiple times in multiple contexts. For example, the
\href{https://electronjs.org/}{Electron} runtime runs multiple instances
of Node.js in a single process. Each instance will have its own
\texttt{require()} cache, and thus each instance will need a native
addon to behave correctly when loaded via \texttt{require()}. This means
that the addon must support multiple initializations.

A context-aware addon can be constructed by using the macro
\texttt{NODE\_MODULE\_INITIALIZER}, which expands to the name of a
function which Node.js will expect to find when it loads an addon. An
addon can thus be initialized as in the following example:

\begin{Shaded}
\begin{Highlighting}[]
\KeywordTok{using} \KeywordTok{namespace}\NormalTok{ v8}\OperatorTok{;}

\AttributeTok{extern} \StringTok{"C"}\NormalTok{ NODE\_MODULE\_EXPORT }\DataTypeTok{void}
\NormalTok{NODE\_MODULE\_INITIALIZER}\OperatorTok{(}\NormalTok{Local}\OperatorTok{\textless{}}\NormalTok{Object}\OperatorTok{\textgreater{}}\NormalTok{ exports}\OperatorTok{,}
\NormalTok{                        Local}\OperatorTok{\textless{}}\NormalTok{Value}\OperatorTok{\textgreater{}} \KeywordTok{module}\OperatorTok{,}
\NormalTok{                        Local}\OperatorTok{\textless{}}\NormalTok{Context}\OperatorTok{\textgreater{}}\NormalTok{ context}\OperatorTok{)} \OperatorTok{\{}
  \CommentTok{/* Perform addon initialization steps here. */}
\OperatorTok{\}}
\end{Highlighting}
\end{Shaded}

Another option is to use the macro \texttt{NODE\_MODULE\_INIT()}, which
will also construct a context-aware addon. Unlike
\texttt{NODE\_MODULE()}, which is used to construct an addon around a
given addon initializer function, \texttt{NODE\_MODULE\_INIT()} serves
as the declaration of such an initializer to be followed by a function
body.

The following three variables may be used inside the function body
following an invocation of \texttt{NODE\_MODULE\_INIT()}:

\begin{itemize}
\tightlist
\item
  \texttt{Local\textless{}Object\textgreater{}\ exports},
\item
  \texttt{Local\textless{}Value\textgreater{}\ module}, and
\item
  \texttt{Local\textless{}Context\textgreater{}\ context}
\end{itemize}

The choice to build a context-aware addon carries with it the
responsibility of carefully managing global static data. Since the addon
may be loaded multiple times, potentially even from different threads,
any global static data stored in the addon must be properly protected,
and must not contain any persistent references to JavaScript objects.
The reason for this is that JavaScript objects are only valid in one
context, and will likely cause a crash when accessed from the wrong
context or from a different thread than the one on which they were
created.

The context-aware addon can be structured to avoid global static data by
performing the following steps:

\begin{itemize}
\item
  Define a class which will hold per-addon-instance data and which has a
  static member of the form

\begin{Shaded}
\begin{Highlighting}[]
\AttributeTok{static} \DataTypeTok{void}\NormalTok{ DeleteInstance}\OperatorTok{(}\DataTypeTok{void}\OperatorTok{*}\NormalTok{ data}\OperatorTok{)} \OperatorTok{\{}
  \CommentTok{// Cast \textasciigrave{}data\textasciigrave{} to an instance of the class and delete it.}
\OperatorTok{\}}
\end{Highlighting}
\end{Shaded}
\item
  Heap-allocate an instance of this class in the addon initializer. This
  can be accomplished using the \texttt{new} keyword.
\item
  Call \texttt{node::AddEnvironmentCleanupHook()}, passing it the
  above-created instance and a pointer to \texttt{DeleteInstance()}.
  This will ensure the instance is deleted when the environment is torn
  down.
\item
  Store the instance of the class in a \texttt{v8::External}, and
\item
  Pass the \texttt{v8::External} to all methods exposed to JavaScript by
  passing it to \texttt{v8::FunctionTemplate::New()} or
  \texttt{v8::Function::New()} which creates the native-backed
  JavaScript functions. The third parameter of
  \texttt{v8::FunctionTemplate::New()} or \texttt{v8::Function::New()}
  accepts the \texttt{v8::External} and makes it available in the native
  callback using the \texttt{v8::FunctionCallbackInfo::Data()} method.
\end{itemize}

This will ensure that the per-addon-instance data reaches each binding
that can be called from JavaScript. The per-addon-instance data must
also be passed into any asynchronous callbacks the addon may create.

The following example illustrates the implementation of a context-aware
addon:

\begin{Shaded}
\begin{Highlighting}[]
\PreprocessorTok{\#include }\ImportTok{\textless{}node.h\textgreater{}}

\KeywordTok{using} \KeywordTok{namespace}\NormalTok{ v8}\OperatorTok{;}

\KeywordTok{class}\NormalTok{ AddonData }\OperatorTok{\{}
 \KeywordTok{public}\OperatorTok{:}
  \KeywordTok{explicit}\NormalTok{ AddonData}\OperatorTok{(}\NormalTok{Isolate}\OperatorTok{*}\NormalTok{ isolate}\OperatorTok{):}
\NormalTok{      call\_count}\OperatorTok{(}\DecValTok{0}\OperatorTok{)} \OperatorTok{\{}
    \CommentTok{// Ensure this per{-}addon{-}instance data is deleted at environment cleanup.}
\NormalTok{    node}\OperatorTok{::}\NormalTok{AddEnvironmentCleanupHook}\OperatorTok{(}\NormalTok{isolate}\OperatorTok{,}\NormalTok{ DeleteInstance}\OperatorTok{,} \KeywordTok{this}\OperatorTok{);}
  \OperatorTok{\}}

  \CommentTok{// Per{-}addon data.}
  \DataTypeTok{int}\NormalTok{ call\_count}\OperatorTok{;}

  \AttributeTok{static} \DataTypeTok{void}\NormalTok{ DeleteInstance}\OperatorTok{(}\DataTypeTok{void}\OperatorTok{*}\NormalTok{ data}\OperatorTok{)} \OperatorTok{\{}
    \KeywordTok{delete} \KeywordTok{static\_cast}\OperatorTok{\textless{}}\NormalTok{AddonData}\OperatorTok{*\textgreater{}(}\NormalTok{data}\OperatorTok{);}
  \OperatorTok{\}}
\OperatorTok{\};}

\AttributeTok{static} \DataTypeTok{void}\NormalTok{ Method}\OperatorTok{(}\AttributeTok{const}\NormalTok{ v8}\OperatorTok{::}\NormalTok{FunctionCallbackInfo}\OperatorTok{\textless{}}\NormalTok{v8}\OperatorTok{::}\NormalTok{Value}\OperatorTok{\textgreater{}\&}\NormalTok{ info}\OperatorTok{)} \OperatorTok{\{}
  \CommentTok{// Retrieve the per{-}addon{-}instance data.}
\NormalTok{  AddonData}\OperatorTok{*}\NormalTok{ data }\OperatorTok{=}
      \KeywordTok{reinterpret\_cast}\OperatorTok{\textless{}}\NormalTok{AddonData}\OperatorTok{*\textgreater{}(}\NormalTok{info}\OperatorTok{.}\NormalTok{Data}\OperatorTok{().}\NormalTok{As}\OperatorTok{\textless{}}\NormalTok{External}\OperatorTok{\textgreater{}(){-}\textgreater{}}\NormalTok{Value}\OperatorTok{());}
\NormalTok{  data}\OperatorTok{{-}\textgreater{}}\NormalTok{call\_count}\OperatorTok{++;}
\NormalTok{  info}\OperatorTok{.}\NormalTok{GetReturnValue}\OperatorTok{().}\NormalTok{Set}\OperatorTok{((}\DataTypeTok{double}\OperatorTok{)}\NormalTok{data}\OperatorTok{{-}\textgreater{}}\NormalTok{call\_count}\OperatorTok{);}
\OperatorTok{\}}

\CommentTok{// Initialize this addon to be context{-}aware.}
\NormalTok{NODE\_MODULE\_INIT}\OperatorTok{(}\CommentTok{/* exports, module, context */}\OperatorTok{)} \OperatorTok{\{}
\NormalTok{  Isolate}\OperatorTok{*}\NormalTok{ isolate }\OperatorTok{=}\NormalTok{ context}\OperatorTok{{-}\textgreater{}}\NormalTok{GetIsolate}\OperatorTok{();}

  \CommentTok{// Create a new instance of \textasciigrave{}AddonData\textasciigrave{} for this instance of the addon and}
  \CommentTok{// tie its life cycle to that of the Node.js environment.}
\NormalTok{  AddonData}\OperatorTok{*}\NormalTok{ data }\OperatorTok{=} \KeywordTok{new}\NormalTok{ AddonData}\OperatorTok{(}\NormalTok{isolate}\OperatorTok{);}

  \CommentTok{// Wrap the data in a \textasciigrave{}v8::External\textasciigrave{} so we can pass it to the method we}
  \CommentTok{// expose.}
\NormalTok{  Local}\OperatorTok{\textless{}}\NormalTok{External}\OperatorTok{\textgreater{}}\NormalTok{ external }\OperatorTok{=}\NormalTok{ External}\OperatorTok{::}\NormalTok{New}\OperatorTok{(}\NormalTok{isolate}\OperatorTok{,}\NormalTok{ data}\OperatorTok{);}

  \CommentTok{// Expose the method \textasciigrave{}Method\textasciigrave{} to JavaScript, and make sure it receives the}
  \CommentTok{// per{-}addon{-}instance data we created above by passing \textasciigrave{}external\textasciigrave{} as the}
  \CommentTok{// third parameter to the \textasciigrave{}FunctionTemplate\textasciigrave{} constructor.}
\NormalTok{  exports}\OperatorTok{{-}\textgreater{}}\NormalTok{Set}\OperatorTok{(}\NormalTok{context}\OperatorTok{,}
\NormalTok{               String}\OperatorTok{::}\NormalTok{NewFromUtf8}\OperatorTok{(}\NormalTok{isolate}\OperatorTok{,} \StringTok{"method"}\OperatorTok{).}\NormalTok{ToLocalChecked}\OperatorTok{(),}
\NormalTok{               FunctionTemplate}\OperatorTok{::}\NormalTok{New}\OperatorTok{(}\NormalTok{isolate}\OperatorTok{,}\NormalTok{ Method}\OperatorTok{,}\NormalTok{ external}\OperatorTok{)}
                  \OperatorTok{{-}\textgreater{}}\NormalTok{GetFunction}\OperatorTok{(}\NormalTok{context}\OperatorTok{).}\NormalTok{ToLocalChecked}\OperatorTok{()).}\NormalTok{FromJust}\OperatorTok{();}
\OperatorTok{\}}
\end{Highlighting}
\end{Shaded}

\paragraph{Worker support}\label{worker-support}

In order to be loaded from multiple Node.js environments, such as a main
thread and a Worker thread, an add-on needs to either:

\begin{itemize}
\tightlist
\item
  Be an Node-API addon, or
\item
  Be declared as context-aware using \texttt{NODE\_MODULE\_INIT()} as
  described above
\end{itemize}

In order to support
\href{worker_threads.md\#class-worker}{\texttt{Worker}} threads, addons
need to clean up any resources they may have allocated when such a
thread exists. This can be achieved through the usage of the
\texttt{AddEnvironmentCleanupHook()} function:

\begin{Shaded}
\begin{Highlighting}[]
\DataTypeTok{void}\NormalTok{ AddEnvironmentCleanupHook}\OperatorTok{(}\NormalTok{v8}\OperatorTok{::}\NormalTok{Isolate}\OperatorTok{*}\NormalTok{ isolate}\OperatorTok{,}
                               \DataTypeTok{void} \OperatorTok{(*}\NormalTok{fun}\OperatorTok{)(}\DataTypeTok{void}\OperatorTok{*}\NormalTok{ arg}\OperatorTok{),}
                               \DataTypeTok{void}\OperatorTok{*}\NormalTok{ arg}\OperatorTok{);}
\end{Highlighting}
\end{Shaded}

This function adds a hook that will run before a given Node.js instance
shuts down. If necessary, such hooks can be removed before they are run
using \texttt{RemoveEnvironmentCleanupHook()}, which has the same
signature. Callbacks are run in last-in first-out order.

If necessary, there is an additional pair of
\texttt{AddEnvironmentCleanupHook()} and
\texttt{RemoveEnvironmentCleanupHook()} overloads, where the cleanup
hook takes a callback function. This can be used for shutting down
asynchronous resources, such as any libuv handles registered by the
addon.

The following \texttt{addon.cc} uses \texttt{AddEnvironmentCleanupHook}:

\begin{Shaded}
\begin{Highlighting}[]
\CommentTok{// addon.cc}
\PreprocessorTok{\#include }\ImportTok{\textless{}node.h\textgreater{}}
\PreprocessorTok{\#include }\ImportTok{\textless{}assert.h\textgreater{}}
\PreprocessorTok{\#include }\ImportTok{\textless{}stdlib.h\textgreater{}}

\KeywordTok{using}\NormalTok{ node}\OperatorTok{::}\NormalTok{AddEnvironmentCleanupHook}\OperatorTok{;}
\KeywordTok{using}\NormalTok{ v8}\OperatorTok{::}\NormalTok{HandleScope}\OperatorTok{;}
\KeywordTok{using}\NormalTok{ v8}\OperatorTok{::}\NormalTok{Isolate}\OperatorTok{;}
\KeywordTok{using}\NormalTok{ v8}\OperatorTok{::}\NormalTok{Local}\OperatorTok{;}
\KeywordTok{using}\NormalTok{ v8}\OperatorTok{::}\NormalTok{Object}\OperatorTok{;}

\CommentTok{// Note: In a real{-}world application, do not rely on static/global data.}
\AttributeTok{static} \DataTypeTok{char}\NormalTok{ cookie}\OperatorTok{[]} \OperatorTok{=} \StringTok{"yum yum"}\OperatorTok{;}
\AttributeTok{static} \DataTypeTok{int}\NormalTok{ cleanup\_cb1\_called }\OperatorTok{=} \DecValTok{0}\OperatorTok{;}
\AttributeTok{static} \DataTypeTok{int}\NormalTok{ cleanup\_cb2\_called }\OperatorTok{=} \DecValTok{0}\OperatorTok{;}

\AttributeTok{static} \DataTypeTok{void}\NormalTok{ cleanup\_cb1}\OperatorTok{(}\DataTypeTok{void}\OperatorTok{*}\NormalTok{ arg}\OperatorTok{)} \OperatorTok{\{}
\NormalTok{  Isolate}\OperatorTok{*}\NormalTok{ isolate }\OperatorTok{=} \KeywordTok{static\_cast}\OperatorTok{\textless{}}\NormalTok{Isolate}\OperatorTok{*\textgreater{}(}\NormalTok{arg}\OperatorTok{);}
\NormalTok{  HandleScope scope}\OperatorTok{(}\NormalTok{isolate}\OperatorTok{);}
\NormalTok{  Local}\OperatorTok{\textless{}}\NormalTok{Object}\OperatorTok{\textgreater{}}\NormalTok{ obj }\OperatorTok{=}\NormalTok{ Object}\OperatorTok{::}\NormalTok{New}\OperatorTok{(}\NormalTok{isolate}\OperatorTok{);}
  \OtherTok{assert}\OperatorTok{(!}\NormalTok{obj}\OperatorTok{.}\NormalTok{IsEmpty}\OperatorTok{());}  \CommentTok{// assert VM is still alive}
  \OtherTok{assert}\OperatorTok{(}\NormalTok{obj}\OperatorTok{{-}\textgreater{}}\NormalTok{IsObject}\OperatorTok{());}
\NormalTok{  cleanup\_cb1\_called}\OperatorTok{++;}
\OperatorTok{\}}

\AttributeTok{static} \DataTypeTok{void}\NormalTok{ cleanup\_cb2}\OperatorTok{(}\DataTypeTok{void}\OperatorTok{*}\NormalTok{ arg}\OperatorTok{)} \OperatorTok{\{}
  \OtherTok{assert}\OperatorTok{(}\NormalTok{arg }\OperatorTok{==} \KeywordTok{static\_cast}\OperatorTok{\textless{}}\DataTypeTok{void}\OperatorTok{*\textgreater{}(}\NormalTok{cookie}\OperatorTok{));}
\NormalTok{  cleanup\_cb2\_called}\OperatorTok{++;}
\OperatorTok{\}}

\AttributeTok{static} \DataTypeTok{void}\NormalTok{ sanity\_check}\OperatorTok{(}\DataTypeTok{void}\OperatorTok{*)} \OperatorTok{\{}
  \OtherTok{assert}\OperatorTok{(}\NormalTok{cleanup\_cb1\_called }\OperatorTok{==} \DecValTok{1}\OperatorTok{);}
  \OtherTok{assert}\OperatorTok{(}\NormalTok{cleanup\_cb2\_called }\OperatorTok{==} \DecValTok{1}\OperatorTok{);}
\OperatorTok{\}}

\CommentTok{// Initialize this addon to be context{-}aware.}
\NormalTok{NODE\_MODULE\_INIT}\OperatorTok{(}\CommentTok{/* exports, module, context */}\OperatorTok{)} \OperatorTok{\{}
\NormalTok{  Isolate}\OperatorTok{*}\NormalTok{ isolate }\OperatorTok{=}\NormalTok{ context}\OperatorTok{{-}\textgreater{}}\NormalTok{GetIsolate}\OperatorTok{();}

\NormalTok{  AddEnvironmentCleanupHook}\OperatorTok{(}\NormalTok{isolate}\OperatorTok{,}\NormalTok{ sanity\_check}\OperatorTok{,} \KeywordTok{nullptr}\OperatorTok{);}
\NormalTok{  AddEnvironmentCleanupHook}\OperatorTok{(}\NormalTok{isolate}\OperatorTok{,}\NormalTok{ cleanup\_cb2}\OperatorTok{,}\NormalTok{ cookie}\OperatorTok{);}
\NormalTok{  AddEnvironmentCleanupHook}\OperatorTok{(}\NormalTok{isolate}\OperatorTok{,}\NormalTok{ cleanup\_cb1}\OperatorTok{,}\NormalTok{ isolate}\OperatorTok{);}
\OperatorTok{\}}
\end{Highlighting}
\end{Shaded}

Test in JavaScript by running:

\begin{Shaded}
\begin{Highlighting}[]
\CommentTok{// test.js}
\PreprocessorTok{require}\NormalTok{(}\StringTok{\textquotesingle{}./build/Release/addon\textquotesingle{}}\NormalTok{)}\OperatorTok{;}
\end{Highlighting}
\end{Shaded}

\subsubsection{Building}\label{building}

Once the source code has been written, it must be compiled into the
binary \texttt{addon.node} file. To do so, create a file called
\texttt{binding.gyp} in the top-level of the project describing the
build configuration of the module using a JSON-like format. This file is
used by \href{https://github.com/nodejs/node-gyp}{node-gyp}, a tool
written specifically to compile Node.js addons.

\begin{Shaded}
\begin{Highlighting}[]
\FunctionTok{\{}
  \DataTypeTok{"targets"}\FunctionTok{:} \OtherTok{[}
    \FunctionTok{\{}
      \DataTypeTok{"target\_name"}\FunctionTok{:} \StringTok{"addon"}\FunctionTok{,}
      \DataTypeTok{"sources"}\FunctionTok{:} \OtherTok{[} \StringTok{"hello.cc"} \OtherTok{]}
    \FunctionTok{\}}
  \OtherTok{]}
\FunctionTok{\}}
\end{Highlighting}
\end{Shaded}

A version of the \texttt{node-gyp} utility is bundled and distributed
with Node.js as part of \texttt{npm}. This version is not made directly
available for developers to use and is intended only to support the
ability to use the \texttt{npm\ install} command to compile and install
addons. Developers who wish to use \texttt{node-gyp} directly can
install it using the command \texttt{npm\ install\ -g\ node-gyp}. See
the \texttt{node-gyp}
\href{https://github.com/nodejs/node-gyp\#installation}{installation
instructions} for more information, including platform-specific
requirements.

Once the \texttt{binding.gyp} file has been created, use
\texttt{node-gyp\ configure} to generate the appropriate project build
files for the current platform. This will generate either a
\texttt{Makefile} (on Unix platforms) or a \texttt{vcxproj} file (on
Windows) in the \texttt{build/} directory.

Next, invoke the \texttt{node-gyp\ build} command to generate the
compiled \texttt{addon.node} file. This will be put into the
\texttt{build/Release/} directory.

When using \texttt{npm\ install} to install a Node.js addon, npm uses
its own bundled version of \texttt{node-gyp} to perform this same set of
actions, generating a compiled version of the addon for the user's
platform on demand.

Once built, the binary addon can be used from within Node.js by pointing
\href{modules.md\#requireid}{\texttt{require()}} to the built
\texttt{addon.node} module:

\begin{Shaded}
\begin{Highlighting}[]
\CommentTok{// hello.js}
\KeywordTok{const}\NormalTok{ addon }\OperatorTok{=} \PreprocessorTok{require}\NormalTok{(}\StringTok{\textquotesingle{}./build/Release/addon\textquotesingle{}}\NormalTok{)}\OperatorTok{;}

\BuiltInTok{console}\OperatorTok{.}\FunctionTok{log}\NormalTok{(addon}\OperatorTok{.}\FunctionTok{hello}\NormalTok{())}\OperatorTok{;}
\CommentTok{// Prints: \textquotesingle{}world\textquotesingle{}}
\end{Highlighting}
\end{Shaded}

Because the exact path to the compiled addon binary can vary depending
on how it is compiled (i.e.~sometimes it may be in
\texttt{./build/Debug/}), addons can use the
\href{https://github.com/TooTallNate/node-bindings}{bindings} package to
load the compiled module.

While the \texttt{bindings} package implementation is more sophisticated
in how it locates addon modules, it is essentially using a
\texttt{try…catch} pattern similar to:

\begin{Shaded}
\begin{Highlighting}[]
\ControlFlowTok{try}\NormalTok{ \{}
  \ControlFlowTok{return} \PreprocessorTok{require}\NormalTok{(}\StringTok{\textquotesingle{}./build/Release/addon.node\textquotesingle{}}\NormalTok{)}\OperatorTok{;}
\NormalTok{\} }\ControlFlowTok{catch}\NormalTok{ (err) \{}
  \ControlFlowTok{return} \PreprocessorTok{require}\NormalTok{(}\StringTok{\textquotesingle{}./build/Debug/addon.node\textquotesingle{}}\NormalTok{)}\OperatorTok{;}
\NormalTok{\}}
\end{Highlighting}
\end{Shaded}

\subsubsection{Linking to libraries included with
Node.js}\label{linking-to-libraries-included-with-node.js}

Node.js uses statically linked libraries such as V8, libuv, and OpenSSL.
All addons are required to link to V8 and may link to any of the other
dependencies as well. Typically, this is as simple as including the
appropriate \texttt{\#include\ \textless{}...\textgreater{}} statements
(e.g.~\texttt{\#include\ \textless{}v8.h\textgreater{}}) and
\texttt{node-gyp} will locate the appropriate headers automatically.
However, there are a few caveats to be aware of:

\begin{itemize}
\item
  When \texttt{node-gyp} runs, it will detect the specific release
  version of Node.js and download either the full source tarball or just
  the headers. If the full source is downloaded, addons will have
  complete access to the full set of Node.js dependencies. However, if
  only the Node.js headers are downloaded, then only the symbols
  exported by Node.js will be available.
\item
  \texttt{node-gyp} can be run using the \texttt{-\/-nodedir} flag
  pointing at a local Node.js source image. Using this option, the addon
  will have access to the full set of dependencies.
\end{itemize}

\subsubsection{\texorpdfstring{Loading addons using
\texttt{require()}}{Loading addons using require()}}\label{loading-addons-using-require}

The filename extension of the compiled addon binary is \texttt{.node}
(as opposed to \texttt{.dll} or \texttt{.so}). The
\href{modules.md\#requireid}{\texttt{require()}} function is written to
look for files with the \texttt{.node} file extension and initialize
those as dynamically-linked libraries.

When calling \href{modules.md\#requireid}{\texttt{require()}}, the
\texttt{.node} extension can usually be omitted and Node.js will still
find and initialize the addon. One caveat, however, is that Node.js will
first attempt to locate and load modules or JavaScript files that happen
to share the same base name. For instance, if there is a file
\texttt{addon.js} in the same directory as the binary
\texttt{addon.node}, then
\href{modules.md\#requireid}{\texttt{require(\textquotesingle{}addon\textquotesingle{})}}
will give precedence to the \texttt{addon.js} file and load it instead.

\subsection{Native abstractions for
Node.js}\label{native-abstractions-for-node.js}

Each of the examples illustrated in this document directly use the
Node.js and V8 APIs for implementing addons. The V8 API can, and has,
changed dramatically from one V8 release to the next (and one major
Node.js release to the next). With each change, addons may need to be
updated and recompiled in order to continue functioning. The Node.js
release schedule is designed to minimize the frequency and impact of
such changes but there is little that Node.js can do to ensure stability
of the V8 APIs.

The \href{https://github.com/nodejs/nan}{Native Abstractions for
Node.js} (or \texttt{nan}) provide a set of tools that addon developers
are recommended to use to keep compatibility between past and future
releases of V8 and Node.js. See the \texttt{nan}
\href{https://github.com/nodejs/nan/tree/HEAD/examples/}{examples} for
an illustration of how it can be used.

\subsection{Node-API}\label{node-api}

\begin{quote}
Stability: 2 - Stable
\end{quote}

Node-API is an API for building native addons. It is independent from
the underlying JavaScript runtime (e.g.~V8) and is maintained as part of
Node.js itself. This API will be Application Binary Interface (ABI)
stable across versions of Node.js. It is intended to insulate addons
from changes in the underlying JavaScript engine and allow modules
compiled for one version to run on later versions of Node.js without
recompilation. Addons are built/packaged with the same approach/tools
outlined in this document (node-gyp, etc.). The only difference is the
set of APIs that are used by the native code. Instead of using the V8 or
\href{https://github.com/nodejs/nan}{Native Abstractions for Node.js}
APIs, the functions available in the Node-API are used.

Creating and maintaining an addon that benefits from the ABI stability
provided by Node-API carries with it certain
\href{n-api.md\#implications-of-abi-stability}{implementation
considerations}.

To use Node-API in the above ``Hello world'' example, replace the
content of \texttt{hello.cc} with the following. All other instructions
remain the same.

\begin{Shaded}
\begin{Highlighting}[]
\CommentTok{// hello.cc using Node{-}API}
\PreprocessorTok{\#include }\ImportTok{\textless{}node\_api.h\textgreater{}}

\KeywordTok{namespace}\NormalTok{ demo }\OperatorTok{\{}

\NormalTok{napi\_value Method}\OperatorTok{(}\NormalTok{napi\_env env}\OperatorTok{,}\NormalTok{ napi\_callback\_info args}\OperatorTok{)} \OperatorTok{\{}
\NormalTok{  napi\_value greeting}\OperatorTok{;}
\NormalTok{  napi\_status status}\OperatorTok{;}

\NormalTok{  status }\OperatorTok{=}\NormalTok{ napi\_create\_string\_utf8}\OperatorTok{(}\NormalTok{env}\OperatorTok{,} \StringTok{"world"}\OperatorTok{,}\NormalTok{ NAPI\_AUTO\_LENGTH}\OperatorTok{,} \OperatorTok{\&}\NormalTok{greeting}\OperatorTok{);}
  \ControlFlowTok{if} \OperatorTok{(}\NormalTok{status }\OperatorTok{!=}\NormalTok{ napi\_ok}\OperatorTok{)} \ControlFlowTok{return} \KeywordTok{nullptr}\OperatorTok{;}
  \ControlFlowTok{return}\NormalTok{ greeting}\OperatorTok{;}
\OperatorTok{\}}

\NormalTok{napi\_value init}\OperatorTok{(}\NormalTok{napi\_env env}\OperatorTok{,}\NormalTok{ napi\_value exports}\OperatorTok{)} \OperatorTok{\{}
\NormalTok{  napi\_status status}\OperatorTok{;}
\NormalTok{  napi\_value fn}\OperatorTok{;}

\NormalTok{  status }\OperatorTok{=}\NormalTok{ napi\_create\_function}\OperatorTok{(}\NormalTok{env}\OperatorTok{,} \KeywordTok{nullptr}\OperatorTok{,} \DecValTok{0}\OperatorTok{,}\NormalTok{ Method}\OperatorTok{,} \KeywordTok{nullptr}\OperatorTok{,} \OperatorTok{\&}\NormalTok{fn}\OperatorTok{);}
  \ControlFlowTok{if} \OperatorTok{(}\NormalTok{status }\OperatorTok{!=}\NormalTok{ napi\_ok}\OperatorTok{)} \ControlFlowTok{return} \KeywordTok{nullptr}\OperatorTok{;}

\NormalTok{  status }\OperatorTok{=}\NormalTok{ napi\_set\_named\_property}\OperatorTok{(}\NormalTok{env}\OperatorTok{,}\NormalTok{ exports}\OperatorTok{,} \StringTok{"hello"}\OperatorTok{,}\NormalTok{ fn}\OperatorTok{);}
  \ControlFlowTok{if} \OperatorTok{(}\NormalTok{status }\OperatorTok{!=}\NormalTok{ napi\_ok}\OperatorTok{)} \ControlFlowTok{return} \KeywordTok{nullptr}\OperatorTok{;}
  \ControlFlowTok{return}\NormalTok{ exports}\OperatorTok{;}
\OperatorTok{\}}

\NormalTok{NAPI\_MODULE}\OperatorTok{(}\NormalTok{NODE\_GYP\_MODULE\_NAME}\OperatorTok{,}\NormalTok{ init}\OperatorTok{)}

\OperatorTok{\}}  \CommentTok{// namespace demo}
\end{Highlighting}
\end{Shaded}

The functions available and how to use them are documented in
\href{n-api.md}{C/C++ addons with Node-API}.

\subsection{Addon examples}\label{addon-examples}

Following are some example addons intended to help developers get
started. The examples use the V8 APIs. Refer to the online
\href{https://v8docs.nodesource.com/}{V8 reference} for help with the
various V8 calls, and V8's \href{https://v8.dev/docs/embed}{Embedder's
Guide} for an explanation of several concepts used such as handles,
scopes, function templates, etc.

Each of these examples using the following \texttt{binding.gyp} file:

\begin{Shaded}
\begin{Highlighting}[]
\FunctionTok{\{}
  \DataTypeTok{"targets"}\FunctionTok{:} \OtherTok{[}
    \FunctionTok{\{}
      \DataTypeTok{"target\_name"}\FunctionTok{:} \StringTok{"addon"}\FunctionTok{,}
      \DataTypeTok{"sources"}\FunctionTok{:} \OtherTok{[} \StringTok{"addon.cc"} \OtherTok{]}
    \FunctionTok{\}}
  \OtherTok{]}
\FunctionTok{\}}
\end{Highlighting}
\end{Shaded}

In cases where there is more than one \texttt{.cc} file, simply add the
additional filename to the \texttt{sources} array:

\begin{Shaded}
\begin{Highlighting}[]
\ErrorTok{"sources":} \OtherTok{[}\StringTok{"addon.cc"}\OtherTok{,} \StringTok{"myexample.cc"}\OtherTok{]}
\end{Highlighting}
\end{Shaded}

Once the \texttt{binding.gyp} file is ready, the example addons can be
configured and built using \texttt{node-gyp}:

\begin{Shaded}
\begin{Highlighting}[]
\ExtensionTok{node{-}gyp}\NormalTok{ configure build}
\end{Highlighting}
\end{Shaded}

\subsubsection{Function arguments}\label{function-arguments}

Addons will typically expose objects and functions that can be accessed
from JavaScript running within Node.js. When functions are invoked from
JavaScript, the input arguments and return value must be mapped to and
from the C/C++ code.

The following example illustrates how to read function arguments passed
from JavaScript and how to return a result:

\begin{Shaded}
\begin{Highlighting}[]
\CommentTok{// addon.cc}
\PreprocessorTok{\#include }\ImportTok{\textless{}node.h\textgreater{}}

\KeywordTok{namespace}\NormalTok{ demo }\OperatorTok{\{}

\KeywordTok{using}\NormalTok{ v8}\OperatorTok{::}\NormalTok{Exception}\OperatorTok{;}
\KeywordTok{using}\NormalTok{ v8}\OperatorTok{::}\NormalTok{FunctionCallbackInfo}\OperatorTok{;}
\KeywordTok{using}\NormalTok{ v8}\OperatorTok{::}\NormalTok{Isolate}\OperatorTok{;}
\KeywordTok{using}\NormalTok{ v8}\OperatorTok{::}\NormalTok{Local}\OperatorTok{;}
\KeywordTok{using}\NormalTok{ v8}\OperatorTok{::}\NormalTok{Number}\OperatorTok{;}
\KeywordTok{using}\NormalTok{ v8}\OperatorTok{::}\NormalTok{Object}\OperatorTok{;}
\KeywordTok{using}\NormalTok{ v8}\OperatorTok{::}\NormalTok{String}\OperatorTok{;}
\KeywordTok{using}\NormalTok{ v8}\OperatorTok{::}\NormalTok{Value}\OperatorTok{;}

\CommentTok{// This is the implementation of the "add" method}
\CommentTok{// Input arguments are passed using the}
\CommentTok{// const FunctionCallbackInfo\textless{}Value\textgreater{}\& args struct}
\DataTypeTok{void}\NormalTok{ Add}\OperatorTok{(}\AttributeTok{const}\NormalTok{ FunctionCallbackInfo}\OperatorTok{\textless{}}\NormalTok{Value}\OperatorTok{\textgreater{}\&}\NormalTok{ args}\OperatorTok{)} \OperatorTok{\{}
\NormalTok{  Isolate}\OperatorTok{*}\NormalTok{ isolate }\OperatorTok{=}\NormalTok{ args}\OperatorTok{.}\NormalTok{GetIsolate}\OperatorTok{();}

  \CommentTok{// Check the number of arguments passed.}
  \ControlFlowTok{if} \OperatorTok{(}\NormalTok{args}\OperatorTok{.}\NormalTok{Length}\OperatorTok{()} \OperatorTok{\textless{}} \DecValTok{2}\OperatorTok{)} \OperatorTok{\{}
    \CommentTok{// Throw an Error that is passed back to JavaScript}
\NormalTok{    isolate}\OperatorTok{{-}\textgreater{}}\NormalTok{ThrowException}\OperatorTok{(}\NormalTok{Exception}\OperatorTok{::}\NormalTok{TypeError}\OperatorTok{(}
\NormalTok{        String}\OperatorTok{::}\NormalTok{NewFromUtf8}\OperatorTok{(}\NormalTok{isolate}\OperatorTok{,}
                            \StringTok{"Wrong number of arguments"}\OperatorTok{).}\NormalTok{ToLocalChecked}\OperatorTok{()));}
    \ControlFlowTok{return}\OperatorTok{;}
  \OperatorTok{\}}

  \CommentTok{// Check the argument types}
  \ControlFlowTok{if} \OperatorTok{(!}\NormalTok{args}\OperatorTok{[}\DecValTok{0}\OperatorTok{]{-}\textgreater{}}\NormalTok{IsNumber}\OperatorTok{()} \OperatorTok{||} \OperatorTok{!}\NormalTok{args}\OperatorTok{[}\DecValTok{1}\OperatorTok{]{-}\textgreater{}}\NormalTok{IsNumber}\OperatorTok{())} \OperatorTok{\{}
\NormalTok{    isolate}\OperatorTok{{-}\textgreater{}}\NormalTok{ThrowException}\OperatorTok{(}\NormalTok{Exception}\OperatorTok{::}\NormalTok{TypeError}\OperatorTok{(}
\NormalTok{        String}\OperatorTok{::}\NormalTok{NewFromUtf8}\OperatorTok{(}\NormalTok{isolate}\OperatorTok{,}
                            \StringTok{"Wrong arguments"}\OperatorTok{).}\NormalTok{ToLocalChecked}\OperatorTok{()));}
    \ControlFlowTok{return}\OperatorTok{;}
  \OperatorTok{\}}

  \CommentTok{// Perform the operation}
  \DataTypeTok{double}\NormalTok{ value }\OperatorTok{=}
\NormalTok{      args}\OperatorTok{[}\DecValTok{0}\OperatorTok{].}\NormalTok{As}\OperatorTok{\textless{}}\NormalTok{Number}\OperatorTok{\textgreater{}(){-}\textgreater{}}\NormalTok{Value}\OperatorTok{()} \OperatorTok{+}\NormalTok{ args}\OperatorTok{[}\DecValTok{1}\OperatorTok{].}\NormalTok{As}\OperatorTok{\textless{}}\NormalTok{Number}\OperatorTok{\textgreater{}(){-}\textgreater{}}\NormalTok{Value}\OperatorTok{();}
\NormalTok{  Local}\OperatorTok{\textless{}}\NormalTok{Number}\OperatorTok{\textgreater{}}\NormalTok{ num }\OperatorTok{=}\NormalTok{ Number}\OperatorTok{::}\NormalTok{New}\OperatorTok{(}\NormalTok{isolate}\OperatorTok{,}\NormalTok{ value}\OperatorTok{);}

  \CommentTok{// Set the return value (using the passed in}
  \CommentTok{// FunctionCallbackInfo\textless{}Value\textgreater{}\&)}
\NormalTok{  args}\OperatorTok{.}\NormalTok{GetReturnValue}\OperatorTok{().}\NormalTok{Set}\OperatorTok{(}\NormalTok{num}\OperatorTok{);}
\OperatorTok{\}}

\DataTypeTok{void}\NormalTok{ Init}\OperatorTok{(}\NormalTok{Local}\OperatorTok{\textless{}}\NormalTok{Object}\OperatorTok{\textgreater{}}\NormalTok{ exports}\OperatorTok{)} \OperatorTok{\{}
\NormalTok{  NODE\_SET\_METHOD}\OperatorTok{(}\NormalTok{exports}\OperatorTok{,} \StringTok{"add"}\OperatorTok{,}\NormalTok{ Add}\OperatorTok{);}
\OperatorTok{\}}

\NormalTok{NODE\_MODULE}\OperatorTok{(}\NormalTok{NODE\_GYP\_MODULE\_NAME}\OperatorTok{,}\NormalTok{ Init}\OperatorTok{)}

\OperatorTok{\}}  \CommentTok{// namespace demo}
\end{Highlighting}
\end{Shaded}

Once compiled, the example addon can be required and used from within
Node.js:

\begin{Shaded}
\begin{Highlighting}[]
\CommentTok{// test.js}
\KeywordTok{const}\NormalTok{ addon }\OperatorTok{=} \PreprocessorTok{require}\NormalTok{(}\StringTok{\textquotesingle{}./build/Release/addon\textquotesingle{}}\NormalTok{)}\OperatorTok{;}

\BuiltInTok{console}\OperatorTok{.}\FunctionTok{log}\NormalTok{(}\StringTok{\textquotesingle{}This should be eight:\textquotesingle{}}\OperatorTok{,}\NormalTok{ addon}\OperatorTok{.}\FunctionTok{add}\NormalTok{(}\DecValTok{3}\OperatorTok{,} \DecValTok{5}\NormalTok{))}\OperatorTok{;}
\end{Highlighting}
\end{Shaded}

\subsubsection{Callbacks}\label{callbacks}

It is common practice within addons to pass JavaScript functions to a
C++ function and execute them from there. The following example
illustrates how to invoke such callbacks:

\begin{Shaded}
\begin{Highlighting}[]
\CommentTok{// addon.cc}
\PreprocessorTok{\#include }\ImportTok{\textless{}node.h\textgreater{}}

\KeywordTok{namespace}\NormalTok{ demo }\OperatorTok{\{}

\KeywordTok{using}\NormalTok{ v8}\OperatorTok{::}\NormalTok{Context}\OperatorTok{;}
\KeywordTok{using}\NormalTok{ v8}\OperatorTok{::}\NormalTok{Function}\OperatorTok{;}
\KeywordTok{using}\NormalTok{ v8}\OperatorTok{::}\NormalTok{FunctionCallbackInfo}\OperatorTok{;}
\KeywordTok{using}\NormalTok{ v8}\OperatorTok{::}\NormalTok{Isolate}\OperatorTok{;}
\KeywordTok{using}\NormalTok{ v8}\OperatorTok{::}\NormalTok{Local}\OperatorTok{;}
\KeywordTok{using}\NormalTok{ v8}\OperatorTok{::}\NormalTok{Null}\OperatorTok{;}
\KeywordTok{using}\NormalTok{ v8}\OperatorTok{::}\NormalTok{Object}\OperatorTok{;}
\KeywordTok{using}\NormalTok{ v8}\OperatorTok{::}\NormalTok{String}\OperatorTok{;}
\KeywordTok{using}\NormalTok{ v8}\OperatorTok{::}\NormalTok{Value}\OperatorTok{;}

\DataTypeTok{void}\NormalTok{ RunCallback}\OperatorTok{(}\AttributeTok{const}\NormalTok{ FunctionCallbackInfo}\OperatorTok{\textless{}}\NormalTok{Value}\OperatorTok{\textgreater{}\&}\NormalTok{ args}\OperatorTok{)} \OperatorTok{\{}
\NormalTok{  Isolate}\OperatorTok{*}\NormalTok{ isolate }\OperatorTok{=}\NormalTok{ args}\OperatorTok{.}\NormalTok{GetIsolate}\OperatorTok{();}
\NormalTok{  Local}\OperatorTok{\textless{}}\NormalTok{Context}\OperatorTok{\textgreater{}}\NormalTok{ context }\OperatorTok{=}\NormalTok{ isolate}\OperatorTok{{-}\textgreater{}}\NormalTok{GetCurrentContext}\OperatorTok{();}
\NormalTok{  Local}\OperatorTok{\textless{}}\NormalTok{Function}\OperatorTok{\textgreater{}}\NormalTok{ cb }\OperatorTok{=}\NormalTok{ Local}\OperatorTok{\textless{}}\NormalTok{Function}\OperatorTok{\textgreater{}::}\NormalTok{Cast}\OperatorTok{(}\NormalTok{args}\OperatorTok{[}\DecValTok{0}\OperatorTok{]);}
  \AttributeTok{const} \DataTypeTok{unsigned}\NormalTok{ argc }\OperatorTok{=} \DecValTok{1}\OperatorTok{;}
\NormalTok{  Local}\OperatorTok{\textless{}}\NormalTok{Value}\OperatorTok{\textgreater{}}\NormalTok{ argv}\OperatorTok{[}\NormalTok{argc}\OperatorTok{]} \OperatorTok{=} \OperatorTok{\{}
\NormalTok{      String}\OperatorTok{::}\NormalTok{NewFromUtf8}\OperatorTok{(}\NormalTok{isolate}\OperatorTok{,}
                          \StringTok{"hello world"}\OperatorTok{).}\NormalTok{ToLocalChecked}\OperatorTok{()} \OperatorTok{\};}
\NormalTok{  cb}\OperatorTok{{-}\textgreater{}}\NormalTok{Call}\OperatorTok{(}\NormalTok{context}\OperatorTok{,}\NormalTok{ Null}\OperatorTok{(}\NormalTok{isolate}\OperatorTok{),}\NormalTok{ argc}\OperatorTok{,}\NormalTok{ argv}\OperatorTok{).}\NormalTok{ToLocalChecked}\OperatorTok{();}
\OperatorTok{\}}

\DataTypeTok{void}\NormalTok{ Init}\OperatorTok{(}\NormalTok{Local}\OperatorTok{\textless{}}\NormalTok{Object}\OperatorTok{\textgreater{}}\NormalTok{ exports}\OperatorTok{,}\NormalTok{ Local}\OperatorTok{\textless{}}\NormalTok{Object}\OperatorTok{\textgreater{}} \KeywordTok{module}\OperatorTok{)} \OperatorTok{\{}
\NormalTok{  NODE\_SET\_METHOD}\OperatorTok{(}\KeywordTok{module}\OperatorTok{,} \StringTok{"exports"}\OperatorTok{,}\NormalTok{ RunCallback}\OperatorTok{);}
\OperatorTok{\}}

\NormalTok{NODE\_MODULE}\OperatorTok{(}\NormalTok{NODE\_GYP\_MODULE\_NAME}\OperatorTok{,}\NormalTok{ Init}\OperatorTok{)}

\OperatorTok{\}}  \CommentTok{// namespace demo}
\end{Highlighting}
\end{Shaded}

This example uses a two-argument form of \texttt{Init()} that receives
the full \texttt{module} object as the second argument. This allows the
addon to completely overwrite \texttt{exports} with a single function
instead of adding the function as a property of \texttt{exports}.

To test it, run the following JavaScript:

\begin{Shaded}
\begin{Highlighting}[]
\CommentTok{// test.js}
\KeywordTok{const}\NormalTok{ addon }\OperatorTok{=} \PreprocessorTok{require}\NormalTok{(}\StringTok{\textquotesingle{}./build/Release/addon\textquotesingle{}}\NormalTok{)}\OperatorTok{;}

\FunctionTok{addon}\NormalTok{((msg) }\KeywordTok{=\textgreater{}}\NormalTok{ \{}
  \BuiltInTok{console}\OperatorTok{.}\FunctionTok{log}\NormalTok{(msg)}\OperatorTok{;}
\CommentTok{// Prints: \textquotesingle{}hello world\textquotesingle{}}
\NormalTok{\})}\OperatorTok{;}
\end{Highlighting}
\end{Shaded}

In this example, the callback function is invoked synchronously.

\subsubsection{Object factory}\label{object-factory}

Addons can create and return new objects from within a C++ function as
illustrated in the following example. An object is created and returned
with a property \texttt{msg} that echoes the string passed to
\texttt{createObject()}:

\begin{Shaded}
\begin{Highlighting}[]
\CommentTok{// addon.cc}
\PreprocessorTok{\#include }\ImportTok{\textless{}node.h\textgreater{}}

\KeywordTok{namespace}\NormalTok{ demo }\OperatorTok{\{}

\KeywordTok{using}\NormalTok{ v8}\OperatorTok{::}\NormalTok{Context}\OperatorTok{;}
\KeywordTok{using}\NormalTok{ v8}\OperatorTok{::}\NormalTok{FunctionCallbackInfo}\OperatorTok{;}
\KeywordTok{using}\NormalTok{ v8}\OperatorTok{::}\NormalTok{Isolate}\OperatorTok{;}
\KeywordTok{using}\NormalTok{ v8}\OperatorTok{::}\NormalTok{Local}\OperatorTok{;}
\KeywordTok{using}\NormalTok{ v8}\OperatorTok{::}\NormalTok{Object}\OperatorTok{;}
\KeywordTok{using}\NormalTok{ v8}\OperatorTok{::}\NormalTok{String}\OperatorTok{;}
\KeywordTok{using}\NormalTok{ v8}\OperatorTok{::}\NormalTok{Value}\OperatorTok{;}

\DataTypeTok{void}\NormalTok{ CreateObject}\OperatorTok{(}\AttributeTok{const}\NormalTok{ FunctionCallbackInfo}\OperatorTok{\textless{}}\NormalTok{Value}\OperatorTok{\textgreater{}\&}\NormalTok{ args}\OperatorTok{)} \OperatorTok{\{}
\NormalTok{  Isolate}\OperatorTok{*}\NormalTok{ isolate }\OperatorTok{=}\NormalTok{ args}\OperatorTok{.}\NormalTok{GetIsolate}\OperatorTok{();}
\NormalTok{  Local}\OperatorTok{\textless{}}\NormalTok{Context}\OperatorTok{\textgreater{}}\NormalTok{ context }\OperatorTok{=}\NormalTok{ isolate}\OperatorTok{{-}\textgreater{}}\NormalTok{GetCurrentContext}\OperatorTok{();}

\NormalTok{  Local}\OperatorTok{\textless{}}\NormalTok{Object}\OperatorTok{\textgreater{}}\NormalTok{ obj }\OperatorTok{=}\NormalTok{ Object}\OperatorTok{::}\NormalTok{New}\OperatorTok{(}\NormalTok{isolate}\OperatorTok{);}
\NormalTok{  obj}\OperatorTok{{-}\textgreater{}}\NormalTok{Set}\OperatorTok{(}\NormalTok{context}\OperatorTok{,}
\NormalTok{           String}\OperatorTok{::}\NormalTok{NewFromUtf8}\OperatorTok{(}\NormalTok{isolate}\OperatorTok{,}
                               \StringTok{"msg"}\OperatorTok{).}\NormalTok{ToLocalChecked}\OperatorTok{(),}
\NormalTok{                               args}\OperatorTok{[}\DecValTok{0}\OperatorTok{]{-}\textgreater{}}\NormalTok{ToString}\OperatorTok{(}\NormalTok{context}\OperatorTok{).}\NormalTok{ToLocalChecked}\OperatorTok{())}
           \OperatorTok{.}\NormalTok{FromJust}\OperatorTok{();}

\NormalTok{  args}\OperatorTok{.}\NormalTok{GetReturnValue}\OperatorTok{().}\NormalTok{Set}\OperatorTok{(}\NormalTok{obj}\OperatorTok{);}
\OperatorTok{\}}

\DataTypeTok{void}\NormalTok{ Init}\OperatorTok{(}\NormalTok{Local}\OperatorTok{\textless{}}\NormalTok{Object}\OperatorTok{\textgreater{}}\NormalTok{ exports}\OperatorTok{,}\NormalTok{ Local}\OperatorTok{\textless{}}\NormalTok{Object}\OperatorTok{\textgreater{}} \KeywordTok{module}\OperatorTok{)} \OperatorTok{\{}
\NormalTok{  NODE\_SET\_METHOD}\OperatorTok{(}\KeywordTok{module}\OperatorTok{,} \StringTok{"exports"}\OperatorTok{,}\NormalTok{ CreateObject}\OperatorTok{);}
\OperatorTok{\}}

\NormalTok{NODE\_MODULE}\OperatorTok{(}\NormalTok{NODE\_GYP\_MODULE\_NAME}\OperatorTok{,}\NormalTok{ Init}\OperatorTok{)}

\OperatorTok{\}}  \CommentTok{// namespace demo}
\end{Highlighting}
\end{Shaded}

To test it in JavaScript:

\begin{Shaded}
\begin{Highlighting}[]
\CommentTok{// test.js}
\KeywordTok{const}\NormalTok{ addon }\OperatorTok{=} \PreprocessorTok{require}\NormalTok{(}\StringTok{\textquotesingle{}./build/Release/addon\textquotesingle{}}\NormalTok{)}\OperatorTok{;}

\KeywordTok{const}\NormalTok{ obj1 }\OperatorTok{=} \FunctionTok{addon}\NormalTok{(}\StringTok{\textquotesingle{}hello\textquotesingle{}}\NormalTok{)}\OperatorTok{;}
\KeywordTok{const}\NormalTok{ obj2 }\OperatorTok{=} \FunctionTok{addon}\NormalTok{(}\StringTok{\textquotesingle{}world\textquotesingle{}}\NormalTok{)}\OperatorTok{;}
\BuiltInTok{console}\OperatorTok{.}\FunctionTok{log}\NormalTok{(obj1}\OperatorTok{.}\AttributeTok{msg}\OperatorTok{,}\NormalTok{ obj2}\OperatorTok{.}\AttributeTok{msg}\NormalTok{)}\OperatorTok{;}
\CommentTok{// Prints: \textquotesingle{}hello world\textquotesingle{}}
\end{Highlighting}
\end{Shaded}

\subsubsection{Function factory}\label{function-factory}

Another common scenario is creating JavaScript functions that wrap C++
functions and returning those back to JavaScript:

\begin{Shaded}
\begin{Highlighting}[]
\CommentTok{// addon.cc}
\PreprocessorTok{\#include }\ImportTok{\textless{}node.h\textgreater{}}

\KeywordTok{namespace}\NormalTok{ demo }\OperatorTok{\{}

\KeywordTok{using}\NormalTok{ v8}\OperatorTok{::}\NormalTok{Context}\OperatorTok{;}
\KeywordTok{using}\NormalTok{ v8}\OperatorTok{::}\NormalTok{Function}\OperatorTok{;}
\KeywordTok{using}\NormalTok{ v8}\OperatorTok{::}\NormalTok{FunctionCallbackInfo}\OperatorTok{;}
\KeywordTok{using}\NormalTok{ v8}\OperatorTok{::}\NormalTok{FunctionTemplate}\OperatorTok{;}
\KeywordTok{using}\NormalTok{ v8}\OperatorTok{::}\NormalTok{Isolate}\OperatorTok{;}
\KeywordTok{using}\NormalTok{ v8}\OperatorTok{::}\NormalTok{Local}\OperatorTok{;}
\KeywordTok{using}\NormalTok{ v8}\OperatorTok{::}\NormalTok{Object}\OperatorTok{;}
\KeywordTok{using}\NormalTok{ v8}\OperatorTok{::}\NormalTok{String}\OperatorTok{;}
\KeywordTok{using}\NormalTok{ v8}\OperatorTok{::}\NormalTok{Value}\OperatorTok{;}

\DataTypeTok{void}\NormalTok{ MyFunction}\OperatorTok{(}\AttributeTok{const}\NormalTok{ FunctionCallbackInfo}\OperatorTok{\textless{}}\NormalTok{Value}\OperatorTok{\textgreater{}\&}\NormalTok{ args}\OperatorTok{)} \OperatorTok{\{}
\NormalTok{  Isolate}\OperatorTok{*}\NormalTok{ isolate }\OperatorTok{=}\NormalTok{ args}\OperatorTok{.}\NormalTok{GetIsolate}\OperatorTok{();}
\NormalTok{  args}\OperatorTok{.}\NormalTok{GetReturnValue}\OperatorTok{().}\NormalTok{Set}\OperatorTok{(}\NormalTok{String}\OperatorTok{::}\NormalTok{NewFromUtf8}\OperatorTok{(}
\NormalTok{      isolate}\OperatorTok{,} \StringTok{"hello world"}\OperatorTok{).}\NormalTok{ToLocalChecked}\OperatorTok{());}
\OperatorTok{\}}

\DataTypeTok{void}\NormalTok{ CreateFunction}\OperatorTok{(}\AttributeTok{const}\NormalTok{ FunctionCallbackInfo}\OperatorTok{\textless{}}\NormalTok{Value}\OperatorTok{\textgreater{}\&}\NormalTok{ args}\OperatorTok{)} \OperatorTok{\{}
\NormalTok{  Isolate}\OperatorTok{*}\NormalTok{ isolate }\OperatorTok{=}\NormalTok{ args}\OperatorTok{.}\NormalTok{GetIsolate}\OperatorTok{();}

\NormalTok{  Local}\OperatorTok{\textless{}}\NormalTok{Context}\OperatorTok{\textgreater{}}\NormalTok{ context }\OperatorTok{=}\NormalTok{ isolate}\OperatorTok{{-}\textgreater{}}\NormalTok{GetCurrentContext}\OperatorTok{();}
\NormalTok{  Local}\OperatorTok{\textless{}}\NormalTok{FunctionTemplate}\OperatorTok{\textgreater{}}\NormalTok{ tpl }\OperatorTok{=}\NormalTok{ FunctionTemplate}\OperatorTok{::}\NormalTok{New}\OperatorTok{(}\NormalTok{isolate}\OperatorTok{,}\NormalTok{ MyFunction}\OperatorTok{);}
\NormalTok{  Local}\OperatorTok{\textless{}}\NormalTok{Function}\OperatorTok{\textgreater{}}\NormalTok{ fn }\OperatorTok{=}\NormalTok{ tpl}\OperatorTok{{-}\textgreater{}}\NormalTok{GetFunction}\OperatorTok{(}\NormalTok{context}\OperatorTok{).}\NormalTok{ToLocalChecked}\OperatorTok{();}

  \CommentTok{// omit this to make it anonymous}
\NormalTok{  fn}\OperatorTok{{-}\textgreater{}}\NormalTok{SetName}\OperatorTok{(}\NormalTok{String}\OperatorTok{::}\NormalTok{NewFromUtf8}\OperatorTok{(}
\NormalTok{      isolate}\OperatorTok{,} \StringTok{"theFunction"}\OperatorTok{).}\NormalTok{ToLocalChecked}\OperatorTok{());}

\NormalTok{  args}\OperatorTok{.}\NormalTok{GetReturnValue}\OperatorTok{().}\NormalTok{Set}\OperatorTok{(}\NormalTok{fn}\OperatorTok{);}
\OperatorTok{\}}

\DataTypeTok{void}\NormalTok{ Init}\OperatorTok{(}\NormalTok{Local}\OperatorTok{\textless{}}\NormalTok{Object}\OperatorTok{\textgreater{}}\NormalTok{ exports}\OperatorTok{,}\NormalTok{ Local}\OperatorTok{\textless{}}\NormalTok{Object}\OperatorTok{\textgreater{}} \KeywordTok{module}\OperatorTok{)} \OperatorTok{\{}
\NormalTok{  NODE\_SET\_METHOD}\OperatorTok{(}\KeywordTok{module}\OperatorTok{,} \StringTok{"exports"}\OperatorTok{,}\NormalTok{ CreateFunction}\OperatorTok{);}
\OperatorTok{\}}

\NormalTok{NODE\_MODULE}\OperatorTok{(}\NormalTok{NODE\_GYP\_MODULE\_NAME}\OperatorTok{,}\NormalTok{ Init}\OperatorTok{)}

\OperatorTok{\}}  \CommentTok{// namespace demo}
\end{Highlighting}
\end{Shaded}

To test:

\begin{Shaded}
\begin{Highlighting}[]
\CommentTok{// test.js}
\KeywordTok{const}\NormalTok{ addon }\OperatorTok{=} \PreprocessorTok{require}\NormalTok{(}\StringTok{\textquotesingle{}./build/Release/addon\textquotesingle{}}\NormalTok{)}\OperatorTok{;}

\KeywordTok{const}\NormalTok{ fn }\OperatorTok{=} \FunctionTok{addon}\NormalTok{()}\OperatorTok{;}
\BuiltInTok{console}\OperatorTok{.}\FunctionTok{log}\NormalTok{(}\FunctionTok{fn}\NormalTok{())}\OperatorTok{;}
\CommentTok{// Prints: \textquotesingle{}hello world\textquotesingle{}}
\end{Highlighting}
\end{Shaded}

\subsubsection{Wrapping C++ objects}\label{wrapping-c-objects}

It is also possible to wrap C++ objects/classes in a way that allows new
instances to be created using the JavaScript \texttt{new} operator:

\begin{Shaded}
\begin{Highlighting}[]
\CommentTok{// addon.cc}
\PreprocessorTok{\#include }\ImportTok{\textless{}node.h\textgreater{}}
\PreprocessorTok{\#include }\ImportTok{"myobject.h"}

\KeywordTok{namespace}\NormalTok{ demo }\OperatorTok{\{}

\KeywordTok{using}\NormalTok{ v8}\OperatorTok{::}\NormalTok{Local}\OperatorTok{;}
\KeywordTok{using}\NormalTok{ v8}\OperatorTok{::}\NormalTok{Object}\OperatorTok{;}

\DataTypeTok{void}\NormalTok{ InitAll}\OperatorTok{(}\NormalTok{Local}\OperatorTok{\textless{}}\NormalTok{Object}\OperatorTok{\textgreater{}}\NormalTok{ exports}\OperatorTok{)} \OperatorTok{\{}
\NormalTok{  MyObject}\OperatorTok{::}\NormalTok{Init}\OperatorTok{(}\NormalTok{exports}\OperatorTok{);}
\OperatorTok{\}}

\NormalTok{NODE\_MODULE}\OperatorTok{(}\NormalTok{NODE\_GYP\_MODULE\_NAME}\OperatorTok{,}\NormalTok{ InitAll}\OperatorTok{)}

\OperatorTok{\}}  \CommentTok{// namespace demo}
\end{Highlighting}
\end{Shaded}

Then, in \texttt{myobject.h}, the wrapper class inherits from
\texttt{node::ObjectWrap}:

\begin{Shaded}
\begin{Highlighting}[]
\CommentTok{// myobject.h}
\PreprocessorTok{\#ifndef MYOBJECT\_H}
\PreprocessorTok{\#define MYOBJECT\_H}

\PreprocessorTok{\#include }\ImportTok{\textless{}node.h\textgreater{}}
\PreprocessorTok{\#include }\ImportTok{\textless{}node\_object\_wrap.h\textgreater{}}

\KeywordTok{namespace}\NormalTok{ demo }\OperatorTok{\{}

\KeywordTok{class}\NormalTok{ MyObject }\OperatorTok{:} \KeywordTok{public}\NormalTok{ node}\OperatorTok{::}\NormalTok{ObjectWrap }\OperatorTok{\{}
 \KeywordTok{public}\OperatorTok{:}
  \AttributeTok{static} \DataTypeTok{void}\NormalTok{ Init}\OperatorTok{(}\NormalTok{v8}\OperatorTok{::}\NormalTok{Local}\OperatorTok{\textless{}}\NormalTok{v8}\OperatorTok{::}\NormalTok{Object}\OperatorTok{\textgreater{}}\NormalTok{ exports}\OperatorTok{);}

 \KeywordTok{private}\OperatorTok{:}
  \KeywordTok{explicit}\NormalTok{ MyObject}\OperatorTok{(}\DataTypeTok{double}\NormalTok{ value }\OperatorTok{=} \DecValTok{0}\OperatorTok{);}
  \OperatorTok{\textasciitilde{}}\NormalTok{MyObject}\OperatorTok{();}

  \AttributeTok{static} \DataTypeTok{void}\NormalTok{ New}\OperatorTok{(}\AttributeTok{const}\NormalTok{ v8}\OperatorTok{::}\NormalTok{FunctionCallbackInfo}\OperatorTok{\textless{}}\NormalTok{v8}\OperatorTok{::}\NormalTok{Value}\OperatorTok{\textgreater{}\&}\NormalTok{ args}\OperatorTok{);}
  \AttributeTok{static} \DataTypeTok{void}\NormalTok{ PlusOne}\OperatorTok{(}\AttributeTok{const}\NormalTok{ v8}\OperatorTok{::}\NormalTok{FunctionCallbackInfo}\OperatorTok{\textless{}}\NormalTok{v8}\OperatorTok{::}\NormalTok{Value}\OperatorTok{\textgreater{}\&}\NormalTok{ args}\OperatorTok{);}

  \DataTypeTok{double} \VariableTok{value\_}\OperatorTok{;}
\OperatorTok{\};}

\OperatorTok{\}}  \CommentTok{// namespace demo}

\PreprocessorTok{\#endif}
\end{Highlighting}
\end{Shaded}

In \texttt{myobject.cc}, implement the various methods that are to be
exposed. Below, the method \texttt{plusOne()} is exposed by adding it to
the constructor's prototype:

\begin{Shaded}
\begin{Highlighting}[]
\CommentTok{// myobject.cc}
\PreprocessorTok{\#include }\ImportTok{"myobject.h"}

\KeywordTok{namespace}\NormalTok{ demo }\OperatorTok{\{}

\KeywordTok{using}\NormalTok{ v8}\OperatorTok{::}\NormalTok{Context}\OperatorTok{;}
\KeywordTok{using}\NormalTok{ v8}\OperatorTok{::}\NormalTok{Function}\OperatorTok{;}
\KeywordTok{using}\NormalTok{ v8}\OperatorTok{::}\NormalTok{FunctionCallbackInfo}\OperatorTok{;}
\KeywordTok{using}\NormalTok{ v8}\OperatorTok{::}\NormalTok{FunctionTemplate}\OperatorTok{;}
\KeywordTok{using}\NormalTok{ v8}\OperatorTok{::}\NormalTok{Isolate}\OperatorTok{;}
\KeywordTok{using}\NormalTok{ v8}\OperatorTok{::}\NormalTok{Local}\OperatorTok{;}
\KeywordTok{using}\NormalTok{ v8}\OperatorTok{::}\NormalTok{Number}\OperatorTok{;}
\KeywordTok{using}\NormalTok{ v8}\OperatorTok{::}\NormalTok{Object}\OperatorTok{;}
\KeywordTok{using}\NormalTok{ v8}\OperatorTok{::}\NormalTok{ObjectTemplate}\OperatorTok{;}
\KeywordTok{using}\NormalTok{ v8}\OperatorTok{::}\NormalTok{String}\OperatorTok{;}
\KeywordTok{using}\NormalTok{ v8}\OperatorTok{::}\NormalTok{Value}\OperatorTok{;}

\NormalTok{MyObject}\OperatorTok{::}\NormalTok{MyObject}\OperatorTok{(}\DataTypeTok{double}\NormalTok{ value}\OperatorTok{)} \OperatorTok{:} \VariableTok{value\_}\OperatorTok{(}\NormalTok{value}\OperatorTok{)} \OperatorTok{\{}
\OperatorTok{\}}

\NormalTok{MyObject}\OperatorTok{::\textasciitilde{}}\NormalTok{MyObject}\OperatorTok{()} \OperatorTok{\{}
\OperatorTok{\}}

\DataTypeTok{void}\NormalTok{ MyObject}\OperatorTok{::}\NormalTok{Init}\OperatorTok{(}\NormalTok{Local}\OperatorTok{\textless{}}\NormalTok{Object}\OperatorTok{\textgreater{}}\NormalTok{ exports}\OperatorTok{)} \OperatorTok{\{}
\NormalTok{  Isolate}\OperatorTok{*}\NormalTok{ isolate }\OperatorTok{=}\NormalTok{ exports}\OperatorTok{{-}\textgreater{}}\NormalTok{GetIsolate}\OperatorTok{();}
\NormalTok{  Local}\OperatorTok{\textless{}}\NormalTok{Context}\OperatorTok{\textgreater{}}\NormalTok{ context }\OperatorTok{=}\NormalTok{ isolate}\OperatorTok{{-}\textgreater{}}\NormalTok{GetCurrentContext}\OperatorTok{();}

\NormalTok{  Local}\OperatorTok{\textless{}}\NormalTok{ObjectTemplate}\OperatorTok{\textgreater{}}\NormalTok{ addon\_data\_tpl }\OperatorTok{=}\NormalTok{ ObjectTemplate}\OperatorTok{::}\NormalTok{New}\OperatorTok{(}\NormalTok{isolate}\OperatorTok{);}
\NormalTok{  addon\_data\_tpl}\OperatorTok{{-}\textgreater{}}\NormalTok{SetInternalFieldCount}\OperatorTok{(}\DecValTok{1}\OperatorTok{);}  \CommentTok{// 1 field for the MyObject::New()}
\NormalTok{  Local}\OperatorTok{\textless{}}\NormalTok{Object}\OperatorTok{\textgreater{}}\NormalTok{ addon\_data }\OperatorTok{=}
\NormalTok{      addon\_data\_tpl}\OperatorTok{{-}\textgreater{}}\NormalTok{NewInstance}\OperatorTok{(}\NormalTok{context}\OperatorTok{).}\NormalTok{ToLocalChecked}\OperatorTok{();}

  \CommentTok{// Prepare constructor template}
\NormalTok{  Local}\OperatorTok{\textless{}}\NormalTok{FunctionTemplate}\OperatorTok{\textgreater{}}\NormalTok{ tpl }\OperatorTok{=}\NormalTok{ FunctionTemplate}\OperatorTok{::}\NormalTok{New}\OperatorTok{(}\NormalTok{isolate}\OperatorTok{,}\NormalTok{ New}\OperatorTok{,}\NormalTok{ addon\_data}\OperatorTok{);}
\NormalTok{  tpl}\OperatorTok{{-}\textgreater{}}\NormalTok{SetClassName}\OperatorTok{(}\NormalTok{String}\OperatorTok{::}\NormalTok{NewFromUtf8}\OperatorTok{(}\NormalTok{isolate}\OperatorTok{,} \StringTok{"MyObject"}\OperatorTok{).}\NormalTok{ToLocalChecked}\OperatorTok{());}
\NormalTok{  tpl}\OperatorTok{{-}\textgreater{}}\NormalTok{InstanceTemplate}\OperatorTok{(){-}\textgreater{}}\NormalTok{SetInternalFieldCount}\OperatorTok{(}\DecValTok{1}\OperatorTok{);}

  \CommentTok{// Prototype}
\NormalTok{  NODE\_SET\_PROTOTYPE\_METHOD}\OperatorTok{(}\NormalTok{tpl}\OperatorTok{,} \StringTok{"plusOne"}\OperatorTok{,}\NormalTok{ PlusOne}\OperatorTok{);}

\NormalTok{  Local}\OperatorTok{\textless{}}\NormalTok{Function}\OperatorTok{\textgreater{}}\NormalTok{ constructor }\OperatorTok{=}\NormalTok{ tpl}\OperatorTok{{-}\textgreater{}}\NormalTok{GetFunction}\OperatorTok{(}\NormalTok{context}\OperatorTok{).}\NormalTok{ToLocalChecked}\OperatorTok{();}
\NormalTok{  addon\_data}\OperatorTok{{-}\textgreater{}}\NormalTok{SetInternalField}\OperatorTok{(}\DecValTok{0}\OperatorTok{,}\NormalTok{ constructor}\OperatorTok{);}
\NormalTok{  exports}\OperatorTok{{-}\textgreater{}}\NormalTok{Set}\OperatorTok{(}\NormalTok{context}\OperatorTok{,}\NormalTok{ String}\OperatorTok{::}\NormalTok{NewFromUtf8}\OperatorTok{(}
\NormalTok{      isolate}\OperatorTok{,} \StringTok{"MyObject"}\OperatorTok{).}\NormalTok{ToLocalChecked}\OperatorTok{(),}
\NormalTok{      constructor}\OperatorTok{).}\NormalTok{FromJust}\OperatorTok{();}
\OperatorTok{\}}

\DataTypeTok{void}\NormalTok{ MyObject}\OperatorTok{::}\NormalTok{New}\OperatorTok{(}\AttributeTok{const}\NormalTok{ FunctionCallbackInfo}\OperatorTok{\textless{}}\NormalTok{Value}\OperatorTok{\textgreater{}\&}\NormalTok{ args}\OperatorTok{)} \OperatorTok{\{}
\NormalTok{  Isolate}\OperatorTok{*}\NormalTok{ isolate }\OperatorTok{=}\NormalTok{ args}\OperatorTok{.}\NormalTok{GetIsolate}\OperatorTok{();}
\NormalTok{  Local}\OperatorTok{\textless{}}\NormalTok{Context}\OperatorTok{\textgreater{}}\NormalTok{ context }\OperatorTok{=}\NormalTok{ isolate}\OperatorTok{{-}\textgreater{}}\NormalTok{GetCurrentContext}\OperatorTok{();}

  \ControlFlowTok{if} \OperatorTok{(}\NormalTok{args}\OperatorTok{.}\NormalTok{IsConstructCall}\OperatorTok{())} \OperatorTok{\{}
    \CommentTok{// Invoked as constructor: \textasciigrave{}new MyObject(...)\textasciigrave{}}
    \DataTypeTok{double}\NormalTok{ value }\OperatorTok{=}\NormalTok{ args}\OperatorTok{[}\DecValTok{0}\OperatorTok{]{-}\textgreater{}}\NormalTok{IsUndefined}\OperatorTok{()} \OperatorTok{?}
        \DecValTok{0} \OperatorTok{:}\NormalTok{ args}\OperatorTok{[}\DecValTok{0}\OperatorTok{]{-}\textgreater{}}\NormalTok{NumberValue}\OperatorTok{(}\NormalTok{context}\OperatorTok{).}\NormalTok{FromMaybe}\OperatorTok{(}\DecValTok{0}\OperatorTok{);}
\NormalTok{    MyObject}\OperatorTok{*}\NormalTok{ obj }\OperatorTok{=} \KeywordTok{new}\NormalTok{ MyObject}\OperatorTok{(}\NormalTok{value}\OperatorTok{);}
\NormalTok{    obj}\OperatorTok{{-}\textgreater{}}\NormalTok{Wrap}\OperatorTok{(}\NormalTok{args}\OperatorTok{.}\NormalTok{This}\OperatorTok{());}
\NormalTok{    args}\OperatorTok{.}\NormalTok{GetReturnValue}\OperatorTok{().}\NormalTok{Set}\OperatorTok{(}\NormalTok{args}\OperatorTok{.}\NormalTok{This}\OperatorTok{());}
  \OperatorTok{\}} \ControlFlowTok{else} \OperatorTok{\{}
    \CommentTok{// Invoked as plain function \textasciigrave{}MyObject(...)\textasciigrave{}, turn into construct call.}
    \AttributeTok{const} \DataTypeTok{int}\NormalTok{ argc }\OperatorTok{=} \DecValTok{1}\OperatorTok{;}
\NormalTok{    Local}\OperatorTok{\textless{}}\NormalTok{Value}\OperatorTok{\textgreater{}}\NormalTok{ argv}\OperatorTok{[}\NormalTok{argc}\OperatorTok{]} \OperatorTok{=} \OperatorTok{\{}\NormalTok{ args}\OperatorTok{[}\DecValTok{0}\OperatorTok{]} \OperatorTok{\};}
\NormalTok{    Local}\OperatorTok{\textless{}}\NormalTok{Function}\OperatorTok{\textgreater{}}\NormalTok{ cons }\OperatorTok{=}
\NormalTok{        args}\OperatorTok{.}\NormalTok{Data}\OperatorTok{().}\NormalTok{As}\OperatorTok{\textless{}}\NormalTok{Object}\OperatorTok{\textgreater{}(){-}\textgreater{}}\NormalTok{GetInternalField}\OperatorTok{(}\DecValTok{0}\OperatorTok{)}
            \OperatorTok{.}\NormalTok{As}\OperatorTok{\textless{}}\NormalTok{Value}\OperatorTok{\textgreater{}().}\NormalTok{As}\OperatorTok{\textless{}}\NormalTok{Function}\OperatorTok{\textgreater{}();}
\NormalTok{    Local}\OperatorTok{\textless{}}\NormalTok{Object}\OperatorTok{\textgreater{}}\NormalTok{ result }\OperatorTok{=}
\NormalTok{        cons}\OperatorTok{{-}\textgreater{}}\NormalTok{NewInstance}\OperatorTok{(}\NormalTok{context}\OperatorTok{,}\NormalTok{ argc}\OperatorTok{,}\NormalTok{ argv}\OperatorTok{).}\NormalTok{ToLocalChecked}\OperatorTok{();}
\NormalTok{    args}\OperatorTok{.}\NormalTok{GetReturnValue}\OperatorTok{().}\NormalTok{Set}\OperatorTok{(}\NormalTok{result}\OperatorTok{);}
  \OperatorTok{\}}
\OperatorTok{\}}

\DataTypeTok{void}\NormalTok{ MyObject}\OperatorTok{::}\NormalTok{PlusOne}\OperatorTok{(}\AttributeTok{const}\NormalTok{ FunctionCallbackInfo}\OperatorTok{\textless{}}\NormalTok{Value}\OperatorTok{\textgreater{}\&}\NormalTok{ args}\OperatorTok{)} \OperatorTok{\{}
\NormalTok{  Isolate}\OperatorTok{*}\NormalTok{ isolate }\OperatorTok{=}\NormalTok{ args}\OperatorTok{.}\NormalTok{GetIsolate}\OperatorTok{();}

\NormalTok{  MyObject}\OperatorTok{*}\NormalTok{ obj }\OperatorTok{=}\NormalTok{ ObjectWrap}\OperatorTok{::}\NormalTok{Unwrap}\OperatorTok{\textless{}}\NormalTok{MyObject}\OperatorTok{\textgreater{}(}\NormalTok{args}\OperatorTok{.}\NormalTok{Holder}\OperatorTok{());}
\NormalTok{  obj}\OperatorTok{{-}\textgreater{}}\VariableTok{value\_} \OperatorTok{+=} \DecValTok{1}\OperatorTok{;}

\NormalTok{  args}\OperatorTok{.}\NormalTok{GetReturnValue}\OperatorTok{().}\NormalTok{Set}\OperatorTok{(}\NormalTok{Number}\OperatorTok{::}\NormalTok{New}\OperatorTok{(}\NormalTok{isolate}\OperatorTok{,}\NormalTok{ obj}\OperatorTok{{-}\textgreater{}}\VariableTok{value\_}\OperatorTok{));}
\OperatorTok{\}}

\OperatorTok{\}}  \CommentTok{// namespace demo}
\end{Highlighting}
\end{Shaded}

To build this example, the \texttt{myobject.cc} file must be added to
the \texttt{binding.gyp}:

\begin{Shaded}
\begin{Highlighting}[]
\FunctionTok{\{}
  \DataTypeTok{"targets"}\FunctionTok{:} \OtherTok{[}
    \FunctionTok{\{}
      \DataTypeTok{"target\_name"}\FunctionTok{:} \StringTok{"addon"}\FunctionTok{,}
      \DataTypeTok{"sources"}\FunctionTok{:} \OtherTok{[}
        \StringTok{"addon.cc"}\OtherTok{,}
        \StringTok{"myobject.cc"}
      \OtherTok{]}
    \FunctionTok{\}}
  \OtherTok{]}
\FunctionTok{\}}
\end{Highlighting}
\end{Shaded}

Test it with:

\begin{Shaded}
\begin{Highlighting}[]
\CommentTok{// test.js}
\KeywordTok{const}\NormalTok{ addon }\OperatorTok{=} \PreprocessorTok{require}\NormalTok{(}\StringTok{\textquotesingle{}./build/Release/addon\textquotesingle{}}\NormalTok{)}\OperatorTok{;}

\KeywordTok{const}\NormalTok{ obj }\OperatorTok{=} \KeywordTok{new}\NormalTok{ addon}\OperatorTok{.}\FunctionTok{MyObject}\NormalTok{(}\DecValTok{10}\NormalTok{)}\OperatorTok{;}
\BuiltInTok{console}\OperatorTok{.}\FunctionTok{log}\NormalTok{(obj}\OperatorTok{.}\FunctionTok{plusOne}\NormalTok{())}\OperatorTok{;}
\CommentTok{// Prints: 11}
\BuiltInTok{console}\OperatorTok{.}\FunctionTok{log}\NormalTok{(obj}\OperatorTok{.}\FunctionTok{plusOne}\NormalTok{())}\OperatorTok{;}
\CommentTok{// Prints: 12}
\BuiltInTok{console}\OperatorTok{.}\FunctionTok{log}\NormalTok{(obj}\OperatorTok{.}\FunctionTok{plusOne}\NormalTok{())}\OperatorTok{;}
\CommentTok{// Prints: 13}
\end{Highlighting}
\end{Shaded}

The destructor for a wrapper object will run when the object is
garbage-collected. For destructor testing, there are command-line flags
that can be used to make it possible to force garbage collection. These
flags are provided by the underlying V8 JavaScript engine. They are
subject to change or removal at any time. They are not documented by
Node.js or V8, and they should never be used outside of testing.

During shutdown of the process or worker threads destructors are not
called by the JS engine. Therefore it's the responsibility of the user
to track these objects and ensure proper destruction to avoid resource
leaks.

\subsubsection{Factory of wrapped
objects}\label{factory-of-wrapped-objects}

Alternatively, it is possible to use a factory pattern to avoid
explicitly creating object instances using the JavaScript \texttt{new}
operator:

\begin{Shaded}
\begin{Highlighting}[]
\KeywordTok{const}\NormalTok{ obj }\OperatorTok{=}\NormalTok{ addon}\OperatorTok{.}\FunctionTok{createObject}\NormalTok{()}\OperatorTok{;}
\CommentTok{// instead of:}
\CommentTok{// const obj = new addon.Object();}
\end{Highlighting}
\end{Shaded}

First, the \texttt{createObject()} method is implemented in
\texttt{addon.cc}:

\begin{Shaded}
\begin{Highlighting}[]
\CommentTok{// addon.cc}
\PreprocessorTok{\#include }\ImportTok{\textless{}node.h\textgreater{}}
\PreprocessorTok{\#include }\ImportTok{"myobject.h"}

\KeywordTok{namespace}\NormalTok{ demo }\OperatorTok{\{}

\KeywordTok{using}\NormalTok{ v8}\OperatorTok{::}\NormalTok{FunctionCallbackInfo}\OperatorTok{;}
\KeywordTok{using}\NormalTok{ v8}\OperatorTok{::}\NormalTok{Isolate}\OperatorTok{;}
\KeywordTok{using}\NormalTok{ v8}\OperatorTok{::}\NormalTok{Local}\OperatorTok{;}
\KeywordTok{using}\NormalTok{ v8}\OperatorTok{::}\NormalTok{Object}\OperatorTok{;}
\KeywordTok{using}\NormalTok{ v8}\OperatorTok{::}\NormalTok{String}\OperatorTok{;}
\KeywordTok{using}\NormalTok{ v8}\OperatorTok{::}\NormalTok{Value}\OperatorTok{;}

\DataTypeTok{void}\NormalTok{ CreateObject}\OperatorTok{(}\AttributeTok{const}\NormalTok{ FunctionCallbackInfo}\OperatorTok{\textless{}}\NormalTok{Value}\OperatorTok{\textgreater{}\&}\NormalTok{ args}\OperatorTok{)} \OperatorTok{\{}
\NormalTok{  MyObject}\OperatorTok{::}\NormalTok{NewInstance}\OperatorTok{(}\NormalTok{args}\OperatorTok{);}
\OperatorTok{\}}

\DataTypeTok{void}\NormalTok{ InitAll}\OperatorTok{(}\NormalTok{Local}\OperatorTok{\textless{}}\NormalTok{Object}\OperatorTok{\textgreater{}}\NormalTok{ exports}\OperatorTok{,}\NormalTok{ Local}\OperatorTok{\textless{}}\NormalTok{Object}\OperatorTok{\textgreater{}} \KeywordTok{module}\OperatorTok{)} \OperatorTok{\{}
\NormalTok{  MyObject}\OperatorTok{::}\NormalTok{Init}\OperatorTok{(}\NormalTok{exports}\OperatorTok{{-}\textgreater{}}\NormalTok{GetIsolate}\OperatorTok{());}

\NormalTok{  NODE\_SET\_METHOD}\OperatorTok{(}\KeywordTok{module}\OperatorTok{,} \StringTok{"exports"}\OperatorTok{,}\NormalTok{ CreateObject}\OperatorTok{);}
\OperatorTok{\}}

\NormalTok{NODE\_MODULE}\OperatorTok{(}\NormalTok{NODE\_GYP\_MODULE\_NAME}\OperatorTok{,}\NormalTok{ InitAll}\OperatorTok{)}

\OperatorTok{\}}  \CommentTok{// namespace demo}
\end{Highlighting}
\end{Shaded}

In \texttt{myobject.h}, the static method \texttt{NewInstance()} is
added to handle instantiating the object. This method takes the place of
using \texttt{new} in JavaScript:

\begin{Shaded}
\begin{Highlighting}[]
\CommentTok{// myobject.h}
\PreprocessorTok{\#ifndef MYOBJECT\_H}
\PreprocessorTok{\#define MYOBJECT\_H}

\PreprocessorTok{\#include }\ImportTok{\textless{}node.h\textgreater{}}
\PreprocessorTok{\#include }\ImportTok{\textless{}node\_object\_wrap.h\textgreater{}}

\KeywordTok{namespace}\NormalTok{ demo }\OperatorTok{\{}

\KeywordTok{class}\NormalTok{ MyObject }\OperatorTok{:} \KeywordTok{public}\NormalTok{ node}\OperatorTok{::}\NormalTok{ObjectWrap }\OperatorTok{\{}
 \KeywordTok{public}\OperatorTok{:}
  \AttributeTok{static} \DataTypeTok{void}\NormalTok{ Init}\OperatorTok{(}\NormalTok{v8}\OperatorTok{::}\NormalTok{Isolate}\OperatorTok{*}\NormalTok{ isolate}\OperatorTok{);}
  \AttributeTok{static} \DataTypeTok{void}\NormalTok{ NewInstance}\OperatorTok{(}\AttributeTok{const}\NormalTok{ v8}\OperatorTok{::}\NormalTok{FunctionCallbackInfo}\OperatorTok{\textless{}}\NormalTok{v8}\OperatorTok{::}\NormalTok{Value}\OperatorTok{\textgreater{}\&}\NormalTok{ args}\OperatorTok{);}

 \KeywordTok{private}\OperatorTok{:}
  \KeywordTok{explicit}\NormalTok{ MyObject}\OperatorTok{(}\DataTypeTok{double}\NormalTok{ value }\OperatorTok{=} \DecValTok{0}\OperatorTok{);}
  \OperatorTok{\textasciitilde{}}\NormalTok{MyObject}\OperatorTok{();}

  \AttributeTok{static} \DataTypeTok{void}\NormalTok{ New}\OperatorTok{(}\AttributeTok{const}\NormalTok{ v8}\OperatorTok{::}\NormalTok{FunctionCallbackInfo}\OperatorTok{\textless{}}\NormalTok{v8}\OperatorTok{::}\NormalTok{Value}\OperatorTok{\textgreater{}\&}\NormalTok{ args}\OperatorTok{);}
  \AttributeTok{static} \DataTypeTok{void}\NormalTok{ PlusOne}\OperatorTok{(}\AttributeTok{const}\NormalTok{ v8}\OperatorTok{::}\NormalTok{FunctionCallbackInfo}\OperatorTok{\textless{}}\NormalTok{v8}\OperatorTok{::}\NormalTok{Value}\OperatorTok{\textgreater{}\&}\NormalTok{ args}\OperatorTok{);}
  \AttributeTok{static}\NormalTok{ v8}\OperatorTok{::}\NormalTok{Global}\OperatorTok{\textless{}}\NormalTok{v8}\OperatorTok{::}\NormalTok{Function}\OperatorTok{\textgreater{}}\NormalTok{ constructor}\OperatorTok{;}
  \DataTypeTok{double} \VariableTok{value\_}\OperatorTok{;}
\OperatorTok{\};}

\OperatorTok{\}}  \CommentTok{// namespace demo}

\PreprocessorTok{\#endif}
\end{Highlighting}
\end{Shaded}

The implementation in \texttt{myobject.cc} is similar to the previous
example:

\begin{Shaded}
\begin{Highlighting}[]
\CommentTok{// myobject.cc}
\PreprocessorTok{\#include }\ImportTok{\textless{}node.h\textgreater{}}
\PreprocessorTok{\#include }\ImportTok{"myobject.h"}

\KeywordTok{namespace}\NormalTok{ demo }\OperatorTok{\{}

\KeywordTok{using}\NormalTok{ node}\OperatorTok{::}\NormalTok{AddEnvironmentCleanupHook}\OperatorTok{;}
\KeywordTok{using}\NormalTok{ v8}\OperatorTok{::}\NormalTok{Context}\OperatorTok{;}
\KeywordTok{using}\NormalTok{ v8}\OperatorTok{::}\NormalTok{Function}\OperatorTok{;}
\KeywordTok{using}\NormalTok{ v8}\OperatorTok{::}\NormalTok{FunctionCallbackInfo}\OperatorTok{;}
\KeywordTok{using}\NormalTok{ v8}\OperatorTok{::}\NormalTok{FunctionTemplate}\OperatorTok{;}
\KeywordTok{using}\NormalTok{ v8}\OperatorTok{::}\NormalTok{Global}\OperatorTok{;}
\KeywordTok{using}\NormalTok{ v8}\OperatorTok{::}\NormalTok{Isolate}\OperatorTok{;}
\KeywordTok{using}\NormalTok{ v8}\OperatorTok{::}\NormalTok{Local}\OperatorTok{;}
\KeywordTok{using}\NormalTok{ v8}\OperatorTok{::}\NormalTok{Number}\OperatorTok{;}
\KeywordTok{using}\NormalTok{ v8}\OperatorTok{::}\NormalTok{Object}\OperatorTok{;}
\KeywordTok{using}\NormalTok{ v8}\OperatorTok{::}\NormalTok{String}\OperatorTok{;}
\KeywordTok{using}\NormalTok{ v8}\OperatorTok{::}\NormalTok{Value}\OperatorTok{;}

\CommentTok{// Warning! This is not thread{-}safe, this addon cannot be used for worker}
\CommentTok{// threads.}
\NormalTok{Global}\OperatorTok{\textless{}}\NormalTok{Function}\OperatorTok{\textgreater{}}\NormalTok{ MyObject}\OperatorTok{::}\NormalTok{constructor}\OperatorTok{;}

\NormalTok{MyObject}\OperatorTok{::}\NormalTok{MyObject}\OperatorTok{(}\DataTypeTok{double}\NormalTok{ value}\OperatorTok{)} \OperatorTok{:} \VariableTok{value\_}\OperatorTok{(}\NormalTok{value}\OperatorTok{)} \OperatorTok{\{}
\OperatorTok{\}}

\NormalTok{MyObject}\OperatorTok{::\textasciitilde{}}\NormalTok{MyObject}\OperatorTok{()} \OperatorTok{\{}
\OperatorTok{\}}

\DataTypeTok{void}\NormalTok{ MyObject}\OperatorTok{::}\NormalTok{Init}\OperatorTok{(}\NormalTok{Isolate}\OperatorTok{*}\NormalTok{ isolate}\OperatorTok{)} \OperatorTok{\{}
  \CommentTok{// Prepare constructor template}
\NormalTok{  Local}\OperatorTok{\textless{}}\NormalTok{FunctionTemplate}\OperatorTok{\textgreater{}}\NormalTok{ tpl }\OperatorTok{=}\NormalTok{ FunctionTemplate}\OperatorTok{::}\NormalTok{New}\OperatorTok{(}\NormalTok{isolate}\OperatorTok{,}\NormalTok{ New}\OperatorTok{);}
\NormalTok{  tpl}\OperatorTok{{-}\textgreater{}}\NormalTok{SetClassName}\OperatorTok{(}\NormalTok{String}\OperatorTok{::}\NormalTok{NewFromUtf8}\OperatorTok{(}\NormalTok{isolate}\OperatorTok{,} \StringTok{"MyObject"}\OperatorTok{).}\NormalTok{ToLocalChecked}\OperatorTok{());}
\NormalTok{  tpl}\OperatorTok{{-}\textgreater{}}\NormalTok{InstanceTemplate}\OperatorTok{(){-}\textgreater{}}\NormalTok{SetInternalFieldCount}\OperatorTok{(}\DecValTok{1}\OperatorTok{);}

  \CommentTok{// Prototype}
\NormalTok{  NODE\_SET\_PROTOTYPE\_METHOD}\OperatorTok{(}\NormalTok{tpl}\OperatorTok{,} \StringTok{"plusOne"}\OperatorTok{,}\NormalTok{ PlusOne}\OperatorTok{);}

\NormalTok{  Local}\OperatorTok{\textless{}}\NormalTok{Context}\OperatorTok{\textgreater{}}\NormalTok{ context }\OperatorTok{=}\NormalTok{ isolate}\OperatorTok{{-}\textgreater{}}\NormalTok{GetCurrentContext}\OperatorTok{();}
\NormalTok{  constructor}\OperatorTok{.}\NormalTok{Reset}\OperatorTok{(}\NormalTok{isolate}\OperatorTok{,}\NormalTok{ tpl}\OperatorTok{{-}\textgreater{}}\NormalTok{GetFunction}\OperatorTok{(}\NormalTok{context}\OperatorTok{).}\NormalTok{ToLocalChecked}\OperatorTok{());}

\NormalTok{  AddEnvironmentCleanupHook}\OperatorTok{(}\NormalTok{isolate}\OperatorTok{,} \OperatorTok{[](}\DataTypeTok{void}\OperatorTok{*)} \OperatorTok{\{}
\NormalTok{    constructor}\OperatorTok{.}\NormalTok{Reset}\OperatorTok{();}
  \OperatorTok{\},} \KeywordTok{nullptr}\OperatorTok{);}
\OperatorTok{\}}

\DataTypeTok{void}\NormalTok{ MyObject}\OperatorTok{::}\NormalTok{New}\OperatorTok{(}\AttributeTok{const}\NormalTok{ FunctionCallbackInfo}\OperatorTok{\textless{}}\NormalTok{Value}\OperatorTok{\textgreater{}\&}\NormalTok{ args}\OperatorTok{)} \OperatorTok{\{}
\NormalTok{  Isolate}\OperatorTok{*}\NormalTok{ isolate }\OperatorTok{=}\NormalTok{ args}\OperatorTok{.}\NormalTok{GetIsolate}\OperatorTok{();}
\NormalTok{  Local}\OperatorTok{\textless{}}\NormalTok{Context}\OperatorTok{\textgreater{}}\NormalTok{ context }\OperatorTok{=}\NormalTok{ isolate}\OperatorTok{{-}\textgreater{}}\NormalTok{GetCurrentContext}\OperatorTok{();}

  \ControlFlowTok{if} \OperatorTok{(}\NormalTok{args}\OperatorTok{.}\NormalTok{IsConstructCall}\OperatorTok{())} \OperatorTok{\{}
    \CommentTok{// Invoked as constructor: \textasciigrave{}new MyObject(...)\textasciigrave{}}
    \DataTypeTok{double}\NormalTok{ value }\OperatorTok{=}\NormalTok{ args}\OperatorTok{[}\DecValTok{0}\OperatorTok{]{-}\textgreater{}}\NormalTok{IsUndefined}\OperatorTok{()} \OperatorTok{?}
        \DecValTok{0} \OperatorTok{:}\NormalTok{ args}\OperatorTok{[}\DecValTok{0}\OperatorTok{]{-}\textgreater{}}\NormalTok{NumberValue}\OperatorTok{(}\NormalTok{context}\OperatorTok{).}\NormalTok{FromMaybe}\OperatorTok{(}\DecValTok{0}\OperatorTok{);}
\NormalTok{    MyObject}\OperatorTok{*}\NormalTok{ obj }\OperatorTok{=} \KeywordTok{new}\NormalTok{ MyObject}\OperatorTok{(}\NormalTok{value}\OperatorTok{);}
\NormalTok{    obj}\OperatorTok{{-}\textgreater{}}\NormalTok{Wrap}\OperatorTok{(}\NormalTok{args}\OperatorTok{.}\NormalTok{This}\OperatorTok{());}
\NormalTok{    args}\OperatorTok{.}\NormalTok{GetReturnValue}\OperatorTok{().}\NormalTok{Set}\OperatorTok{(}\NormalTok{args}\OperatorTok{.}\NormalTok{This}\OperatorTok{());}
  \OperatorTok{\}} \ControlFlowTok{else} \OperatorTok{\{}
    \CommentTok{// Invoked as plain function \textasciigrave{}MyObject(...)\textasciigrave{}, turn into construct call.}
    \AttributeTok{const} \DataTypeTok{int}\NormalTok{ argc }\OperatorTok{=} \DecValTok{1}\OperatorTok{;}
\NormalTok{    Local}\OperatorTok{\textless{}}\NormalTok{Value}\OperatorTok{\textgreater{}}\NormalTok{ argv}\OperatorTok{[}\NormalTok{argc}\OperatorTok{]} \OperatorTok{=} \OperatorTok{\{}\NormalTok{ args}\OperatorTok{[}\DecValTok{0}\OperatorTok{]} \OperatorTok{\};}
\NormalTok{    Local}\OperatorTok{\textless{}}\NormalTok{Function}\OperatorTok{\textgreater{}}\NormalTok{ cons }\OperatorTok{=}\NormalTok{ Local}\OperatorTok{\textless{}}\NormalTok{Function}\OperatorTok{\textgreater{}::}\NormalTok{New}\OperatorTok{(}\NormalTok{isolate}\OperatorTok{,}\NormalTok{ constructor}\OperatorTok{);}
\NormalTok{    Local}\OperatorTok{\textless{}}\NormalTok{Object}\OperatorTok{\textgreater{}}\NormalTok{ instance }\OperatorTok{=}
\NormalTok{        cons}\OperatorTok{{-}\textgreater{}}\NormalTok{NewInstance}\OperatorTok{(}\NormalTok{context}\OperatorTok{,}\NormalTok{ argc}\OperatorTok{,}\NormalTok{ argv}\OperatorTok{).}\NormalTok{ToLocalChecked}\OperatorTok{();}
\NormalTok{    args}\OperatorTok{.}\NormalTok{GetReturnValue}\OperatorTok{().}\NormalTok{Set}\OperatorTok{(}\NormalTok{instance}\OperatorTok{);}
  \OperatorTok{\}}
\OperatorTok{\}}

\DataTypeTok{void}\NormalTok{ MyObject}\OperatorTok{::}\NormalTok{NewInstance}\OperatorTok{(}\AttributeTok{const}\NormalTok{ FunctionCallbackInfo}\OperatorTok{\textless{}}\NormalTok{Value}\OperatorTok{\textgreater{}\&}\NormalTok{ args}\OperatorTok{)} \OperatorTok{\{}
\NormalTok{  Isolate}\OperatorTok{*}\NormalTok{ isolate }\OperatorTok{=}\NormalTok{ args}\OperatorTok{.}\NormalTok{GetIsolate}\OperatorTok{();}

  \AttributeTok{const} \DataTypeTok{unsigned}\NormalTok{ argc }\OperatorTok{=} \DecValTok{1}\OperatorTok{;}
\NormalTok{  Local}\OperatorTok{\textless{}}\NormalTok{Value}\OperatorTok{\textgreater{}}\NormalTok{ argv}\OperatorTok{[}\NormalTok{argc}\OperatorTok{]} \OperatorTok{=} \OperatorTok{\{}\NormalTok{ args}\OperatorTok{[}\DecValTok{0}\OperatorTok{]} \OperatorTok{\};}
\NormalTok{  Local}\OperatorTok{\textless{}}\NormalTok{Function}\OperatorTok{\textgreater{}}\NormalTok{ cons }\OperatorTok{=}\NormalTok{ Local}\OperatorTok{\textless{}}\NormalTok{Function}\OperatorTok{\textgreater{}::}\NormalTok{New}\OperatorTok{(}\NormalTok{isolate}\OperatorTok{,}\NormalTok{ constructor}\OperatorTok{);}
\NormalTok{  Local}\OperatorTok{\textless{}}\NormalTok{Context}\OperatorTok{\textgreater{}}\NormalTok{ context }\OperatorTok{=}\NormalTok{ isolate}\OperatorTok{{-}\textgreater{}}\NormalTok{GetCurrentContext}\OperatorTok{();}
\NormalTok{  Local}\OperatorTok{\textless{}}\NormalTok{Object}\OperatorTok{\textgreater{}}\NormalTok{ instance }\OperatorTok{=}
\NormalTok{      cons}\OperatorTok{{-}\textgreater{}}\NormalTok{NewInstance}\OperatorTok{(}\NormalTok{context}\OperatorTok{,}\NormalTok{ argc}\OperatorTok{,}\NormalTok{ argv}\OperatorTok{).}\NormalTok{ToLocalChecked}\OperatorTok{();}

\NormalTok{  args}\OperatorTok{.}\NormalTok{GetReturnValue}\OperatorTok{().}\NormalTok{Set}\OperatorTok{(}\NormalTok{instance}\OperatorTok{);}
\OperatorTok{\}}

\DataTypeTok{void}\NormalTok{ MyObject}\OperatorTok{::}\NormalTok{PlusOne}\OperatorTok{(}\AttributeTok{const}\NormalTok{ FunctionCallbackInfo}\OperatorTok{\textless{}}\NormalTok{Value}\OperatorTok{\textgreater{}\&}\NormalTok{ args}\OperatorTok{)} \OperatorTok{\{}
\NormalTok{  Isolate}\OperatorTok{*}\NormalTok{ isolate }\OperatorTok{=}\NormalTok{ args}\OperatorTok{.}\NormalTok{GetIsolate}\OperatorTok{();}

\NormalTok{  MyObject}\OperatorTok{*}\NormalTok{ obj }\OperatorTok{=}\NormalTok{ ObjectWrap}\OperatorTok{::}\NormalTok{Unwrap}\OperatorTok{\textless{}}\NormalTok{MyObject}\OperatorTok{\textgreater{}(}\NormalTok{args}\OperatorTok{.}\NormalTok{Holder}\OperatorTok{());}
\NormalTok{  obj}\OperatorTok{{-}\textgreater{}}\VariableTok{value\_} \OperatorTok{+=} \DecValTok{1}\OperatorTok{;}

\NormalTok{  args}\OperatorTok{.}\NormalTok{GetReturnValue}\OperatorTok{().}\NormalTok{Set}\OperatorTok{(}\NormalTok{Number}\OperatorTok{::}\NormalTok{New}\OperatorTok{(}\NormalTok{isolate}\OperatorTok{,}\NormalTok{ obj}\OperatorTok{{-}\textgreater{}}\VariableTok{value\_}\OperatorTok{));}
\OperatorTok{\}}

\OperatorTok{\}}  \CommentTok{// namespace demo}
\end{Highlighting}
\end{Shaded}

Once again, to build this example, the \texttt{myobject.cc} file must be
added to the \texttt{binding.gyp}:

\begin{Shaded}
\begin{Highlighting}[]
\FunctionTok{\{}
  \DataTypeTok{"targets"}\FunctionTok{:} \OtherTok{[}
    \FunctionTok{\{}
      \DataTypeTok{"target\_name"}\FunctionTok{:} \StringTok{"addon"}\FunctionTok{,}
      \DataTypeTok{"sources"}\FunctionTok{:} \OtherTok{[}
        \StringTok{"addon.cc"}\OtherTok{,}
        \StringTok{"myobject.cc"}
      \OtherTok{]}
    \FunctionTok{\}}
  \OtherTok{]}
\FunctionTok{\}}
\end{Highlighting}
\end{Shaded}

Test it with:

\begin{Shaded}
\begin{Highlighting}[]
\CommentTok{// test.js}
\KeywordTok{const}\NormalTok{ createObject }\OperatorTok{=} \PreprocessorTok{require}\NormalTok{(}\StringTok{\textquotesingle{}./build/Release/addon\textquotesingle{}}\NormalTok{)}\OperatorTok{;}

\KeywordTok{const}\NormalTok{ obj }\OperatorTok{=} \FunctionTok{createObject}\NormalTok{(}\DecValTok{10}\NormalTok{)}\OperatorTok{;}
\BuiltInTok{console}\OperatorTok{.}\FunctionTok{log}\NormalTok{(obj}\OperatorTok{.}\FunctionTok{plusOne}\NormalTok{())}\OperatorTok{;}
\CommentTok{// Prints: 11}
\BuiltInTok{console}\OperatorTok{.}\FunctionTok{log}\NormalTok{(obj}\OperatorTok{.}\FunctionTok{plusOne}\NormalTok{())}\OperatorTok{;}
\CommentTok{// Prints: 12}
\BuiltInTok{console}\OperatorTok{.}\FunctionTok{log}\NormalTok{(obj}\OperatorTok{.}\FunctionTok{plusOne}\NormalTok{())}\OperatorTok{;}
\CommentTok{// Prints: 13}

\KeywordTok{const}\NormalTok{ obj2 }\OperatorTok{=} \FunctionTok{createObject}\NormalTok{(}\DecValTok{20}\NormalTok{)}\OperatorTok{;}
\BuiltInTok{console}\OperatorTok{.}\FunctionTok{log}\NormalTok{(obj2}\OperatorTok{.}\FunctionTok{plusOne}\NormalTok{())}\OperatorTok{;}
\CommentTok{// Prints: 21}
\BuiltInTok{console}\OperatorTok{.}\FunctionTok{log}\NormalTok{(obj2}\OperatorTok{.}\FunctionTok{plusOne}\NormalTok{())}\OperatorTok{;}
\CommentTok{// Prints: 22}
\BuiltInTok{console}\OperatorTok{.}\FunctionTok{log}\NormalTok{(obj2}\OperatorTok{.}\FunctionTok{plusOne}\NormalTok{())}\OperatorTok{;}
\CommentTok{// Prints: 23}
\end{Highlighting}
\end{Shaded}

\subsubsection{Passing wrapped objects
around}\label{passing-wrapped-objects-around}

In addition to wrapping and returning C++ objects, it is possible to
pass wrapped objects around by unwrapping them with the Node.js helper
function \texttt{node::ObjectWrap::Unwrap}. The following examples shows
a function \texttt{add()} that can take two \texttt{MyObject} objects as
input arguments:

\begin{Shaded}
\begin{Highlighting}[]
\CommentTok{// addon.cc}
\PreprocessorTok{\#include }\ImportTok{\textless{}node.h\textgreater{}}
\PreprocessorTok{\#include }\ImportTok{\textless{}node\_object\_wrap.h\textgreater{}}
\PreprocessorTok{\#include }\ImportTok{"myobject.h"}

\KeywordTok{namespace}\NormalTok{ demo }\OperatorTok{\{}

\KeywordTok{using}\NormalTok{ v8}\OperatorTok{::}\NormalTok{Context}\OperatorTok{;}
\KeywordTok{using}\NormalTok{ v8}\OperatorTok{::}\NormalTok{FunctionCallbackInfo}\OperatorTok{;}
\KeywordTok{using}\NormalTok{ v8}\OperatorTok{::}\NormalTok{Isolate}\OperatorTok{;}
\KeywordTok{using}\NormalTok{ v8}\OperatorTok{::}\NormalTok{Local}\OperatorTok{;}
\KeywordTok{using}\NormalTok{ v8}\OperatorTok{::}\NormalTok{Number}\OperatorTok{;}
\KeywordTok{using}\NormalTok{ v8}\OperatorTok{::}\NormalTok{Object}\OperatorTok{;}
\KeywordTok{using}\NormalTok{ v8}\OperatorTok{::}\NormalTok{String}\OperatorTok{;}
\KeywordTok{using}\NormalTok{ v8}\OperatorTok{::}\NormalTok{Value}\OperatorTok{;}

\DataTypeTok{void}\NormalTok{ CreateObject}\OperatorTok{(}\AttributeTok{const}\NormalTok{ FunctionCallbackInfo}\OperatorTok{\textless{}}\NormalTok{Value}\OperatorTok{\textgreater{}\&}\NormalTok{ args}\OperatorTok{)} \OperatorTok{\{}
\NormalTok{  MyObject}\OperatorTok{::}\NormalTok{NewInstance}\OperatorTok{(}\NormalTok{args}\OperatorTok{);}
\OperatorTok{\}}

\DataTypeTok{void}\NormalTok{ Add}\OperatorTok{(}\AttributeTok{const}\NormalTok{ FunctionCallbackInfo}\OperatorTok{\textless{}}\NormalTok{Value}\OperatorTok{\textgreater{}\&}\NormalTok{ args}\OperatorTok{)} \OperatorTok{\{}
\NormalTok{  Isolate}\OperatorTok{*}\NormalTok{ isolate }\OperatorTok{=}\NormalTok{ args}\OperatorTok{.}\NormalTok{GetIsolate}\OperatorTok{();}
\NormalTok{  Local}\OperatorTok{\textless{}}\NormalTok{Context}\OperatorTok{\textgreater{}}\NormalTok{ context }\OperatorTok{=}\NormalTok{ isolate}\OperatorTok{{-}\textgreater{}}\NormalTok{GetCurrentContext}\OperatorTok{();}

\NormalTok{  MyObject}\OperatorTok{*}\NormalTok{ obj1 }\OperatorTok{=}\NormalTok{ node}\OperatorTok{::}\NormalTok{ObjectWrap}\OperatorTok{::}\NormalTok{Unwrap}\OperatorTok{\textless{}}\NormalTok{MyObject}\OperatorTok{\textgreater{}(}
\NormalTok{      args}\OperatorTok{[}\DecValTok{0}\OperatorTok{]{-}\textgreater{}}\NormalTok{ToObject}\OperatorTok{(}\NormalTok{context}\OperatorTok{).}\NormalTok{ToLocalChecked}\OperatorTok{());}
\NormalTok{  MyObject}\OperatorTok{*}\NormalTok{ obj2 }\OperatorTok{=}\NormalTok{ node}\OperatorTok{::}\NormalTok{ObjectWrap}\OperatorTok{::}\NormalTok{Unwrap}\OperatorTok{\textless{}}\NormalTok{MyObject}\OperatorTok{\textgreater{}(}
\NormalTok{      args}\OperatorTok{[}\DecValTok{1}\OperatorTok{]{-}\textgreater{}}\NormalTok{ToObject}\OperatorTok{(}\NormalTok{context}\OperatorTok{).}\NormalTok{ToLocalChecked}\OperatorTok{());}

  \DataTypeTok{double}\NormalTok{ sum }\OperatorTok{=}\NormalTok{ obj1}\OperatorTok{{-}\textgreater{}}\NormalTok{value}\OperatorTok{()} \OperatorTok{+}\NormalTok{ obj2}\OperatorTok{{-}\textgreater{}}\NormalTok{value}\OperatorTok{();}
\NormalTok{  args}\OperatorTok{.}\NormalTok{GetReturnValue}\OperatorTok{().}\NormalTok{Set}\OperatorTok{(}\NormalTok{Number}\OperatorTok{::}\NormalTok{New}\OperatorTok{(}\NormalTok{isolate}\OperatorTok{,}\NormalTok{ sum}\OperatorTok{));}
\OperatorTok{\}}

\DataTypeTok{void}\NormalTok{ InitAll}\OperatorTok{(}\NormalTok{Local}\OperatorTok{\textless{}}\NormalTok{Object}\OperatorTok{\textgreater{}}\NormalTok{ exports}\OperatorTok{)} \OperatorTok{\{}
\NormalTok{  MyObject}\OperatorTok{::}\NormalTok{Init}\OperatorTok{(}\NormalTok{exports}\OperatorTok{{-}\textgreater{}}\NormalTok{GetIsolate}\OperatorTok{());}

\NormalTok{  NODE\_SET\_METHOD}\OperatorTok{(}\NormalTok{exports}\OperatorTok{,} \StringTok{"createObject"}\OperatorTok{,}\NormalTok{ CreateObject}\OperatorTok{);}
\NormalTok{  NODE\_SET\_METHOD}\OperatorTok{(}\NormalTok{exports}\OperatorTok{,} \StringTok{"add"}\OperatorTok{,}\NormalTok{ Add}\OperatorTok{);}
\OperatorTok{\}}

\NormalTok{NODE\_MODULE}\OperatorTok{(}\NormalTok{NODE\_GYP\_MODULE\_NAME}\OperatorTok{,}\NormalTok{ InitAll}\OperatorTok{)}

\OperatorTok{\}}  \CommentTok{// namespace demo}
\end{Highlighting}
\end{Shaded}

In \texttt{myobject.h}, a new public method is added to allow access to
private values after unwrapping the object.

\begin{Shaded}
\begin{Highlighting}[]
\CommentTok{// myobject.h}
\PreprocessorTok{\#ifndef MYOBJECT\_H}
\PreprocessorTok{\#define MYOBJECT\_H}

\PreprocessorTok{\#include }\ImportTok{\textless{}node.h\textgreater{}}
\PreprocessorTok{\#include }\ImportTok{\textless{}node\_object\_wrap.h\textgreater{}}

\KeywordTok{namespace}\NormalTok{ demo }\OperatorTok{\{}

\KeywordTok{class}\NormalTok{ MyObject }\OperatorTok{:} \KeywordTok{public}\NormalTok{ node}\OperatorTok{::}\NormalTok{ObjectWrap }\OperatorTok{\{}
 \KeywordTok{public}\OperatorTok{:}
  \AttributeTok{static} \DataTypeTok{void}\NormalTok{ Init}\OperatorTok{(}\NormalTok{v8}\OperatorTok{::}\NormalTok{Isolate}\OperatorTok{*}\NormalTok{ isolate}\OperatorTok{);}
  \AttributeTok{static} \DataTypeTok{void}\NormalTok{ NewInstance}\OperatorTok{(}\AttributeTok{const}\NormalTok{ v8}\OperatorTok{::}\NormalTok{FunctionCallbackInfo}\OperatorTok{\textless{}}\NormalTok{v8}\OperatorTok{::}\NormalTok{Value}\OperatorTok{\textgreater{}\&}\NormalTok{ args}\OperatorTok{);}
  \KeywordTok{inline} \DataTypeTok{double}\NormalTok{ value}\OperatorTok{()} \AttributeTok{const} \OperatorTok{\{} \ControlFlowTok{return} \VariableTok{value\_}\OperatorTok{;} \OperatorTok{\}}

 \KeywordTok{private}\OperatorTok{:}
  \KeywordTok{explicit}\NormalTok{ MyObject}\OperatorTok{(}\DataTypeTok{double}\NormalTok{ value }\OperatorTok{=} \DecValTok{0}\OperatorTok{);}
  \OperatorTok{\textasciitilde{}}\NormalTok{MyObject}\OperatorTok{();}

  \AttributeTok{static} \DataTypeTok{void}\NormalTok{ New}\OperatorTok{(}\AttributeTok{const}\NormalTok{ v8}\OperatorTok{::}\NormalTok{FunctionCallbackInfo}\OperatorTok{\textless{}}\NormalTok{v8}\OperatorTok{::}\NormalTok{Value}\OperatorTok{\textgreater{}\&}\NormalTok{ args}\OperatorTok{);}
  \AttributeTok{static}\NormalTok{ v8}\OperatorTok{::}\NormalTok{Global}\OperatorTok{\textless{}}\NormalTok{v8}\OperatorTok{::}\NormalTok{Function}\OperatorTok{\textgreater{}}\NormalTok{ constructor}\OperatorTok{;}
  \DataTypeTok{double} \VariableTok{value\_}\OperatorTok{;}
\OperatorTok{\};}

\OperatorTok{\}}  \CommentTok{// namespace demo}

\PreprocessorTok{\#endif}
\end{Highlighting}
\end{Shaded}

The implementation of \texttt{myobject.cc} is similar to before:

\begin{Shaded}
\begin{Highlighting}[]
\CommentTok{// myobject.cc}
\PreprocessorTok{\#include }\ImportTok{\textless{}node.h\textgreater{}}
\PreprocessorTok{\#include }\ImportTok{"myobject.h"}

\KeywordTok{namespace}\NormalTok{ demo }\OperatorTok{\{}

\KeywordTok{using}\NormalTok{ node}\OperatorTok{::}\NormalTok{AddEnvironmentCleanupHook}\OperatorTok{;}
\KeywordTok{using}\NormalTok{ v8}\OperatorTok{::}\NormalTok{Context}\OperatorTok{;}
\KeywordTok{using}\NormalTok{ v8}\OperatorTok{::}\NormalTok{Function}\OperatorTok{;}
\KeywordTok{using}\NormalTok{ v8}\OperatorTok{::}\NormalTok{FunctionCallbackInfo}\OperatorTok{;}
\KeywordTok{using}\NormalTok{ v8}\OperatorTok{::}\NormalTok{FunctionTemplate}\OperatorTok{;}
\KeywordTok{using}\NormalTok{ v8}\OperatorTok{::}\NormalTok{Global}\OperatorTok{;}
\KeywordTok{using}\NormalTok{ v8}\OperatorTok{::}\NormalTok{Isolate}\OperatorTok{;}
\KeywordTok{using}\NormalTok{ v8}\OperatorTok{::}\NormalTok{Local}\OperatorTok{;}
\KeywordTok{using}\NormalTok{ v8}\OperatorTok{::}\NormalTok{Object}\OperatorTok{;}
\KeywordTok{using}\NormalTok{ v8}\OperatorTok{::}\NormalTok{String}\OperatorTok{;}
\KeywordTok{using}\NormalTok{ v8}\OperatorTok{::}\NormalTok{Value}\OperatorTok{;}

\CommentTok{// Warning! This is not thread{-}safe, this addon cannot be used for worker}
\CommentTok{// threads.}
\NormalTok{Global}\OperatorTok{\textless{}}\NormalTok{Function}\OperatorTok{\textgreater{}}\NormalTok{ MyObject}\OperatorTok{::}\NormalTok{constructor}\OperatorTok{;}

\NormalTok{MyObject}\OperatorTok{::}\NormalTok{MyObject}\OperatorTok{(}\DataTypeTok{double}\NormalTok{ value}\OperatorTok{)} \OperatorTok{:} \VariableTok{value\_}\OperatorTok{(}\NormalTok{value}\OperatorTok{)} \OperatorTok{\{}
\OperatorTok{\}}

\NormalTok{MyObject}\OperatorTok{::\textasciitilde{}}\NormalTok{MyObject}\OperatorTok{()} \OperatorTok{\{}
\OperatorTok{\}}

\DataTypeTok{void}\NormalTok{ MyObject}\OperatorTok{::}\NormalTok{Init}\OperatorTok{(}\NormalTok{Isolate}\OperatorTok{*}\NormalTok{ isolate}\OperatorTok{)} \OperatorTok{\{}
  \CommentTok{// Prepare constructor template}
\NormalTok{  Local}\OperatorTok{\textless{}}\NormalTok{FunctionTemplate}\OperatorTok{\textgreater{}}\NormalTok{ tpl }\OperatorTok{=}\NormalTok{ FunctionTemplate}\OperatorTok{::}\NormalTok{New}\OperatorTok{(}\NormalTok{isolate}\OperatorTok{,}\NormalTok{ New}\OperatorTok{);}
\NormalTok{  tpl}\OperatorTok{{-}\textgreater{}}\NormalTok{SetClassName}\OperatorTok{(}\NormalTok{String}\OperatorTok{::}\NormalTok{NewFromUtf8}\OperatorTok{(}\NormalTok{isolate}\OperatorTok{,} \StringTok{"MyObject"}\OperatorTok{).}\NormalTok{ToLocalChecked}\OperatorTok{());}
\NormalTok{  tpl}\OperatorTok{{-}\textgreater{}}\NormalTok{InstanceTemplate}\OperatorTok{(){-}\textgreater{}}\NormalTok{SetInternalFieldCount}\OperatorTok{(}\DecValTok{1}\OperatorTok{);}

\NormalTok{  Local}\OperatorTok{\textless{}}\NormalTok{Context}\OperatorTok{\textgreater{}}\NormalTok{ context }\OperatorTok{=}\NormalTok{ isolate}\OperatorTok{{-}\textgreater{}}\NormalTok{GetCurrentContext}\OperatorTok{();}
\NormalTok{  constructor}\OperatorTok{.}\NormalTok{Reset}\OperatorTok{(}\NormalTok{isolate}\OperatorTok{,}\NormalTok{ tpl}\OperatorTok{{-}\textgreater{}}\NormalTok{GetFunction}\OperatorTok{(}\NormalTok{context}\OperatorTok{).}\NormalTok{ToLocalChecked}\OperatorTok{());}

\NormalTok{  AddEnvironmentCleanupHook}\OperatorTok{(}\NormalTok{isolate}\OperatorTok{,} \OperatorTok{[](}\DataTypeTok{void}\OperatorTok{*)} \OperatorTok{\{}
\NormalTok{    constructor}\OperatorTok{.}\NormalTok{Reset}\OperatorTok{();}
  \OperatorTok{\},} \KeywordTok{nullptr}\OperatorTok{);}
\OperatorTok{\}}

\DataTypeTok{void}\NormalTok{ MyObject}\OperatorTok{::}\NormalTok{New}\OperatorTok{(}\AttributeTok{const}\NormalTok{ FunctionCallbackInfo}\OperatorTok{\textless{}}\NormalTok{Value}\OperatorTok{\textgreater{}\&}\NormalTok{ args}\OperatorTok{)} \OperatorTok{\{}
\NormalTok{  Isolate}\OperatorTok{*}\NormalTok{ isolate }\OperatorTok{=}\NormalTok{ args}\OperatorTok{.}\NormalTok{GetIsolate}\OperatorTok{();}
\NormalTok{  Local}\OperatorTok{\textless{}}\NormalTok{Context}\OperatorTok{\textgreater{}}\NormalTok{ context }\OperatorTok{=}\NormalTok{ isolate}\OperatorTok{{-}\textgreater{}}\NormalTok{GetCurrentContext}\OperatorTok{();}

  \ControlFlowTok{if} \OperatorTok{(}\NormalTok{args}\OperatorTok{.}\NormalTok{IsConstructCall}\OperatorTok{())} \OperatorTok{\{}
    \CommentTok{// Invoked as constructor: \textasciigrave{}new MyObject(...)\textasciigrave{}}
    \DataTypeTok{double}\NormalTok{ value }\OperatorTok{=}\NormalTok{ args}\OperatorTok{[}\DecValTok{0}\OperatorTok{]{-}\textgreater{}}\NormalTok{IsUndefined}\OperatorTok{()} \OperatorTok{?}
        \DecValTok{0} \OperatorTok{:}\NormalTok{ args}\OperatorTok{[}\DecValTok{0}\OperatorTok{]{-}\textgreater{}}\NormalTok{NumberValue}\OperatorTok{(}\NormalTok{context}\OperatorTok{).}\NormalTok{FromMaybe}\OperatorTok{(}\DecValTok{0}\OperatorTok{);}
\NormalTok{    MyObject}\OperatorTok{*}\NormalTok{ obj }\OperatorTok{=} \KeywordTok{new}\NormalTok{ MyObject}\OperatorTok{(}\NormalTok{value}\OperatorTok{);}
\NormalTok{    obj}\OperatorTok{{-}\textgreater{}}\NormalTok{Wrap}\OperatorTok{(}\NormalTok{args}\OperatorTok{.}\NormalTok{This}\OperatorTok{());}
\NormalTok{    args}\OperatorTok{.}\NormalTok{GetReturnValue}\OperatorTok{().}\NormalTok{Set}\OperatorTok{(}\NormalTok{args}\OperatorTok{.}\NormalTok{This}\OperatorTok{());}
  \OperatorTok{\}} \ControlFlowTok{else} \OperatorTok{\{}
    \CommentTok{// Invoked as plain function \textasciigrave{}MyObject(...)\textasciigrave{}, turn into construct call.}
    \AttributeTok{const} \DataTypeTok{int}\NormalTok{ argc }\OperatorTok{=} \DecValTok{1}\OperatorTok{;}
\NormalTok{    Local}\OperatorTok{\textless{}}\NormalTok{Value}\OperatorTok{\textgreater{}}\NormalTok{ argv}\OperatorTok{[}\NormalTok{argc}\OperatorTok{]} \OperatorTok{=} \OperatorTok{\{}\NormalTok{ args}\OperatorTok{[}\DecValTok{0}\OperatorTok{]} \OperatorTok{\};}
\NormalTok{    Local}\OperatorTok{\textless{}}\NormalTok{Function}\OperatorTok{\textgreater{}}\NormalTok{ cons }\OperatorTok{=}\NormalTok{ Local}\OperatorTok{\textless{}}\NormalTok{Function}\OperatorTok{\textgreater{}::}\NormalTok{New}\OperatorTok{(}\NormalTok{isolate}\OperatorTok{,}\NormalTok{ constructor}\OperatorTok{);}
\NormalTok{    Local}\OperatorTok{\textless{}}\NormalTok{Object}\OperatorTok{\textgreater{}}\NormalTok{ instance }\OperatorTok{=}
\NormalTok{        cons}\OperatorTok{{-}\textgreater{}}\NormalTok{NewInstance}\OperatorTok{(}\NormalTok{context}\OperatorTok{,}\NormalTok{ argc}\OperatorTok{,}\NormalTok{ argv}\OperatorTok{).}\NormalTok{ToLocalChecked}\OperatorTok{();}
\NormalTok{    args}\OperatorTok{.}\NormalTok{GetReturnValue}\OperatorTok{().}\NormalTok{Set}\OperatorTok{(}\NormalTok{instance}\OperatorTok{);}
  \OperatorTok{\}}
\OperatorTok{\}}

\DataTypeTok{void}\NormalTok{ MyObject}\OperatorTok{::}\NormalTok{NewInstance}\OperatorTok{(}\AttributeTok{const}\NormalTok{ FunctionCallbackInfo}\OperatorTok{\textless{}}\NormalTok{Value}\OperatorTok{\textgreater{}\&}\NormalTok{ args}\OperatorTok{)} \OperatorTok{\{}
\NormalTok{  Isolate}\OperatorTok{*}\NormalTok{ isolate }\OperatorTok{=}\NormalTok{ args}\OperatorTok{.}\NormalTok{GetIsolate}\OperatorTok{();}

  \AttributeTok{const} \DataTypeTok{unsigned}\NormalTok{ argc }\OperatorTok{=} \DecValTok{1}\OperatorTok{;}
\NormalTok{  Local}\OperatorTok{\textless{}}\NormalTok{Value}\OperatorTok{\textgreater{}}\NormalTok{ argv}\OperatorTok{[}\NormalTok{argc}\OperatorTok{]} \OperatorTok{=} \OperatorTok{\{}\NormalTok{ args}\OperatorTok{[}\DecValTok{0}\OperatorTok{]} \OperatorTok{\};}
\NormalTok{  Local}\OperatorTok{\textless{}}\NormalTok{Function}\OperatorTok{\textgreater{}}\NormalTok{ cons }\OperatorTok{=}\NormalTok{ Local}\OperatorTok{\textless{}}\NormalTok{Function}\OperatorTok{\textgreater{}::}\NormalTok{New}\OperatorTok{(}\NormalTok{isolate}\OperatorTok{,}\NormalTok{ constructor}\OperatorTok{);}
\NormalTok{  Local}\OperatorTok{\textless{}}\NormalTok{Context}\OperatorTok{\textgreater{}}\NormalTok{ context }\OperatorTok{=}\NormalTok{ isolate}\OperatorTok{{-}\textgreater{}}\NormalTok{GetCurrentContext}\OperatorTok{();}
\NormalTok{  Local}\OperatorTok{\textless{}}\NormalTok{Object}\OperatorTok{\textgreater{}}\NormalTok{ instance }\OperatorTok{=}
\NormalTok{      cons}\OperatorTok{{-}\textgreater{}}\NormalTok{NewInstance}\OperatorTok{(}\NormalTok{context}\OperatorTok{,}\NormalTok{ argc}\OperatorTok{,}\NormalTok{ argv}\OperatorTok{).}\NormalTok{ToLocalChecked}\OperatorTok{();}

\NormalTok{  args}\OperatorTok{.}\NormalTok{GetReturnValue}\OperatorTok{().}\NormalTok{Set}\OperatorTok{(}\NormalTok{instance}\OperatorTok{);}
\OperatorTok{\}}

\OperatorTok{\}}  \CommentTok{// namespace demo}
\end{Highlighting}
\end{Shaded}

Test it with:

\begin{Shaded}
\begin{Highlighting}[]
\CommentTok{// test.js}
\KeywordTok{const}\NormalTok{ addon }\OperatorTok{=} \PreprocessorTok{require}\NormalTok{(}\StringTok{\textquotesingle{}./build/Release/addon\textquotesingle{}}\NormalTok{)}\OperatorTok{;}

\KeywordTok{const}\NormalTok{ obj1 }\OperatorTok{=}\NormalTok{ addon}\OperatorTok{.}\FunctionTok{createObject}\NormalTok{(}\DecValTok{10}\NormalTok{)}\OperatorTok{;}
\KeywordTok{const}\NormalTok{ obj2 }\OperatorTok{=}\NormalTok{ addon}\OperatorTok{.}\FunctionTok{createObject}\NormalTok{(}\DecValTok{20}\NormalTok{)}\OperatorTok{;}
\KeywordTok{const}\NormalTok{ result }\OperatorTok{=}\NormalTok{ addon}\OperatorTok{.}\FunctionTok{add}\NormalTok{(obj1}\OperatorTok{,}\NormalTok{ obj2)}\OperatorTok{;}

\BuiltInTok{console}\OperatorTok{.}\FunctionTok{log}\NormalTok{(result)}\OperatorTok{;}
\CommentTok{// Prints: 30}
\end{Highlighting}
\end{Shaded}
