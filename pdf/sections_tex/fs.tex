\section{File system}\label{file-system}

\begin{quote}
Stability: 2 - Stable
\end{quote}

The \texttt{node:fs} module enables interacting with the file system in
a way modeled on standard POSIX functions.

To use the promise-based APIs:

\begin{Shaded}
\begin{Highlighting}[]
\ImportTok{import} \OperatorTok{*} \ImportTok{as}\NormalTok{ fs }\ImportTok{from} \StringTok{\textquotesingle{}node:fs/promises\textquotesingle{}}\OperatorTok{;}
\end{Highlighting}
\end{Shaded}

\begin{Shaded}
\begin{Highlighting}[]
\KeywordTok{const}\NormalTok{ fs }\OperatorTok{=} \PreprocessorTok{require}\NormalTok{(}\StringTok{\textquotesingle{}node:fs/promises\textquotesingle{}}\NormalTok{)}\OperatorTok{;}
\end{Highlighting}
\end{Shaded}

To use the callback and sync APIs:

\begin{Shaded}
\begin{Highlighting}[]
\ImportTok{import} \OperatorTok{*} \ImportTok{as}\NormalTok{ fs }\ImportTok{from} \StringTok{\textquotesingle{}node:fs\textquotesingle{}}\OperatorTok{;}
\end{Highlighting}
\end{Shaded}

\begin{Shaded}
\begin{Highlighting}[]
\KeywordTok{const}\NormalTok{ fs }\OperatorTok{=} \PreprocessorTok{require}\NormalTok{(}\StringTok{\textquotesingle{}node:fs\textquotesingle{}}\NormalTok{)}\OperatorTok{;}
\end{Highlighting}
\end{Shaded}

All file system operations have synchronous, callback, and promise-based
forms, and are accessible using both CommonJS syntax and ES6 Modules
(ESM).

\subsection{Promise example}\label{promise-example}

Promise-based operations return a promise that is fulfilled when the
asynchronous operation is complete.

\begin{Shaded}
\begin{Highlighting}[]
\ImportTok{import}\NormalTok{ \{ unlink \} }\ImportTok{from} \StringTok{\textquotesingle{}node:fs/promises\textquotesingle{}}\OperatorTok{;}

\ControlFlowTok{try}\NormalTok{ \{}
  \ControlFlowTok{await} \FunctionTok{unlink}\NormalTok{(}\StringTok{\textquotesingle{}/tmp/hello\textquotesingle{}}\NormalTok{)}\OperatorTok{;}
  \BuiltInTok{console}\OperatorTok{.}\FunctionTok{log}\NormalTok{(}\StringTok{\textquotesingle{}successfully deleted /tmp/hello\textquotesingle{}}\NormalTok{)}\OperatorTok{;}
\NormalTok{\} }\ControlFlowTok{catch}\NormalTok{ (error) \{}
  \BuiltInTok{console}\OperatorTok{.}\FunctionTok{error}\NormalTok{(}\StringTok{\textquotesingle{}there was an error:\textquotesingle{}}\OperatorTok{,}\NormalTok{ error}\OperatorTok{.}\AttributeTok{message}\NormalTok{)}\OperatorTok{;}
\NormalTok{\}}
\end{Highlighting}
\end{Shaded}

\begin{Shaded}
\begin{Highlighting}[]
\KeywordTok{const}\NormalTok{ \{ unlink \} }\OperatorTok{=} \PreprocessorTok{require}\NormalTok{(}\StringTok{\textquotesingle{}node:fs/promises\textquotesingle{}}\NormalTok{)}\OperatorTok{;}

\NormalTok{(}\KeywordTok{async} \KeywordTok{function}\NormalTok{(path) \{}
  \ControlFlowTok{try}\NormalTok{ \{}
    \ControlFlowTok{await} \FunctionTok{unlink}\NormalTok{(path)}\OperatorTok{;}
    \BuiltInTok{console}\OperatorTok{.}\FunctionTok{log}\NormalTok{(}\VerbatimStringTok{\textasciigrave{}successfully deleted }\SpecialCharTok{$\{}\NormalTok{path}\SpecialCharTok{\}}\VerbatimStringTok{\textasciigrave{}}\NormalTok{)}\OperatorTok{;}
\NormalTok{  \} }\ControlFlowTok{catch}\NormalTok{ (error) \{}
    \BuiltInTok{console}\OperatorTok{.}\FunctionTok{error}\NormalTok{(}\StringTok{\textquotesingle{}there was an error:\textquotesingle{}}\OperatorTok{,}\NormalTok{ error}\OperatorTok{.}\AttributeTok{message}\NormalTok{)}\OperatorTok{;}
\NormalTok{  \}}
\NormalTok{\})(}\StringTok{\textquotesingle{}/tmp/hello\textquotesingle{}}\NormalTok{)}\OperatorTok{;}
\end{Highlighting}
\end{Shaded}

\subsection{Callback example}\label{callback-example}

The callback form takes a completion callback function as its last
argument and invokes the operation asynchronously. The arguments passed
to the completion callback depend on the method, but the first argument
is always reserved for an exception. If the operation is completed
successfully, then the first argument is \texttt{null} or
\texttt{undefined}.

\begin{Shaded}
\begin{Highlighting}[]
\ImportTok{import}\NormalTok{ \{ unlink \} }\ImportTok{from} \StringTok{\textquotesingle{}node:fs\textquotesingle{}}\OperatorTok{;}

\FunctionTok{unlink}\NormalTok{(}\StringTok{\textquotesingle{}/tmp/hello\textquotesingle{}}\OperatorTok{,}\NormalTok{ (err) }\KeywordTok{=\textgreater{}}\NormalTok{ \{}
  \ControlFlowTok{if}\NormalTok{ (err) }\ControlFlowTok{throw}\NormalTok{ err}\OperatorTok{;}
  \BuiltInTok{console}\OperatorTok{.}\FunctionTok{log}\NormalTok{(}\StringTok{\textquotesingle{}successfully deleted /tmp/hello\textquotesingle{}}\NormalTok{)}\OperatorTok{;}
\NormalTok{\})}\OperatorTok{;}
\end{Highlighting}
\end{Shaded}

\begin{Shaded}
\begin{Highlighting}[]
\KeywordTok{const}\NormalTok{ \{ unlink \} }\OperatorTok{=} \PreprocessorTok{require}\NormalTok{(}\StringTok{\textquotesingle{}node:fs\textquotesingle{}}\NormalTok{)}\OperatorTok{;}

\FunctionTok{unlink}\NormalTok{(}\StringTok{\textquotesingle{}/tmp/hello\textquotesingle{}}\OperatorTok{,}\NormalTok{ (err) }\KeywordTok{=\textgreater{}}\NormalTok{ \{}
  \ControlFlowTok{if}\NormalTok{ (err) }\ControlFlowTok{throw}\NormalTok{ err}\OperatorTok{;}
  \BuiltInTok{console}\OperatorTok{.}\FunctionTok{log}\NormalTok{(}\StringTok{\textquotesingle{}successfully deleted /tmp/hello\textquotesingle{}}\NormalTok{)}\OperatorTok{;}
\NormalTok{\})}\OperatorTok{;}
\end{Highlighting}
\end{Shaded}

The callback-based versions of the \texttt{node:fs} module APIs are
preferable over the use of the promise APIs when maximal performance
(both in terms of execution time and memory allocation) is required.

\subsection{Synchronous example}\label{synchronous-example}

The synchronous APIs block the Node.js event loop and further JavaScript
execution until the operation is complete. Exceptions are thrown
immediately and can be handled using \texttt{try…catch}, or can be
allowed to bubble up.

\begin{Shaded}
\begin{Highlighting}[]
\ImportTok{import}\NormalTok{ \{ unlinkSync \} }\ImportTok{from} \StringTok{\textquotesingle{}node:fs\textquotesingle{}}\OperatorTok{;}

\ControlFlowTok{try}\NormalTok{ \{}
  \FunctionTok{unlinkSync}\NormalTok{(}\StringTok{\textquotesingle{}/tmp/hello\textquotesingle{}}\NormalTok{)}\OperatorTok{;}
  \BuiltInTok{console}\OperatorTok{.}\FunctionTok{log}\NormalTok{(}\StringTok{\textquotesingle{}successfully deleted /tmp/hello\textquotesingle{}}\NormalTok{)}\OperatorTok{;}
\NormalTok{\} }\ControlFlowTok{catch}\NormalTok{ (err) \{}
  \CommentTok{// handle the error}
\NormalTok{\}}
\end{Highlighting}
\end{Shaded}

\begin{Shaded}
\begin{Highlighting}[]
\KeywordTok{const}\NormalTok{ \{ unlinkSync \} }\OperatorTok{=} \PreprocessorTok{require}\NormalTok{(}\StringTok{\textquotesingle{}node:fs\textquotesingle{}}\NormalTok{)}\OperatorTok{;}

\ControlFlowTok{try}\NormalTok{ \{}
  \FunctionTok{unlinkSync}\NormalTok{(}\StringTok{\textquotesingle{}/tmp/hello\textquotesingle{}}\NormalTok{)}\OperatorTok{;}
  \BuiltInTok{console}\OperatorTok{.}\FunctionTok{log}\NormalTok{(}\StringTok{\textquotesingle{}successfully deleted /tmp/hello\textquotesingle{}}\NormalTok{)}\OperatorTok{;}
\NormalTok{\} }\ControlFlowTok{catch}\NormalTok{ (err) \{}
  \CommentTok{// handle the error}
\NormalTok{\}}
\end{Highlighting}
\end{Shaded}

\subsection{Promises API}\label{promises-api}

The \texttt{fs/promises} API provides asynchronous file system methods
that return promises.

The promise APIs use the underlying Node.js threadpool to perform file
system operations off the event loop thread. These operations are not
synchronized or threadsafe. Care must be taken when performing multiple
concurrent modifications on the same file or data corruption may occur.

\subsubsection{\texorpdfstring{Class:
\texttt{FileHandle}}{Class: FileHandle}}\label{class-filehandle}

A \{FileHandle\} object is an object wrapper for a numeric file
descriptor.

Instances of the \{FileHandle\} object are created by the
\texttt{fsPromises.open()} method.

All \{FileHandle\} objects are \{EventEmitter\}s.

If a \{FileHandle\} is not closed using the \texttt{filehandle.close()}
method, it will try to automatically close the file descriptor and emit
a process warning, helping to prevent memory leaks. Please do not rely
on this behavior because it can be unreliable and the file may not be
closed. Instead, always explicitly close \{FileHandle\}s. Node.js may
change this behavior in the future.

\paragraph{\texorpdfstring{Event:
\texttt{\textquotesingle{}close\textquotesingle{}}}{Event: \textquotesingle close\textquotesingle{}}}\label{event-close}

The \texttt{\textquotesingle{}close\textquotesingle{}} event is emitted
when the \{FileHandle\} has been closed and can no longer be used.

\paragraph{\texorpdfstring{\texttt{filehandle.appendFile(data{[},\ options{]})}}{filehandle.appendFile(data{[}, options{]})}}\label{filehandle.appendfiledata-options}

\begin{itemize}
\tightlist
\item
  \texttt{data}
  \{string\textbar Buffer\textbar TypedArray\textbar DataView\textbar AsyncIterable\textbar Iterable\textbar Stream\}
\item
  \texttt{options} \{Object\textbar string\}

  \begin{itemize}
  \tightlist
  \item
    \texttt{encoding} \{string\textbar null\} \textbf{Default:}
    \texttt{\textquotesingle{}utf8\textquotesingle{}}
  \item
    \texttt{flush} \{boolean\} If \texttt{true}, the underlying file
    descriptor is flushed prior to closing it. \textbf{Default:}
    \texttt{false}.
  \end{itemize}
\item
  Returns: \{Promise\} Fulfills with \texttt{undefined} upon success.
\end{itemize}

Alias of
\hyperref[filehandlewritefiledata-options]{\texttt{filehandle.writeFile()}}.

When operating on file handles, the mode cannot be changed from what it
was set to with
\hyperref[fspromisesopenpath-flags-mode]{\texttt{fsPromises.open()}}.
Therefore, this is equivalent to
\hyperref[filehandlewritefiledata-options]{\texttt{filehandle.writeFile()}}.

\paragraph{\texorpdfstring{\texttt{filehandle.chmod(mode)}}{filehandle.chmod(mode)}}\label{filehandle.chmodmode}

\begin{itemize}
\tightlist
\item
  \texttt{mode} \{integer\} the file mode bit mask.
\item
  Returns: \{Promise\} Fulfills with \texttt{undefined} upon success.
\end{itemize}

Modifies the permissions on the file. See chmod(2).

\paragraph{\texorpdfstring{\texttt{filehandle.chown(uid,\ gid)}}{filehandle.chown(uid, gid)}}\label{filehandle.chownuid-gid}

\begin{itemize}
\tightlist
\item
  \texttt{uid} \{integer\} The file's new owner's user id.
\item
  \texttt{gid} \{integer\} The file's new group's group id.
\item
  Returns: \{Promise\} Fulfills with \texttt{undefined} upon success.
\end{itemize}

Changes the ownership of the file. A wrapper for chown(2).

\paragraph{\texorpdfstring{\texttt{filehandle.close()}}{filehandle.close()}}\label{filehandle.close}

\begin{itemize}
\tightlist
\item
  Returns: \{Promise\} Fulfills with \texttt{undefined} upon success.
\end{itemize}

Closes the file handle after waiting for any pending operation on the
handle to complete.

\begin{Shaded}
\begin{Highlighting}[]
\ImportTok{import}\NormalTok{ \{ open \} }\ImportTok{from} \StringTok{\textquotesingle{}node:fs/promises\textquotesingle{}}\OperatorTok{;}

\KeywordTok{let}\NormalTok{ filehandle}\OperatorTok{;}
\ControlFlowTok{try}\NormalTok{ \{}
\NormalTok{  filehandle }\OperatorTok{=} \ControlFlowTok{await} \FunctionTok{open}\NormalTok{(}\StringTok{\textquotesingle{}thefile.txt\textquotesingle{}}\OperatorTok{,} \StringTok{\textquotesingle{}r\textquotesingle{}}\NormalTok{)}\OperatorTok{;}
\NormalTok{\} }\ControlFlowTok{finally}\NormalTok{ \{}
  \ControlFlowTok{await}\NormalTok{ filehandle}\OperatorTok{?.}\FunctionTok{close}\NormalTok{()}\OperatorTok{;}
\NormalTok{\}}
\end{Highlighting}
\end{Shaded}

\paragraph{\texorpdfstring{\texttt{filehandle.createReadStream({[}options{]})}}{filehandle.createReadStream({[}options{]})}}\label{filehandle.createreadstreamoptions}

\begin{itemize}
\tightlist
\item
  \texttt{options} \{Object\}

  \begin{itemize}
  \tightlist
  \item
    \texttt{encoding} \{string\} \textbf{Default:} \texttt{null}
  \item
    \texttt{autoClose} \{boolean\} \textbf{Default:} \texttt{true}
  \item
    \texttt{emitClose} \{boolean\} \textbf{Default:} \texttt{true}
  \item
    \texttt{start} \{integer\}
  \item
    \texttt{end} \{integer\} \textbf{Default:} \texttt{Infinity}
  \item
    \texttt{highWaterMark} \{integer\} \textbf{Default:}
    \texttt{64\ *\ 1024}
  \end{itemize}
\item
  Returns: \{fs.ReadStream\}
\end{itemize}

Unlike the 16 KiB default \texttt{highWaterMark} for a
\{stream.Readable\}, the stream returned by this method has a default
\texttt{highWaterMark} of 64 KiB.

\texttt{options} can include \texttt{start} and \texttt{end} values to
read a range of bytes from the file instead of the entire file. Both
\texttt{start} and \texttt{end} are inclusive and start counting at 0,
allowed values are in the {[}0,
\href{https://developer.mozilla.org/en-US/docs/Web/JavaScript/Reference/Global_Objects/Number/MAX_SAFE_INTEGER}{\texttt{Number.MAX\_SAFE\_INTEGER}}{]}
range. If \texttt{start} is omitted or \texttt{undefined},
\texttt{filehandle.createReadStream()} reads sequentially from the
current file position. The \texttt{encoding} can be any one of those
accepted by \{Buffer\}.

If the \texttt{FileHandle} points to a character device that only
supports blocking reads (such as keyboard or sound card), read
operations do not finish until data is available. This can prevent the
process from exiting and the stream from closing naturally.

By default, the stream will emit a
\texttt{\textquotesingle{}close\textquotesingle{}} event after it has
been destroyed. Set the \texttt{emitClose} option to \texttt{false} to
change this behavior.

\begin{Shaded}
\begin{Highlighting}[]
\ImportTok{import}\NormalTok{ \{ open \} }\ImportTok{from} \StringTok{\textquotesingle{}node:fs/promises\textquotesingle{}}\OperatorTok{;}

\KeywordTok{const}\NormalTok{ fd }\OperatorTok{=} \ControlFlowTok{await} \FunctionTok{open}\NormalTok{(}\StringTok{\textquotesingle{}/dev/input/event0\textquotesingle{}}\NormalTok{)}\OperatorTok{;}
\CommentTok{// Create a stream from some character device.}
\KeywordTok{const}\NormalTok{ stream }\OperatorTok{=}\NormalTok{ fd}\OperatorTok{.}\FunctionTok{createReadStream}\NormalTok{()}\OperatorTok{;}
\PreprocessorTok{setTimeout}\NormalTok{(() }\KeywordTok{=\textgreater{}}\NormalTok{ \{}
\NormalTok{  stream}\OperatorTok{.}\FunctionTok{close}\NormalTok{()}\OperatorTok{;} \CommentTok{// This may not close the stream.}
  \CommentTok{// Artificially marking end{-}of{-}stream, as if the underlying resource had}
  \CommentTok{// indicated end{-}of{-}file by itself, allows the stream to close.}
  \CommentTok{// This does not cancel pending read operations, and if there is such an}
  \CommentTok{// operation, the process may still not be able to exit successfully}
  \CommentTok{// until it finishes.}
\NormalTok{  stream}\OperatorTok{.}\FunctionTok{push}\NormalTok{(}\KeywordTok{null}\NormalTok{)}\OperatorTok{;}
\NormalTok{  stream}\OperatorTok{.}\FunctionTok{read}\NormalTok{(}\DecValTok{0}\NormalTok{)}\OperatorTok{;}
\NormalTok{\}}\OperatorTok{,} \DecValTok{100}\NormalTok{)}\OperatorTok{;}
\end{Highlighting}
\end{Shaded}

If \texttt{autoClose} is false, then the file descriptor won't be
closed, even if there's an error. It is the application's responsibility
to close it and make sure there's no file descriptor leak. If
\texttt{autoClose} is set to true (default behavior), on
\texttt{\textquotesingle{}error\textquotesingle{}} or
\texttt{\textquotesingle{}end\textquotesingle{}} the file descriptor
will be closed automatically.

An example to read the last 10 bytes of a file which is 100 bytes long:

\begin{Shaded}
\begin{Highlighting}[]
\ImportTok{import}\NormalTok{ \{ open \} }\ImportTok{from} \StringTok{\textquotesingle{}node:fs/promises\textquotesingle{}}\OperatorTok{;}

\KeywordTok{const}\NormalTok{ fd }\OperatorTok{=} \ControlFlowTok{await} \FunctionTok{open}\NormalTok{(}\StringTok{\textquotesingle{}sample.txt\textquotesingle{}}\NormalTok{)}\OperatorTok{;}
\NormalTok{fd}\OperatorTok{.}\FunctionTok{createReadStream}\NormalTok{(\{ }\DataTypeTok{start}\OperatorTok{:} \DecValTok{90}\OperatorTok{,} \DataTypeTok{end}\OperatorTok{:} \DecValTok{99}\NormalTok{ \})}\OperatorTok{;}
\end{Highlighting}
\end{Shaded}

\paragraph{\texorpdfstring{\texttt{filehandle.createWriteStream({[}options{]})}}{filehandle.createWriteStream({[}options{]})}}\label{filehandle.createwritestreamoptions}

\begin{itemize}
\tightlist
\item
  \texttt{options} \{Object\}

  \begin{itemize}
  \tightlist
  \item
    \texttt{encoding} \{string\} \textbf{Default:}
    \texttt{\textquotesingle{}utf8\textquotesingle{}}
  \item
    \texttt{autoClose} \{boolean\} \textbf{Default:} \texttt{true}
  \item
    \texttt{emitClose} \{boolean\} \textbf{Default:} \texttt{true}
  \item
    \texttt{start} \{integer\}
  \item
    \texttt{highWaterMark} \{number\} \textbf{Default:} \texttt{16384}
  \item
    \texttt{flush} \{boolean\} If \texttt{true}, the underlying file
    descriptor is flushed prior to closing it. \textbf{Default:}
    \texttt{false}.
  \end{itemize}
\item
  Returns: \{fs.WriteStream\}
\end{itemize}

\texttt{options} may also include a \texttt{start} option to allow
writing data at some position past the beginning of the file, allowed
values are in the {[}0,
\href{https://developer.mozilla.org/en-US/docs/Web/JavaScript/Reference/Global_Objects/Number/MAX_SAFE_INTEGER}{\texttt{Number.MAX\_SAFE\_INTEGER}}{]}
range. Modifying a file rather than replacing it may require the
\texttt{flags} \texttt{open} option to be set to \texttt{r+} rather than
the default \texttt{r}. The \texttt{encoding} can be any one of those
accepted by \{Buffer\}.

If \texttt{autoClose} is set to true (default behavior) on
\texttt{\textquotesingle{}error\textquotesingle{}} or
\texttt{\textquotesingle{}finish\textquotesingle{}} the file descriptor
will be closed automatically. If \texttt{autoClose} is false, then the
file descriptor won't be closed, even if there's an error. It is the
application's responsibility to close it and make sure there's no file
descriptor leak.

By default, the stream will emit a
\texttt{\textquotesingle{}close\textquotesingle{}} event after it has
been destroyed. Set the \texttt{emitClose} option to \texttt{false} to
change this behavior.

\paragraph{\texorpdfstring{\texttt{filehandle.datasync()}}{filehandle.datasync()}}\label{filehandle.datasync}

\begin{itemize}
\tightlist
\item
  Returns: \{Promise\} Fulfills with \texttt{undefined} upon success.
\end{itemize}

Forces all currently queued I/O operations associated with the file to
the operating system's synchronized I/O completion state. Refer to the
POSIX fdatasync(2) documentation for details.

Unlike \texttt{filehandle.sync} this method does not flush modified
metadata.

\paragraph{\texorpdfstring{\texttt{filehandle.fd}}{filehandle.fd}}\label{filehandle.fd}

\begin{itemize}
\tightlist
\item
  \{number\} The numeric file descriptor managed by the \{FileHandle\}
  object.
\end{itemize}

\paragraph{\texorpdfstring{\texttt{filehandle.read(buffer,\ offset,\ length,\ position)}}{filehandle.read(buffer, offset, length, position)}}\label{filehandle.readbuffer-offset-length-position}

\begin{itemize}
\tightlist
\item
  \texttt{buffer} \{Buffer\textbar TypedArray\textbar DataView\} A
  buffer that will be filled with the file data read.
\item
  \texttt{offset} \{integer\} The location in the buffer at which to
  start filling.
\item
  \texttt{length} \{integer\} The number of bytes to read.
\item
  \texttt{position} \{integer\textbar bigint\textbar null\} The location
  where to begin reading data from the file. If \texttt{null} or
  \texttt{-1}, data will be read from the current file position, and the
  position will be updated. If \texttt{position} is a non-negative
  integer, the current file position will remain unchanged.
\item
  Returns: \{Promise\} Fulfills upon success with an object with two
  properties:

  \begin{itemize}
  \tightlist
  \item
    \texttt{bytesRead} \{integer\} The number of bytes read
  \item
    \texttt{buffer} \{Buffer\textbar TypedArray\textbar DataView\} A
    reference to the passed in \texttt{buffer} argument.
  \end{itemize}
\end{itemize}

Reads data from the file and stores that in the given buffer.

If the file is not modified concurrently, the end-of-file is reached
when the number of bytes read is zero.

\paragraph{\texorpdfstring{\texttt{filehandle.read({[}options{]})}}{filehandle.read({[}options{]})}}\label{filehandle.readoptions}

\begin{itemize}
\tightlist
\item
  \texttt{options} \{Object\}

  \begin{itemize}
  \tightlist
  \item
    \texttt{buffer} \{Buffer\textbar TypedArray\textbar DataView\} A
    buffer that will be filled with the file data read.
    \textbf{Default:} \texttt{Buffer.alloc(16384)}
  \item
    \texttt{offset} \{integer\} The location in the buffer at which to
    start filling. \textbf{Default:} \texttt{0}
  \item
    \texttt{length} \{integer\} The number of bytes to read.
    \textbf{Default:} \texttt{buffer.byteLength\ -\ offset}
  \item
    \texttt{position} \{integer\textbar bigint\textbar null\} The
    location where to begin reading data from the file. If \texttt{null}
    or \texttt{-1}, data will be read from the current file position,
    and the position will be updated. If \texttt{position} is a
    non-negative integer, the current file position will remain
    unchanged. \textbf{Default:}: \texttt{null}
  \end{itemize}
\item
  Returns: \{Promise\} Fulfills upon success with an object with two
  properties:

  \begin{itemize}
  \tightlist
  \item
    \texttt{bytesRead} \{integer\} The number of bytes read
  \item
    \texttt{buffer} \{Buffer\textbar TypedArray\textbar DataView\} A
    reference to the passed in \texttt{buffer} argument.
  \end{itemize}
\end{itemize}

Reads data from the file and stores that in the given buffer.

If the file is not modified concurrently, the end-of-file is reached
when the number of bytes read is zero.

\paragraph{\texorpdfstring{\texttt{filehandle.read(buffer{[},\ options{]})}}{filehandle.read(buffer{[}, options{]})}}\label{filehandle.readbuffer-options}

\begin{itemize}
\tightlist
\item
  \texttt{buffer} \{Buffer\textbar TypedArray\textbar DataView\} A
  buffer that will be filled with the file data read.
\item
  \texttt{options} \{Object\}

  \begin{itemize}
  \tightlist
  \item
    \texttt{offset} \{integer\} The location in the buffer at which to
    start filling. \textbf{Default:} \texttt{0}
  \item
    \texttt{length} \{integer\} The number of bytes to read.
    \textbf{Default:} \texttt{buffer.byteLength\ -\ offset}
  \item
    \texttt{position} \{integer\textbar bigint\textbar null\} The
    location where to begin reading data from the file. If \texttt{null}
    or \texttt{-1}, data will be read from the current file position,
    and the position will be updated. If \texttt{position} is a
    non-negative integer, the current file position will remain
    unchanged. \textbf{Default:}: \texttt{null}
  \end{itemize}
\item
  Returns: \{Promise\} Fulfills upon success with an object with two
  properties:

  \begin{itemize}
  \tightlist
  \item
    \texttt{bytesRead} \{integer\} The number of bytes read
  \item
    \texttt{buffer} \{Buffer\textbar TypedArray\textbar DataView\} A
    reference to the passed in \texttt{buffer} argument.
  \end{itemize}
\end{itemize}

Reads data from the file and stores that in the given buffer.

If the file is not modified concurrently, the end-of-file is reached
when the number of bytes read is zero.

\paragraph{\texorpdfstring{\texttt{filehandle.readableWebStream({[}options{]})}}{filehandle.readableWebStream({[}options{]})}}\label{filehandle.readablewebstreamoptions}

\begin{quote}
Stability: 1 - Experimental
\end{quote}

\begin{itemize}
\tightlist
\item
  \texttt{options} \{Object\}

  \begin{itemize}
  \tightlist
  \item
    \texttt{type} \{string\textbar undefined\} Whether to open a normal
    or a \texttt{\textquotesingle{}bytes\textquotesingle{}} stream.
    \textbf{Default:} \texttt{undefined}
  \end{itemize}
\item
  Returns: \{ReadableStream\}
\end{itemize}

Returns a \texttt{ReadableStream} that may be used to read the files
data.

An error will be thrown if this method is called more than once or is
called after the \texttt{FileHandle} is closed or closing.

\begin{Shaded}
\begin{Highlighting}[]
\ImportTok{import}\NormalTok{ \{}
\NormalTok{  open}\OperatorTok{,}
\NormalTok{\} }\ImportTok{from} \StringTok{\textquotesingle{}node:fs/promises\textquotesingle{}}\OperatorTok{;}

\KeywordTok{const}\NormalTok{ file }\OperatorTok{=} \ControlFlowTok{await} \FunctionTok{open}\NormalTok{(}\StringTok{\textquotesingle{}./some/file/to/read\textquotesingle{}}\NormalTok{)}\OperatorTok{;}

\ControlFlowTok{for} \ControlFlowTok{await}\NormalTok{ (}\KeywordTok{const}\NormalTok{ chunk }\KeywordTok{of}\NormalTok{ file}\OperatorTok{.}\FunctionTok{readableWebStream}\NormalTok{())}
  \BuiltInTok{console}\OperatorTok{.}\FunctionTok{log}\NormalTok{(chunk)}\OperatorTok{;}

\ControlFlowTok{await}\NormalTok{ file}\OperatorTok{.}\FunctionTok{close}\NormalTok{()}\OperatorTok{;}
\end{Highlighting}
\end{Shaded}

\begin{Shaded}
\begin{Highlighting}[]
\KeywordTok{const}\NormalTok{ \{}
\NormalTok{  open}\OperatorTok{,}
\NormalTok{\} }\OperatorTok{=} \PreprocessorTok{require}\NormalTok{(}\StringTok{\textquotesingle{}node:fs/promises\textquotesingle{}}\NormalTok{)}\OperatorTok{;}

\NormalTok{(}\KeywordTok{async}\NormalTok{ () }\KeywordTok{=\textgreater{}}\NormalTok{ \{}
  \KeywordTok{const}\NormalTok{ file }\OperatorTok{=} \ControlFlowTok{await} \FunctionTok{open}\NormalTok{(}\StringTok{\textquotesingle{}./some/file/to/read\textquotesingle{}}\NormalTok{)}\OperatorTok{;}

  \ControlFlowTok{for} \ControlFlowTok{await}\NormalTok{ (}\KeywordTok{const}\NormalTok{ chunk }\KeywordTok{of}\NormalTok{ file}\OperatorTok{.}\FunctionTok{readableWebStream}\NormalTok{())}
    \BuiltInTok{console}\OperatorTok{.}\FunctionTok{log}\NormalTok{(chunk)}\OperatorTok{;}

  \ControlFlowTok{await}\NormalTok{ file}\OperatorTok{.}\FunctionTok{close}\NormalTok{()}\OperatorTok{;}
\NormalTok{\})()}\OperatorTok{;}
\end{Highlighting}
\end{Shaded}

While the \texttt{ReadableStream} will read the file to completion, it
will not close the \texttt{FileHandle} automatically. User code must
still call the \texttt{fileHandle.close()} method.

\paragraph{\texorpdfstring{\texttt{filehandle.readFile(options)}}{filehandle.readFile(options)}}\label{filehandle.readfileoptions}

\begin{itemize}
\tightlist
\item
  \texttt{options} \{Object\textbar string\}

  \begin{itemize}
  \tightlist
  \item
    \texttt{encoding} \{string\textbar null\} \textbf{Default:}
    \texttt{null}
  \item
    \texttt{signal} \{AbortSignal\} allows aborting an in-progress
    readFile
  \end{itemize}
\item
  Returns: \{Promise\} Fulfills upon a successful read with the contents
  of the file. If no encoding is specified (using
  \texttt{options.encoding}), the data is returned as a \{Buffer\}
  object. Otherwise, the data will be a string.
\end{itemize}

Asynchronously reads the entire contents of a file.

If \texttt{options} is a string, then it specifies the
\texttt{encoding}.

The \{FileHandle\} has to support reading.

If one or more \texttt{filehandle.read()} calls are made on a file
handle and then a \texttt{filehandle.readFile()} call is made, the data
will be read from the current position till the end of the file. It
doesn't always read from the beginning of the file.

\paragraph{\texorpdfstring{\texttt{filehandle.readLines({[}options{]})}}{filehandle.readLines({[}options{]})}}\label{filehandle.readlinesoptions}

\begin{itemize}
\tightlist
\item
  \texttt{options} \{Object\}

  \begin{itemize}
  \tightlist
  \item
    \texttt{encoding} \{string\} \textbf{Default:} \texttt{null}
  \item
    \texttt{autoClose} \{boolean\} \textbf{Default:} \texttt{true}
  \item
    \texttt{emitClose} \{boolean\} \textbf{Default:} \texttt{true}
  \item
    \texttt{start} \{integer\}
  \item
    \texttt{end} \{integer\} \textbf{Default:} \texttt{Infinity}
  \item
    \texttt{highWaterMark} \{integer\} \textbf{Default:}
    \texttt{64\ *\ 1024}
  \end{itemize}
\item
  Returns: \{readline.InterfaceConstructor\}
\end{itemize}

Convenience method to create a \texttt{readline} interface and stream
over the file. See
\hyperref[filehandlecreatereadstreamoptions]{\texttt{filehandle.createReadStream()}}
for the options.

\begin{Shaded}
\begin{Highlighting}[]
\ImportTok{import}\NormalTok{ \{ open \} }\ImportTok{from} \StringTok{\textquotesingle{}node:fs/promises\textquotesingle{}}\OperatorTok{;}

\KeywordTok{const}\NormalTok{ file }\OperatorTok{=} \ControlFlowTok{await} \FunctionTok{open}\NormalTok{(}\StringTok{\textquotesingle{}./some/file/to/read\textquotesingle{}}\NormalTok{)}\OperatorTok{;}

\ControlFlowTok{for} \ControlFlowTok{await}\NormalTok{ (}\KeywordTok{const}\NormalTok{ line }\KeywordTok{of}\NormalTok{ file}\OperatorTok{.}\FunctionTok{readLines}\NormalTok{()) \{}
  \BuiltInTok{console}\OperatorTok{.}\FunctionTok{log}\NormalTok{(line)}\OperatorTok{;}
\NormalTok{\}}
\end{Highlighting}
\end{Shaded}

\begin{Shaded}
\begin{Highlighting}[]
\KeywordTok{const}\NormalTok{ \{ open \} }\OperatorTok{=} \PreprocessorTok{require}\NormalTok{(}\StringTok{\textquotesingle{}node:fs/promises\textquotesingle{}}\NormalTok{)}\OperatorTok{;}

\NormalTok{(}\KeywordTok{async}\NormalTok{ () }\KeywordTok{=\textgreater{}}\NormalTok{ \{}
  \KeywordTok{const}\NormalTok{ file }\OperatorTok{=} \ControlFlowTok{await} \FunctionTok{open}\NormalTok{(}\StringTok{\textquotesingle{}./some/file/to/read\textquotesingle{}}\NormalTok{)}\OperatorTok{;}

  \ControlFlowTok{for} \ControlFlowTok{await}\NormalTok{ (}\KeywordTok{const}\NormalTok{ line }\KeywordTok{of}\NormalTok{ file}\OperatorTok{.}\FunctionTok{readLines}\NormalTok{()) \{}
    \BuiltInTok{console}\OperatorTok{.}\FunctionTok{log}\NormalTok{(line)}\OperatorTok{;}
\NormalTok{  \}}
\NormalTok{\})()}\OperatorTok{;}
\end{Highlighting}
\end{Shaded}

\paragraph{\texorpdfstring{\texttt{filehandle.readv(buffers{[},\ position{]})}}{filehandle.readv(buffers{[}, position{]})}}\label{filehandle.readvbuffers-position}

\begin{itemize}
\tightlist
\item
  \texttt{buffers}
  \{Buffer{[}{]}\textbar TypedArray{[}{]}\textbar DataView{[}{]}\}
\item
  \texttt{position} \{integer\textbar null\} The offset from the
  beginning of the file where the data should be read from. If
  \texttt{position} is not a \texttt{number}, the data will be read from
  the current position. \textbf{Default:} \texttt{null}
\item
  Returns: \{Promise\} Fulfills upon success an object containing two
  properties:

  \begin{itemize}
  \tightlist
  \item
    \texttt{bytesRead} \{integer\} the number of bytes read
  \item
    \texttt{buffers}
    \{Buffer{[}{]}\textbar TypedArray{[}{]}\textbar DataView{[}{]}\}
    property containing a reference to the \texttt{buffers} input.
  \end{itemize}
\end{itemize}

Read from a file and write to an array of \{ArrayBufferView\}s

\paragraph{\texorpdfstring{\texttt{filehandle.stat({[}options{]})}}{filehandle.stat({[}options{]})}}\label{filehandle.statoptions}

\begin{itemize}
\tightlist
\item
  \texttt{options} \{Object\}

  \begin{itemize}
  \tightlist
  \item
    \texttt{bigint} \{boolean\} Whether the numeric values in the
    returned \{fs.Stats\} object should be \texttt{bigint}.
    \textbf{Default:} \texttt{false}.
  \end{itemize}
\item
  Returns: \{Promise\} Fulfills with an \{fs.Stats\} for the file.
\end{itemize}

\paragraph{\texorpdfstring{\texttt{filehandle.sync()}}{filehandle.sync()}}\label{filehandle.sync}

\begin{itemize}
\tightlist
\item
  Returns: \{Promise\} Fulfills with \texttt{undefined} upon success.
\end{itemize}

Request that all data for the open file descriptor is flushed to the
storage device. The specific implementation is operating system and
device specific. Refer to the POSIX fsync(2) documentation for more
detail.

\paragraph{\texorpdfstring{\texttt{filehandle.truncate(len)}}{filehandle.truncate(len)}}\label{filehandle.truncatelen}

\begin{itemize}
\tightlist
\item
  \texttt{len} \{integer\} \textbf{Default:} \texttt{0}
\item
  Returns: \{Promise\} Fulfills with \texttt{undefined} upon success.
\end{itemize}

Truncates the file.

If the file was larger than \texttt{len} bytes, only the first
\texttt{len} bytes will be retained in the file.

The following example retains only the first four bytes of the file:

\begin{Shaded}
\begin{Highlighting}[]
\ImportTok{import}\NormalTok{ \{ open \} }\ImportTok{from} \StringTok{\textquotesingle{}node:fs/promises\textquotesingle{}}\OperatorTok{;}

\KeywordTok{let}\NormalTok{ filehandle }\OperatorTok{=} \KeywordTok{null}\OperatorTok{;}
\ControlFlowTok{try}\NormalTok{ \{}
\NormalTok{  filehandle }\OperatorTok{=} \ControlFlowTok{await} \FunctionTok{open}\NormalTok{(}\StringTok{\textquotesingle{}temp.txt\textquotesingle{}}\OperatorTok{,} \StringTok{\textquotesingle{}r+\textquotesingle{}}\NormalTok{)}\OperatorTok{;}
  \ControlFlowTok{await}\NormalTok{ filehandle}\OperatorTok{.}\FunctionTok{truncate}\NormalTok{(}\DecValTok{4}\NormalTok{)}\OperatorTok{;}
\NormalTok{\} }\ControlFlowTok{finally}\NormalTok{ \{}
  \ControlFlowTok{await}\NormalTok{ filehandle}\OperatorTok{?.}\FunctionTok{close}\NormalTok{()}\OperatorTok{;}
\NormalTok{\}}
\end{Highlighting}
\end{Shaded}

If the file previously was shorter than \texttt{len} bytes, it is
extended, and the extended part is filled with null bytes
(\texttt{\textquotesingle{}\textbackslash{}0\textquotesingle{}}):

If \texttt{len} is negative then \texttt{0} will be used.

\paragraph{\texorpdfstring{\texttt{filehandle.utimes(atime,\ mtime)}}{filehandle.utimes(atime, mtime)}}\label{filehandle.utimesatime-mtime}

\begin{itemize}
\tightlist
\item
  \texttt{atime} \{number\textbar string\textbar Date\}
\item
  \texttt{mtime} \{number\textbar string\textbar Date\}
\item
  Returns: \{Promise\}
\end{itemize}

Change the file system timestamps of the object referenced by the
\{FileHandle\} then fulfills the promise with no arguments upon success.

\paragraph{\texorpdfstring{\texttt{filehandle.write(buffer,\ offset{[},\ length{[},\ position{]}{]})}}{filehandle.write(buffer, offset{[}, length{[}, position{]}{]})}}\label{filehandle.writebuffer-offset-length-position}

\begin{itemize}
\tightlist
\item
  \texttt{buffer} \{Buffer\textbar TypedArray\textbar DataView\}
\item
  \texttt{offset} \{integer\} The start position from within
  \texttt{buffer} where the data to write begins.
\item
  \texttt{length} \{integer\} The number of bytes from \texttt{buffer}
  to write. \textbf{Default:} \texttt{buffer.byteLength\ -\ offset}
\item
  \texttt{position} \{integer\textbar null\} The offset from the
  beginning of the file where the data from \texttt{buffer} should be
  written. If \texttt{position} is not a \texttt{number}, the data will
  be written at the current position. See the POSIX pwrite(2)
  documentation for more detail. \textbf{Default:} \texttt{null}
\item
  Returns: \{Promise\}
\end{itemize}

Write \texttt{buffer} to the file.

The promise is fulfilled with an object containing two properties:

\begin{itemize}
\tightlist
\item
  \texttt{bytesWritten} \{integer\} the number of bytes written
\item
  \texttt{buffer} \{Buffer\textbar TypedArray\textbar DataView\} a
  reference to the \texttt{buffer} written.
\end{itemize}

It is unsafe to use \texttt{filehandle.write()} multiple times on the
same file without waiting for the promise to be fulfilled (or rejected).
For this scenario, use
\hyperref[filehandlecreatewritestreamoptions]{\texttt{filehandle.createWriteStream()}}.

On Linux, positional writes do not work when the file is opened in
append mode. The kernel ignores the position argument and always appends
the data to the end of the file.

\paragraph{\texorpdfstring{\texttt{filehandle.write(buffer{[},\ options{]})}}{filehandle.write(buffer{[}, options{]})}}\label{filehandle.writebuffer-options}

\begin{itemize}
\tightlist
\item
  \texttt{buffer} \{Buffer\textbar TypedArray\textbar DataView\}
\item
  \texttt{options} \{Object\}

  \begin{itemize}
  \tightlist
  \item
    \texttt{offset} \{integer\} \textbf{Default:} \texttt{0}
  \item
    \texttt{length} \{integer\} \textbf{Default:}
    \texttt{buffer.byteLength\ -\ offset}
  \item
    \texttt{position} \{integer\} \textbf{Default:} \texttt{null}
  \end{itemize}
\item
  Returns: \{Promise\}
\end{itemize}

Write \texttt{buffer} to the file.

Similar to the above \texttt{filehandle.write} function, this version
takes an optional \texttt{options} object. If no \texttt{options} object
is specified, it will default with the above values.

\paragraph{\texorpdfstring{\texttt{filehandle.write(string{[},\ position{[},\ encoding{]}{]})}}{filehandle.write(string{[}, position{[}, encoding{]}{]})}}\label{filehandle.writestring-position-encoding}

\begin{itemize}
\tightlist
\item
  \texttt{string} \{string\}
\item
  \texttt{position} \{integer\textbar null\} The offset from the
  beginning of the file where the data from \texttt{string} should be
  written. If \texttt{position} is not a \texttt{number} the data will
  be written at the current position. See the POSIX pwrite(2)
  documentation for more detail. \textbf{Default:} \texttt{null}
\item
  \texttt{encoding} \{string\} The expected string encoding.
  \textbf{Default:} \texttt{\textquotesingle{}utf8\textquotesingle{}}
\item
  Returns: \{Promise\}
\end{itemize}

Write \texttt{string} to the file. If \texttt{string} is not a string,
the promise is rejected with an error.

The promise is fulfilled with an object containing two properties:

\begin{itemize}
\tightlist
\item
  \texttt{bytesWritten} \{integer\} the number of bytes written
\item
  \texttt{buffer} \{string\} a reference to the \texttt{string} written.
\end{itemize}

It is unsafe to use \texttt{filehandle.write()} multiple times on the
same file without waiting for the promise to be fulfilled (or rejected).
For this scenario, use
\hyperref[filehandlecreatewritestreamoptions]{\texttt{filehandle.createWriteStream()}}.

On Linux, positional writes do not work when the file is opened in
append mode. The kernel ignores the position argument and always appends
the data to the end of the file.

\paragraph{\texorpdfstring{\texttt{filehandle.writeFile(data,\ options)}}{filehandle.writeFile(data, options)}}\label{filehandle.writefiledata-options}

\begin{itemize}
\tightlist
\item
  \texttt{data}
  \{string\textbar Buffer\textbar TypedArray\textbar DataView\textbar AsyncIterable\textbar Iterable\textbar Stream\}
\item
  \texttt{options} \{Object\textbar string\}

  \begin{itemize}
  \tightlist
  \item
    \texttt{encoding} \{string\textbar null\} The expected character
    encoding when \texttt{data} is a string. \textbf{Default:}
    \texttt{\textquotesingle{}utf8\textquotesingle{}}
  \end{itemize}
\item
  Returns: \{Promise\}
\end{itemize}

Asynchronously writes data to a file, replacing the file if it already
exists. \texttt{data} can be a string, a buffer, an \{AsyncIterable\},
or an \{Iterable\} object. The promise is fulfilled with no arguments
upon success.

If \texttt{options} is a string, then it specifies the
\texttt{encoding}.

The \{FileHandle\} has to support writing.

It is unsafe to use \texttt{filehandle.writeFile()} multiple times on
the same file without waiting for the promise to be fulfilled (or
rejected).

If one or more \texttt{filehandle.write()} calls are made on a file
handle and then a \texttt{filehandle.writeFile()} call is made, the data
will be written from the current position till the end of the file. It
doesn't always write from the beginning of the file.

\paragraph{\texorpdfstring{\texttt{filehandle.writev(buffers{[},\ position{]})}}{filehandle.writev(buffers{[}, position{]})}}\label{filehandle.writevbuffers-position}

\begin{itemize}
\tightlist
\item
  \texttt{buffers}
  \{Buffer{[}{]}\textbar TypedArray{[}{]}\textbar DataView{[}{]}\}
\item
  \texttt{position} \{integer\textbar null\} The offset from the
  beginning of the file where the data from \texttt{buffers} should be
  written. If \texttt{position} is not a \texttt{number}, the data will
  be written at the current position. \textbf{Default:} \texttt{null}
\item
  Returns: \{Promise\}
\end{itemize}

Write an array of \{ArrayBufferView\}s to the file.

The promise is fulfilled with an object containing a two properties:

\begin{itemize}
\tightlist
\item
  \texttt{bytesWritten} \{integer\} the number of bytes written
\item
  \texttt{buffers}
  \{Buffer{[}{]}\textbar TypedArray{[}{]}\textbar DataView{[}{]}\} a
  reference to the \texttt{buffers} input.
\end{itemize}

It is unsafe to call \texttt{writev()} multiple times on the same file
without waiting for the promise to be fulfilled (or rejected).

On Linux, positional writes don't work when the file is opened in append
mode. The kernel ignores the position argument and always appends the
data to the end of the file.

\paragraph{\texorpdfstring{\texttt{filehandle{[}Symbol.asyncDispose{]}()}}{filehandle{[}Symbol.asyncDispose{]}()}}\label{filehandlesymbol.asyncdispose}

\begin{quote}
Stability: 1 - Experimental
\end{quote}

An alias for \texttt{filehandle.close()}.

\subsubsection{\texorpdfstring{\texttt{fsPromises.access(path{[},\ mode{]})}}{fsPromises.access(path{[}, mode{]})}}\label{fspromises.accesspath-mode}

\begin{itemize}
\tightlist
\item
  \texttt{path} \{string\textbar Buffer\textbar URL\}
\item
  \texttt{mode} \{integer\} \textbf{Default:}
  \texttt{fs.constants.F\_OK}
\item
  Returns: \{Promise\} Fulfills with \texttt{undefined} upon success.
\end{itemize}

Tests a user's permissions for the file or directory specified by
\texttt{path}. The \texttt{mode} argument is an optional integer that
specifies the accessibility checks to be performed. \texttt{mode} should
be either the value \texttt{fs.constants.F\_OK} or a mask consisting of
the bitwise OR of any of \texttt{fs.constants.R\_OK},
\texttt{fs.constants.W\_OK}, and \texttt{fs.constants.X\_OK} (e.g.
\texttt{fs.constants.W\_OK\ \textbar{}\ fs.constants.R\_OK}). Check
\hyperref[file-access-constants]{File access constants} for possible
values of \texttt{mode}.

If the accessibility check is successful, the promise is fulfilled with
no value. If any of the accessibility checks fail, the promise is
rejected with an \{Error\} object. The following example checks if the
file \texttt{/etc/passwd} can be read and written by the current
process.

\begin{Shaded}
\begin{Highlighting}[]
\ImportTok{import}\NormalTok{ \{ access}\OperatorTok{,}\NormalTok{ constants \} }\ImportTok{from} \StringTok{\textquotesingle{}node:fs/promises\textquotesingle{}}\OperatorTok{;}

\ControlFlowTok{try}\NormalTok{ \{}
  \ControlFlowTok{await} \FunctionTok{access}\NormalTok{(}\StringTok{\textquotesingle{}/etc/passwd\textquotesingle{}}\OperatorTok{,}\NormalTok{ constants}\OperatorTok{.}\AttributeTok{R\_OK} \OperatorTok{|}\NormalTok{ constants}\OperatorTok{.}\AttributeTok{W\_OK}\NormalTok{)}\OperatorTok{;}
  \BuiltInTok{console}\OperatorTok{.}\FunctionTok{log}\NormalTok{(}\StringTok{\textquotesingle{}can access\textquotesingle{}}\NormalTok{)}\OperatorTok{;}
\NormalTok{\} }\ControlFlowTok{catch}\NormalTok{ \{}
  \BuiltInTok{console}\OperatorTok{.}\FunctionTok{error}\NormalTok{(}\StringTok{\textquotesingle{}cannot access\textquotesingle{}}\NormalTok{)}\OperatorTok{;}
\NormalTok{\}}
\end{Highlighting}
\end{Shaded}

Using \texttt{fsPromises.access()} to check for the accessibility of a
file before calling \texttt{fsPromises.open()} is not recommended. Doing
so introduces a race condition, since other processes may change the
file's state between the two calls. Instead, user code should
open/read/write the file directly and handle the error raised if the
file is not accessible.

\subsubsection{\texorpdfstring{\texttt{fsPromises.appendFile(path,\ data{[},\ options{]})}}{fsPromises.appendFile(path, data{[}, options{]})}}\label{fspromises.appendfilepath-data-options}

\begin{itemize}
\tightlist
\item
  \texttt{path} \{string\textbar Buffer\textbar URL\textbar FileHandle\}
  filename or \{FileHandle\}
\item
  \texttt{data} \{string\textbar Buffer\}
\item
  \texttt{options} \{Object\textbar string\}

  \begin{itemize}
  \tightlist
  \item
    \texttt{encoding} \{string\textbar null\} \textbf{Default:}
    \texttt{\textquotesingle{}utf8\textquotesingle{}}
  \item
    \texttt{mode} \{integer\} \textbf{Default:} \texttt{0o666}
  \item
    \texttt{flag} \{string\} See \hyperref[file-system-flags]{support of
    file system \texttt{flags}}. \textbf{Default:}
    \texttt{\textquotesingle{}a\textquotesingle{}}.
  \item
    \texttt{flush} \{boolean\} If \texttt{true}, the underlying file
    descriptor is flushed prior to closing it. \textbf{Default:}
    \texttt{false}.
  \end{itemize}
\item
  Returns: \{Promise\} Fulfills with \texttt{undefined} upon success.
\end{itemize}

Asynchronously append data to a file, creating the file if it does not
yet exist. \texttt{data} can be a string or a \{Buffer\}.

If \texttt{options} is a string, then it specifies the
\texttt{encoding}.

The \texttt{mode} option only affects the newly created file. See
\hyperref[fsopenpath-flags-mode-callback]{\texttt{fs.open()}} for more
details.

The \texttt{path} may be specified as a \{FileHandle\} that has been
opened for appending (using \texttt{fsPromises.open()}).

\subsubsection{\texorpdfstring{\texttt{fsPromises.chmod(path,\ mode)}}{fsPromises.chmod(path, mode)}}\label{fspromises.chmodpath-mode}

\begin{itemize}
\tightlist
\item
  \texttt{path} \{string\textbar Buffer\textbar URL\}
\item
  \texttt{mode} \{string\textbar integer\}
\item
  Returns: \{Promise\} Fulfills with \texttt{undefined} upon success.
\end{itemize}

Changes the permissions of a file.

\subsubsection{\texorpdfstring{\texttt{fsPromises.chown(path,\ uid,\ gid)}}{fsPromises.chown(path, uid, gid)}}\label{fspromises.chownpath-uid-gid}

\begin{itemize}
\tightlist
\item
  \texttt{path} \{string\textbar Buffer\textbar URL\}
\item
  \texttt{uid} \{integer\}
\item
  \texttt{gid} \{integer\}
\item
  Returns: \{Promise\} Fulfills with \texttt{undefined} upon success.
\end{itemize}

Changes the ownership of a file.

\subsubsection{\texorpdfstring{\texttt{fsPromises.copyFile(src,\ dest{[},\ mode{]})}}{fsPromises.copyFile(src, dest{[}, mode{]})}}\label{fspromises.copyfilesrc-dest-mode}

\begin{itemize}
\tightlist
\item
  \texttt{src} \{string\textbar Buffer\textbar URL\} source filename to
  copy
\item
  \texttt{dest} \{string\textbar Buffer\textbar URL\} destination
  filename of the copy operation
\item
  \texttt{mode} \{integer\} Optional modifiers that specify the behavior
  of the copy operation. It is possible to create a mask consisting of
  the bitwise OR of two or more values (e.g.
  \texttt{fs.constants.COPYFILE\_EXCL\ \textbar{}\ fs.constants.COPYFILE\_FICLONE})
  \textbf{Default:} \texttt{0}.

  \begin{itemize}
  \tightlist
  \item
    \texttt{fs.constants.COPYFILE\_EXCL}: The copy operation will fail
    if \texttt{dest} already exists.
  \item
    \texttt{fs.constants.COPYFILE\_FICLONE}: The copy operation will
    attempt to create a copy-on-write reflink. If the platform does not
    support copy-on-write, then a fallback copy mechanism is used.
  \item
    \texttt{fs.constants.COPYFILE\_FICLONE\_FORCE}: The copy operation
    will attempt to create a copy-on-write reflink. If the platform does
    not support copy-on-write, then the operation will fail.
  \end{itemize}
\item
  Returns: \{Promise\} Fulfills with \texttt{undefined} upon success.
\end{itemize}

Asynchronously copies \texttt{src} to \texttt{dest}. By default,
\texttt{dest} is overwritten if it already exists.

No guarantees are made about the atomicity of the copy operation. If an
error occurs after the destination file has been opened for writing, an
attempt will be made to remove the destination.

\begin{Shaded}
\begin{Highlighting}[]
\ImportTok{import}\NormalTok{ \{ copyFile}\OperatorTok{,}\NormalTok{ constants \} }\ImportTok{from} \StringTok{\textquotesingle{}node:fs/promises\textquotesingle{}}\OperatorTok{;}

\ControlFlowTok{try}\NormalTok{ \{}
  \ControlFlowTok{await} \FunctionTok{copyFile}\NormalTok{(}\StringTok{\textquotesingle{}source.txt\textquotesingle{}}\OperatorTok{,} \StringTok{\textquotesingle{}destination.txt\textquotesingle{}}\NormalTok{)}\OperatorTok{;}
  \BuiltInTok{console}\OperatorTok{.}\FunctionTok{log}\NormalTok{(}\StringTok{\textquotesingle{}source.txt was copied to destination.txt\textquotesingle{}}\NormalTok{)}\OperatorTok{;}
\NormalTok{\} }\ControlFlowTok{catch}\NormalTok{ \{}
  \BuiltInTok{console}\OperatorTok{.}\FunctionTok{error}\NormalTok{(}\StringTok{\textquotesingle{}The file could not be copied\textquotesingle{}}\NormalTok{)}\OperatorTok{;}
\NormalTok{\}}

\CommentTok{// By using COPYFILE\_EXCL, the operation will fail if destination.txt exists.}
\ControlFlowTok{try}\NormalTok{ \{}
  \ControlFlowTok{await} \FunctionTok{copyFile}\NormalTok{(}\StringTok{\textquotesingle{}source.txt\textquotesingle{}}\OperatorTok{,} \StringTok{\textquotesingle{}destination.txt\textquotesingle{}}\OperatorTok{,}\NormalTok{ constants}\OperatorTok{.}\AttributeTok{COPYFILE\_EXCL}\NormalTok{)}\OperatorTok{;}
  \BuiltInTok{console}\OperatorTok{.}\FunctionTok{log}\NormalTok{(}\StringTok{\textquotesingle{}source.txt was copied to destination.txt\textquotesingle{}}\NormalTok{)}\OperatorTok{;}
\NormalTok{\} }\ControlFlowTok{catch}\NormalTok{ \{}
  \BuiltInTok{console}\OperatorTok{.}\FunctionTok{error}\NormalTok{(}\StringTok{\textquotesingle{}The file could not be copied\textquotesingle{}}\NormalTok{)}\OperatorTok{;}
\NormalTok{\}}
\end{Highlighting}
\end{Shaded}

\subsubsection{\texorpdfstring{\texttt{fsPromises.cp(src,\ dest{[},\ options{]})}}{fsPromises.cp(src, dest{[}, options{]})}}\label{fspromises.cpsrc-dest-options}

\begin{quote}
Stability: 1 - Experimental
\end{quote}

\begin{itemize}
\tightlist
\item
  \texttt{src} \{string\textbar URL\} source path to copy.
\item
  \texttt{dest} \{string\textbar URL\} destination path to copy to.
\item
  \texttt{options} \{Object\}

  \begin{itemize}
  \tightlist
  \item
    \texttt{dereference} \{boolean\} dereference symlinks.
    \textbf{Default:} \texttt{false}.
  \item
    \texttt{errorOnExist} \{boolean\} when \texttt{force} is
    \texttt{false}, and the destination exists, throw an error.
    \textbf{Default:} \texttt{false}.
  \item
    \texttt{filter} \{Function\} Function to filter copied
    files/directories. Return \texttt{true} to copy the item,
    \texttt{false} to ignore it. When ignoring a directory, all of its
    contents will be skipped as well. Can also return a \texttt{Promise}
    that resolves to \texttt{true} or \texttt{false} \textbf{Default:}
    \texttt{undefined}.

    \begin{itemize}
    \tightlist
    \item
      \texttt{src} \{string\} source path to copy.
    \item
      \texttt{dest} \{string\} destination path to copy to.
    \item
      Returns: \{boolean\textbar Promise\}
    \end{itemize}
  \item
    \texttt{force} \{boolean\} overwrite existing file or directory. The
    copy operation will ignore errors if you set this to false and the
    destination exists. Use the \texttt{errorOnExist} option to change
    this behavior. \textbf{Default:} \texttt{true}.
  \item
    \texttt{mode} \{integer\} modifiers for copy operation.
    \textbf{Default:} \texttt{0}. See \texttt{mode} flag of
    \hyperref[fspromisescopyfilesrc-dest-mode]{\texttt{fsPromises.copyFile()}}.
  \item
    \texttt{preserveTimestamps} \{boolean\} When \texttt{true}
    timestamps from \texttt{src} will be preserved. \textbf{Default:}
    \texttt{false}.
  \item
    \texttt{recursive} \{boolean\} copy directories recursively
    \textbf{Default:} \texttt{false}
  \item
    \texttt{verbatimSymlinks} \{boolean\} When \texttt{true}, path
    resolution for symlinks will be skipped. \textbf{Default:}
    \texttt{false}
  \end{itemize}
\item
  Returns: \{Promise\} Fulfills with \texttt{undefined} upon success.
\end{itemize}

Asynchronously copies the entire directory structure from \texttt{src}
to \texttt{dest}, including subdirectories and files.

When copying a directory to another directory, globs are not supported
and behavior is similar to \texttt{cp\ dir1/\ dir2/}.

\subsubsection{\texorpdfstring{\texttt{fsPromises.lchmod(path,\ mode)}}{fsPromises.lchmod(path, mode)}}\label{fspromises.lchmodpath-mode}

\begin{itemize}
\tightlist
\item
  \texttt{path} \{string\textbar Buffer\textbar URL\}
\item
  \texttt{mode} \{integer\}
\item
  Returns: \{Promise\} Fulfills with \texttt{undefined} upon success.
\end{itemize}

Changes the permissions on a symbolic link.

This method is only implemented on macOS.

\subsubsection{\texorpdfstring{\texttt{fsPromises.lchown(path,\ uid,\ gid)}}{fsPromises.lchown(path, uid, gid)}}\label{fspromises.lchownpath-uid-gid}

\begin{itemize}
\tightlist
\item
  \texttt{path} \{string\textbar Buffer\textbar URL\}
\item
  \texttt{uid} \{integer\}
\item
  \texttt{gid} \{integer\}
\item
  Returns: \{Promise\} Fulfills with \texttt{undefined} upon success.
\end{itemize}

Changes the ownership on a symbolic link.

\subsubsection{\texorpdfstring{\texttt{fsPromises.lutimes(path,\ atime,\ mtime)}}{fsPromises.lutimes(path, atime, mtime)}}\label{fspromises.lutimespath-atime-mtime}

\begin{itemize}
\tightlist
\item
  \texttt{path} \{string\textbar Buffer\textbar URL\}
\item
  \texttt{atime} \{number\textbar string\textbar Date\}
\item
  \texttt{mtime} \{number\textbar string\textbar Date\}
\item
  Returns: \{Promise\} Fulfills with \texttt{undefined} upon success.
\end{itemize}

Changes the access and modification times of a file in the same way as
\hyperref[fspromisesutimespath-atime-mtime]{\texttt{fsPromises.utimes()}},
with the difference that if the path refers to a symbolic link, then the
link is not dereferenced: instead, the timestamps of the symbolic link
itself are changed.

\subsubsection{\texorpdfstring{\texttt{fsPromises.link(existingPath,\ newPath)}}{fsPromises.link(existingPath, newPath)}}\label{fspromises.linkexistingpath-newpath}

\begin{itemize}
\tightlist
\item
  \texttt{existingPath} \{string\textbar Buffer\textbar URL\}
\item
  \texttt{newPath} \{string\textbar Buffer\textbar URL\}
\item
  Returns: \{Promise\} Fulfills with \texttt{undefined} upon success.
\end{itemize}

Creates a new link from the \texttt{existingPath} to the
\texttt{newPath}. See the POSIX link(2) documentation for more detail.

\subsubsection{\texorpdfstring{\texttt{fsPromises.lstat(path{[},\ options{]})}}{fsPromises.lstat(path{[}, options{]})}}\label{fspromises.lstatpath-options}

\begin{itemize}
\tightlist
\item
  \texttt{path} \{string\textbar Buffer\textbar URL\}
\item
  \texttt{options} \{Object\}

  \begin{itemize}
  \tightlist
  \item
    \texttt{bigint} \{boolean\} Whether the numeric values in the
    returned \{fs.Stats\} object should be \texttt{bigint}.
    \textbf{Default:} \texttt{false}.
  \end{itemize}
\item
  Returns: \{Promise\} Fulfills with the \{fs.Stats\} object for the
  given symbolic link \texttt{path}.
\end{itemize}

Equivalent to
\hyperref[fspromisesstatpath-options]{\texttt{fsPromises.stat()}} unless
\texttt{path} refers to a symbolic link, in which case the link itself
is stat-ed, not the file that it refers to. Refer to the POSIX lstat(2)
document for more detail.

\subsubsection{\texorpdfstring{\texttt{fsPromises.mkdir(path{[},\ options{]})}}{fsPromises.mkdir(path{[}, options{]})}}\label{fspromises.mkdirpath-options}

\begin{itemize}
\tightlist
\item
  \texttt{path} \{string\textbar Buffer\textbar URL\}
\item
  \texttt{options} \{Object\textbar integer\}

  \begin{itemize}
  \tightlist
  \item
    \texttt{recursive} \{boolean\} \textbf{Default:} \texttt{false}
  \item
    \texttt{mode} \{string\textbar integer\} Not supported on Windows.
    \textbf{Default:} \texttt{0o777}.
  \end{itemize}
\item
  Returns: \{Promise\} Upon success, fulfills with \texttt{undefined} if
  \texttt{recursive} is \texttt{false}, or the first directory path
  created if \texttt{recursive} is \texttt{true}.
\end{itemize}

Asynchronously creates a directory.

The optional \texttt{options} argument can be an integer specifying
\texttt{mode} (permission and sticky bits), or an object with a
\texttt{mode} property and a \texttt{recursive} property indicating
whether parent directories should be created. Calling
\texttt{fsPromises.mkdir()} when \texttt{path} is a directory that
exists results in a rejection only when \texttt{recursive} is false.

\begin{Shaded}
\begin{Highlighting}[]
\ImportTok{import}\NormalTok{ \{ mkdir \} }\ImportTok{from} \StringTok{\textquotesingle{}node:fs/promises\textquotesingle{}}\OperatorTok{;}

\ControlFlowTok{try}\NormalTok{ \{}
  \KeywordTok{const}\NormalTok{ projectFolder }\OperatorTok{=} \KeywordTok{new} \FunctionTok{URL}\NormalTok{(}\StringTok{\textquotesingle{}./test/project/\textquotesingle{}}\OperatorTok{,} \ImportTok{import}\OperatorTok{.}\AttributeTok{meta}\OperatorTok{.}\AttributeTok{url}\NormalTok{)}\OperatorTok{;}
  \KeywordTok{const}\NormalTok{ createDir }\OperatorTok{=} \ControlFlowTok{await} \FunctionTok{mkdir}\NormalTok{(projectFolder}\OperatorTok{,}\NormalTok{ \{ }\DataTypeTok{recursive}\OperatorTok{:} \KeywordTok{true}\NormalTok{ \})}\OperatorTok{;}

  \BuiltInTok{console}\OperatorTok{.}\FunctionTok{log}\NormalTok{(}\VerbatimStringTok{\textasciigrave{}created }\SpecialCharTok{$\{}\NormalTok{createDir}\SpecialCharTok{\}}\VerbatimStringTok{\textasciigrave{}}\NormalTok{)}\OperatorTok{;}
\NormalTok{\} }\ControlFlowTok{catch}\NormalTok{ (err) \{}
  \BuiltInTok{console}\OperatorTok{.}\FunctionTok{error}\NormalTok{(err}\OperatorTok{.}\AttributeTok{message}\NormalTok{)}\OperatorTok{;}
\NormalTok{\}}
\end{Highlighting}
\end{Shaded}

\begin{Shaded}
\begin{Highlighting}[]
\KeywordTok{const}\NormalTok{ \{ mkdir \} }\OperatorTok{=} \PreprocessorTok{require}\NormalTok{(}\StringTok{\textquotesingle{}node:fs/promises\textquotesingle{}}\NormalTok{)}\OperatorTok{;}
\KeywordTok{const}\NormalTok{ \{ join \} }\OperatorTok{=} \PreprocessorTok{require}\NormalTok{(}\StringTok{\textquotesingle{}node:path\textquotesingle{}}\NormalTok{)}\OperatorTok{;}

\KeywordTok{async} \KeywordTok{function} \FunctionTok{makeDirectory}\NormalTok{() \{}
  \KeywordTok{const}\NormalTok{ projectFolder }\OperatorTok{=} \FunctionTok{join}\NormalTok{(}\BuiltInTok{\_\_dirname}\OperatorTok{,} \StringTok{\textquotesingle{}test\textquotesingle{}}\OperatorTok{,} \StringTok{\textquotesingle{}project\textquotesingle{}}\NormalTok{)}\OperatorTok{;}
  \KeywordTok{const}\NormalTok{ dirCreation }\OperatorTok{=} \ControlFlowTok{await} \FunctionTok{mkdir}\NormalTok{(projectFolder}\OperatorTok{,}\NormalTok{ \{ }\DataTypeTok{recursive}\OperatorTok{:} \KeywordTok{true}\NormalTok{ \})}\OperatorTok{;}

  \BuiltInTok{console}\OperatorTok{.}\FunctionTok{log}\NormalTok{(dirCreation)}\OperatorTok{;}
  \ControlFlowTok{return}\NormalTok{ dirCreation}\OperatorTok{;}
\NormalTok{\}}

\FunctionTok{makeDirectory}\NormalTok{()}\OperatorTok{.}\FunctionTok{catch}\NormalTok{(}\BuiltInTok{console}\OperatorTok{.}\FunctionTok{error}\NormalTok{)}\OperatorTok{;}
\end{Highlighting}
\end{Shaded}

\subsubsection{\texorpdfstring{\texttt{fsPromises.mkdtemp(prefix{[},\ options{]})}}{fsPromises.mkdtemp(prefix{[}, options{]})}}\label{fspromises.mkdtempprefix-options}

\begin{itemize}
\tightlist
\item
  \texttt{prefix} \{string\textbar Buffer\textbar URL\}
\item
  \texttt{options} \{string\textbar Object\}

  \begin{itemize}
  \tightlist
  \item
    \texttt{encoding} \{string\} \textbf{Default:}
    \texttt{\textquotesingle{}utf8\textquotesingle{}}
  \end{itemize}
\item
  Returns: \{Promise\} Fulfills with a string containing the file system
  path of the newly created temporary directory.
\end{itemize}

Creates a unique temporary directory. A unique directory name is
generated by appending six random characters to the end of the provided
\texttt{prefix}. Due to platform inconsistencies, avoid trailing
\texttt{X} characters in \texttt{prefix}. Some platforms, notably the
BSDs, can return more than six random characters, and replace trailing
\texttt{X} characters in \texttt{prefix} with random characters.

The optional \texttt{options} argument can be a string specifying an
encoding, or an object with an \texttt{encoding} property specifying the
character encoding to use.

\begin{Shaded}
\begin{Highlighting}[]
\ImportTok{import}\NormalTok{ \{ mkdtemp \} }\ImportTok{from} \StringTok{\textquotesingle{}node:fs/promises\textquotesingle{}}\OperatorTok{;}
\ImportTok{import}\NormalTok{ \{ join \} }\ImportTok{from} \StringTok{\textquotesingle{}node:path\textquotesingle{}}\OperatorTok{;}
\ImportTok{import}\NormalTok{ \{ tmpdir \} }\ImportTok{from} \StringTok{\textquotesingle{}node:os\textquotesingle{}}\OperatorTok{;}

\ControlFlowTok{try}\NormalTok{ \{}
  \ControlFlowTok{await} \FunctionTok{mkdtemp}\NormalTok{(}\FunctionTok{join}\NormalTok{(}\FunctionTok{tmpdir}\NormalTok{()}\OperatorTok{,} \StringTok{\textquotesingle{}foo{-}\textquotesingle{}}\NormalTok{))}\OperatorTok{;}
\NormalTok{\} }\ControlFlowTok{catch}\NormalTok{ (err) \{}
  \BuiltInTok{console}\OperatorTok{.}\FunctionTok{error}\NormalTok{(err)}\OperatorTok{;}
\NormalTok{\}}
\end{Highlighting}
\end{Shaded}

The \texttt{fsPromises.mkdtemp()} method will append the six randomly
selected characters directly to the \texttt{prefix} string. For
instance, given a directory \texttt{/tmp}, if the intention is to create
a temporary directory \emph{within} \texttt{/tmp}, the \texttt{prefix}
must end with a trailing platform-specific path separator
(\texttt{require(\textquotesingle{}node:path\textquotesingle{}).sep}).

\subsubsection{\texorpdfstring{\texttt{fsPromises.open(path,\ flags{[},\ mode{]})}}{fsPromises.open(path, flags{[}, mode{]})}}\label{fspromises.openpath-flags-mode}

\begin{itemize}
\tightlist
\item
  \texttt{path} \{string\textbar Buffer\textbar URL\}
\item
  \texttt{flags} \{string\textbar number\} See
  \hyperref[file-system-flags]{support of file system \texttt{flags}}.
  \textbf{Default:} \texttt{\textquotesingle{}r\textquotesingle{}}.
\item
  \texttt{mode} \{string\textbar integer\} Sets the file mode
  (permission and sticky bits) if the file is created. \textbf{Default:}
  \texttt{0o666} (readable and writable)
\item
  Returns: \{Promise\} Fulfills with a \{FileHandle\} object.
\end{itemize}

Opens a \{FileHandle\}.

Refer to the POSIX open(2) documentation for more detail.

Some characters
(\texttt{\textless{}\ \textgreater{}\ :\ "\ /\ \textbackslash{}\ \textbar{}\ ?\ *})
are reserved under Windows as documented by
\href{https://docs.microsoft.com/en-us/windows/desktop/FileIO/naming-a-file}{Naming
Files, Paths, and Namespaces}. Under NTFS, if the filename contains a
colon, Node.js will open a file system stream, as described by
\href{https://docs.microsoft.com/en-us/windows/desktop/FileIO/using-streams}{this
MSDN page}.

\subsubsection{\texorpdfstring{\texttt{fsPromises.opendir(path{[},\ options{]})}}{fsPromises.opendir(path{[}, options{]})}}\label{fspromises.opendirpath-options}

\begin{itemize}
\tightlist
\item
  \texttt{path} \{string\textbar Buffer\textbar URL\}
\item
  \texttt{options} \{Object\}

  \begin{itemize}
  \tightlist
  \item
    \texttt{encoding} \{string\textbar null\} \textbf{Default:}
    \texttt{\textquotesingle{}utf8\textquotesingle{}}
  \item
    \texttt{bufferSize} \{number\} Number of directory entries that are
    buffered internally when reading from the directory. Higher values
    lead to better performance but higher memory usage.
    \textbf{Default:} \texttt{32}
  \item
    \texttt{recursive} \{boolean\} Resolved \texttt{Dir} will be an
    \{AsyncIterable\} containing all sub files and directories.
    \textbf{Default:} \texttt{false}
  \end{itemize}
\item
  Returns: \{Promise\} Fulfills with an \{fs.Dir\}.
\end{itemize}

Asynchronously open a directory for iterative scanning. See the POSIX
opendir(3) documentation for more detail.

Creates an \{fs.Dir\}, which contains all further functions for reading
from and cleaning up the directory.

The \texttt{encoding} option sets the encoding for the \texttt{path}
while opening the directory and subsequent read operations.

Example using async iteration:

\begin{Shaded}
\begin{Highlighting}[]
\ImportTok{import}\NormalTok{ \{ opendir \} }\ImportTok{from} \StringTok{\textquotesingle{}node:fs/promises\textquotesingle{}}\OperatorTok{;}

\ControlFlowTok{try}\NormalTok{ \{}
  \KeywordTok{const}\NormalTok{ dir }\OperatorTok{=} \ControlFlowTok{await} \FunctionTok{opendir}\NormalTok{(}\StringTok{\textquotesingle{}./\textquotesingle{}}\NormalTok{)}\OperatorTok{;}
  \ControlFlowTok{for} \ControlFlowTok{await}\NormalTok{ (}\KeywordTok{const}\NormalTok{ dirent }\KeywordTok{of}\NormalTok{ dir)}
    \BuiltInTok{console}\OperatorTok{.}\FunctionTok{log}\NormalTok{(dirent}\OperatorTok{.}\AttributeTok{name}\NormalTok{)}\OperatorTok{;}
\NormalTok{\} }\ControlFlowTok{catch}\NormalTok{ (err) \{}
  \BuiltInTok{console}\OperatorTok{.}\FunctionTok{error}\NormalTok{(err)}\OperatorTok{;}
\NormalTok{\}}
\end{Highlighting}
\end{Shaded}

When using the async iterator, the \{fs.Dir\} object will be
automatically closed after the iterator exits.

\subsubsection{\texorpdfstring{\texttt{fsPromises.readdir(path{[},\ options{]})}}{fsPromises.readdir(path{[}, options{]})}}\label{fspromises.readdirpath-options}

\begin{itemize}
\tightlist
\item
  \texttt{path} \{string\textbar Buffer\textbar URL\}
\item
  \texttt{options} \{string\textbar Object\}

  \begin{itemize}
  \tightlist
  \item
    \texttt{encoding} \{string\} \textbf{Default:}
    \texttt{\textquotesingle{}utf8\textquotesingle{}}
  \item
    \texttt{withFileTypes} \{boolean\} \textbf{Default:} \texttt{false}
  \item
    \texttt{recursive} \{boolean\} If \texttt{true}, reads the contents
    of a directory recursively. In recursive mode, it will list all
    files, sub files, and directories. \textbf{Default:} \texttt{false}.
  \end{itemize}
\item
  Returns: \{Promise\} Fulfills with an array of the names of the files
  in the directory excluding
  \texttt{\textquotesingle{}.\textquotesingle{}} and
  \texttt{\textquotesingle{}..\textquotesingle{}}.
\end{itemize}

Reads the contents of a directory.

The optional \texttt{options} argument can be a string specifying an
encoding, or an object with an \texttt{encoding} property specifying the
character encoding to use for the filenames. If the \texttt{encoding} is
set to \texttt{\textquotesingle{}buffer\textquotesingle{}}, the
filenames returned will be passed as \{Buffer\} objects.

If \texttt{options.withFileTypes} is set to \texttt{true}, the returned
array will contain \{fs.Dirent\} objects.

\begin{Shaded}
\begin{Highlighting}[]
\ImportTok{import}\NormalTok{ \{ readdir \} }\ImportTok{from} \StringTok{\textquotesingle{}node:fs/promises\textquotesingle{}}\OperatorTok{;}

\ControlFlowTok{try}\NormalTok{ \{}
  \KeywordTok{const}\NormalTok{ files }\OperatorTok{=} \ControlFlowTok{await} \FunctionTok{readdir}\NormalTok{(path)}\OperatorTok{;}
  \ControlFlowTok{for}\NormalTok{ (}\KeywordTok{const}\NormalTok{ file }\KeywordTok{of}\NormalTok{ files)}
    \BuiltInTok{console}\OperatorTok{.}\FunctionTok{log}\NormalTok{(file)}\OperatorTok{;}
\NormalTok{\} }\ControlFlowTok{catch}\NormalTok{ (err) \{}
  \BuiltInTok{console}\OperatorTok{.}\FunctionTok{error}\NormalTok{(err)}\OperatorTok{;}
\NormalTok{\}}
\end{Highlighting}
\end{Shaded}

\subsubsection{\texorpdfstring{\texttt{fsPromises.readFile(path{[},\ options{]})}}{fsPromises.readFile(path{[}, options{]})}}\label{fspromises.readfilepath-options}

\begin{itemize}
\tightlist
\item
  \texttt{path} \{string\textbar Buffer\textbar URL\textbar FileHandle\}
  filename or \texttt{FileHandle}
\item
  \texttt{options} \{Object\textbar string\}

  \begin{itemize}
  \tightlist
  \item
    \texttt{encoding} \{string\textbar null\} \textbf{Default:}
    \texttt{null}
  \item
    \texttt{flag} \{string\} See \hyperref[file-system-flags]{support of
    file system \texttt{flags}}. \textbf{Default:}
    \texttt{\textquotesingle{}r\textquotesingle{}}.
  \item
    \texttt{signal} \{AbortSignal\} allows aborting an in-progress
    readFile
  \end{itemize}
\item
  Returns: \{Promise\} Fulfills with the contents of the file.
\end{itemize}

Asynchronously reads the entire contents of a file.

If no encoding is specified (using \texttt{options.encoding}), the data
is returned as a \{Buffer\} object. Otherwise, the data will be a
string.

If \texttt{options} is a string, then it specifies the encoding.

When the \texttt{path} is a directory, the behavior of
\texttt{fsPromises.readFile()} is platform-specific. On macOS, Linux,
and Windows, the promise will be rejected with an error. On FreeBSD, a
representation of the directory's contents will be returned.

An example of reading a \texttt{package.json} file located in the same
directory of the running code:

\begin{Shaded}
\begin{Highlighting}[]
\ImportTok{import}\NormalTok{ \{ readFile \} }\ImportTok{from} \StringTok{\textquotesingle{}node:fs/promises\textquotesingle{}}\OperatorTok{;}
\ControlFlowTok{try}\NormalTok{ \{}
  \KeywordTok{const}\NormalTok{ filePath }\OperatorTok{=} \KeywordTok{new} \FunctionTok{URL}\NormalTok{(}\StringTok{\textquotesingle{}./package.json\textquotesingle{}}\OperatorTok{,} \ImportTok{import}\OperatorTok{.}\AttributeTok{meta}\OperatorTok{.}\AttributeTok{url}\NormalTok{)}\OperatorTok{;}
  \KeywordTok{const}\NormalTok{ contents }\OperatorTok{=} \ControlFlowTok{await} \FunctionTok{readFile}\NormalTok{(filePath}\OperatorTok{,}\NormalTok{ \{ }\DataTypeTok{encoding}\OperatorTok{:} \StringTok{\textquotesingle{}utf8\textquotesingle{}}\NormalTok{ \})}\OperatorTok{;}
  \BuiltInTok{console}\OperatorTok{.}\FunctionTok{log}\NormalTok{(contents)}\OperatorTok{;}
\NormalTok{\} }\ControlFlowTok{catch}\NormalTok{ (err) \{}
  \BuiltInTok{console}\OperatorTok{.}\FunctionTok{error}\NormalTok{(err}\OperatorTok{.}\AttributeTok{message}\NormalTok{)}\OperatorTok{;}
\NormalTok{\}}
\end{Highlighting}
\end{Shaded}

\begin{Shaded}
\begin{Highlighting}[]
\KeywordTok{const}\NormalTok{ \{ readFile \} }\OperatorTok{=} \PreprocessorTok{require}\NormalTok{(}\StringTok{\textquotesingle{}node:fs/promises\textquotesingle{}}\NormalTok{)}\OperatorTok{;}
\KeywordTok{const}\NormalTok{ \{ resolve \} }\OperatorTok{=} \PreprocessorTok{require}\NormalTok{(}\StringTok{\textquotesingle{}node:path\textquotesingle{}}\NormalTok{)}\OperatorTok{;}
\KeywordTok{async} \KeywordTok{function} \FunctionTok{logFile}\NormalTok{() \{}
  \ControlFlowTok{try}\NormalTok{ \{}
    \KeywordTok{const}\NormalTok{ filePath }\OperatorTok{=} \FunctionTok{resolve}\NormalTok{(}\StringTok{\textquotesingle{}./package.json\textquotesingle{}}\NormalTok{)}\OperatorTok{;}
    \KeywordTok{const}\NormalTok{ contents }\OperatorTok{=} \ControlFlowTok{await} \FunctionTok{readFile}\NormalTok{(filePath}\OperatorTok{,}\NormalTok{ \{ }\DataTypeTok{encoding}\OperatorTok{:} \StringTok{\textquotesingle{}utf8\textquotesingle{}}\NormalTok{ \})}\OperatorTok{;}
    \BuiltInTok{console}\OperatorTok{.}\FunctionTok{log}\NormalTok{(contents)}\OperatorTok{;}
\NormalTok{  \} }\ControlFlowTok{catch}\NormalTok{ (err) \{}
    \BuiltInTok{console}\OperatorTok{.}\FunctionTok{error}\NormalTok{(err}\OperatorTok{.}\AttributeTok{message}\NormalTok{)}\OperatorTok{;}
\NormalTok{  \}}
\NormalTok{\}}
\FunctionTok{logFile}\NormalTok{()}\OperatorTok{;}
\end{Highlighting}
\end{Shaded}

It is possible to abort an ongoing \texttt{readFile} using an
\{AbortSignal\}. If a request is aborted the promise returned is
rejected with an \texttt{AbortError}:

\begin{Shaded}
\begin{Highlighting}[]
\ImportTok{import}\NormalTok{ \{ readFile \} }\ImportTok{from} \StringTok{\textquotesingle{}node:fs/promises\textquotesingle{}}\OperatorTok{;}

\ControlFlowTok{try}\NormalTok{ \{}
  \KeywordTok{const}\NormalTok{ controller }\OperatorTok{=} \KeywordTok{new} \FunctionTok{AbortController}\NormalTok{()}\OperatorTok{;}
  \KeywordTok{const}\NormalTok{ \{ signal \} }\OperatorTok{=}\NormalTok{ controller}\OperatorTok{;}
  \KeywordTok{const}\NormalTok{ promise }\OperatorTok{=} \FunctionTok{readFile}\NormalTok{(fileName}\OperatorTok{,}\NormalTok{ \{ signal \})}\OperatorTok{;}

  \CommentTok{// Abort the request before the promise settles.}
\NormalTok{  controller}\OperatorTok{.}\FunctionTok{abort}\NormalTok{()}\OperatorTok{;}

  \ControlFlowTok{await}\NormalTok{ promise}\OperatorTok{;}
\NormalTok{\} }\ControlFlowTok{catch}\NormalTok{ (err) \{}
  \CommentTok{// When a request is aborted {-} err is an AbortError}
  \BuiltInTok{console}\OperatorTok{.}\FunctionTok{error}\NormalTok{(err)}\OperatorTok{;}
\NormalTok{\}}
\end{Highlighting}
\end{Shaded}

Aborting an ongoing request does not abort individual operating system
requests but rather the internal buffering \texttt{fs.readFile}
performs.

Any specified \{FileHandle\} has to support reading.

\subsubsection{\texorpdfstring{\texttt{fsPromises.readlink(path{[},\ options{]})}}{fsPromises.readlink(path{[}, options{]})}}\label{fspromises.readlinkpath-options}

\begin{itemize}
\tightlist
\item
  \texttt{path} \{string\textbar Buffer\textbar URL\}
\item
  \texttt{options} \{string\textbar Object\}

  \begin{itemize}
  \tightlist
  \item
    \texttt{encoding} \{string\} \textbf{Default:}
    \texttt{\textquotesingle{}utf8\textquotesingle{}}
  \end{itemize}
\item
  Returns: \{Promise\} Fulfills with the \texttt{linkString} upon
  success.
\end{itemize}

Reads the contents of the symbolic link referred to by \texttt{path}.
See the POSIX readlink(2) documentation for more detail. The promise is
fulfilled with the \texttt{linkString} upon success.

The optional \texttt{options} argument can be a string specifying an
encoding, or an object with an \texttt{encoding} property specifying the
character encoding to use for the link path returned. If the
\texttt{encoding} is set to
\texttt{\textquotesingle{}buffer\textquotesingle{}}, the link path
returned will be passed as a \{Buffer\} object.

\subsubsection{\texorpdfstring{\texttt{fsPromises.realpath(path{[},\ options{]})}}{fsPromises.realpath(path{[}, options{]})}}\label{fspromises.realpathpath-options}

\begin{itemize}
\tightlist
\item
  \texttt{path} \{string\textbar Buffer\textbar URL\}
\item
  \texttt{options} \{string\textbar Object\}

  \begin{itemize}
  \tightlist
  \item
    \texttt{encoding} \{string\} \textbf{Default:}
    \texttt{\textquotesingle{}utf8\textquotesingle{}}
  \end{itemize}
\item
  Returns: \{Promise\} Fulfills with the resolved path upon success.
\end{itemize}

Determines the actual location of \texttt{path} using the same semantics
as the \texttt{fs.realpath.native()} function.

Only paths that can be converted to UTF8 strings are supported.

The optional \texttt{options} argument can be a string specifying an
encoding, or an object with an \texttt{encoding} property specifying the
character encoding to use for the path. If the \texttt{encoding} is set
to \texttt{\textquotesingle{}buffer\textquotesingle{}}, the path
returned will be passed as a \{Buffer\} object.

On Linux, when Node.js is linked against musl libc, the procfs file
system must be mounted on \texttt{/proc} in order for this function to
work. Glibc does not have this restriction.

\subsubsection{\texorpdfstring{\texttt{fsPromises.rename(oldPath,\ newPath)}}{fsPromises.rename(oldPath, newPath)}}\label{fspromises.renameoldpath-newpath}

\begin{itemize}
\tightlist
\item
  \texttt{oldPath} \{string\textbar Buffer\textbar URL\}
\item
  \texttt{newPath} \{string\textbar Buffer\textbar URL\}
\item
  Returns: \{Promise\} Fulfills with \texttt{undefined} upon success.
\end{itemize}

Renames \texttt{oldPath} to \texttt{newPath}.

\subsubsection{\texorpdfstring{\texttt{fsPromises.rmdir(path{[},\ options{]})}}{fsPromises.rmdir(path{[}, options{]})}}\label{fspromises.rmdirpath-options}

\begin{itemize}
\tightlist
\item
  \texttt{path} \{string\textbar Buffer\textbar URL\}
\item
  \texttt{options} \{Object\}

  \begin{itemize}
  \tightlist
  \item
    \texttt{maxRetries} \{integer\} If an \texttt{EBUSY},
    \texttt{EMFILE}, \texttt{ENFILE}, \texttt{ENOTEMPTY}, or
    \texttt{EPERM} error is encountered, Node.js retries the operation
    with a linear backoff wait of \texttt{retryDelay} milliseconds
    longer on each try. This option represents the number of retries.
    This option is ignored if the \texttt{recursive} option is not
    \texttt{true}. \textbf{Default:} \texttt{0}.
  \item
    \texttt{recursive} \{boolean\} If \texttt{true}, perform a recursive
    directory removal. In recursive mode, operations are retried on
    failure. \textbf{Default:} \texttt{false}. \textbf{Deprecated.}
  \item
    \texttt{retryDelay} \{integer\} The amount of time in milliseconds
    to wait between retries. This option is ignored if the
    \texttt{recursive} option is not \texttt{true}. \textbf{Default:}
    \texttt{100}.
  \end{itemize}
\item
  Returns: \{Promise\} Fulfills with \texttt{undefined} upon success.
\end{itemize}

Removes the directory identified by \texttt{path}.

Using \texttt{fsPromises.rmdir()} on a file (not a directory) results in
the promise being rejected with an \texttt{ENOENT} error on Windows and
an \texttt{ENOTDIR} error on POSIX.

To get a behavior similar to the \texttt{rm\ -rf} Unix command, use
\hyperref[fspromisesrmpath-options]{\texttt{fsPromises.rm()}} with
options \texttt{\{\ recursive:\ true,\ force:\ true\ \}}.

\subsubsection{\texorpdfstring{\texttt{fsPromises.rm(path{[},\ options{]})}}{fsPromises.rm(path{[}, options{]})}}\label{fspromises.rmpath-options}

\begin{itemize}
\tightlist
\item
  \texttt{path} \{string\textbar Buffer\textbar URL\}
\item
  \texttt{options} \{Object\}

  \begin{itemize}
  \tightlist
  \item
    \texttt{force} \{boolean\} When \texttt{true}, exceptions will be
    ignored if \texttt{path} does not exist. \textbf{Default:}
    \texttt{false}.
  \item
    \texttt{maxRetries} \{integer\} If an \texttt{EBUSY},
    \texttt{EMFILE}, \texttt{ENFILE}, \texttt{ENOTEMPTY}, or
    \texttt{EPERM} error is encountered, Node.js will retry the
    operation with a linear backoff wait of \texttt{retryDelay}
    milliseconds longer on each try. This option represents the number
    of retries. This option is ignored if the \texttt{recursive} option
    is not \texttt{true}. \textbf{Default:} \texttt{0}.
  \item
    \texttt{recursive} \{boolean\} If \texttt{true}, perform a recursive
    directory removal. In recursive mode operations are retried on
    failure. \textbf{Default:} \texttt{false}.
  \item
    \texttt{retryDelay} \{integer\} The amount of time in milliseconds
    to wait between retries. This option is ignored if the
    \texttt{recursive} option is not \texttt{true}. \textbf{Default:}
    \texttt{100}.
  \end{itemize}
\item
  Returns: \{Promise\} Fulfills with \texttt{undefined} upon success.
\end{itemize}

Removes files and directories (modeled on the standard POSIX \texttt{rm}
utility).

\subsubsection{\texorpdfstring{\texttt{fsPromises.stat(path{[},\ options{]})}}{fsPromises.stat(path{[}, options{]})}}\label{fspromises.statpath-options}

\begin{itemize}
\tightlist
\item
  \texttt{path} \{string\textbar Buffer\textbar URL\}
\item
  \texttt{options} \{Object\}

  \begin{itemize}
  \tightlist
  \item
    \texttt{bigint} \{boolean\} Whether the numeric values in the
    returned \{fs.Stats\} object should be \texttt{bigint}.
    \textbf{Default:} \texttt{false}.
  \end{itemize}
\item
  Returns: \{Promise\} Fulfills with the \{fs.Stats\} object for the
  given \texttt{path}.
\end{itemize}

\subsubsection{\texorpdfstring{\texttt{fsPromises.statfs(path{[},\ options{]})}}{fsPromises.statfs(path{[}, options{]})}}\label{fspromises.statfspath-options}

\begin{itemize}
\tightlist
\item
  \texttt{path} \{string\textbar Buffer\textbar URL\}
\item
  \texttt{options} \{Object\}

  \begin{itemize}
  \tightlist
  \item
    \texttt{bigint} \{boolean\} Whether the numeric values in the
    returned \{fs.StatFs\} object should be \texttt{bigint}.
    \textbf{Default:} \texttt{false}.
  \end{itemize}
\item
  Returns: \{Promise\} Fulfills with the \{fs.StatFs\} object for the
  given \texttt{path}.
\end{itemize}

\subsubsection{\texorpdfstring{\texttt{fsPromises.symlink(target,\ path{[},\ type{]})}}{fsPromises.symlink(target, path{[}, type{]})}}\label{fspromises.symlinktarget-path-type}

\begin{itemize}
\tightlist
\item
  \texttt{target} \{string\textbar Buffer\textbar URL\}
\item
  \texttt{path} \{string\textbar Buffer\textbar URL\}
\item
  \texttt{type} \{string\textbar null\} \textbf{Default:} \texttt{null}
\item
  Returns: \{Promise\} Fulfills with \texttt{undefined} upon success.
\end{itemize}

Creates a symbolic link.

The \texttt{type} argument is only used on Windows platforms and can be
one of \texttt{\textquotesingle{}dir\textquotesingle{}},
\texttt{\textquotesingle{}file\textquotesingle{}}, or
\texttt{\textquotesingle{}junction\textquotesingle{}}. If the
\texttt{type} argument is not a string, Node.js will autodetect
\texttt{target} type and use
\texttt{\textquotesingle{}file\textquotesingle{}} or
\texttt{\textquotesingle{}dir\textquotesingle{}}. If the \texttt{target}
does not exist, \texttt{\textquotesingle{}file\textquotesingle{}} will
be used. Windows junction points require the destination path to be
absolute. When using
\texttt{\textquotesingle{}junction\textquotesingle{}}, the
\texttt{target} argument will automatically be normalized to absolute
path. Junction points on NTFS volumes can only point to directories.

\subsubsection{\texorpdfstring{\texttt{fsPromises.truncate(path{[},\ len{]})}}{fsPromises.truncate(path{[}, len{]})}}\label{fspromises.truncatepath-len}

\begin{itemize}
\tightlist
\item
  \texttt{path} \{string\textbar Buffer\textbar URL\}
\item
  \texttt{len} \{integer\} \textbf{Default:} \texttt{0}
\item
  Returns: \{Promise\} Fulfills with \texttt{undefined} upon success.
\end{itemize}

Truncates (shortens or extends the length) of the content at
\texttt{path} to \texttt{len} bytes.

\subsubsection{\texorpdfstring{\texttt{fsPromises.unlink(path)}}{fsPromises.unlink(path)}}\label{fspromises.unlinkpath}

\begin{itemize}
\tightlist
\item
  \texttt{path} \{string\textbar Buffer\textbar URL\}
\item
  Returns: \{Promise\} Fulfills with \texttt{undefined} upon success.
\end{itemize}

If \texttt{path} refers to a symbolic link, then the link is removed
without affecting the file or directory to which that link refers. If
the \texttt{path} refers to a file path that is not a symbolic link, the
file is deleted. See the POSIX unlink(2) documentation for more detail.

\subsubsection{\texorpdfstring{\texttt{fsPromises.utimes(path,\ atime,\ mtime)}}{fsPromises.utimes(path, atime, mtime)}}\label{fspromises.utimespath-atime-mtime}

\begin{itemize}
\tightlist
\item
  \texttt{path} \{string\textbar Buffer\textbar URL\}
\item
  \texttt{atime} \{number\textbar string\textbar Date\}
\item
  \texttt{mtime} \{number\textbar string\textbar Date\}
\item
  Returns: \{Promise\} Fulfills with \texttt{undefined} upon success.
\end{itemize}

Change the file system timestamps of the object referenced by
\texttt{path}.

The \texttt{atime} and \texttt{mtime} arguments follow these rules:

\begin{itemize}
\tightlist
\item
  Values can be either numbers representing Unix epoch time,
  \texttt{Date}s, or a numeric string like
  \texttt{\textquotesingle{}123456789.0\textquotesingle{}}.
\item
  If the value can not be converted to a number, or is \texttt{NaN},
  \texttt{Infinity}, or \texttt{-Infinity}, an \texttt{Error} will be
  thrown.
\end{itemize}

\subsubsection{\texorpdfstring{\texttt{fsPromises.watch(filename{[},\ options{]})}}{fsPromises.watch(filename{[}, options{]})}}\label{fspromises.watchfilename-options}

\begin{itemize}
\tightlist
\item
  \texttt{filename} \{string\textbar Buffer\textbar URL\}
\item
  \texttt{options} \{string\textbar Object\}

  \begin{itemize}
  \tightlist
  \item
    \texttt{persistent} \{boolean\} Indicates whether the process should
    continue to run as long as files are being watched.
    \textbf{Default:} \texttt{true}.
  \item
    \texttt{recursive} \{boolean\} Indicates whether all subdirectories
    should be watched, or only the current directory. This applies when
    a directory is specified, and only on supported platforms (See
    \hyperref[caveats]{caveats}). \textbf{Default:} \texttt{false}.
  \item
    \texttt{encoding} \{string\} Specifies the character encoding to be
    used for the filename passed to the listener. \textbf{Default:}
    \texttt{\textquotesingle{}utf8\textquotesingle{}}.
  \item
    \texttt{signal} \{AbortSignal\} An \{AbortSignal\} used to signal
    when the watcher should stop.
  \end{itemize}
\item
  Returns: \{AsyncIterator\} of objects with the properties:

  \begin{itemize}
  \tightlist
  \item
    \texttt{eventType} \{string\} The type of change
  \item
    \texttt{filename} \{string\textbar Buffer\textbar null\} The name of
    the file changed.
  \end{itemize}
\end{itemize}

Returns an async iterator that watches for changes on \texttt{filename},
where \texttt{filename} is either a file or a directory.

\begin{Shaded}
\begin{Highlighting}[]
\KeywordTok{const}\NormalTok{ \{ watch \} }\OperatorTok{=} \PreprocessorTok{require}\NormalTok{(}\StringTok{\textquotesingle{}node:fs/promises\textquotesingle{}}\NormalTok{)}\OperatorTok{;}

\KeywordTok{const}\NormalTok{ ac }\OperatorTok{=} \KeywordTok{new} \FunctionTok{AbortController}\NormalTok{()}\OperatorTok{;}
\KeywordTok{const}\NormalTok{ \{ signal \} }\OperatorTok{=}\NormalTok{ ac}\OperatorTok{;}
\PreprocessorTok{setTimeout}\NormalTok{(() }\KeywordTok{=\textgreater{}}\NormalTok{ ac}\OperatorTok{.}\FunctionTok{abort}\NormalTok{()}\OperatorTok{,} \DecValTok{10000}\NormalTok{)}\OperatorTok{;}

\NormalTok{(}\KeywordTok{async}\NormalTok{ () }\KeywordTok{=\textgreater{}}\NormalTok{ \{}
  \ControlFlowTok{try}\NormalTok{ \{}
    \KeywordTok{const}\NormalTok{ watcher }\OperatorTok{=} \FunctionTok{watch}\NormalTok{(}\BuiltInTok{\_\_filename}\OperatorTok{,}\NormalTok{ \{ signal \})}\OperatorTok{;}
    \ControlFlowTok{for} \ControlFlowTok{await}\NormalTok{ (}\KeywordTok{const} \BuiltInTok{event} \KeywordTok{of}\NormalTok{ watcher)}
      \BuiltInTok{console}\OperatorTok{.}\FunctionTok{log}\NormalTok{(}\BuiltInTok{event}\NormalTok{)}\OperatorTok{;}
\NormalTok{  \} }\ControlFlowTok{catch}\NormalTok{ (err) \{}
    \ControlFlowTok{if}\NormalTok{ (err}\OperatorTok{.}\AttributeTok{name} \OperatorTok{===} \StringTok{\textquotesingle{}AbortError\textquotesingle{}}\NormalTok{)}
      \ControlFlowTok{return}\OperatorTok{;}
    \ControlFlowTok{throw}\NormalTok{ err}\OperatorTok{;}
\NormalTok{  \}}
\NormalTok{\})()}\OperatorTok{;}
\end{Highlighting}
\end{Shaded}

On most platforms, \texttt{\textquotesingle{}rename\textquotesingle{}}
is emitted whenever a filename appears or disappears in the directory.

All the \hyperref[caveats]{caveats} for \texttt{fs.watch()} also apply
to \texttt{fsPromises.watch()}.

\subsubsection{\texorpdfstring{\texttt{fsPromises.writeFile(file,\ data{[},\ options{]})}}{fsPromises.writeFile(file, data{[}, options{]})}}\label{fspromises.writefilefile-data-options}

\begin{itemize}
\tightlist
\item
  \texttt{file} \{string\textbar Buffer\textbar URL\textbar FileHandle\}
  filename or \texttt{FileHandle}
\item
  \texttt{data}
  \{string\textbar Buffer\textbar TypedArray\textbar DataView\textbar AsyncIterable\textbar Iterable\textbar Stream\}
\item
  \texttt{options} \{Object\textbar string\}

  \begin{itemize}
  \tightlist
  \item
    \texttt{encoding} \{string\textbar null\} \textbf{Default:}
    \texttt{\textquotesingle{}utf8\textquotesingle{}}
  \item
    \texttt{mode} \{integer\} \textbf{Default:} \texttt{0o666}
  \item
    \texttt{flag} \{string\} See \hyperref[file-system-flags]{support of
    file system \texttt{flags}}. \textbf{Default:}
    \texttt{\textquotesingle{}w\textquotesingle{}}.
  \item
    \texttt{flush} \{boolean\} If all data is successfully written to
    the file, and \texttt{flush} is \texttt{true},
    \texttt{filehandle.sync()} is used to flush the data.
    \textbf{Default:} \texttt{false}.
  \item
    \texttt{signal} \{AbortSignal\} allows aborting an in-progress
    writeFile
  \end{itemize}
\item
  Returns: \{Promise\} Fulfills with \texttt{undefined} upon success.
\end{itemize}

Asynchronously writes data to a file, replacing the file if it already
exists. \texttt{data} can be a string, a buffer, an \{AsyncIterable\},
or an \{Iterable\} object.

The \texttt{encoding} option is ignored if \texttt{data} is a buffer.

If \texttt{options} is a string, then it specifies the encoding.

The \texttt{mode} option only affects the newly created file. See
\hyperref[fsopenpath-flags-mode-callback]{\texttt{fs.open()}} for more
details.

Any specified \{FileHandle\} has to support writing.

It is unsafe to use \texttt{fsPromises.writeFile()} multiple times on
the same file without waiting for the promise to be settled.

Similarly to \texttt{fsPromises.readFile} -
\texttt{fsPromises.writeFile} is a convenience method that performs
multiple \texttt{write} calls internally to write the buffer passed to
it. For performance sensitive code consider using
\hyperref[fscreatewritestreampath-options]{\texttt{fs.createWriteStream()}}
or
\hyperref[filehandlecreatewritestreamoptions]{\texttt{filehandle.createWriteStream()}}.

It is possible to use an \{AbortSignal\} to cancel an
\texttt{fsPromises.writeFile()}. Cancelation is ``best effort'', and
some amount of data is likely still to be written.

\begin{Shaded}
\begin{Highlighting}[]
\ImportTok{import}\NormalTok{ \{ writeFile \} }\ImportTok{from} \StringTok{\textquotesingle{}node:fs/promises\textquotesingle{}}\OperatorTok{;}
\ImportTok{import}\NormalTok{ \{ }\BuiltInTok{Buffer}\NormalTok{ \} }\ImportTok{from} \StringTok{\textquotesingle{}node:buffer\textquotesingle{}}\OperatorTok{;}

\ControlFlowTok{try}\NormalTok{ \{}
  \KeywordTok{const}\NormalTok{ controller }\OperatorTok{=} \KeywordTok{new} \FunctionTok{AbortController}\NormalTok{()}\OperatorTok{;}
  \KeywordTok{const}\NormalTok{ \{ signal \} }\OperatorTok{=}\NormalTok{ controller}\OperatorTok{;}
  \KeywordTok{const}\NormalTok{ data }\OperatorTok{=} \KeywordTok{new} \BuiltInTok{Uint8Array}\NormalTok{(}\BuiltInTok{Buffer}\OperatorTok{.}\FunctionTok{from}\NormalTok{(}\StringTok{\textquotesingle{}Hello Node.js\textquotesingle{}}\NormalTok{))}\OperatorTok{;}
  \KeywordTok{const}\NormalTok{ promise }\OperatorTok{=} \FunctionTok{writeFile}\NormalTok{(}\StringTok{\textquotesingle{}message.txt\textquotesingle{}}\OperatorTok{,}\NormalTok{ data}\OperatorTok{,}\NormalTok{ \{ signal \})}\OperatorTok{;}

  \CommentTok{// Abort the request before the promise settles.}
\NormalTok{  controller}\OperatorTok{.}\FunctionTok{abort}\NormalTok{()}\OperatorTok{;}

  \ControlFlowTok{await}\NormalTok{ promise}\OperatorTok{;}
\NormalTok{\} }\ControlFlowTok{catch}\NormalTok{ (err) \{}
  \CommentTok{// When a request is aborted {-} err is an AbortError}
  \BuiltInTok{console}\OperatorTok{.}\FunctionTok{error}\NormalTok{(err)}\OperatorTok{;}
\NormalTok{\}}
\end{Highlighting}
\end{Shaded}

Aborting an ongoing request does not abort individual operating system
requests but rather the internal buffering \texttt{fs.writeFile}
performs.

\subsubsection{\texorpdfstring{\texttt{fsPromises.constants}}{fsPromises.constants}}\label{fspromises.constants}

\begin{itemize}
\tightlist
\item
  \{Object\}
\end{itemize}

Returns an object containing commonly used constants for file system
operations. The object is the same as \texttt{fs.constants}. See
\hyperref[fs-constants]{FS constants} for more details.

\subsection{Callback API}\label{callback-api}

The callback APIs perform all operations asynchronously, without
blocking the event loop, then invoke a callback function upon completion
or error.

The callback APIs use the underlying Node.js threadpool to perform file
system operations off the event loop thread. These operations are not
synchronized or threadsafe. Care must be taken when performing multiple
concurrent modifications on the same file or data corruption may occur.

\subsubsection{\texorpdfstring{\texttt{fs.access(path{[},\ mode{]},\ callback)}}{fs.access(path{[}, mode{]}, callback)}}\label{fs.accesspath-mode-callback}

\begin{itemize}
\tightlist
\item
  \texttt{path} \{string\textbar Buffer\textbar URL\}
\item
  \texttt{mode} \{integer\} \textbf{Default:}
  \texttt{fs.constants.F\_OK}
\item
  \texttt{callback} \{Function\}

  \begin{itemize}
  \tightlist
  \item
    \texttt{err} \{Error\}
  \end{itemize}
\end{itemize}

Tests a user's permissions for the file or directory specified by
\texttt{path}. The \texttt{mode} argument is an optional integer that
specifies the accessibility checks to be performed. \texttt{mode} should
be either the value \texttt{fs.constants.F\_OK} or a mask consisting of
the bitwise OR of any of \texttt{fs.constants.R\_OK},
\texttt{fs.constants.W\_OK}, and \texttt{fs.constants.X\_OK} (e.g.
\texttt{fs.constants.W\_OK\ \textbar{}\ fs.constants.R\_OK}). Check
\hyperref[file-access-constants]{File access constants} for possible
values of \texttt{mode}.

The final argument, \texttt{callback}, is a callback function that is
invoked with a possible error argument. If any of the accessibility
checks fail, the error argument will be an \texttt{Error} object. The
following examples check if \texttt{package.json} exists, and if it is
readable or writable.

\begin{Shaded}
\begin{Highlighting}[]
\ImportTok{import}\NormalTok{ \{ access}\OperatorTok{,}\NormalTok{ constants \} }\ImportTok{from} \StringTok{\textquotesingle{}node:fs\textquotesingle{}}\OperatorTok{;}

\KeywordTok{const}\NormalTok{ file }\OperatorTok{=} \StringTok{\textquotesingle{}package.json\textquotesingle{}}\OperatorTok{;}

\CommentTok{// Check if the file exists in the current directory.}
\FunctionTok{access}\NormalTok{(file}\OperatorTok{,}\NormalTok{ constants}\OperatorTok{.}\AttributeTok{F\_OK}\OperatorTok{,}\NormalTok{ (err) }\KeywordTok{=\textgreater{}}\NormalTok{ \{}
  \BuiltInTok{console}\OperatorTok{.}\FunctionTok{log}\NormalTok{(}\VerbatimStringTok{\textasciigrave{}}\SpecialCharTok{$\{}\NormalTok{file}\SpecialCharTok{\}}\VerbatimStringTok{ }\SpecialCharTok{$\{}\NormalTok{err }\OperatorTok{?} \StringTok{\textquotesingle{}does not exist\textquotesingle{}} \OperatorTok{:} \StringTok{\textquotesingle{}exists\textquotesingle{}}\SpecialCharTok{\}}\VerbatimStringTok{\textasciigrave{}}\NormalTok{)}\OperatorTok{;}
\NormalTok{\})}\OperatorTok{;}

\CommentTok{// Check if the file is readable.}
\FunctionTok{access}\NormalTok{(file}\OperatorTok{,}\NormalTok{ constants}\OperatorTok{.}\AttributeTok{R\_OK}\OperatorTok{,}\NormalTok{ (err) }\KeywordTok{=\textgreater{}}\NormalTok{ \{}
  \BuiltInTok{console}\OperatorTok{.}\FunctionTok{log}\NormalTok{(}\VerbatimStringTok{\textasciigrave{}}\SpecialCharTok{$\{}\NormalTok{file}\SpecialCharTok{\}}\VerbatimStringTok{ }\SpecialCharTok{$\{}\NormalTok{err }\OperatorTok{?} \StringTok{\textquotesingle{}is not readable\textquotesingle{}} \OperatorTok{:} \StringTok{\textquotesingle{}is readable\textquotesingle{}}\SpecialCharTok{\}}\VerbatimStringTok{\textasciigrave{}}\NormalTok{)}\OperatorTok{;}
\NormalTok{\})}\OperatorTok{;}

\CommentTok{// Check if the file is writable.}
\FunctionTok{access}\NormalTok{(file}\OperatorTok{,}\NormalTok{ constants}\OperatorTok{.}\AttributeTok{W\_OK}\OperatorTok{,}\NormalTok{ (err) }\KeywordTok{=\textgreater{}}\NormalTok{ \{}
  \BuiltInTok{console}\OperatorTok{.}\FunctionTok{log}\NormalTok{(}\VerbatimStringTok{\textasciigrave{}}\SpecialCharTok{$\{}\NormalTok{file}\SpecialCharTok{\}}\VerbatimStringTok{ }\SpecialCharTok{$\{}\NormalTok{err }\OperatorTok{?} \StringTok{\textquotesingle{}is not writable\textquotesingle{}} \OperatorTok{:} \StringTok{\textquotesingle{}is writable\textquotesingle{}}\SpecialCharTok{\}}\VerbatimStringTok{\textasciigrave{}}\NormalTok{)}\OperatorTok{;}
\NormalTok{\})}\OperatorTok{;}

\CommentTok{// Check if the file is readable and writable.}
\FunctionTok{access}\NormalTok{(file}\OperatorTok{,}\NormalTok{ constants}\OperatorTok{.}\AttributeTok{R\_OK} \OperatorTok{|}\NormalTok{ constants}\OperatorTok{.}\AttributeTok{W\_OK}\OperatorTok{,}\NormalTok{ (err) }\KeywordTok{=\textgreater{}}\NormalTok{ \{}
  \BuiltInTok{console}\OperatorTok{.}\FunctionTok{log}\NormalTok{(}\VerbatimStringTok{\textasciigrave{}}\SpecialCharTok{$\{}\NormalTok{file}\SpecialCharTok{\}}\VerbatimStringTok{ }\SpecialCharTok{$\{}\NormalTok{err }\OperatorTok{?} \StringTok{\textquotesingle{}is not\textquotesingle{}} \OperatorTok{:} \StringTok{\textquotesingle{}is\textquotesingle{}}\SpecialCharTok{\}}\VerbatimStringTok{ readable and writable\textasciigrave{}}\NormalTok{)}\OperatorTok{;}
\NormalTok{\})}\OperatorTok{;}
\end{Highlighting}
\end{Shaded}

Do not use \texttt{fs.access()} to check for the accessibility of a file
before calling \texttt{fs.open()}, \texttt{fs.readFile()}, or
\texttt{fs.writeFile()}. Doing so introduces a race condition, since
other processes may change the file's state between the two calls.
Instead, user code should open/read/write the file directly and handle
the error raised if the file is not accessible.

\textbf{write (NOT RECOMMENDED)}

\begin{Shaded}
\begin{Highlighting}[]
\ImportTok{import}\NormalTok{ \{ access}\OperatorTok{,}\NormalTok{ open}\OperatorTok{,}\NormalTok{ close \} }\ImportTok{from} \StringTok{\textquotesingle{}node:fs\textquotesingle{}}\OperatorTok{;}

\FunctionTok{access}\NormalTok{(}\StringTok{\textquotesingle{}myfile\textquotesingle{}}\OperatorTok{,}\NormalTok{ (err) }\KeywordTok{=\textgreater{}}\NormalTok{ \{}
  \ControlFlowTok{if}\NormalTok{ (}\OperatorTok{!}\NormalTok{err) \{}
    \BuiltInTok{console}\OperatorTok{.}\FunctionTok{error}\NormalTok{(}\StringTok{\textquotesingle{}myfile already exists\textquotesingle{}}\NormalTok{)}\OperatorTok{;}
    \ControlFlowTok{return}\OperatorTok{;}
\NormalTok{  \}}

  \FunctionTok{open}\NormalTok{(}\StringTok{\textquotesingle{}myfile\textquotesingle{}}\OperatorTok{,} \StringTok{\textquotesingle{}wx\textquotesingle{}}\OperatorTok{,}\NormalTok{ (err}\OperatorTok{,}\NormalTok{ fd) }\KeywordTok{=\textgreater{}}\NormalTok{ \{}
    \ControlFlowTok{if}\NormalTok{ (err) }\ControlFlowTok{throw}\NormalTok{ err}\OperatorTok{;}

    \ControlFlowTok{try}\NormalTok{ \{}
      \FunctionTok{writeMyData}\NormalTok{(fd)}\OperatorTok{;}
\NormalTok{    \} }\ControlFlowTok{finally}\NormalTok{ \{}
      \FunctionTok{close}\NormalTok{(fd}\OperatorTok{,}\NormalTok{ (err) }\KeywordTok{=\textgreater{}}\NormalTok{ \{}
        \ControlFlowTok{if}\NormalTok{ (err) }\ControlFlowTok{throw}\NormalTok{ err}\OperatorTok{;}
\NormalTok{      \})}\OperatorTok{;}
\NormalTok{    \}}
\NormalTok{  \})}\OperatorTok{;}
\NormalTok{\})}\OperatorTok{;}
\end{Highlighting}
\end{Shaded}

\textbf{write (RECOMMENDED)}

\begin{Shaded}
\begin{Highlighting}[]
\ImportTok{import}\NormalTok{ \{ open}\OperatorTok{,}\NormalTok{ close \} }\ImportTok{from} \StringTok{\textquotesingle{}node:fs\textquotesingle{}}\OperatorTok{;}

\FunctionTok{open}\NormalTok{(}\StringTok{\textquotesingle{}myfile\textquotesingle{}}\OperatorTok{,} \StringTok{\textquotesingle{}wx\textquotesingle{}}\OperatorTok{,}\NormalTok{ (err}\OperatorTok{,}\NormalTok{ fd) }\KeywordTok{=\textgreater{}}\NormalTok{ \{}
  \ControlFlowTok{if}\NormalTok{ (err) \{}
    \ControlFlowTok{if}\NormalTok{ (err}\OperatorTok{.}\AttributeTok{code} \OperatorTok{===} \StringTok{\textquotesingle{}EEXIST\textquotesingle{}}\NormalTok{) \{}
      \BuiltInTok{console}\OperatorTok{.}\FunctionTok{error}\NormalTok{(}\StringTok{\textquotesingle{}myfile already exists\textquotesingle{}}\NormalTok{)}\OperatorTok{;}
      \ControlFlowTok{return}\OperatorTok{;}
\NormalTok{    \}}

    \ControlFlowTok{throw}\NormalTok{ err}\OperatorTok{;}
\NormalTok{  \}}

  \ControlFlowTok{try}\NormalTok{ \{}
    \FunctionTok{writeMyData}\NormalTok{(fd)}\OperatorTok{;}
\NormalTok{  \} }\ControlFlowTok{finally}\NormalTok{ \{}
    \FunctionTok{close}\NormalTok{(fd}\OperatorTok{,}\NormalTok{ (err) }\KeywordTok{=\textgreater{}}\NormalTok{ \{}
      \ControlFlowTok{if}\NormalTok{ (err) }\ControlFlowTok{throw}\NormalTok{ err}\OperatorTok{;}
\NormalTok{    \})}\OperatorTok{;}
\NormalTok{  \}}
\NormalTok{\})}\OperatorTok{;}
\end{Highlighting}
\end{Shaded}

\textbf{read (NOT RECOMMENDED)}

\begin{Shaded}
\begin{Highlighting}[]
\ImportTok{import}\NormalTok{ \{ access}\OperatorTok{,}\NormalTok{ open}\OperatorTok{,}\NormalTok{ close \} }\ImportTok{from} \StringTok{\textquotesingle{}node:fs\textquotesingle{}}\OperatorTok{;}
\FunctionTok{access}\NormalTok{(}\StringTok{\textquotesingle{}myfile\textquotesingle{}}\OperatorTok{,}\NormalTok{ (err) }\KeywordTok{=\textgreater{}}\NormalTok{ \{}
  \ControlFlowTok{if}\NormalTok{ (err) \{}
    \ControlFlowTok{if}\NormalTok{ (err}\OperatorTok{.}\AttributeTok{code} \OperatorTok{===} \StringTok{\textquotesingle{}ENOENT\textquotesingle{}}\NormalTok{) \{}
      \BuiltInTok{console}\OperatorTok{.}\FunctionTok{error}\NormalTok{(}\StringTok{\textquotesingle{}myfile does not exist\textquotesingle{}}\NormalTok{)}\OperatorTok{;}
      \ControlFlowTok{return}\OperatorTok{;}
\NormalTok{    \}}

    \ControlFlowTok{throw}\NormalTok{ err}\OperatorTok{;}
\NormalTok{  \}}

  \FunctionTok{open}\NormalTok{(}\StringTok{\textquotesingle{}myfile\textquotesingle{}}\OperatorTok{,} \StringTok{\textquotesingle{}r\textquotesingle{}}\OperatorTok{,}\NormalTok{ (err}\OperatorTok{,}\NormalTok{ fd) }\KeywordTok{=\textgreater{}}\NormalTok{ \{}
    \ControlFlowTok{if}\NormalTok{ (err) }\ControlFlowTok{throw}\NormalTok{ err}\OperatorTok{;}

    \ControlFlowTok{try}\NormalTok{ \{}
      \FunctionTok{readMyData}\NormalTok{(fd)}\OperatorTok{;}
\NormalTok{    \} }\ControlFlowTok{finally}\NormalTok{ \{}
      \FunctionTok{close}\NormalTok{(fd}\OperatorTok{,}\NormalTok{ (err) }\KeywordTok{=\textgreater{}}\NormalTok{ \{}
        \ControlFlowTok{if}\NormalTok{ (err) }\ControlFlowTok{throw}\NormalTok{ err}\OperatorTok{;}
\NormalTok{      \})}\OperatorTok{;}
\NormalTok{    \}}
\NormalTok{  \})}\OperatorTok{;}
\NormalTok{\})}\OperatorTok{;}
\end{Highlighting}
\end{Shaded}

\textbf{read (RECOMMENDED)}

\begin{Shaded}
\begin{Highlighting}[]
\ImportTok{import}\NormalTok{ \{ open}\OperatorTok{,}\NormalTok{ close \} }\ImportTok{from} \StringTok{\textquotesingle{}node:fs\textquotesingle{}}\OperatorTok{;}

\FunctionTok{open}\NormalTok{(}\StringTok{\textquotesingle{}myfile\textquotesingle{}}\OperatorTok{,} \StringTok{\textquotesingle{}r\textquotesingle{}}\OperatorTok{,}\NormalTok{ (err}\OperatorTok{,}\NormalTok{ fd) }\KeywordTok{=\textgreater{}}\NormalTok{ \{}
  \ControlFlowTok{if}\NormalTok{ (err) \{}
    \ControlFlowTok{if}\NormalTok{ (err}\OperatorTok{.}\AttributeTok{code} \OperatorTok{===} \StringTok{\textquotesingle{}ENOENT\textquotesingle{}}\NormalTok{) \{}
      \BuiltInTok{console}\OperatorTok{.}\FunctionTok{error}\NormalTok{(}\StringTok{\textquotesingle{}myfile does not exist\textquotesingle{}}\NormalTok{)}\OperatorTok{;}
      \ControlFlowTok{return}\OperatorTok{;}
\NormalTok{    \}}

    \ControlFlowTok{throw}\NormalTok{ err}\OperatorTok{;}
\NormalTok{  \}}

  \ControlFlowTok{try}\NormalTok{ \{}
    \FunctionTok{readMyData}\NormalTok{(fd)}\OperatorTok{;}
\NormalTok{  \} }\ControlFlowTok{finally}\NormalTok{ \{}
    \FunctionTok{close}\NormalTok{(fd}\OperatorTok{,}\NormalTok{ (err) }\KeywordTok{=\textgreater{}}\NormalTok{ \{}
      \ControlFlowTok{if}\NormalTok{ (err) }\ControlFlowTok{throw}\NormalTok{ err}\OperatorTok{;}
\NormalTok{    \})}\OperatorTok{;}
\NormalTok{  \}}
\NormalTok{\})}\OperatorTok{;}
\end{Highlighting}
\end{Shaded}

The ``not recommended'' examples above check for accessibility and then
use the file; the ``recommended'' examples are better because they use
the file directly and handle the error, if any.

In general, check for the accessibility of a file only if the file will
not be used directly, for example when its accessibility is a signal
from another process.

On Windows, access-control policies (ACLs) on a directory may limit
access to a file or directory. The \texttt{fs.access()} function,
however, does not check the ACL and therefore may report that a path is
accessible even if the ACL restricts the user from reading or writing to
it.

\subsubsection{\texorpdfstring{\texttt{fs.appendFile(path,\ data{[},\ options{]},\ callback)}}{fs.appendFile(path, data{[}, options{]}, callback)}}\label{fs.appendfilepath-data-options-callback}

\begin{itemize}
\tightlist
\item
  \texttt{path} \{string\textbar Buffer\textbar URL\textbar number\}
  filename or file descriptor
\item
  \texttt{data} \{string\textbar Buffer\}
\item
  \texttt{options} \{Object\textbar string\}

  \begin{itemize}
  \tightlist
  \item
    \texttt{encoding} \{string\textbar null\} \textbf{Default:}
    \texttt{\textquotesingle{}utf8\textquotesingle{}}
  \item
    \texttt{mode} \{integer\} \textbf{Default:} \texttt{0o666}
  \item
    \texttt{flag} \{string\} See \hyperref[file-system-flags]{support of
    file system \texttt{flags}}. \textbf{Default:}
    \texttt{\textquotesingle{}a\textquotesingle{}}.
  \item
    \texttt{flush} \{boolean\} If \texttt{true}, the underlying file
    descriptor is flushed prior to closing it. \textbf{Default:}
    \texttt{false}.
  \end{itemize}
\item
  \texttt{callback} \{Function\}

  \begin{itemize}
  \tightlist
  \item
    \texttt{err} \{Error\}
  \end{itemize}
\end{itemize}

Asynchronously append data to a file, creating the file if it does not
yet exist. \texttt{data} can be a string or a \{Buffer\}.

The \texttt{mode} option only affects the newly created file. See
\hyperref[fsopenpath-flags-mode-callback]{\texttt{fs.open()}} for more
details.

\begin{Shaded}
\begin{Highlighting}[]
\ImportTok{import}\NormalTok{ \{ appendFile \} }\ImportTok{from} \StringTok{\textquotesingle{}node:fs\textquotesingle{}}\OperatorTok{;}

\FunctionTok{appendFile}\NormalTok{(}\StringTok{\textquotesingle{}message.txt\textquotesingle{}}\OperatorTok{,} \StringTok{\textquotesingle{}data to append\textquotesingle{}}\OperatorTok{,}\NormalTok{ (err) }\KeywordTok{=\textgreater{}}\NormalTok{ \{}
  \ControlFlowTok{if}\NormalTok{ (err) }\ControlFlowTok{throw}\NormalTok{ err}\OperatorTok{;}
  \BuiltInTok{console}\OperatorTok{.}\FunctionTok{log}\NormalTok{(}\StringTok{\textquotesingle{}The "data to append" was appended to file!\textquotesingle{}}\NormalTok{)}\OperatorTok{;}
\NormalTok{\})}\OperatorTok{;}
\end{Highlighting}
\end{Shaded}

If \texttt{options} is a string, then it specifies the encoding:

\begin{Shaded}
\begin{Highlighting}[]
\ImportTok{import}\NormalTok{ \{ appendFile \} }\ImportTok{from} \StringTok{\textquotesingle{}node:fs\textquotesingle{}}\OperatorTok{;}

\FunctionTok{appendFile}\NormalTok{(}\StringTok{\textquotesingle{}message.txt\textquotesingle{}}\OperatorTok{,} \StringTok{\textquotesingle{}data to append\textquotesingle{}}\OperatorTok{,} \StringTok{\textquotesingle{}utf8\textquotesingle{}}\OperatorTok{,}\NormalTok{ callback)}\OperatorTok{;}
\end{Highlighting}
\end{Shaded}

The \texttt{path} may be specified as a numeric file descriptor that has
been opened for appending (using \texttt{fs.open()} or
\texttt{fs.openSync()}). The file descriptor will not be closed
automatically.

\begin{Shaded}
\begin{Highlighting}[]
\ImportTok{import}\NormalTok{ \{ open}\OperatorTok{,}\NormalTok{ close}\OperatorTok{,}\NormalTok{ appendFile \} }\ImportTok{from} \StringTok{\textquotesingle{}node:fs\textquotesingle{}}\OperatorTok{;}

\KeywordTok{function} \FunctionTok{closeFd}\NormalTok{(fd) \{}
  \FunctionTok{close}\NormalTok{(fd}\OperatorTok{,}\NormalTok{ (err) }\KeywordTok{=\textgreater{}}\NormalTok{ \{}
    \ControlFlowTok{if}\NormalTok{ (err) }\ControlFlowTok{throw}\NormalTok{ err}\OperatorTok{;}
\NormalTok{  \})}\OperatorTok{;}
\NormalTok{\}}

\FunctionTok{open}\NormalTok{(}\StringTok{\textquotesingle{}message.txt\textquotesingle{}}\OperatorTok{,} \StringTok{\textquotesingle{}a\textquotesingle{}}\OperatorTok{,}\NormalTok{ (err}\OperatorTok{,}\NormalTok{ fd) }\KeywordTok{=\textgreater{}}\NormalTok{ \{}
  \ControlFlowTok{if}\NormalTok{ (err) }\ControlFlowTok{throw}\NormalTok{ err}\OperatorTok{;}

  \ControlFlowTok{try}\NormalTok{ \{}
    \FunctionTok{appendFile}\NormalTok{(fd}\OperatorTok{,} \StringTok{\textquotesingle{}data to append\textquotesingle{}}\OperatorTok{,} \StringTok{\textquotesingle{}utf8\textquotesingle{}}\OperatorTok{,}\NormalTok{ (err) }\KeywordTok{=\textgreater{}}\NormalTok{ \{}
      \FunctionTok{closeFd}\NormalTok{(fd)}\OperatorTok{;}
      \ControlFlowTok{if}\NormalTok{ (err) }\ControlFlowTok{throw}\NormalTok{ err}\OperatorTok{;}
\NormalTok{    \})}\OperatorTok{;}
\NormalTok{  \} }\ControlFlowTok{catch}\NormalTok{ (err) \{}
    \FunctionTok{closeFd}\NormalTok{(fd)}\OperatorTok{;}
    \ControlFlowTok{throw}\NormalTok{ err}\OperatorTok{;}
\NormalTok{  \}}
\NormalTok{\})}\OperatorTok{;}
\end{Highlighting}
\end{Shaded}

\subsubsection{\texorpdfstring{\texttt{fs.chmod(path,\ mode,\ callback)}}{fs.chmod(path, mode, callback)}}\label{fs.chmodpath-mode-callback}

\begin{itemize}
\tightlist
\item
  \texttt{path} \{string\textbar Buffer\textbar URL\}
\item
  \texttt{mode} \{string\textbar integer\}
\item
  \texttt{callback} \{Function\}

  \begin{itemize}
  \tightlist
  \item
    \texttt{err} \{Error\}
  \end{itemize}
\end{itemize}

Asynchronously changes the permissions of a file. No arguments other
than a possible exception are given to the completion callback.

See the POSIX chmod(2) documentation for more detail.

\begin{Shaded}
\begin{Highlighting}[]
\ImportTok{import}\NormalTok{ \{ chmod \} }\ImportTok{from} \StringTok{\textquotesingle{}node:fs\textquotesingle{}}\OperatorTok{;}

\FunctionTok{chmod}\NormalTok{(}\StringTok{\textquotesingle{}my\_file.txt\textquotesingle{}}\OperatorTok{,} \BaseNTok{0o775}\OperatorTok{,}\NormalTok{ (err) }\KeywordTok{=\textgreater{}}\NormalTok{ \{}
  \ControlFlowTok{if}\NormalTok{ (err) }\ControlFlowTok{throw}\NormalTok{ err}\OperatorTok{;}
  \BuiltInTok{console}\OperatorTok{.}\FunctionTok{log}\NormalTok{(}\StringTok{\textquotesingle{}The permissions for file "my\_file.txt" have been changed!\textquotesingle{}}\NormalTok{)}\OperatorTok{;}
\NormalTok{\})}\OperatorTok{;}
\end{Highlighting}
\end{Shaded}

\paragraph{File modes}\label{file-modes}

The \texttt{mode} argument used in both the \texttt{fs.chmod()} and
\texttt{fs.chmodSync()} methods is a numeric bitmask created using a
logical OR of the following constants:

\begin{longtable}[]{@{}lll@{}}
\toprule\noalign{}
Constant & Octal & Description \\
\midrule\noalign{}
\endhead
\bottomrule\noalign{}
\endlastfoot
\texttt{fs.constants.S\_IRUSR} & \texttt{0o400} & read by owner \\
\texttt{fs.constants.S\_IWUSR} & \texttt{0o200} & write by owner \\
\texttt{fs.constants.S\_IXUSR} & \texttt{0o100} & execute/search by
owner \\
\texttt{fs.constants.S\_IRGRP} & \texttt{0o40} & read by group \\
\texttt{fs.constants.S\_IWGRP} & \texttt{0o20} & write by group \\
\texttt{fs.constants.S\_IXGRP} & \texttt{0o10} & execute/search by
group \\
\texttt{fs.constants.S\_IROTH} & \texttt{0o4} & read by others \\
\texttt{fs.constants.S\_IWOTH} & \texttt{0o2} & write by others \\
\texttt{fs.constants.S\_IXOTH} & \texttt{0o1} & execute/search by
others \\
\end{longtable}

An easier method of constructing the \texttt{mode} is to use a sequence
of three octal digits (e.g.~\texttt{765}). The left-most digit
(\texttt{7} in the example), specifies the permissions for the file
owner. The middle digit (\texttt{6} in the example), specifies
permissions for the group. The right-most digit (\texttt{5} in the
example), specifies the permissions for others.

\begin{longtable}[]{@{}ll@{}}
\toprule\noalign{}
Number & Description \\
\midrule\noalign{}
\endhead
\bottomrule\noalign{}
\endlastfoot
\texttt{7} & read, write, and execute \\
\texttt{6} & read and write \\
\texttt{5} & read and execute \\
\texttt{4} & read only \\
\texttt{3} & write and execute \\
\texttt{2} & write only \\
\texttt{1} & execute only \\
\texttt{0} & no permission \\
\end{longtable}

For example, the octal value \texttt{0o765} means:

\begin{itemize}
\tightlist
\item
  The owner may read, write, and execute the file.
\item
  The group may read and write the file.
\item
  Others may read and execute the file.
\end{itemize}

When using raw numbers where file modes are expected, any value larger
than \texttt{0o777} may result in platform-specific behaviors that are
not supported to work consistently. Therefore constants like
\texttt{S\_ISVTX}, \texttt{S\_ISGID}, or \texttt{S\_ISUID} are not
exposed in \texttt{fs.constants}.

Caveats: on Windows only the write permission can be changed, and the
distinction among the permissions of group, owner, or others is not
implemented.

\subsubsection{\texorpdfstring{\texttt{fs.chown(path,\ uid,\ gid,\ callback)}}{fs.chown(path, uid, gid, callback)}}\label{fs.chownpath-uid-gid-callback}

\begin{itemize}
\tightlist
\item
  \texttt{path} \{string\textbar Buffer\textbar URL\}
\item
  \texttt{uid} \{integer\}
\item
  \texttt{gid} \{integer\}
\item
  \texttt{callback} \{Function\}

  \begin{itemize}
  \tightlist
  \item
    \texttt{err} \{Error\}
  \end{itemize}
\end{itemize}

Asynchronously changes owner and group of a file. No arguments other
than a possible exception are given to the completion callback.

See the POSIX chown(2) documentation for more detail.

\subsubsection{\texorpdfstring{\texttt{fs.close(fd{[},\ callback{]})}}{fs.close(fd{[}, callback{]})}}\label{fs.closefd-callback}

\begin{itemize}
\tightlist
\item
  \texttt{fd} \{integer\}
\item
  \texttt{callback} \{Function\}

  \begin{itemize}
  \tightlist
  \item
    \texttt{err} \{Error\}
  \end{itemize}
\end{itemize}

Closes the file descriptor. No arguments other than a possible exception
are given to the completion callback.

Calling \texttt{fs.close()} on any file descriptor (\texttt{fd}) that is
currently in use through any other \texttt{fs} operation may lead to
undefined behavior.

See the POSIX close(2) documentation for more detail.

\subsubsection{\texorpdfstring{\texttt{fs.copyFile(src,\ dest{[},\ mode{]},\ callback)}}{fs.copyFile(src, dest{[}, mode{]}, callback)}}\label{fs.copyfilesrc-dest-mode-callback}

\begin{itemize}
\tightlist
\item
  \texttt{src} \{string\textbar Buffer\textbar URL\} source filename to
  copy
\item
  \texttt{dest} \{string\textbar Buffer\textbar URL\} destination
  filename of the copy operation
\item
  \texttt{mode} \{integer\} modifiers for copy operation.
  \textbf{Default:} \texttt{0}.
\item
  \texttt{callback} \{Function\}
\end{itemize}

Asynchronously copies \texttt{src} to \texttt{dest}. By default,
\texttt{dest} is overwritten if it already exists. No arguments other
than a possible exception are given to the callback function. Node.js
makes no guarantees about the atomicity of the copy operation. If an
error occurs after the destination file has been opened for writing,
Node.js will attempt to remove the destination.

\texttt{mode} is an optional integer that specifies the behavior of the
copy operation. It is possible to create a mask consisting of the
bitwise OR of two or more values (e.g.
\texttt{fs.constants.COPYFILE\_EXCL\ \textbar{}\ fs.constants.COPYFILE\_FICLONE}).

\begin{itemize}
\tightlist
\item
  \texttt{fs.constants.COPYFILE\_EXCL}: The copy operation will fail if
  \texttt{dest} already exists.
\item
  \texttt{fs.constants.COPYFILE\_FICLONE}: The copy operation will
  attempt to create a copy-on-write reflink. If the platform does not
  support copy-on-write, then a fallback copy mechanism is used.
\item
  \texttt{fs.constants.COPYFILE\_FICLONE\_FORCE}: The copy operation
  will attempt to create a copy-on-write reflink. If the platform does
  not support copy-on-write, then the operation will fail.
\end{itemize}

\begin{Shaded}
\begin{Highlighting}[]
\ImportTok{import}\NormalTok{ \{ copyFile}\OperatorTok{,}\NormalTok{ constants \} }\ImportTok{from} \StringTok{\textquotesingle{}node:fs\textquotesingle{}}\OperatorTok{;}

\KeywordTok{function} \FunctionTok{callback}\NormalTok{(err) \{}
  \ControlFlowTok{if}\NormalTok{ (err) }\ControlFlowTok{throw}\NormalTok{ err}\OperatorTok{;}
  \BuiltInTok{console}\OperatorTok{.}\FunctionTok{log}\NormalTok{(}\StringTok{\textquotesingle{}source.txt was copied to destination.txt\textquotesingle{}}\NormalTok{)}\OperatorTok{;}
\NormalTok{\}}

\CommentTok{// destination.txt will be created or overwritten by default.}
\FunctionTok{copyFile}\NormalTok{(}\StringTok{\textquotesingle{}source.txt\textquotesingle{}}\OperatorTok{,} \StringTok{\textquotesingle{}destination.txt\textquotesingle{}}\OperatorTok{,}\NormalTok{ callback)}\OperatorTok{;}

\CommentTok{// By using COPYFILE\_EXCL, the operation will fail if destination.txt exists.}
\FunctionTok{copyFile}\NormalTok{(}\StringTok{\textquotesingle{}source.txt\textquotesingle{}}\OperatorTok{,} \StringTok{\textquotesingle{}destination.txt\textquotesingle{}}\OperatorTok{,}\NormalTok{ constants}\OperatorTok{.}\AttributeTok{COPYFILE\_EXCL}\OperatorTok{,}\NormalTok{ callback)}\OperatorTok{;}
\end{Highlighting}
\end{Shaded}

\subsubsection{\texorpdfstring{\texttt{fs.cp(src,\ dest{[},\ options{]},\ callback)}}{fs.cp(src, dest{[}, options{]}, callback)}}\label{fs.cpsrc-dest-options-callback}

\begin{quote}
Stability: 1 - Experimental
\end{quote}

\begin{itemize}
\tightlist
\item
  \texttt{src} \{string\textbar URL\} source path to copy.
\item
  \texttt{dest} \{string\textbar URL\} destination path to copy to.
\item
  \texttt{options} \{Object\}

  \begin{itemize}
  \tightlist
  \item
    \texttt{dereference} \{boolean\} dereference symlinks.
    \textbf{Default:} \texttt{false}.
  \item
    \texttt{errorOnExist} \{boolean\} when \texttt{force} is
    \texttt{false}, and the destination exists, throw an error.
    \textbf{Default:} \texttt{false}.
  \item
    \texttt{filter} \{Function\} Function to filter copied
    files/directories. Return \texttt{true} to copy the item,
    \texttt{false} to ignore it. When ignoring a directory, all of its
    contents will be skipped as well. Can also return a \texttt{Promise}
    that resolves to \texttt{true} or \texttt{false} \textbf{Default:}
    \texttt{undefined}.

    \begin{itemize}
    \tightlist
    \item
      \texttt{src} \{string\} source path to copy.
    \item
      \texttt{dest} \{string\} destination path to copy to.
    \item
      Returns: \{boolean\textbar Promise\}
    \end{itemize}
  \item
    \texttt{force} \{boolean\} overwrite existing file or directory. The
    copy operation will ignore errors if you set this to false and the
    destination exists. Use the \texttt{errorOnExist} option to change
    this behavior. \textbf{Default:} \texttt{true}.
  \item
    \texttt{mode} \{integer\} modifiers for copy operation.
    \textbf{Default:} \texttt{0}. See \texttt{mode} flag of
    \hyperref[fscopyfilesrc-dest-mode-callback]{\texttt{fs.copyFile()}}.
  \item
    \texttt{preserveTimestamps} \{boolean\} When \texttt{true}
    timestamps from \texttt{src} will be preserved. \textbf{Default:}
    \texttt{false}.
  \item
    \texttt{recursive} \{boolean\} copy directories recursively
    \textbf{Default:} \texttt{false}
  \item
    \texttt{verbatimSymlinks} \{boolean\} When \texttt{true}, path
    resolution for symlinks will be skipped. \textbf{Default:}
    \texttt{false}
  \end{itemize}
\item
  \texttt{callback} \{Function\}
\end{itemize}

Asynchronously copies the entire directory structure from \texttt{src}
to \texttt{dest}, including subdirectories and files.

When copying a directory to another directory, globs are not supported
and behavior is similar to \texttt{cp\ dir1/\ dir2/}.

\subsubsection{\texorpdfstring{\texttt{fs.createReadStream(path{[},\ options{]})}}{fs.createReadStream(path{[}, options{]})}}\label{fs.createreadstreampath-options}

\begin{itemize}
\tightlist
\item
  \texttt{path} \{string\textbar Buffer\textbar URL\}
\item
  \texttt{options} \{string\textbar Object\}

  \begin{itemize}
  \tightlist
  \item
    \texttt{flags} \{string\} See \hyperref[file-system-flags]{support
    of file system \texttt{flags}}. \textbf{Default:}
    \texttt{\textquotesingle{}r\textquotesingle{}}.
  \item
    \texttt{encoding} \{string\} \textbf{Default:} \texttt{null}
  \item
    \texttt{fd} \{integer\textbar FileHandle\} \textbf{Default:}
    \texttt{null}
  \item
    \texttt{mode} \{integer\} \textbf{Default:} \texttt{0o666}
  \item
    \texttt{autoClose} \{boolean\} \textbf{Default:} \texttt{true}
  \item
    \texttt{emitClose} \{boolean\} \textbf{Default:} \texttt{true}
  \item
    \texttt{start} \{integer\}
  \item
    \texttt{end} \{integer\} \textbf{Default:} \texttt{Infinity}
  \item
    \texttt{highWaterMark} \{integer\} \textbf{Default:}
    \texttt{64\ *\ 1024}
  \item
    \texttt{fs} \{Object\textbar null\} \textbf{Default:} \texttt{null}
  \item
    \texttt{signal} \{AbortSignal\textbar null\} \textbf{Default:}
    \texttt{null}
  \end{itemize}
\item
  Returns: \{fs.ReadStream\}
\end{itemize}

Unlike the 16 KiB default \texttt{highWaterMark} for a
\{stream.Readable\}, the stream returned by this method has a default
\texttt{highWaterMark} of 64 KiB.

\texttt{options} can include \texttt{start} and \texttt{end} values to
read a range of bytes from the file instead of the entire file. Both
\texttt{start} and \texttt{end} are inclusive and start counting at 0,
allowed values are in the {[}0,
\href{https://developer.mozilla.org/en-US/docs/Web/JavaScript/Reference/Global_Objects/Number/MAX_SAFE_INTEGER}{\texttt{Number.MAX\_SAFE\_INTEGER}}{]}
range. If \texttt{fd} is specified and \texttt{start} is omitted or
\texttt{undefined}, \texttt{fs.createReadStream()} reads sequentially
from the current file position. The \texttt{encoding} can be any one of
those accepted by \{Buffer\}.

If \texttt{fd} is specified, \texttt{ReadStream} will ignore the
\texttt{path} argument and will use the specified file descriptor. This
means that no \texttt{\textquotesingle{}open\textquotesingle{}} event
will be emitted. \texttt{fd} should be blocking; non-blocking
\texttt{fd}s should be passed to \{net.Socket\}.

If \texttt{fd} points to a character device that only supports blocking
reads (such as keyboard or sound card), read operations do not finish
until data is available. This can prevent the process from exiting and
the stream from closing naturally.

By default, the stream will emit a
\texttt{\textquotesingle{}close\textquotesingle{}} event after it has
been destroyed. Set the \texttt{emitClose} option to \texttt{false} to
change this behavior.

By providing the \texttt{fs} option, it is possible to override the
corresponding \texttt{fs} implementations for \texttt{open},
\texttt{read}, and \texttt{close}. When providing the \texttt{fs}
option, an override for \texttt{read} is required. If no \texttt{fd} is
provided, an override for \texttt{open} is also required. If
\texttt{autoClose} is \texttt{true}, an override for \texttt{close} is
also required.

\begin{Shaded}
\begin{Highlighting}[]
\ImportTok{import}\NormalTok{ \{ createReadStream \} }\ImportTok{from} \StringTok{\textquotesingle{}node:fs\textquotesingle{}}\OperatorTok{;}

\CommentTok{// Create a stream from some character device.}
\KeywordTok{const}\NormalTok{ stream }\OperatorTok{=} \FunctionTok{createReadStream}\NormalTok{(}\StringTok{\textquotesingle{}/dev/input/event0\textquotesingle{}}\NormalTok{)}\OperatorTok{;}
\PreprocessorTok{setTimeout}\NormalTok{(() }\KeywordTok{=\textgreater{}}\NormalTok{ \{}
\NormalTok{  stream}\OperatorTok{.}\FunctionTok{close}\NormalTok{()}\OperatorTok{;} \CommentTok{// This may not close the stream.}
  \CommentTok{// Artificially marking end{-}of{-}stream, as if the underlying resource had}
  \CommentTok{// indicated end{-}of{-}file by itself, allows the stream to close.}
  \CommentTok{// This does not cancel pending read operations, and if there is such an}
  \CommentTok{// operation, the process may still not be able to exit successfully}
  \CommentTok{// until it finishes.}
\NormalTok{  stream}\OperatorTok{.}\FunctionTok{push}\NormalTok{(}\KeywordTok{null}\NormalTok{)}\OperatorTok{;}
\NormalTok{  stream}\OperatorTok{.}\FunctionTok{read}\NormalTok{(}\DecValTok{0}\NormalTok{)}\OperatorTok{;}
\NormalTok{\}}\OperatorTok{,} \DecValTok{100}\NormalTok{)}\OperatorTok{;}
\end{Highlighting}
\end{Shaded}

If \texttt{autoClose} is false, then the file descriptor won't be
closed, even if there's an error. It is the application's responsibility
to close it and make sure there's no file descriptor leak. If
\texttt{autoClose} is set to true (default behavior), on
\texttt{\textquotesingle{}error\textquotesingle{}} or
\texttt{\textquotesingle{}end\textquotesingle{}} the file descriptor
will be closed automatically.

\texttt{mode} sets the file mode (permission and sticky bits), but only
if the file was created.

An example to read the last 10 bytes of a file which is 100 bytes long:

\begin{Shaded}
\begin{Highlighting}[]
\ImportTok{import}\NormalTok{ \{ createReadStream \} }\ImportTok{from} \StringTok{\textquotesingle{}node:fs\textquotesingle{}}\OperatorTok{;}

\FunctionTok{createReadStream}\NormalTok{(}\StringTok{\textquotesingle{}sample.txt\textquotesingle{}}\OperatorTok{,}\NormalTok{ \{ }\DataTypeTok{start}\OperatorTok{:} \DecValTok{90}\OperatorTok{,} \DataTypeTok{end}\OperatorTok{:} \DecValTok{99}\NormalTok{ \})}\OperatorTok{;}
\end{Highlighting}
\end{Shaded}

If \texttt{options} is a string, then it specifies the encoding.

\subsubsection{\texorpdfstring{\texttt{fs.createWriteStream(path{[},\ options{]})}}{fs.createWriteStream(path{[}, options{]})}}\label{fs.createwritestreampath-options}

\begin{itemize}
\tightlist
\item
  \texttt{path} \{string\textbar Buffer\textbar URL\}
\item
  \texttt{options} \{string\textbar Object\}

  \begin{itemize}
  \tightlist
  \item
    \texttt{flags} \{string\} See \hyperref[file-system-flags]{support
    of file system \texttt{flags}}. \textbf{Default:}
    \texttt{\textquotesingle{}w\textquotesingle{}}.
  \item
    \texttt{encoding} \{string\} \textbf{Default:}
    \texttt{\textquotesingle{}utf8\textquotesingle{}}
  \item
    \texttt{fd} \{integer\textbar FileHandle\} \textbf{Default:}
    \texttt{null}
  \item
    \texttt{mode} \{integer\} \textbf{Default:} \texttt{0o666}
  \item
    \texttt{autoClose} \{boolean\} \textbf{Default:} \texttt{true}
  \item
    \texttt{emitClose} \{boolean\} \textbf{Default:} \texttt{true}
  \item
    \texttt{start} \{integer\}
  \item
    \texttt{fs} \{Object\textbar null\} \textbf{Default:} \texttt{null}
  \item
    \texttt{signal} \{AbortSignal\textbar null\} \textbf{Default:}
    \texttt{null}
  \item
    \texttt{highWaterMark} \{number\} \textbf{Default:} \texttt{16384}
  \item
    \texttt{flush} \{boolean\} If \texttt{true}, the underlying file
    descriptor is flushed prior to closing it. \textbf{Default:}
    \texttt{false}.
  \end{itemize}
\item
  Returns: \{fs.WriteStream\}
\end{itemize}

\texttt{options} may also include a \texttt{start} option to allow
writing data at some position past the beginning of the file, allowed
values are in the {[}0,
\href{https://developer.mozilla.org/en-US/docs/Web/JavaScript/Reference/Global_Objects/Number/MAX_SAFE_INTEGER}{\texttt{Number.MAX\_SAFE\_INTEGER}}{]}
range. Modifying a file rather than replacing it may require the
\texttt{flags} option to be set to \texttt{r+} rather than the default
\texttt{w}. The \texttt{encoding} can be any one of those accepted by
\{Buffer\}.

If \texttt{autoClose} is set to true (default behavior) on
\texttt{\textquotesingle{}error\textquotesingle{}} or
\texttt{\textquotesingle{}finish\textquotesingle{}} the file descriptor
will be closed automatically. If \texttt{autoClose} is false, then the
file descriptor won't be closed, even if there's an error. It is the
application's responsibility to close it and make sure there's no file
descriptor leak.

By default, the stream will emit a
\texttt{\textquotesingle{}close\textquotesingle{}} event after it has
been destroyed. Set the \texttt{emitClose} option to \texttt{false} to
change this behavior.

By providing the \texttt{fs} option it is possible to override the
corresponding \texttt{fs} implementations for \texttt{open},
\texttt{write}, \texttt{writev}, and \texttt{close}. Overriding
\texttt{write()} without \texttt{writev()} can reduce performance as
some optimizations (\texttt{\_writev()}) will be disabled. When
providing the \texttt{fs} option, overrides for at least one of
\texttt{write} and \texttt{writev} are required. If no \texttt{fd}
option is supplied, an override for \texttt{open} is also required. If
\texttt{autoClose} is \texttt{true}, an override for \texttt{close} is
also required.

Like \{fs.ReadStream\}, if \texttt{fd} is specified, \{fs.WriteStream\}
will ignore the \texttt{path} argument and will use the specified file
descriptor. This means that no
\texttt{\textquotesingle{}open\textquotesingle{}} event will be emitted.
\texttt{fd} should be blocking; non-blocking \texttt{fd}s should be
passed to \{net.Socket\}.

If \texttt{options} is a string, then it specifies the encoding.

\subsubsection{\texorpdfstring{\texttt{fs.exists(path,\ callback)}}{fs.exists(path, callback)}}\label{fs.existspath-callback}

\begin{quote}
Stability: 0 - Deprecated: Use
\hyperref[fsstatpath-options-callback]{\texttt{fs.stat()}} or
\hyperref[fsaccesspath-mode-callback]{\texttt{fs.access()}} instead.
\end{quote}

\begin{itemize}
\tightlist
\item
  \texttt{path} \{string\textbar Buffer\textbar URL\}
\item
  \texttt{callback} \{Function\}

  \begin{itemize}
  \tightlist
  \item
    \texttt{exists} \{boolean\}
  \end{itemize}
\end{itemize}

Test whether or not the given path exists by checking with the file
system. Then call the \texttt{callback} argument with either true or
false:

\begin{Shaded}
\begin{Highlighting}[]
\ImportTok{import}\NormalTok{ \{ exists \} }\ImportTok{from} \StringTok{\textquotesingle{}node:fs\textquotesingle{}}\OperatorTok{;}

\FunctionTok{exists}\NormalTok{(}\StringTok{\textquotesingle{}/etc/passwd\textquotesingle{}}\OperatorTok{,}\NormalTok{ (e) }\KeywordTok{=\textgreater{}}\NormalTok{ \{}
  \BuiltInTok{console}\OperatorTok{.}\FunctionTok{log}\NormalTok{(e }\OperatorTok{?} \StringTok{\textquotesingle{}it exists\textquotesingle{}} \OperatorTok{:} \StringTok{\textquotesingle{}no passwd!\textquotesingle{}}\NormalTok{)}\OperatorTok{;}
\NormalTok{\})}\OperatorTok{;}
\end{Highlighting}
\end{Shaded}

\textbf{The parameters for this callback are not consistent with other
Node.js callbacks.} Normally, the first parameter to a Node.js callback
is an \texttt{err} parameter, optionally followed by other parameters.
The \texttt{fs.exists()} callback has only one boolean parameter. This
is one reason \texttt{fs.access()} is recommended instead of
\texttt{fs.exists()}.

Using \texttt{fs.exists()} to check for the existence of a file before
calling \texttt{fs.open()}, \texttt{fs.readFile()}, or
\texttt{fs.writeFile()} is not recommended. Doing so introduces a race
condition, since other processes may change the file's state between the
two calls. Instead, user code should open/read/write the file directly
and handle the error raised if the file does not exist.

\textbf{write (NOT RECOMMENDED)}

\begin{Shaded}
\begin{Highlighting}[]
\ImportTok{import}\NormalTok{ \{ exists}\OperatorTok{,}\NormalTok{ open}\OperatorTok{,}\NormalTok{ close \} }\ImportTok{from} \StringTok{\textquotesingle{}node:fs\textquotesingle{}}\OperatorTok{;}

\FunctionTok{exists}\NormalTok{(}\StringTok{\textquotesingle{}myfile\textquotesingle{}}\OperatorTok{,}\NormalTok{ (e) }\KeywordTok{=\textgreater{}}\NormalTok{ \{}
  \ControlFlowTok{if}\NormalTok{ (e) \{}
    \BuiltInTok{console}\OperatorTok{.}\FunctionTok{error}\NormalTok{(}\StringTok{\textquotesingle{}myfile already exists\textquotesingle{}}\NormalTok{)}\OperatorTok{;}
\NormalTok{  \} }\ControlFlowTok{else}\NormalTok{ \{}
    \FunctionTok{open}\NormalTok{(}\StringTok{\textquotesingle{}myfile\textquotesingle{}}\OperatorTok{,} \StringTok{\textquotesingle{}wx\textquotesingle{}}\OperatorTok{,}\NormalTok{ (err}\OperatorTok{,}\NormalTok{ fd) }\KeywordTok{=\textgreater{}}\NormalTok{ \{}
      \ControlFlowTok{if}\NormalTok{ (err) }\ControlFlowTok{throw}\NormalTok{ err}\OperatorTok{;}

      \ControlFlowTok{try}\NormalTok{ \{}
        \FunctionTok{writeMyData}\NormalTok{(fd)}\OperatorTok{;}
\NormalTok{      \} }\ControlFlowTok{finally}\NormalTok{ \{}
        \FunctionTok{close}\NormalTok{(fd}\OperatorTok{,}\NormalTok{ (err) }\KeywordTok{=\textgreater{}}\NormalTok{ \{}
          \ControlFlowTok{if}\NormalTok{ (err) }\ControlFlowTok{throw}\NormalTok{ err}\OperatorTok{;}
\NormalTok{        \})}\OperatorTok{;}
\NormalTok{      \}}
\NormalTok{    \})}\OperatorTok{;}
\NormalTok{  \}}
\NormalTok{\})}\OperatorTok{;}
\end{Highlighting}
\end{Shaded}

\textbf{write (RECOMMENDED)}

\begin{Shaded}
\begin{Highlighting}[]
\ImportTok{import}\NormalTok{ \{ open}\OperatorTok{,}\NormalTok{ close \} }\ImportTok{from} \StringTok{\textquotesingle{}node:fs\textquotesingle{}}\OperatorTok{;}
\FunctionTok{open}\NormalTok{(}\StringTok{\textquotesingle{}myfile\textquotesingle{}}\OperatorTok{,} \StringTok{\textquotesingle{}wx\textquotesingle{}}\OperatorTok{,}\NormalTok{ (err}\OperatorTok{,}\NormalTok{ fd) }\KeywordTok{=\textgreater{}}\NormalTok{ \{}
  \ControlFlowTok{if}\NormalTok{ (err) \{}
    \ControlFlowTok{if}\NormalTok{ (err}\OperatorTok{.}\AttributeTok{code} \OperatorTok{===} \StringTok{\textquotesingle{}EEXIST\textquotesingle{}}\NormalTok{) \{}
      \BuiltInTok{console}\OperatorTok{.}\FunctionTok{error}\NormalTok{(}\StringTok{\textquotesingle{}myfile already exists\textquotesingle{}}\NormalTok{)}\OperatorTok{;}
      \ControlFlowTok{return}\OperatorTok{;}
\NormalTok{    \}}

    \ControlFlowTok{throw}\NormalTok{ err}\OperatorTok{;}
\NormalTok{  \}}

  \ControlFlowTok{try}\NormalTok{ \{}
    \FunctionTok{writeMyData}\NormalTok{(fd)}\OperatorTok{;}
\NormalTok{  \} }\ControlFlowTok{finally}\NormalTok{ \{}
    \FunctionTok{close}\NormalTok{(fd}\OperatorTok{,}\NormalTok{ (err) }\KeywordTok{=\textgreater{}}\NormalTok{ \{}
      \ControlFlowTok{if}\NormalTok{ (err) }\ControlFlowTok{throw}\NormalTok{ err}\OperatorTok{;}
\NormalTok{    \})}\OperatorTok{;}
\NormalTok{  \}}
\NormalTok{\})}\OperatorTok{;}
\end{Highlighting}
\end{Shaded}

\textbf{read (NOT RECOMMENDED)}

\begin{Shaded}
\begin{Highlighting}[]
\ImportTok{import}\NormalTok{ \{ open}\OperatorTok{,}\NormalTok{ close}\OperatorTok{,}\NormalTok{ exists \} }\ImportTok{from} \StringTok{\textquotesingle{}node:fs\textquotesingle{}}\OperatorTok{;}

\FunctionTok{exists}\NormalTok{(}\StringTok{\textquotesingle{}myfile\textquotesingle{}}\OperatorTok{,}\NormalTok{ (e) }\KeywordTok{=\textgreater{}}\NormalTok{ \{}
  \ControlFlowTok{if}\NormalTok{ (e) \{}
    \FunctionTok{open}\NormalTok{(}\StringTok{\textquotesingle{}myfile\textquotesingle{}}\OperatorTok{,} \StringTok{\textquotesingle{}r\textquotesingle{}}\OperatorTok{,}\NormalTok{ (err}\OperatorTok{,}\NormalTok{ fd) }\KeywordTok{=\textgreater{}}\NormalTok{ \{}
      \ControlFlowTok{if}\NormalTok{ (err) }\ControlFlowTok{throw}\NormalTok{ err}\OperatorTok{;}

      \ControlFlowTok{try}\NormalTok{ \{}
        \FunctionTok{readMyData}\NormalTok{(fd)}\OperatorTok{;}
\NormalTok{      \} }\ControlFlowTok{finally}\NormalTok{ \{}
        \FunctionTok{close}\NormalTok{(fd}\OperatorTok{,}\NormalTok{ (err) }\KeywordTok{=\textgreater{}}\NormalTok{ \{}
          \ControlFlowTok{if}\NormalTok{ (err) }\ControlFlowTok{throw}\NormalTok{ err}\OperatorTok{;}
\NormalTok{        \})}\OperatorTok{;}
\NormalTok{      \}}
\NormalTok{    \})}\OperatorTok{;}
\NormalTok{  \} }\ControlFlowTok{else}\NormalTok{ \{}
    \BuiltInTok{console}\OperatorTok{.}\FunctionTok{error}\NormalTok{(}\StringTok{\textquotesingle{}myfile does not exist\textquotesingle{}}\NormalTok{)}\OperatorTok{;}
\NormalTok{  \}}
\NormalTok{\})}\OperatorTok{;}
\end{Highlighting}
\end{Shaded}

\textbf{read (RECOMMENDED)}

\begin{Shaded}
\begin{Highlighting}[]
\ImportTok{import}\NormalTok{ \{ open}\OperatorTok{,}\NormalTok{ close \} }\ImportTok{from} \StringTok{\textquotesingle{}node:fs\textquotesingle{}}\OperatorTok{;}

\FunctionTok{open}\NormalTok{(}\StringTok{\textquotesingle{}myfile\textquotesingle{}}\OperatorTok{,} \StringTok{\textquotesingle{}r\textquotesingle{}}\OperatorTok{,}\NormalTok{ (err}\OperatorTok{,}\NormalTok{ fd) }\KeywordTok{=\textgreater{}}\NormalTok{ \{}
  \ControlFlowTok{if}\NormalTok{ (err) \{}
    \ControlFlowTok{if}\NormalTok{ (err}\OperatorTok{.}\AttributeTok{code} \OperatorTok{===} \StringTok{\textquotesingle{}ENOENT\textquotesingle{}}\NormalTok{) \{}
      \BuiltInTok{console}\OperatorTok{.}\FunctionTok{error}\NormalTok{(}\StringTok{\textquotesingle{}myfile does not exist\textquotesingle{}}\NormalTok{)}\OperatorTok{;}
      \ControlFlowTok{return}\OperatorTok{;}
\NormalTok{    \}}

    \ControlFlowTok{throw}\NormalTok{ err}\OperatorTok{;}
\NormalTok{  \}}

  \ControlFlowTok{try}\NormalTok{ \{}
    \FunctionTok{readMyData}\NormalTok{(fd)}\OperatorTok{;}
\NormalTok{  \} }\ControlFlowTok{finally}\NormalTok{ \{}
    \FunctionTok{close}\NormalTok{(fd}\OperatorTok{,}\NormalTok{ (err) }\KeywordTok{=\textgreater{}}\NormalTok{ \{}
      \ControlFlowTok{if}\NormalTok{ (err) }\ControlFlowTok{throw}\NormalTok{ err}\OperatorTok{;}
\NormalTok{    \})}\OperatorTok{;}
\NormalTok{  \}}
\NormalTok{\})}\OperatorTok{;}
\end{Highlighting}
\end{Shaded}

The ``not recommended'' examples above check for existence and then use
the file; the ``recommended'' examples are better because they use the
file directly and handle the error, if any.

In general, check for the existence of a file only if the file won't be
used directly, for example when its existence is a signal from another
process.

\subsubsection{\texorpdfstring{\texttt{fs.fchmod(fd,\ mode,\ callback)}}{fs.fchmod(fd, mode, callback)}}\label{fs.fchmodfd-mode-callback}

\begin{itemize}
\tightlist
\item
  \texttt{fd} \{integer\}
\item
  \texttt{mode} \{string\textbar integer\}
\item
  \texttt{callback} \{Function\}

  \begin{itemize}
  \tightlist
  \item
    \texttt{err} \{Error\}
  \end{itemize}
\end{itemize}

Sets the permissions on the file. No arguments other than a possible
exception are given to the completion callback.

See the POSIX fchmod(2) documentation for more detail.

\subsubsection{\texorpdfstring{\texttt{fs.fchown(fd,\ uid,\ gid,\ callback)}}{fs.fchown(fd, uid, gid, callback)}}\label{fs.fchownfd-uid-gid-callback}

\begin{itemize}
\tightlist
\item
  \texttt{fd} \{integer\}
\item
  \texttt{uid} \{integer\}
\item
  \texttt{gid} \{integer\}
\item
  \texttt{callback} \{Function\}

  \begin{itemize}
  \tightlist
  \item
    \texttt{err} \{Error\}
  \end{itemize}
\end{itemize}

Sets the owner of the file. No arguments other than a possible exception
are given to the completion callback.

See the POSIX fchown(2) documentation for more detail.

\subsubsection{\texorpdfstring{\texttt{fs.fdatasync(fd,\ callback)}}{fs.fdatasync(fd, callback)}}\label{fs.fdatasyncfd-callback}

\begin{itemize}
\tightlist
\item
  \texttt{fd} \{integer\}
\item
  \texttt{callback} \{Function\}

  \begin{itemize}
  \tightlist
  \item
    \texttt{err} \{Error\}
  \end{itemize}
\end{itemize}

Forces all currently queued I/O operations associated with the file to
the operating system's synchronized I/O completion state. Refer to the
POSIX fdatasync(2) documentation for details. No arguments other than a
possible exception are given to the completion callback.

\subsubsection{\texorpdfstring{\texttt{fs.fstat(fd{[},\ options{]},\ callback)}}{fs.fstat(fd{[}, options{]}, callback)}}\label{fs.fstatfd-options-callback}

\begin{itemize}
\tightlist
\item
  \texttt{fd} \{integer\}
\item
  \texttt{options} \{Object\}

  \begin{itemize}
  \tightlist
  \item
    \texttt{bigint} \{boolean\} Whether the numeric values in the
    returned \{fs.Stats\} object should be \texttt{bigint}.
    \textbf{Default:} \texttt{false}.
  \end{itemize}
\item
  \texttt{callback} \{Function\}

  \begin{itemize}
  \tightlist
  \item
    \texttt{err} \{Error\}
  \item
    \texttt{stats} \{fs.Stats\}
  \end{itemize}
\end{itemize}

Invokes the callback with the \{fs.Stats\} for the file descriptor.

See the POSIX fstat(2) documentation for more detail.

\subsubsection{\texorpdfstring{\texttt{fs.fsync(fd,\ callback)}}{fs.fsync(fd, callback)}}\label{fs.fsyncfd-callback}

\begin{itemize}
\tightlist
\item
  \texttt{fd} \{integer\}
\item
  \texttt{callback} \{Function\}

  \begin{itemize}
  \tightlist
  \item
    \texttt{err} \{Error\}
  \end{itemize}
\end{itemize}

Request that all data for the open file descriptor is flushed to the
storage device. The specific implementation is operating system and
device specific. Refer to the POSIX fsync(2) documentation for more
detail. No arguments other than a possible exception are given to the
completion callback.

\subsubsection{\texorpdfstring{\texttt{fs.ftruncate(fd{[},\ len{]},\ callback)}}{fs.ftruncate(fd{[}, len{]}, callback)}}\label{fs.ftruncatefd-len-callback}

\begin{itemize}
\tightlist
\item
  \texttt{fd} \{integer\}
\item
  \texttt{len} \{integer\} \textbf{Default:} \texttt{0}
\item
  \texttt{callback} \{Function\}

  \begin{itemize}
  \tightlist
  \item
    \texttt{err} \{Error\}
  \end{itemize}
\end{itemize}

Truncates the file descriptor. No arguments other than a possible
exception are given to the completion callback.

See the POSIX ftruncate(2) documentation for more detail.

If the file referred to by the file descriptor was larger than
\texttt{len} bytes, only the first \texttt{len} bytes will be retained
in the file.

For example, the following program retains only the first four bytes of
the file:

\begin{Shaded}
\begin{Highlighting}[]
\ImportTok{import}\NormalTok{ \{ open}\OperatorTok{,}\NormalTok{ close}\OperatorTok{,}\NormalTok{ ftruncate \} }\ImportTok{from} \StringTok{\textquotesingle{}node:fs\textquotesingle{}}\OperatorTok{;}

\KeywordTok{function} \FunctionTok{closeFd}\NormalTok{(fd) \{}
  \FunctionTok{close}\NormalTok{(fd}\OperatorTok{,}\NormalTok{ (err) }\KeywordTok{=\textgreater{}}\NormalTok{ \{}
    \ControlFlowTok{if}\NormalTok{ (err) }\ControlFlowTok{throw}\NormalTok{ err}\OperatorTok{;}
\NormalTok{  \})}\OperatorTok{;}
\NormalTok{\}}

\FunctionTok{open}\NormalTok{(}\StringTok{\textquotesingle{}temp.txt\textquotesingle{}}\OperatorTok{,} \StringTok{\textquotesingle{}r+\textquotesingle{}}\OperatorTok{,}\NormalTok{ (err}\OperatorTok{,}\NormalTok{ fd) }\KeywordTok{=\textgreater{}}\NormalTok{ \{}
  \ControlFlowTok{if}\NormalTok{ (err) }\ControlFlowTok{throw}\NormalTok{ err}\OperatorTok{;}

  \ControlFlowTok{try}\NormalTok{ \{}
    \FunctionTok{ftruncate}\NormalTok{(fd}\OperatorTok{,} \DecValTok{4}\OperatorTok{,}\NormalTok{ (err) }\KeywordTok{=\textgreater{}}\NormalTok{ \{}
      \FunctionTok{closeFd}\NormalTok{(fd)}\OperatorTok{;}
      \ControlFlowTok{if}\NormalTok{ (err) }\ControlFlowTok{throw}\NormalTok{ err}\OperatorTok{;}
\NormalTok{    \})}\OperatorTok{;}
\NormalTok{  \} }\ControlFlowTok{catch}\NormalTok{ (err) \{}
    \FunctionTok{closeFd}\NormalTok{(fd)}\OperatorTok{;}
    \ControlFlowTok{if}\NormalTok{ (err) }\ControlFlowTok{throw}\NormalTok{ err}\OperatorTok{;}
\NormalTok{  \}}
\NormalTok{\})}\OperatorTok{;}
\end{Highlighting}
\end{Shaded}

If the file previously was shorter than \texttt{len} bytes, it is
extended, and the extended part is filled with null bytes
(\texttt{\textquotesingle{}\textbackslash{}0\textquotesingle{}}):

If \texttt{len} is negative then \texttt{0} will be used.

\subsubsection{\texorpdfstring{\texttt{fs.futimes(fd,\ atime,\ mtime,\ callback)}}{fs.futimes(fd, atime, mtime, callback)}}\label{fs.futimesfd-atime-mtime-callback}

\begin{itemize}
\tightlist
\item
  \texttt{fd} \{integer\}
\item
  \texttt{atime} \{number\textbar string\textbar Date\}
\item
  \texttt{mtime} \{number\textbar string\textbar Date\}
\item
  \texttt{callback} \{Function\}

  \begin{itemize}
  \tightlist
  \item
    \texttt{err} \{Error\}
  \end{itemize}
\end{itemize}

Change the file system timestamps of the object referenced by the
supplied file descriptor. See
\hyperref[fsutimespath-atime-mtime-callback]{\texttt{fs.utimes()}}.

\subsubsection{\texorpdfstring{\texttt{fs.lchmod(path,\ mode,\ callback)}}{fs.lchmod(path, mode, callback)}}\label{fs.lchmodpath-mode-callback}

\begin{itemize}
\tightlist
\item
  \texttt{path} \{string\textbar Buffer\textbar URL\}
\item
  \texttt{mode} \{integer\}
\item
  \texttt{callback} \{Function\}

  \begin{itemize}
  \tightlist
  \item
    \texttt{err} \{Error\textbar AggregateError\}
  \end{itemize}
\end{itemize}

Changes the permissions on a symbolic link. No arguments other than a
possible exception are given to the completion callback.

This method is only implemented on macOS.

See the POSIX lchmod(2) documentation for more detail.

\subsubsection{\texorpdfstring{\texttt{fs.lchown(path,\ uid,\ gid,\ callback)}}{fs.lchown(path, uid, gid, callback)}}\label{fs.lchownpath-uid-gid-callback}

\begin{itemize}
\tightlist
\item
  \texttt{path} \{string\textbar Buffer\textbar URL\}
\item
  \texttt{uid} \{integer\}
\item
  \texttt{gid} \{integer\}
\item
  \texttt{callback} \{Function\}

  \begin{itemize}
  \tightlist
  \item
    \texttt{err} \{Error\}
  \end{itemize}
\end{itemize}

Set the owner of the symbolic link. No arguments other than a possible
exception are given to the completion callback.

See the POSIX lchown(2) documentation for more detail.

\subsubsection{\texorpdfstring{\texttt{fs.lutimes(path,\ atime,\ mtime,\ callback)}}{fs.lutimes(path, atime, mtime, callback)}}\label{fs.lutimespath-atime-mtime-callback}

\begin{itemize}
\tightlist
\item
  \texttt{path} \{string\textbar Buffer\textbar URL\}
\item
  \texttt{atime} \{number\textbar string\textbar Date\}
\item
  \texttt{mtime} \{number\textbar string\textbar Date\}
\item
  \texttt{callback} \{Function\}

  \begin{itemize}
  \tightlist
  \item
    \texttt{err} \{Error\}
  \end{itemize}
\end{itemize}

Changes the access and modification times of a file in the same way as
\hyperref[fsutimespath-atime-mtime-callback]{\texttt{fs.utimes()}}, with
the difference that if the path refers to a symbolic link, then the link
is not dereferenced: instead, the timestamps of the symbolic link itself
are changed.

No arguments other than a possible exception are given to the completion
callback.

\subsubsection{\texorpdfstring{\texttt{fs.link(existingPath,\ newPath,\ callback)}}{fs.link(existingPath, newPath, callback)}}\label{fs.linkexistingpath-newpath-callback}

\begin{itemize}
\tightlist
\item
  \texttt{existingPath} \{string\textbar Buffer\textbar URL\}
\item
  \texttt{newPath} \{string\textbar Buffer\textbar URL\}
\item
  \texttt{callback} \{Function\}

  \begin{itemize}
  \tightlist
  \item
    \texttt{err} \{Error\}
  \end{itemize}
\end{itemize}

Creates a new link from the \texttt{existingPath} to the
\texttt{newPath}. See the POSIX link(2) documentation for more detail.
No arguments other than a possible exception are given to the completion
callback.

\subsubsection{\texorpdfstring{\texttt{fs.lstat(path{[},\ options{]},\ callback)}}{fs.lstat(path{[}, options{]}, callback)}}\label{fs.lstatpath-options-callback}

\begin{itemize}
\tightlist
\item
  \texttt{path} \{string\textbar Buffer\textbar URL\}
\item
  \texttt{options} \{Object\}

  \begin{itemize}
  \tightlist
  \item
    \texttt{bigint} \{boolean\} Whether the numeric values in the
    returned \{fs.Stats\} object should be \texttt{bigint}.
    \textbf{Default:} \texttt{false}.
  \end{itemize}
\item
  \texttt{callback} \{Function\}

  \begin{itemize}
  \tightlist
  \item
    \texttt{err} \{Error\}
  \item
    \texttt{stats} \{fs.Stats\}
  \end{itemize}
\end{itemize}

Retrieves the \{fs.Stats\} for the symbolic link referred to by the
path. The callback gets two arguments \texttt{(err,\ stats)} where
\texttt{stats} is a \{fs.Stats\} object. \texttt{lstat()} is identical
to \texttt{stat()}, except that if \texttt{path} is a symbolic link,
then the link itself is stat-ed, not the file that it refers to.

See the POSIX lstat(2) documentation for more details.

\subsubsection{\texorpdfstring{\texttt{fs.mkdir(path{[},\ options{]},\ callback)}}{fs.mkdir(path{[}, options{]}, callback)}}\label{fs.mkdirpath-options-callback}

\begin{itemize}
\tightlist
\item
  \texttt{path} \{string\textbar Buffer\textbar URL\}
\item
  \texttt{options} \{Object\textbar integer\}

  \begin{itemize}
  \tightlist
  \item
    \texttt{recursive} \{boolean\} \textbf{Default:} \texttt{false}
  \item
    \texttt{mode} \{string\textbar integer\} Not supported on Windows.
    \textbf{Default:} \texttt{0o777}.
  \end{itemize}
\item
  \texttt{callback} \{Function\}

  \begin{itemize}
  \tightlist
  \item
    \texttt{err} \{Error\}
  \item
    \texttt{path} \{string\textbar undefined\} Present only if a
    directory is created with \texttt{recursive} set to \texttt{true}.
  \end{itemize}
\end{itemize}

Asynchronously creates a directory.

The callback is given a possible exception and, if \texttt{recursive} is
\texttt{true}, the first directory path created,
\texttt{(err{[},\ path{]})}. \texttt{path} can still be
\texttt{undefined} when \texttt{recursive} is \texttt{true}, if no
directory was created (for instance, if it was previously created).

The optional \texttt{options} argument can be an integer specifying
\texttt{mode} (permission and sticky bits), or an object with a
\texttt{mode} property and a \texttt{recursive} property indicating
whether parent directories should be created. Calling
\texttt{fs.mkdir()} when \texttt{path} is a directory that exists
results in an error only when \texttt{recursive} is false. If
\texttt{recursive} is false and the directory exists, an \texttt{EEXIST}
error occurs.

\begin{Shaded}
\begin{Highlighting}[]
\ImportTok{import}\NormalTok{ \{ mkdir \} }\ImportTok{from} \StringTok{\textquotesingle{}node:fs\textquotesingle{}}\OperatorTok{;}

\CommentTok{// Create ./tmp/a/apple, regardless of whether ./tmp and ./tmp/a exist.}
\FunctionTok{mkdir}\NormalTok{(}\StringTok{\textquotesingle{}./tmp/a/apple\textquotesingle{}}\OperatorTok{,}\NormalTok{ \{ }\DataTypeTok{recursive}\OperatorTok{:} \KeywordTok{true}\NormalTok{ \}}\OperatorTok{,}\NormalTok{ (err) }\KeywordTok{=\textgreater{}}\NormalTok{ \{}
  \ControlFlowTok{if}\NormalTok{ (err) }\ControlFlowTok{throw}\NormalTok{ err}\OperatorTok{;}
\NormalTok{\})}\OperatorTok{;}
\end{Highlighting}
\end{Shaded}

On Windows, using \texttt{fs.mkdir()} on the root directory even with
recursion will result in an error:

\begin{Shaded}
\begin{Highlighting}[]
\ImportTok{import}\NormalTok{ \{ mkdir \} }\ImportTok{from} \StringTok{\textquotesingle{}node:fs\textquotesingle{}}\OperatorTok{;}

\FunctionTok{mkdir}\NormalTok{(}\StringTok{\textquotesingle{}/\textquotesingle{}}\OperatorTok{,}\NormalTok{ \{ }\DataTypeTok{recursive}\OperatorTok{:} \KeywordTok{true}\NormalTok{ \}}\OperatorTok{,}\NormalTok{ (err) }\KeywordTok{=\textgreater{}}\NormalTok{ \{}
  \CommentTok{// =\textgreater{} [Error: EPERM: operation not permitted, mkdir \textquotesingle{}C:\textbackslash{}\textquotesingle{}]}
\NormalTok{\})}\OperatorTok{;}
\end{Highlighting}
\end{Shaded}

See the POSIX mkdir(2) documentation for more details.

\subsubsection{\texorpdfstring{\texttt{fs.mkdtemp(prefix{[},\ options{]},\ callback)}}{fs.mkdtemp(prefix{[}, options{]}, callback)}}\label{fs.mkdtempprefix-options-callback}

\begin{itemize}
\tightlist
\item
  \texttt{prefix} \{string\textbar Buffer\textbar URL\}
\item
  \texttt{options} \{string\textbar Object\}

  \begin{itemize}
  \tightlist
  \item
    \texttt{encoding} \{string\} \textbf{Default:}
    \texttt{\textquotesingle{}utf8\textquotesingle{}}
  \end{itemize}
\item
  \texttt{callback} \{Function\}

  \begin{itemize}
  \tightlist
  \item
    \texttt{err} \{Error\}
  \item
    \texttt{directory} \{string\}
  \end{itemize}
\end{itemize}

Creates a unique temporary directory.

Generates six random characters to be appended behind a required
\texttt{prefix} to create a unique temporary directory. Due to platform
inconsistencies, avoid trailing \texttt{X} characters in
\texttt{prefix}. Some platforms, notably the BSDs, can return more than
six random characters, and replace trailing \texttt{X} characters in
\texttt{prefix} with random characters.

The created directory path is passed as a string to the callback's
second parameter.

The optional \texttt{options} argument can be a string specifying an
encoding, or an object with an \texttt{encoding} property specifying the
character encoding to use.

\begin{Shaded}
\begin{Highlighting}[]
\ImportTok{import}\NormalTok{ \{ mkdtemp \} }\ImportTok{from} \StringTok{\textquotesingle{}node:fs\textquotesingle{}}\OperatorTok{;}
\ImportTok{import}\NormalTok{ \{ join \} }\ImportTok{from} \StringTok{\textquotesingle{}node:path\textquotesingle{}}\OperatorTok{;}
\ImportTok{import}\NormalTok{ \{ tmpdir \} }\ImportTok{from} \StringTok{\textquotesingle{}node:os\textquotesingle{}}\OperatorTok{;}

\FunctionTok{mkdtemp}\NormalTok{(}\FunctionTok{join}\NormalTok{(}\FunctionTok{tmpdir}\NormalTok{()}\OperatorTok{,} \StringTok{\textquotesingle{}foo{-}\textquotesingle{}}\NormalTok{)}\OperatorTok{,}\NormalTok{ (err}\OperatorTok{,}\NormalTok{ directory) }\KeywordTok{=\textgreater{}}\NormalTok{ \{}
  \ControlFlowTok{if}\NormalTok{ (err) }\ControlFlowTok{throw}\NormalTok{ err}\OperatorTok{;}
  \BuiltInTok{console}\OperatorTok{.}\FunctionTok{log}\NormalTok{(directory)}\OperatorTok{;}
  \CommentTok{// Prints: /tmp/foo{-}itXde2 or C:\textbackslash{}Users\textbackslash{}...\textbackslash{}AppData\textbackslash{}Local\textbackslash{}Temp\textbackslash{}foo{-}itXde2}
\NormalTok{\})}\OperatorTok{;}
\end{Highlighting}
\end{Shaded}

The \texttt{fs.mkdtemp()} method will append the six randomly selected
characters directly to the \texttt{prefix} string. For instance, given a
directory \texttt{/tmp}, if the intention is to create a temporary
directory \emph{within} \texttt{/tmp}, the \texttt{prefix} must end with
a trailing platform-specific path separator
(\texttt{require(\textquotesingle{}node:path\textquotesingle{}).sep}).

\begin{Shaded}
\begin{Highlighting}[]
\ImportTok{import}\NormalTok{ \{ tmpdir \} }\ImportTok{from} \StringTok{\textquotesingle{}node:os\textquotesingle{}}\OperatorTok{;}
\ImportTok{import}\NormalTok{ \{ mkdtemp \} }\ImportTok{from} \StringTok{\textquotesingle{}node:fs\textquotesingle{}}\OperatorTok{;}

\CommentTok{// The parent directory for the new temporary directory}
\KeywordTok{const}\NormalTok{ tmpDir }\OperatorTok{=} \FunctionTok{tmpdir}\NormalTok{()}\OperatorTok{;}

\CommentTok{// This method is *INCORRECT*:}
\FunctionTok{mkdtemp}\NormalTok{(tmpDir}\OperatorTok{,}\NormalTok{ (err}\OperatorTok{,}\NormalTok{ directory) }\KeywordTok{=\textgreater{}}\NormalTok{ \{}
  \ControlFlowTok{if}\NormalTok{ (err) }\ControlFlowTok{throw}\NormalTok{ err}\OperatorTok{;}
  \BuiltInTok{console}\OperatorTok{.}\FunctionTok{log}\NormalTok{(directory)}\OperatorTok{;}
  \CommentTok{// Will print something similar to \textasciigrave{}/tmpabc123\textasciigrave{}.}
  \CommentTok{// A new temporary directory is created at the file system root}
  \CommentTok{// rather than *within* the /tmp directory.}
\NormalTok{\})}\OperatorTok{;}

\CommentTok{// This method is *CORRECT*:}
\ImportTok{import}\NormalTok{ \{ sep \} }\ImportTok{from} \StringTok{\textquotesingle{}node:path\textquotesingle{}}\OperatorTok{;}
\FunctionTok{mkdtemp}\NormalTok{(}\VerbatimStringTok{\textasciigrave{}}\SpecialCharTok{$\{}\NormalTok{tmpDir}\SpecialCharTok{\}$\{}\NormalTok{sep}\SpecialCharTok{\}}\VerbatimStringTok{\textasciigrave{}}\OperatorTok{,}\NormalTok{ (err}\OperatorTok{,}\NormalTok{ directory) }\KeywordTok{=\textgreater{}}\NormalTok{ \{}
  \ControlFlowTok{if}\NormalTok{ (err) }\ControlFlowTok{throw}\NormalTok{ err}\OperatorTok{;}
  \BuiltInTok{console}\OperatorTok{.}\FunctionTok{log}\NormalTok{(directory)}\OperatorTok{;}
  \CommentTok{// Will print something similar to \textasciigrave{}/tmp/abc123\textasciigrave{}.}
  \CommentTok{// A new temporary directory is created within}
  \CommentTok{// the /tmp directory.}
\NormalTok{\})}\OperatorTok{;}
\end{Highlighting}
\end{Shaded}

\subsubsection{\texorpdfstring{\texttt{fs.open(path{[},\ flags{[},\ mode{]}{]},\ callback)}}{fs.open(path{[}, flags{[}, mode{]}{]}, callback)}}\label{fs.openpath-flags-mode-callback}

\begin{itemize}
\tightlist
\item
  \texttt{path} \{string\textbar Buffer\textbar URL\}
\item
  \texttt{flags} \{string\textbar number\} See
  \hyperref[file-system-flags]{support of file system \texttt{flags}}.
  \textbf{Default:} \texttt{\textquotesingle{}r\textquotesingle{}}.
\item
  \texttt{mode} \{string\textbar integer\} \textbf{Default:}
  \texttt{0o666} (readable and writable)
\item
  \texttt{callback} \{Function\}

  \begin{itemize}
  \tightlist
  \item
    \texttt{err} \{Error\}
  \item
    \texttt{fd} \{integer\}
  \end{itemize}
\end{itemize}

Asynchronous file open. See the POSIX open(2) documentation for more
details.

\texttt{mode} sets the file mode (permission and sticky bits), but only
if the file was created. On Windows, only the write permission can be
manipulated; see
\hyperref[fschmodpath-mode-callback]{\texttt{fs.chmod()}}.

The callback gets two arguments \texttt{(err,\ fd)}.

Some characters
(\texttt{\textless{}\ \textgreater{}\ :\ "\ /\ \textbackslash{}\ \textbar{}\ ?\ *})
are reserved under Windows as documented by
\href{https://docs.microsoft.com/en-us/windows/desktop/FileIO/naming-a-file}{Naming
Files, Paths, and Namespaces}. Under NTFS, if the filename contains a
colon, Node.js will open a file system stream, as described by
\href{https://docs.microsoft.com/en-us/windows/desktop/FileIO/using-streams}{this
MSDN page}.

Functions based on \texttt{fs.open()} exhibit this behavior as well:
\texttt{fs.writeFile()}, \texttt{fs.readFile()}, etc.

\subsubsection{\texorpdfstring{\texttt{fs.openAsBlob(path{[},\ options{]})}}{fs.openAsBlob(path{[}, options{]})}}\label{fs.openasblobpath-options}

\begin{quote}
Stability: 1 - Experimental
\end{quote}

\begin{itemize}
\tightlist
\item
  \texttt{path} \{string\textbar Buffer\textbar URL\}
\item
  \texttt{options} \{Object\}

  \begin{itemize}
  \tightlist
  \item
    \texttt{type} \{string\} An optional mime type for the blob.
  \end{itemize}
\item
  Return: \{Promise\} containing \{Blob\}
\end{itemize}

Returns a \{Blob\} whose data is backed by the given file.

The file must not be modified after the \{Blob\} is created. Any
modifications will cause reading the \{Blob\} data to fail with a
\texttt{DOMException} error. Synchronous stat operations on the file
when the \texttt{Blob} is created, and before each read in order to
detect whether the file data has been modified on disk.

\begin{Shaded}
\begin{Highlighting}[]
\ImportTok{import}\NormalTok{ \{ openAsBlob \} }\ImportTok{from} \StringTok{\textquotesingle{}node:fs\textquotesingle{}}\OperatorTok{;}

\KeywordTok{const}\NormalTok{ blob }\OperatorTok{=} \ControlFlowTok{await} \FunctionTok{openAsBlob}\NormalTok{(}\StringTok{\textquotesingle{}the.file.txt\textquotesingle{}}\NormalTok{)}\OperatorTok{;}
\KeywordTok{const}\NormalTok{ ab }\OperatorTok{=} \ControlFlowTok{await}\NormalTok{ blob}\OperatorTok{.}\FunctionTok{arrayBuffer}\NormalTok{()}\OperatorTok{;}
\NormalTok{blob}\OperatorTok{.}\FunctionTok{stream}\NormalTok{()}\OperatorTok{;}
\end{Highlighting}
\end{Shaded}

\begin{Shaded}
\begin{Highlighting}[]
\KeywordTok{const}\NormalTok{ \{ openAsBlob \} }\OperatorTok{=} \PreprocessorTok{require}\NormalTok{(}\StringTok{\textquotesingle{}node:fs\textquotesingle{}}\NormalTok{)}\OperatorTok{;}

\NormalTok{(}\KeywordTok{async}\NormalTok{ () }\KeywordTok{=\textgreater{}}\NormalTok{ \{}
  \KeywordTok{const}\NormalTok{ blob }\OperatorTok{=} \ControlFlowTok{await} \FunctionTok{openAsBlob}\NormalTok{(}\StringTok{\textquotesingle{}the.file.txt\textquotesingle{}}\NormalTok{)}\OperatorTok{;}
  \KeywordTok{const}\NormalTok{ ab }\OperatorTok{=} \ControlFlowTok{await}\NormalTok{ blob}\OperatorTok{.}\FunctionTok{arrayBuffer}\NormalTok{()}\OperatorTok{;}
\NormalTok{  blob}\OperatorTok{.}\FunctionTok{stream}\NormalTok{()}\OperatorTok{;}
\NormalTok{\})()}\OperatorTok{;}
\end{Highlighting}
\end{Shaded}

\subsubsection{\texorpdfstring{\texttt{fs.opendir(path{[},\ options{]},\ callback)}}{fs.opendir(path{[}, options{]}, callback)}}\label{fs.opendirpath-options-callback}

\begin{itemize}
\tightlist
\item
  \texttt{path} \{string\textbar Buffer\textbar URL\}
\item
  \texttt{options} \{Object\}

  \begin{itemize}
  \tightlist
  \item
    \texttt{encoding} \{string\textbar null\} \textbf{Default:}
    \texttt{\textquotesingle{}utf8\textquotesingle{}}
  \item
    \texttt{bufferSize} \{number\} Number of directory entries that are
    buffered internally when reading from the directory. Higher values
    lead to better performance but higher memory usage.
    \textbf{Default:} \texttt{32}
  \item
    \texttt{recursive} \{boolean\} \textbf{Default:} \texttt{false}
  \end{itemize}
\item
  \texttt{callback} \{Function\}

  \begin{itemize}
  \tightlist
  \item
    \texttt{err} \{Error\}
  \item
    \texttt{dir} \{fs.Dir\}
  \end{itemize}
\end{itemize}

Asynchronously open a directory. See the POSIX opendir(3) documentation
for more details.

Creates an \{fs.Dir\}, which contains all further functions for reading
from and cleaning up the directory.

The \texttt{encoding} option sets the encoding for the \texttt{path}
while opening the directory and subsequent read operations.

\subsubsection{\texorpdfstring{\texttt{fs.read(fd,\ buffer,\ offset,\ length,\ position,\ callback)}}{fs.read(fd, buffer, offset, length, position, callback)}}\label{fs.readfd-buffer-offset-length-position-callback}

\begin{itemize}
\tightlist
\item
  \texttt{fd} \{integer\}
\item
  \texttt{buffer} \{Buffer\textbar TypedArray\textbar DataView\} The
  buffer that the data will be written to.
\item
  \texttt{offset} \{integer\} The position in \texttt{buffer} to write
  the data to.
\item
  \texttt{length} \{integer\} The number of bytes to read.
\item
  \texttt{position} \{integer\textbar bigint\textbar null\} Specifies
  where to begin reading from in the file. If \texttt{position} is
  \texttt{null} or \texttt{-1}, data will be read from the current file
  position, and the file position will be updated. If \texttt{position}
  is a non-negative integer, the file position will be unchanged.
\item
  \texttt{callback} \{Function\}

  \begin{itemize}
  \tightlist
  \item
    \texttt{err} \{Error\}
  \item
    \texttt{bytesRead} \{integer\}
  \item
    \texttt{buffer} \{Buffer\}
  \end{itemize}
\end{itemize}

Read data from the file specified by \texttt{fd}.

The callback is given the three arguments,
\texttt{(err,\ bytesRead,\ buffer)}.

If the file is not modified concurrently, the end-of-file is reached
when the number of bytes read is zero.

If this method is invoked as its
\href{util.md\#utilpromisifyoriginal}{\texttt{util.promisify()}}ed
version, it returns a promise for an \texttt{Object} with
\texttt{bytesRead} and \texttt{buffer} properties.

\subsubsection{\texorpdfstring{\texttt{fs.read(fd{[},\ options{]},\ callback)}}{fs.read(fd{[}, options{]}, callback)}}\label{fs.readfd-options-callback}

\begin{itemize}
\tightlist
\item
  \texttt{fd} \{integer\}
\item
  \texttt{options} \{Object\}

  \begin{itemize}
  \tightlist
  \item
    \texttt{buffer} \{Buffer\textbar TypedArray\textbar DataView\}
    \textbf{Default:} \texttt{Buffer.alloc(16384)}
  \item
    \texttt{offset} \{integer\} \textbf{Default:} \texttt{0}
  \item
    \texttt{length} \{integer\} \textbf{Default:}
    \texttt{buffer.byteLength\ -\ offset}
  \item
    \texttt{position} \{integer\textbar bigint\textbar null\}
    \textbf{Default:} \texttt{null}
  \end{itemize}
\item
  \texttt{callback} \{Function\}

  \begin{itemize}
  \tightlist
  \item
    \texttt{err} \{Error\}
  \item
    \texttt{bytesRead} \{integer\}
  \item
    \texttt{buffer} \{Buffer\}
  \end{itemize}
\end{itemize}

Similar to the
\hyperref[fsreadfd-buffer-offset-length-position-callback]{\texttt{fs.read()}}
function, this version takes an optional \texttt{options} object. If no
\texttt{options} object is specified, it will default with the above
values.

\subsubsection{\texorpdfstring{\texttt{fs.read(fd,\ buffer{[},\ options{]},\ callback)}}{fs.read(fd, buffer{[}, options{]}, callback)}}\label{fs.readfd-buffer-options-callback}

\begin{itemize}
\tightlist
\item
  \texttt{fd} \{integer\}
\item
  \texttt{buffer} \{Buffer\textbar TypedArray\textbar DataView\} The
  buffer that the data will be written to.
\item
  \texttt{options} \{Object\}

  \begin{itemize}
  \tightlist
  \item
    \texttt{offset} \{integer\} \textbf{Default:} \texttt{0}
  \item
    \texttt{length} \{integer\} \textbf{Default:}
    \texttt{buffer.byteLength\ -\ offset}
  \item
    \texttt{position} \{integer\textbar bigint\} \textbf{Default:}
    \texttt{null}
  \end{itemize}
\item
  \texttt{callback} \{Function\}

  \begin{itemize}
  \tightlist
  \item
    \texttt{err} \{Error\}
  \item
    \texttt{bytesRead} \{integer\}
  \item
    \texttt{buffer} \{Buffer\}
  \end{itemize}
\end{itemize}

Similar to the
\hyperref[fsreadfd-buffer-offset-length-position-callback]{\texttt{fs.read()}}
function, this version takes an optional \texttt{options} object. If no
\texttt{options} object is specified, it will default with the above
values.

\subsubsection{\texorpdfstring{\texttt{fs.readdir(path{[},\ options{]},\ callback)}}{fs.readdir(path{[}, options{]}, callback)}}\label{fs.readdirpath-options-callback}

\begin{itemize}
\tightlist
\item
  \texttt{path} \{string\textbar Buffer\textbar URL\}
\item
  \texttt{options} \{string\textbar Object\}

  \begin{itemize}
  \tightlist
  \item
    \texttt{encoding} \{string\} \textbf{Default:}
    \texttt{\textquotesingle{}utf8\textquotesingle{}}
  \item
    \texttt{withFileTypes} \{boolean\} \textbf{Default:} \texttt{false}
  \item
    \texttt{recursive} \{boolean\} If \texttt{true}, reads the contents
    of a directory recursively. In recursive mode, it will list all
    files, sub files and directories. \textbf{Default:} \texttt{false}.
  \end{itemize}
\item
  \texttt{callback} \{Function\}

  \begin{itemize}
  \tightlist
  \item
    \texttt{err} \{Error\}
  \item
    \texttt{files}
    \{string{[}{]}\textbar Buffer{[}{]}\textbar fs.Dirent{[}{]}\}
  \end{itemize}
\end{itemize}

Reads the contents of a directory. The callback gets two arguments
\texttt{(err,\ files)} where \texttt{files} is an array of the names of
the files in the directory excluding
\texttt{\textquotesingle{}.\textquotesingle{}} and
\texttt{\textquotesingle{}..\textquotesingle{}}.

See the POSIX readdir(3) documentation for more details.

The optional \texttt{options} argument can be a string specifying an
encoding, or an object with an \texttt{encoding} property specifying the
character encoding to use for the filenames passed to the callback. If
the \texttt{encoding} is set to
\texttt{\textquotesingle{}buffer\textquotesingle{}}, the filenames
returned will be passed as \{Buffer\} objects.

If \texttt{options.withFileTypes} is set to \texttt{true}, the
\texttt{files} array will contain \{fs.Dirent\} objects.

\subsubsection{\texorpdfstring{\texttt{fs.readFile(path{[},\ options{]},\ callback)}}{fs.readFile(path{[}, options{]}, callback)}}\label{fs.readfilepath-options-callback}

\begin{itemize}
\tightlist
\item
  \texttt{path} \{string\textbar Buffer\textbar URL\textbar integer\}
  filename or file descriptor
\item
  \texttt{options} \{Object\textbar string\}

  \begin{itemize}
  \tightlist
  \item
    \texttt{encoding} \{string\textbar null\} \textbf{Default:}
    \texttt{null}
  \item
    \texttt{flag} \{string\} See \hyperref[file-system-flags]{support of
    file system \texttt{flags}}. \textbf{Default:}
    \texttt{\textquotesingle{}r\textquotesingle{}}.
  \item
    \texttt{signal} \{AbortSignal\} allows aborting an in-progress
    readFile
  \end{itemize}
\item
  \texttt{callback} \{Function\}

  \begin{itemize}
  \tightlist
  \item
    \texttt{err} \{Error\textbar AggregateError\}
  \item
    \texttt{data} \{string\textbar Buffer\}
  \end{itemize}
\end{itemize}

Asynchronously reads the entire contents of a file.

\begin{Shaded}
\begin{Highlighting}[]
\ImportTok{import}\NormalTok{ \{ readFile \} }\ImportTok{from} \StringTok{\textquotesingle{}node:fs\textquotesingle{}}\OperatorTok{;}

\FunctionTok{readFile}\NormalTok{(}\StringTok{\textquotesingle{}/etc/passwd\textquotesingle{}}\OperatorTok{,}\NormalTok{ (err}\OperatorTok{,}\NormalTok{ data) }\KeywordTok{=\textgreater{}}\NormalTok{ \{}
  \ControlFlowTok{if}\NormalTok{ (err) }\ControlFlowTok{throw}\NormalTok{ err}\OperatorTok{;}
  \BuiltInTok{console}\OperatorTok{.}\FunctionTok{log}\NormalTok{(data)}\OperatorTok{;}
\NormalTok{\})}\OperatorTok{;}
\end{Highlighting}
\end{Shaded}

The callback is passed two arguments \texttt{(err,\ data)}, where
\texttt{data} is the contents of the file.

If no encoding is specified, then the raw buffer is returned.

If \texttt{options} is a string, then it specifies the encoding:

\begin{Shaded}
\begin{Highlighting}[]
\ImportTok{import}\NormalTok{ \{ readFile \} }\ImportTok{from} \StringTok{\textquotesingle{}node:fs\textquotesingle{}}\OperatorTok{;}

\FunctionTok{readFile}\NormalTok{(}\StringTok{\textquotesingle{}/etc/passwd\textquotesingle{}}\OperatorTok{,} \StringTok{\textquotesingle{}utf8\textquotesingle{}}\OperatorTok{,}\NormalTok{ callback)}\OperatorTok{;}
\end{Highlighting}
\end{Shaded}

When the path is a directory, the behavior of \texttt{fs.readFile()} and
\hyperref[fsreadfilesyncpath-options]{\texttt{fs.readFileSync()}} is
platform-specific. On macOS, Linux, and Windows, an error will be
returned. On FreeBSD, a representation of the directory's contents will
be returned.

\begin{Shaded}
\begin{Highlighting}[]
\ImportTok{import}\NormalTok{ \{ readFile \} }\ImportTok{from} \StringTok{\textquotesingle{}node:fs\textquotesingle{}}\OperatorTok{;}

\CommentTok{// macOS, Linux, and Windows}
\FunctionTok{readFile}\NormalTok{(}\StringTok{\textquotesingle{}\textless{}directory\textgreater{}\textquotesingle{}}\OperatorTok{,}\NormalTok{ (err}\OperatorTok{,}\NormalTok{ data) }\KeywordTok{=\textgreater{}}\NormalTok{ \{}
  \CommentTok{// =\textgreater{} [Error: EISDIR: illegal operation on a directory, read \textless{}directory\textgreater{}]}
\NormalTok{\})}\OperatorTok{;}

\CommentTok{//  FreeBSD}
\FunctionTok{readFile}\NormalTok{(}\StringTok{\textquotesingle{}\textless{}directory\textgreater{}\textquotesingle{}}\OperatorTok{,}\NormalTok{ (err}\OperatorTok{,}\NormalTok{ data) }\KeywordTok{=\textgreater{}}\NormalTok{ \{}
  \CommentTok{// =\textgreater{} null, \textless{}data\textgreater{}}
\NormalTok{\})}\OperatorTok{;}
\end{Highlighting}
\end{Shaded}

It is possible to abort an ongoing request using an
\texttt{AbortSignal}. If a request is aborted the callback is called
with an \texttt{AbortError}:

\begin{Shaded}
\begin{Highlighting}[]
\ImportTok{import}\NormalTok{ \{ readFile \} }\ImportTok{from} \StringTok{\textquotesingle{}node:fs\textquotesingle{}}\OperatorTok{;}

\KeywordTok{const}\NormalTok{ controller }\OperatorTok{=} \KeywordTok{new} \FunctionTok{AbortController}\NormalTok{()}\OperatorTok{;}
\KeywordTok{const}\NormalTok{ signal }\OperatorTok{=}\NormalTok{ controller}\OperatorTok{.}\AttributeTok{signal}\OperatorTok{;}
\FunctionTok{readFile}\NormalTok{(fileInfo[}\DecValTok{0}\NormalTok{]}\OperatorTok{.}\AttributeTok{name}\OperatorTok{,}\NormalTok{ \{ signal \}}\OperatorTok{,}\NormalTok{ (err}\OperatorTok{,}\NormalTok{ buf) }\KeywordTok{=\textgreater{}}\NormalTok{ \{}
  \CommentTok{// ...}
\NormalTok{\})}\OperatorTok{;}
\CommentTok{// When you want to abort the request}
\NormalTok{controller}\OperatorTok{.}\FunctionTok{abort}\NormalTok{()}\OperatorTok{;}
\end{Highlighting}
\end{Shaded}

The \texttt{fs.readFile()} function buffers the entire file. To minimize
memory costs, when possible prefer streaming via
\texttt{fs.createReadStream()}.

Aborting an ongoing request does not abort individual operating system
requests but rather the internal buffering \texttt{fs.readFile}
performs.

\paragraph{File descriptors}\label{file-descriptors}

\begin{enumerate}
\def\labelenumi{\arabic{enumi}.}
\tightlist
\item
  Any specified file descriptor has to support reading.
\item
  If a file descriptor is specified as the \texttt{path}, it will not be
  closed automatically.
\item
  The reading will begin at the current position. For example, if the
  file already had
  \texttt{\textquotesingle{}Hello\ World\textquotesingle{}} and six
  bytes are read with the file descriptor, the call to
  \texttt{fs.readFile()} with the same file descriptor, would give
  \texttt{\textquotesingle{}World\textquotesingle{}}, rather than
  \texttt{\textquotesingle{}Hello\ World\textquotesingle{}}.
\end{enumerate}

\paragraph{Performance Considerations}\label{performance-considerations}

The \texttt{fs.readFile()} method asynchronously reads the contents of a
file into memory one chunk at a time, allowing the event loop to turn
between each chunk. This allows the read operation to have less impact
on other activity that may be using the underlying libuv thread pool but
means that it will take longer to read a complete file into memory.

The additional read overhead can vary broadly on different systems and
depends on the type of file being read. If the file type is not a
regular file (a pipe for instance) and Node.js is unable to determine an
actual file size, each read operation will load on 64 KiB of data. For
regular files, each read will process 512 KiB of data.

For applications that require as-fast-as-possible reading of file
contents, it is better to use \texttt{fs.read()} directly and for
application code to manage reading the full contents of the file itself.

The Node.js GitHub issue
\href{https://github.com/nodejs/node/issues/25741}{\#25741} provides
more information and a detailed analysis on the performance of
\texttt{fs.readFile()} for multiple file sizes in different Node.js
versions.

\subsubsection{\texorpdfstring{\texttt{fs.readlink(path{[},\ options{]},\ callback)}}{fs.readlink(path{[}, options{]}, callback)}}\label{fs.readlinkpath-options-callback}

\begin{itemize}
\tightlist
\item
  \texttt{path} \{string\textbar Buffer\textbar URL\}
\item
  \texttt{options} \{string\textbar Object\}

  \begin{itemize}
  \tightlist
  \item
    \texttt{encoding} \{string\} \textbf{Default:}
    \texttt{\textquotesingle{}utf8\textquotesingle{}}
  \end{itemize}
\item
  \texttt{callback} \{Function\}

  \begin{itemize}
  \tightlist
  \item
    \texttt{err} \{Error\}
  \item
    \texttt{linkString} \{string\textbar Buffer\}
  \end{itemize}
\end{itemize}

Reads the contents of the symbolic link referred to by \texttt{path}.
The callback gets two arguments \texttt{(err,\ linkString)}.

See the POSIX readlink(2) documentation for more details.

The optional \texttt{options} argument can be a string specifying an
encoding, or an object with an \texttt{encoding} property specifying the
character encoding to use for the link path passed to the callback. If
the \texttt{encoding} is set to
\texttt{\textquotesingle{}buffer\textquotesingle{}}, the link path
returned will be passed as a \{Buffer\} object.

\subsubsection{\texorpdfstring{\texttt{fs.readv(fd,\ buffers{[},\ position{]},\ callback)}}{fs.readv(fd, buffers{[}, position{]}, callback)}}\label{fs.readvfd-buffers-position-callback}

\begin{itemize}
\tightlist
\item
  \texttt{fd} \{integer\}
\item
  \texttt{buffers} \{ArrayBufferView{[}{]}\}
\item
  \texttt{position} \{integer\textbar null\} \textbf{Default:}
  \texttt{null}
\item
  \texttt{callback} \{Function\}

  \begin{itemize}
  \tightlist
  \item
    \texttt{err} \{Error\}
  \item
    \texttt{bytesRead} \{integer\}
  \item
    \texttt{buffers} \{ArrayBufferView{[}{]}\}
  \end{itemize}
\end{itemize}

Read from a file specified by \texttt{fd} and write to an array of
\texttt{ArrayBufferView}s using \texttt{readv()}.

\texttt{position} is the offset from the beginning of the file from
where data should be read. If
\texttt{typeof\ position\ !==\ \textquotesingle{}number\textquotesingle{}},
the data will be read from the current position.

The callback will be given three arguments: \texttt{err},
\texttt{bytesRead}, and \texttt{buffers}. \texttt{bytesRead} is how many
bytes were read from the file.

If this method is invoked as its
\href{util.md\#utilpromisifyoriginal}{\texttt{util.promisify()}}ed
version, it returns a promise for an \texttt{Object} with
\texttt{bytesRead} and \texttt{buffers} properties.

\subsubsection{\texorpdfstring{\texttt{fs.realpath(path{[},\ options{]},\ callback)}}{fs.realpath(path{[}, options{]}, callback)}}\label{fs.realpathpath-options-callback}

\begin{itemize}
\tightlist
\item
  \texttt{path} \{string\textbar Buffer\textbar URL\}
\item
  \texttt{options} \{string\textbar Object\}

  \begin{itemize}
  \tightlist
  \item
    \texttt{encoding} \{string\} \textbf{Default:}
    \texttt{\textquotesingle{}utf8\textquotesingle{}}
  \end{itemize}
\item
  \texttt{callback} \{Function\}

  \begin{itemize}
  \tightlist
  \item
    \texttt{err} \{Error\}
  \item
    \texttt{resolvedPath} \{string\textbar Buffer\}
  \end{itemize}
\end{itemize}

Asynchronously computes the canonical pathname by resolving \texttt{.},
\texttt{..}, and symbolic links.

A canonical pathname is not necessarily unique. Hard links and bind
mounts can expose a file system entity through many pathnames.

This function behaves like realpath(3), with some exceptions:

\begin{enumerate}
\def\labelenumi{\arabic{enumi}.}
\item
  No case conversion is performed on case-insensitive file systems.
\item
  The maximum number of symbolic links is platform-independent and
  generally (much) higher than what the native realpath(3)
  implementation supports.
\end{enumerate}

The \texttt{callback} gets two arguments \texttt{(err,\ resolvedPath)}.
May use \texttt{process.cwd} to resolve relative paths.

Only paths that can be converted to UTF8 strings are supported.

The optional \texttt{options} argument can be a string specifying an
encoding, or an object with an \texttt{encoding} property specifying the
character encoding to use for the path passed to the callback. If the
\texttt{encoding} is set to
\texttt{\textquotesingle{}buffer\textquotesingle{}}, the path returned
will be passed as a \{Buffer\} object.

If \texttt{path} resolves to a socket or a pipe, the function will
return a system dependent name for that object.

\subsubsection{\texorpdfstring{\texttt{fs.realpath.native(path{[},\ options{]},\ callback)}}{fs.realpath.native(path{[}, options{]}, callback)}}\label{fs.realpath.nativepath-options-callback}

\begin{itemize}
\tightlist
\item
  \texttt{path} \{string\textbar Buffer\textbar URL\}
\item
  \texttt{options} \{string\textbar Object\}

  \begin{itemize}
  \tightlist
  \item
    \texttt{encoding} \{string\} \textbf{Default:}
    \texttt{\textquotesingle{}utf8\textquotesingle{}}
  \end{itemize}
\item
  \texttt{callback} \{Function\}

  \begin{itemize}
  \tightlist
  \item
    \texttt{err} \{Error\}
  \item
    \texttt{resolvedPath} \{string\textbar Buffer\}
  \end{itemize}
\end{itemize}

Asynchronous realpath(3).

The \texttt{callback} gets two arguments \texttt{(err,\ resolvedPath)}.

Only paths that can be converted to UTF8 strings are supported.

The optional \texttt{options} argument can be a string specifying an
encoding, or an object with an \texttt{encoding} property specifying the
character encoding to use for the path passed to the callback. If the
\texttt{encoding} is set to
\texttt{\textquotesingle{}buffer\textquotesingle{}}, the path returned
will be passed as a \{Buffer\} object.

On Linux, when Node.js is linked against musl libc, the procfs file
system must be mounted on \texttt{/proc} in order for this function to
work. Glibc does not have this restriction.

\subsubsection{\texorpdfstring{\texttt{fs.rename(oldPath,\ newPath,\ callback)}}{fs.rename(oldPath, newPath, callback)}}\label{fs.renameoldpath-newpath-callback}

\begin{itemize}
\tightlist
\item
  \texttt{oldPath} \{string\textbar Buffer\textbar URL\}
\item
  \texttt{newPath} \{string\textbar Buffer\textbar URL\}
\item
  \texttt{callback} \{Function\}

  \begin{itemize}
  \tightlist
  \item
    \texttt{err} \{Error\}
  \end{itemize}
\end{itemize}

Asynchronously rename file at \texttt{oldPath} to the pathname provided
as \texttt{newPath}. In the case that \texttt{newPath} already exists,
it will be overwritten. If there is a directory at \texttt{newPath}, an
error will be raised instead. No arguments other than a possible
exception are given to the completion callback.

See also: rename(2).

\begin{Shaded}
\begin{Highlighting}[]
\ImportTok{import}\NormalTok{ \{ rename \} }\ImportTok{from} \StringTok{\textquotesingle{}node:fs\textquotesingle{}}\OperatorTok{;}

\FunctionTok{rename}\NormalTok{(}\StringTok{\textquotesingle{}oldFile.txt\textquotesingle{}}\OperatorTok{,} \StringTok{\textquotesingle{}newFile.txt\textquotesingle{}}\OperatorTok{,}\NormalTok{ (err) }\KeywordTok{=\textgreater{}}\NormalTok{ \{}
  \ControlFlowTok{if}\NormalTok{ (err) }\ControlFlowTok{throw}\NormalTok{ err}\OperatorTok{;}
  \BuiltInTok{console}\OperatorTok{.}\FunctionTok{log}\NormalTok{(}\StringTok{\textquotesingle{}Rename complete!\textquotesingle{}}\NormalTok{)}\OperatorTok{;}
\NormalTok{\})}\OperatorTok{;}
\end{Highlighting}
\end{Shaded}

\subsubsection{\texorpdfstring{\texttt{fs.rmdir(path{[},\ options{]},\ callback)}}{fs.rmdir(path{[}, options{]}, callback)}}\label{fs.rmdirpath-options-callback}

\begin{itemize}
\tightlist
\item
  \texttt{path} \{string\textbar Buffer\textbar URL\}
\item
  \texttt{options} \{Object\}

  \begin{itemize}
  \tightlist
  \item
    \texttt{maxRetries} \{integer\} If an \texttt{EBUSY},
    \texttt{EMFILE}, \texttt{ENFILE}, \texttt{ENOTEMPTY}, or
    \texttt{EPERM} error is encountered, Node.js retries the operation
    with a linear backoff wait of \texttt{retryDelay} milliseconds
    longer on each try. This option represents the number of retries.
    This option is ignored if the \texttt{recursive} option is not
    \texttt{true}. \textbf{Default:} \texttt{0}.
  \item
    \texttt{recursive} \{boolean\} If \texttt{true}, perform a recursive
    directory removal. In recursive mode, operations are retried on
    failure. \textbf{Default:} \texttt{false}. \textbf{Deprecated.}
  \item
    \texttt{retryDelay} \{integer\} The amount of time in milliseconds
    to wait between retries. This option is ignored if the
    \texttt{recursive} option is not \texttt{true}. \textbf{Default:}
    \texttt{100}.
  \end{itemize}
\item
  \texttt{callback} \{Function\}

  \begin{itemize}
  \tightlist
  \item
    \texttt{err} \{Error\}
  \end{itemize}
\end{itemize}

Asynchronous rmdir(2). No arguments other than a possible exception are
given to the completion callback.

Using \texttt{fs.rmdir()} on a file (not a directory) results in an
\texttt{ENOENT} error on Windows and an \texttt{ENOTDIR} error on POSIX.

To get a behavior similar to the \texttt{rm\ -rf} Unix command, use
\hyperref[fsrmpath-options-callback]{\texttt{fs.rm()}} with options
\texttt{\{\ recursive:\ true,\ force:\ true\ \}}.

\subsubsection{\texorpdfstring{\texttt{fs.rm(path{[},\ options{]},\ callback)}}{fs.rm(path{[}, options{]}, callback)}}\label{fs.rmpath-options-callback}

\begin{itemize}
\tightlist
\item
  \texttt{path} \{string\textbar Buffer\textbar URL\}
\item
  \texttt{options} \{Object\}

  \begin{itemize}
  \tightlist
  \item
    \texttt{force} \{boolean\} When \texttt{true}, exceptions will be
    ignored if \texttt{path} does not exist. \textbf{Default:}
    \texttt{false}.
  \item
    \texttt{maxRetries} \{integer\} If an \texttt{EBUSY},
    \texttt{EMFILE}, \texttt{ENFILE}, \texttt{ENOTEMPTY}, or
    \texttt{EPERM} error is encountered, Node.js will retry the
    operation with a linear backoff wait of \texttt{retryDelay}
    milliseconds longer on each try. This option represents the number
    of retries. This option is ignored if the \texttt{recursive} option
    is not \texttt{true}. \textbf{Default:} \texttt{0}.
  \item
    \texttt{recursive} \{boolean\} If \texttt{true}, perform a recursive
    removal. In recursive mode operations are retried on failure.
    \textbf{Default:} \texttt{false}.
  \item
    \texttt{retryDelay} \{integer\} The amount of time in milliseconds
    to wait between retries. This option is ignored if the
    \texttt{recursive} option is not \texttt{true}. \textbf{Default:}
    \texttt{100}.
  \end{itemize}
\item
  \texttt{callback} \{Function\}

  \begin{itemize}
  \tightlist
  \item
    \texttt{err} \{Error\}
  \end{itemize}
\end{itemize}

Asynchronously removes files and directories (modeled on the standard
POSIX \texttt{rm} utility). No arguments other than a possible exception
are given to the completion callback.

\subsubsection{\texorpdfstring{\texttt{fs.stat(path{[},\ options{]},\ callback)}}{fs.stat(path{[}, options{]}, callback)}}\label{fs.statpath-options-callback}

\begin{itemize}
\tightlist
\item
  \texttt{path} \{string\textbar Buffer\textbar URL\}
\item
  \texttt{options} \{Object\}

  \begin{itemize}
  \tightlist
  \item
    \texttt{bigint} \{boolean\} Whether the numeric values in the
    returned \{fs.Stats\} object should be \texttt{bigint}.
    \textbf{Default:} \texttt{false}.
  \end{itemize}
\item
  \texttt{callback} \{Function\}

  \begin{itemize}
  \tightlist
  \item
    \texttt{err} \{Error\}
  \item
    \texttt{stats} \{fs.Stats\}
  \end{itemize}
\end{itemize}

Asynchronous stat(2). The callback gets two arguments
\texttt{(err,\ stats)} where \texttt{stats} is an \{fs.Stats\} object.

In case of an error, the \texttt{err.code} will be one of
\href{errors.md\#common-system-errors}{Common System Errors}.

\hyperref[fsstatpath-options-callback]{\texttt{fs.stat()}} follows
symbolic links. Use
\hyperref[fslstatpath-options-callback]{\texttt{fs.lstat()}} to look at
the links themselves.

Using \texttt{fs.stat()} to check for the existence of a file before
calling \texttt{fs.open()}, \texttt{fs.readFile()}, or
\texttt{fs.writeFile()} is not recommended. Instead, user code should
open/read/write the file directly and handle the error raised if the
file is not available.

To check if a file exists without manipulating it afterwards,
\hyperref[fsaccesspath-mode-callback]{\texttt{fs.access()}} is
recommended.

For example, given the following directory structure:

\begin{Shaded}
\begin{Highlighting}[]
\NormalTok{{-} txtDir}
\NormalTok{{-}{-} file.txt}
\NormalTok{{-} app.js}
\end{Highlighting}
\end{Shaded}

The next program will check for the stats of the given paths:

\begin{Shaded}
\begin{Highlighting}[]
\ImportTok{import}\NormalTok{ \{ stat \} }\ImportTok{from} \StringTok{\textquotesingle{}node:fs\textquotesingle{}}\OperatorTok{;}

\KeywordTok{const}\NormalTok{ pathsToCheck }\OperatorTok{=}\NormalTok{ [}\StringTok{\textquotesingle{}./txtDir\textquotesingle{}}\OperatorTok{,} \StringTok{\textquotesingle{}./txtDir/file.txt\textquotesingle{}}\NormalTok{]}\OperatorTok{;}

\ControlFlowTok{for}\NormalTok{ (}\KeywordTok{let}\NormalTok{ i }\OperatorTok{=} \DecValTok{0}\OperatorTok{;}\NormalTok{ i }\OperatorTok{\textless{}}\NormalTok{ pathsToCheck}\OperatorTok{.}\AttributeTok{length}\OperatorTok{;}\NormalTok{ i}\OperatorTok{++}\NormalTok{) \{}
  \FunctionTok{stat}\NormalTok{(pathsToCheck[i]}\OperatorTok{,}\NormalTok{ (err}\OperatorTok{,}\NormalTok{ stats) }\KeywordTok{=\textgreater{}}\NormalTok{ \{}
    \BuiltInTok{console}\OperatorTok{.}\FunctionTok{log}\NormalTok{(stats}\OperatorTok{.}\FunctionTok{isDirectory}\NormalTok{())}\OperatorTok{;}
    \BuiltInTok{console}\OperatorTok{.}\FunctionTok{log}\NormalTok{(stats)}\OperatorTok{;}
\NormalTok{  \})}\OperatorTok{;}
\NormalTok{\}}
\end{Highlighting}
\end{Shaded}

The resulting output will resemble:

\begin{Shaded}
\begin{Highlighting}[]
\NormalTok{true}
\NormalTok{Stats \{}
\NormalTok{  dev: 16777220,}
\NormalTok{  mode: 16877,}
\NormalTok{  nlink: 3,}
\NormalTok{  uid: 501,}
\NormalTok{  gid: 20,}
\NormalTok{  rdev: 0,}
\NormalTok{  blksize: 4096,}
\NormalTok{  ino: 14214262,}
\NormalTok{  size: 96,}
\NormalTok{  blocks: 0,}
\NormalTok{  atimeMs: 1561174653071.963,}
\NormalTok{  mtimeMs: 1561174614583.3518,}
\NormalTok{  ctimeMs: 1561174626623.5366,}
\NormalTok{  birthtimeMs: 1561174126937.2893,}
\NormalTok{  atime: 2019{-}06{-}22T03:37:33.072Z,}
\NormalTok{  mtime: 2019{-}06{-}22T03:36:54.583Z,}
\NormalTok{  ctime: 2019{-}06{-}22T03:37:06.624Z,}
\NormalTok{  birthtime: 2019{-}06{-}22T03:28:46.937Z}
\NormalTok{\}}
\NormalTok{false}
\NormalTok{Stats \{}
\NormalTok{  dev: 16777220,}
\NormalTok{  mode: 33188,}
\NormalTok{  nlink: 1,}
\NormalTok{  uid: 501,}
\NormalTok{  gid: 20,}
\NormalTok{  rdev: 0,}
\NormalTok{  blksize: 4096,}
\NormalTok{  ino: 14214074,}
\NormalTok{  size: 8,}
\NormalTok{  blocks: 8,}
\NormalTok{  atimeMs: 1561174616618.8555,}
\NormalTok{  mtimeMs: 1561174614584,}
\NormalTok{  ctimeMs: 1561174614583.8145,}
\NormalTok{  birthtimeMs: 1561174007710.7478,}
\NormalTok{  atime: 2019{-}06{-}22T03:36:56.619Z,}
\NormalTok{  mtime: 2019{-}06{-}22T03:36:54.584Z,}
\NormalTok{  ctime: 2019{-}06{-}22T03:36:54.584Z,}
\NormalTok{  birthtime: 2019{-}06{-}22T03:26:47.711Z}
\NormalTok{\}}
\end{Highlighting}
\end{Shaded}

\subsubsection{\texorpdfstring{\texttt{fs.statfs(path{[},\ options{]},\ callback)}}{fs.statfs(path{[}, options{]}, callback)}}\label{fs.statfspath-options-callback}

\begin{itemize}
\tightlist
\item
  \texttt{path} \{string\textbar Buffer\textbar URL\}
\item
  \texttt{options} \{Object\}

  \begin{itemize}
  \tightlist
  \item
    \texttt{bigint} \{boolean\} Whether the numeric values in the
    returned \{fs.StatFs\} object should be \texttt{bigint}.
    \textbf{Default:} \texttt{false}.
  \end{itemize}
\item
  \texttt{callback} \{Function\}

  \begin{itemize}
  \tightlist
  \item
    \texttt{err} \{Error\}
  \item
    \texttt{stats} \{fs.StatFs\}
  \end{itemize}
\end{itemize}

Asynchronous statfs(2). Returns information about the mounted file
system which contains \texttt{path}. The callback gets two arguments
\texttt{(err,\ stats)} where \texttt{stats} is an \{fs.StatFs\} object.

In case of an error, the \texttt{err.code} will be one of
\href{errors.md\#common-system-errors}{Common System Errors}.

\subsubsection{\texorpdfstring{\texttt{fs.symlink(target,\ path{[},\ type{]},\ callback)}}{fs.symlink(target, path{[}, type{]}, callback)}}\label{fs.symlinktarget-path-type-callback}

\begin{itemize}
\tightlist
\item
  \texttt{target} \{string\textbar Buffer\textbar URL\}
\item
  \texttt{path} \{string\textbar Buffer\textbar URL\}
\item
  \texttt{type} \{string\textbar null\} \textbf{Default:} \texttt{null}
\item
  \texttt{callback} \{Function\}

  \begin{itemize}
  \tightlist
  \item
    \texttt{err} \{Error\}
  \end{itemize}
\end{itemize}

Creates the link called \texttt{path} pointing to \texttt{target}. No
arguments other than a possible exception are given to the completion
callback.

See the POSIX symlink(2) documentation for more details.

The \texttt{type} argument is only available on Windows and ignored on
other platforms. It can be set to
\texttt{\textquotesingle{}dir\textquotesingle{}},
\texttt{\textquotesingle{}file\textquotesingle{}}, or
\texttt{\textquotesingle{}junction\textquotesingle{}}. If the
\texttt{type} argument is not a string, Node.js will autodetect
\texttt{target} type and use
\texttt{\textquotesingle{}file\textquotesingle{}} or
\texttt{\textquotesingle{}dir\textquotesingle{}}. If the \texttt{target}
does not exist, \texttt{\textquotesingle{}file\textquotesingle{}} will
be used. Windows junction points require the destination path to be
absolute. When using
\texttt{\textquotesingle{}junction\textquotesingle{}}, the
\texttt{target} argument will automatically be normalized to absolute
path. Junction points on NTFS volumes can only point to directories.

Relative targets are relative to the link's parent directory.

\begin{Shaded}
\begin{Highlighting}[]
\ImportTok{import}\NormalTok{ \{ symlink \} }\ImportTok{from} \StringTok{\textquotesingle{}node:fs\textquotesingle{}}\OperatorTok{;}

\FunctionTok{symlink}\NormalTok{(}\StringTok{\textquotesingle{}./mew\textquotesingle{}}\OperatorTok{,} \StringTok{\textquotesingle{}./mewtwo\textquotesingle{}}\OperatorTok{,}\NormalTok{ callback)}\OperatorTok{;}
\end{Highlighting}
\end{Shaded}

The above example creates a symbolic link \texttt{mewtwo} which points
to \texttt{mew} in the same directory:

\begin{Shaded}
\begin{Highlighting}[]
\ExtensionTok{$}\NormalTok{ tree .}
\BuiltInTok{.}
\ExtensionTok{├──}\NormalTok{ mew}
\ExtensionTok{└──}\NormalTok{ mewtwo }\AttributeTok{{-}}\OperatorTok{\textgreater{}}\NormalTok{ ./mew}
\end{Highlighting}
\end{Shaded}

\subsubsection{\texorpdfstring{\texttt{fs.truncate(path{[},\ len{]},\ callback)}}{fs.truncate(path{[}, len{]}, callback)}}\label{fs.truncatepath-len-callback}

\begin{itemize}
\tightlist
\item
  \texttt{path} \{string\textbar Buffer\textbar URL\}
\item
  \texttt{len} \{integer\} \textbf{Default:} \texttt{0}
\item
  \texttt{callback} \{Function\}

  \begin{itemize}
  \tightlist
  \item
    \texttt{err} \{Error\textbar AggregateError\}
  \end{itemize}
\end{itemize}

Truncates the file. No arguments other than a possible exception are
given to the completion callback. A file descriptor can also be passed
as the first argument. In this case, \texttt{fs.ftruncate()} is called.

\begin{Shaded}
\begin{Highlighting}[]
\ImportTok{import}\NormalTok{ \{ truncate \} }\ImportTok{from} \StringTok{\textquotesingle{}node:fs\textquotesingle{}}\OperatorTok{;}
\CommentTok{// Assuming that \textquotesingle{}path/file.txt\textquotesingle{} is a regular file.}
\FunctionTok{truncate}\NormalTok{(}\StringTok{\textquotesingle{}path/file.txt\textquotesingle{}}\OperatorTok{,}\NormalTok{ (err) }\KeywordTok{=\textgreater{}}\NormalTok{ \{}
  \ControlFlowTok{if}\NormalTok{ (err) }\ControlFlowTok{throw}\NormalTok{ err}\OperatorTok{;}
  \BuiltInTok{console}\OperatorTok{.}\FunctionTok{log}\NormalTok{(}\StringTok{\textquotesingle{}path/file.txt was truncated\textquotesingle{}}\NormalTok{)}\OperatorTok{;}
\NormalTok{\})}\OperatorTok{;}
\end{Highlighting}
\end{Shaded}

\begin{Shaded}
\begin{Highlighting}[]
\KeywordTok{const}\NormalTok{ \{ truncate \} }\OperatorTok{=} \PreprocessorTok{require}\NormalTok{(}\StringTok{\textquotesingle{}node:fs\textquotesingle{}}\NormalTok{)}\OperatorTok{;}
\CommentTok{// Assuming that \textquotesingle{}path/file.txt\textquotesingle{} is a regular file.}
\FunctionTok{truncate}\NormalTok{(}\StringTok{\textquotesingle{}path/file.txt\textquotesingle{}}\OperatorTok{,}\NormalTok{ (err) }\KeywordTok{=\textgreater{}}\NormalTok{ \{}
  \ControlFlowTok{if}\NormalTok{ (err) }\ControlFlowTok{throw}\NormalTok{ err}\OperatorTok{;}
  \BuiltInTok{console}\OperatorTok{.}\FunctionTok{log}\NormalTok{(}\StringTok{\textquotesingle{}path/file.txt was truncated\textquotesingle{}}\NormalTok{)}\OperatorTok{;}
\NormalTok{\})}\OperatorTok{;}
\end{Highlighting}
\end{Shaded}

Passing a file descriptor is deprecated and may result in an error being
thrown in the future.

See the POSIX truncate(2) documentation for more details.

\subsubsection{\texorpdfstring{\texttt{fs.unlink(path,\ callback)}}{fs.unlink(path, callback)}}\label{fs.unlinkpath-callback}

\begin{itemize}
\tightlist
\item
  \texttt{path} \{string\textbar Buffer\textbar URL\}
\item
  \texttt{callback} \{Function\}

  \begin{itemize}
  \tightlist
  \item
    \texttt{err} \{Error\}
  \end{itemize}
\end{itemize}

Asynchronously removes a file or symbolic link. No arguments other than
a possible exception are given to the completion callback.

\begin{Shaded}
\begin{Highlighting}[]
\ImportTok{import}\NormalTok{ \{ unlink \} }\ImportTok{from} \StringTok{\textquotesingle{}node:fs\textquotesingle{}}\OperatorTok{;}
\CommentTok{// Assuming that \textquotesingle{}path/file.txt\textquotesingle{} is a regular file.}
\FunctionTok{unlink}\NormalTok{(}\StringTok{\textquotesingle{}path/file.txt\textquotesingle{}}\OperatorTok{,}\NormalTok{ (err) }\KeywordTok{=\textgreater{}}\NormalTok{ \{}
  \ControlFlowTok{if}\NormalTok{ (err) }\ControlFlowTok{throw}\NormalTok{ err}\OperatorTok{;}
  \BuiltInTok{console}\OperatorTok{.}\FunctionTok{log}\NormalTok{(}\StringTok{\textquotesingle{}path/file.txt was deleted\textquotesingle{}}\NormalTok{)}\OperatorTok{;}
\NormalTok{\})}\OperatorTok{;}
\end{Highlighting}
\end{Shaded}

\texttt{fs.unlink()} will not work on a directory, empty or otherwise.
To remove a directory, use
\hyperref[fsrmdirpath-options-callback]{\texttt{fs.rmdir()}}.

See the POSIX unlink(2) documentation for more details.

\subsubsection{\texorpdfstring{\texttt{fs.unwatchFile(filename{[},\ listener{]})}}{fs.unwatchFile(filename{[}, listener{]})}}\label{fs.unwatchfilefilename-listener}

\begin{itemize}
\tightlist
\item
  \texttt{filename} \{string\textbar Buffer\textbar URL\}
\item
  \texttt{listener} \{Function\} Optional, a listener previously
  attached using \texttt{fs.watchFile()}
\end{itemize}

Stop watching for changes on \texttt{filename}. If \texttt{listener} is
specified, only that particular listener is removed. Otherwise,
\emph{all} listeners are removed, effectively stopping watching of
\texttt{filename}.

Calling \texttt{fs.unwatchFile()} with a filename that is not being
watched is a no-op, not an error.

Using \hyperref[fswatchfilename-options-listener]{\texttt{fs.watch()}}
is more efficient than \texttt{fs.watchFile()} and
\texttt{fs.unwatchFile()}. \texttt{fs.watch()} should be used instead of
\texttt{fs.watchFile()} and \texttt{fs.unwatchFile()} when possible.

\subsubsection{\texorpdfstring{\texttt{fs.utimes(path,\ atime,\ mtime,\ callback)}}{fs.utimes(path, atime, mtime, callback)}}\label{fs.utimespath-atime-mtime-callback}

\begin{itemize}
\tightlist
\item
  \texttt{path} \{string\textbar Buffer\textbar URL\}
\item
  \texttt{atime} \{number\textbar string\textbar Date\}
\item
  \texttt{mtime} \{number\textbar string\textbar Date\}
\item
  \texttt{callback} \{Function\}

  \begin{itemize}
  \tightlist
  \item
    \texttt{err} \{Error\}
  \end{itemize}
\end{itemize}

Change the file system timestamps of the object referenced by
\texttt{path}.

The \texttt{atime} and \texttt{mtime} arguments follow these rules:

\begin{itemize}
\tightlist
\item
  Values can be either numbers representing Unix epoch time in seconds,
  \texttt{Date}s, or a numeric string like
  \texttt{\textquotesingle{}123456789.0\textquotesingle{}}.
\item
  If the value can not be converted to a number, or is \texttt{NaN},
  \texttt{Infinity}, or \texttt{-Infinity}, an \texttt{Error} will be
  thrown.
\end{itemize}

\subsubsection{\texorpdfstring{\texttt{fs.watch(filename{[},\ options{]}{[},\ listener{]})}}{fs.watch(filename{[}, options{]}{[}, listener{]})}}\label{fs.watchfilename-options-listener}

\begin{itemize}
\tightlist
\item
  \texttt{filename} \{string\textbar Buffer\textbar URL\}
\item
  \texttt{options} \{string\textbar Object\}

  \begin{itemize}
  \tightlist
  \item
    \texttt{persistent} \{boolean\} Indicates whether the process should
    continue to run as long as files are being watched.
    \textbf{Default:} \texttt{true}.
  \item
    \texttt{recursive} \{boolean\} Indicates whether all subdirectories
    should be watched, or only the current directory. This applies when
    a directory is specified, and only on supported platforms (See
    \hyperref[caveats]{caveats}). \textbf{Default:} \texttt{false}.
  \item
    \texttt{encoding} \{string\} Specifies the character encoding to be
    used for the filename passed to the listener. \textbf{Default:}
    \texttt{\textquotesingle{}utf8\textquotesingle{}}.
  \item
    \texttt{signal} \{AbortSignal\} allows closing the watcher with an
    AbortSignal.
  \end{itemize}
\item
  \texttt{listener} \{Function\textbar undefined\} \textbf{Default:}
  \texttt{undefined}

  \begin{itemize}
  \tightlist
  \item
    \texttt{eventType} \{string\}
  \item
    \texttt{filename} \{string\textbar Buffer\textbar null\}
  \end{itemize}
\item
  Returns: \{fs.FSWatcher\}
\end{itemize}

Watch for changes on \texttt{filename}, where \texttt{filename} is
either a file or a directory.

The second argument is optional. If \texttt{options} is provided as a
string, it specifies the \texttt{encoding}. Otherwise \texttt{options}
should be passed as an object.

The listener callback gets two arguments
\texttt{(eventType,\ filename)}. \texttt{eventType} is either
\texttt{\textquotesingle{}rename\textquotesingle{}} or
\texttt{\textquotesingle{}change\textquotesingle{}}, and
\texttt{filename} is the name of the file which triggered the event.

On most platforms, \texttt{\textquotesingle{}rename\textquotesingle{}}
is emitted whenever a filename appears or disappears in the directory.

The listener callback is attached to the
\texttt{\textquotesingle{}change\textquotesingle{}} event fired by
\{fs.FSWatcher\}, but it is not the same thing as the
\texttt{\textquotesingle{}change\textquotesingle{}} value of
\texttt{eventType}.

If a \texttt{signal} is passed, aborting the corresponding
AbortController will close the returned \{fs.FSWatcher\}.

\paragraph{Caveats}\label{caveats}

The \texttt{fs.watch} API is not 100\% consistent across platforms, and
is unavailable in some situations.

On Windows, no events will be emitted if the watched directory is moved
or renamed. An \texttt{EPERM} error is reported when the watched
directory is deleted.

\subparagraph{Availability}\label{availability}

This feature depends on the underlying operating system providing a way
to be notified of file system changes.

\begin{itemize}
\tightlist
\item
  On Linux systems, this uses
  \href{https://man7.org/linux/man-pages/man7/inotify.7.html}{\texttt{inotify(7)}}.
\item
  On BSD systems, this uses
  \href{https://www.freebsd.org/cgi/man.cgi?query=kqueue&sektion=2}{\texttt{kqueue(2)}}.
\item
  On macOS, this uses
  \href{https://www.freebsd.org/cgi/man.cgi?query=kqueue&sektion=2}{\texttt{kqueue(2)}}
  for files and
  \href{https://developer.apple.com/documentation/coreservices/file_system_events}{\texttt{FSEvents}}
  for directories.
\item
  On SunOS systems (including Solaris and SmartOS), this uses
  \href{https://illumos.org/man/port_create}{\texttt{event\ ports}}.
\item
  On Windows systems, this feature depends on
  \href{https://docs.microsoft.com/en-us/windows/desktop/api/winbase/nf-winbase-readdirectorychangesw}{\texttt{ReadDirectoryChangesW}}.
\item
  On AIX systems, this feature depends on
  \href{https://developer.ibm.com/articles/au-aix_event_infrastructure/}{\texttt{AHAFS}},
  which must be enabled.
\item
  On IBM i systems, this feature is not supported.
\end{itemize}

If the underlying functionality is not available for some reason, then
\texttt{fs.watch()} will not be able to function and may throw an
exception. For example, watching files or directories can be unreliable,
and in some cases impossible, on network file systems (NFS, SMB, etc) or
host file systems when using virtualization software such as Vagrant or
Docker.

It is still possible to use \texttt{fs.watchFile()}, which uses stat
polling, but this method is slower and less reliable.

\subparagraph{Inodes}\label{inodes}

On Linux and macOS systems, \texttt{fs.watch()} resolves the path to an
\href{https://en.wikipedia.org/wiki/Inode}{inode} and watches the inode.
If the watched path is deleted and recreated, it is assigned a new
inode. The watch will emit an event for the delete but will continue
watching the \emph{original} inode. Events for the new inode will not be
emitted. This is expected behavior.

AIX files retain the same inode for the lifetime of a file. Saving and
closing a watched file on AIX will result in two notifications (one for
adding new content, and one for truncation).

\subparagraph{Filename argument}\label{filename-argument}

Providing \texttt{filename} argument in the callback is only supported
on Linux, macOS, Windows, and AIX. Even on supported platforms,
\texttt{filename} is not always guaranteed to be provided. Therefore,
don't assume that \texttt{filename} argument is always provided in the
callback, and have some fallback logic if it is \texttt{null}.

\begin{Shaded}
\begin{Highlighting}[]
\ImportTok{import}\NormalTok{ \{ watch \} }\ImportTok{from} \StringTok{\textquotesingle{}node:fs\textquotesingle{}}\OperatorTok{;}
\FunctionTok{watch}\NormalTok{(}\StringTok{\textquotesingle{}somedir\textquotesingle{}}\OperatorTok{,}\NormalTok{ (eventType}\OperatorTok{,}\NormalTok{ filename) }\KeywordTok{=\textgreater{}}\NormalTok{ \{}
  \BuiltInTok{console}\OperatorTok{.}\FunctionTok{log}\NormalTok{(}\VerbatimStringTok{\textasciigrave{}event type is: }\SpecialCharTok{$\{}\NormalTok{eventType}\SpecialCharTok{\}}\VerbatimStringTok{\textasciigrave{}}\NormalTok{)}\OperatorTok{;}
  \ControlFlowTok{if}\NormalTok{ (filename) \{}
    \BuiltInTok{console}\OperatorTok{.}\FunctionTok{log}\NormalTok{(}\VerbatimStringTok{\textasciigrave{}filename provided: }\SpecialCharTok{$\{}\NormalTok{filename}\SpecialCharTok{\}}\VerbatimStringTok{\textasciigrave{}}\NormalTok{)}\OperatorTok{;}
\NormalTok{  \} }\ControlFlowTok{else}\NormalTok{ \{}
    \BuiltInTok{console}\OperatorTok{.}\FunctionTok{log}\NormalTok{(}\StringTok{\textquotesingle{}filename not provided\textquotesingle{}}\NormalTok{)}\OperatorTok{;}
\NormalTok{  \}}
\NormalTok{\})}\OperatorTok{;}
\end{Highlighting}
\end{Shaded}

\subsubsection{\texorpdfstring{\texttt{fs.watchFile(filename{[},\ options{]},\ listener)}}{fs.watchFile(filename{[}, options{]}, listener)}}\label{fs.watchfilefilename-options-listener}

\begin{itemize}
\tightlist
\item
  \texttt{filename} \{string\textbar Buffer\textbar URL\}
\item
  \texttt{options} \{Object\}

  \begin{itemize}
  \tightlist
  \item
    \texttt{bigint} \{boolean\} \textbf{Default:} \texttt{false}
  \item
    \texttt{persistent} \{boolean\} \textbf{Default:} \texttt{true}
  \item
    \texttt{interval} \{integer\} \textbf{Default:} \texttt{5007}
  \end{itemize}
\item
  \texttt{listener} \{Function\}

  \begin{itemize}
  \tightlist
  \item
    \texttt{current} \{fs.Stats\}
  \item
    \texttt{previous} \{fs.Stats\}
  \end{itemize}
\item
  Returns: \{fs.StatWatcher\}
\end{itemize}

Watch for changes on \texttt{filename}. The callback \texttt{listener}
will be called each time the file is accessed.

The \texttt{options} argument may be omitted. If provided, it should be
an object. The \texttt{options} object may contain a boolean named
\texttt{persistent} that indicates whether the process should continue
to run as long as files are being watched. The \texttt{options} object
may specify an \texttt{interval} property indicating how often the
target should be polled in milliseconds.

The \texttt{listener} gets two arguments the current stat object and the
previous stat object:

\begin{Shaded}
\begin{Highlighting}[]
\ImportTok{import}\NormalTok{ \{ watchFile \} }\ImportTok{from} \StringTok{\textquotesingle{}node:fs\textquotesingle{}}\OperatorTok{;}

\FunctionTok{watchFile}\NormalTok{(}\StringTok{\textquotesingle{}message.text\textquotesingle{}}\OperatorTok{,}\NormalTok{ (curr}\OperatorTok{,}\NormalTok{ prev) }\KeywordTok{=\textgreater{}}\NormalTok{ \{}
  \BuiltInTok{console}\OperatorTok{.}\FunctionTok{log}\NormalTok{(}\VerbatimStringTok{\textasciigrave{}the current mtime is: }\SpecialCharTok{$\{}\NormalTok{curr}\OperatorTok{.}\AttributeTok{mtime}\SpecialCharTok{\}}\VerbatimStringTok{\textasciigrave{}}\NormalTok{)}\OperatorTok{;}
  \BuiltInTok{console}\OperatorTok{.}\FunctionTok{log}\NormalTok{(}\VerbatimStringTok{\textasciigrave{}the previous mtime was: }\SpecialCharTok{$\{}\NormalTok{prev}\OperatorTok{.}\AttributeTok{mtime}\SpecialCharTok{\}}\VerbatimStringTok{\textasciigrave{}}\NormalTok{)}\OperatorTok{;}
\NormalTok{\})}\OperatorTok{;}
\end{Highlighting}
\end{Shaded}

These stat objects are instances of \texttt{fs.Stat}. If the
\texttt{bigint} option is \texttt{true}, the numeric values in these
objects are specified as \texttt{BigInt}s.

To be notified when the file was modified, not just accessed, it is
necessary to compare \texttt{curr.mtimeMs} and \texttt{prev.mtimeMs}.

When an \texttt{fs.watchFile} operation results in an \texttt{ENOENT}
error, it will invoke the listener once, with all the fields zeroed (or,
for dates, the Unix Epoch). If the file is created later on, the
listener will be called again, with the latest stat objects. This is a
change in functionality since v0.10.

Using \hyperref[fswatchfilename-options-listener]{\texttt{fs.watch()}}
is more efficient than \texttt{fs.watchFile} and
\texttt{fs.unwatchFile}. \texttt{fs.watch} should be used instead of
\texttt{fs.watchFile} and \texttt{fs.unwatchFile} when possible.

When a file being watched by \texttt{fs.watchFile()} disappears and
reappears, then the contents of \texttt{previous} in the second callback
event (the file's reappearance) will be the same as the contents of
\texttt{previous} in the first callback event (its disappearance).

This happens when:

\begin{itemize}
\tightlist
\item
  the file is deleted, followed by a restore
\item
  the file is renamed and then renamed a second time back to its
  original name
\end{itemize}

\subsubsection{\texorpdfstring{\texttt{fs.write(fd,\ buffer,\ offset{[},\ length{[},\ position{]}{]},\ callback)}}{fs.write(fd, buffer, offset{[}, length{[}, position{]}{]}, callback)}}\label{fs.writefd-buffer-offset-length-position-callback}

\begin{itemize}
\tightlist
\item
  \texttt{fd} \{integer\}
\item
  \texttt{buffer} \{Buffer\textbar TypedArray\textbar DataView\}
\item
  \texttt{offset} \{integer\} \textbf{Default:} \texttt{0}
\item
  \texttt{length} \{integer\} \textbf{Default:}
  \texttt{buffer.byteLength\ -\ offset}
\item
  \texttt{position} \{integer\textbar null\} \textbf{Default:}
  \texttt{null}
\item
  \texttt{callback} \{Function\}

  \begin{itemize}
  \tightlist
  \item
    \texttt{err} \{Error\}
  \item
    \texttt{bytesWritten} \{integer\}
  \item
    \texttt{buffer} \{Buffer\textbar TypedArray\textbar DataView\}
  \end{itemize}
\end{itemize}

Write \texttt{buffer} to the file specified by \texttt{fd}.

\texttt{offset} determines the part of the buffer to be written, and
\texttt{length} is an integer specifying the number of bytes to write.

\texttt{position} refers to the offset from the beginning of the file
where this data should be written. If
\texttt{typeof\ position\ !==\ \textquotesingle{}number\textquotesingle{}},
the data will be written at the current position. See pwrite(2).

The callback will be given three arguments
\texttt{(err,\ bytesWritten,\ buffer)} where \texttt{bytesWritten}
specifies how many \emph{bytes} were written from \texttt{buffer}.

If this method is invoked as its
\href{util.md\#utilpromisifyoriginal}{\texttt{util.promisify()}}ed
version, it returns a promise for an \texttt{Object} with
\texttt{bytesWritten} and \texttt{buffer} properties.

It is unsafe to use \texttt{fs.write()} multiple times on the same file
without waiting for the callback. For this scenario,
\hyperref[fscreatewritestreampath-options]{\texttt{fs.createWriteStream()}}
is recommended.

On Linux, positional writes don't work when the file is opened in append
mode. The kernel ignores the position argument and always appends the
data to the end of the file.

\subsubsection{\texorpdfstring{\texttt{fs.write(fd,\ buffer{[},\ options{]},\ callback)}}{fs.write(fd, buffer{[}, options{]}, callback)}}\label{fs.writefd-buffer-options-callback}

\begin{itemize}
\tightlist
\item
  \texttt{fd} \{integer\}
\item
  \texttt{buffer} \{Buffer\textbar TypedArray\textbar DataView\}
\item
  \texttt{options} \{Object\}

  \begin{itemize}
  \tightlist
  \item
    \texttt{offset} \{integer\} \textbf{Default:} \texttt{0}
  \item
    \texttt{length} \{integer\} \textbf{Default:}
    \texttt{buffer.byteLength\ -\ offset}
  \item
    \texttt{position} \{integer\} \textbf{Default:} \texttt{null}
  \end{itemize}
\item
  \texttt{callback} \{Function\}

  \begin{itemize}
  \tightlist
  \item
    \texttt{err} \{Error\}
  \item
    \texttt{bytesWritten} \{integer\}
  \item
    \texttt{buffer} \{Buffer\textbar TypedArray\textbar DataView\}
  \end{itemize}
\end{itemize}

Write \texttt{buffer} to the file specified by \texttt{fd}.

Similar to the above \texttt{fs.write} function, this version takes an
optional \texttt{options} object. If no \texttt{options} object is
specified, it will default with the above values.

\subsubsection{\texorpdfstring{\texttt{fs.write(fd,\ string{[},\ position{[},\ encoding{]}{]},\ callback)}}{fs.write(fd, string{[}, position{[}, encoding{]}{]}, callback)}}\label{fs.writefd-string-position-encoding-callback}

\begin{itemize}
\tightlist
\item
  \texttt{fd} \{integer\}
\item
  \texttt{string} \{string\}
\item
  \texttt{position} \{integer\textbar null\} \textbf{Default:}
  \texttt{null}
\item
  \texttt{encoding} \{string\} \textbf{Default:}
  \texttt{\textquotesingle{}utf8\textquotesingle{}}
\item
  \texttt{callback} \{Function\}

  \begin{itemize}
  \tightlist
  \item
    \texttt{err} \{Error\}
  \item
    \texttt{written} \{integer\}
  \item
    \texttt{string} \{string\}
  \end{itemize}
\end{itemize}

Write \texttt{string} to the file specified by \texttt{fd}. If
\texttt{string} is not a string, an exception is thrown.

\texttt{position} refers to the offset from the beginning of the file
where this data should be written. If
\texttt{typeof\ position\ !==\ \textquotesingle{}number\textquotesingle{}}
the data will be written at the current position. See pwrite(2).

\texttt{encoding} is the expected string encoding.

The callback will receive the arguments
\texttt{(err,\ written,\ string)} where \texttt{written} specifies how
many \emph{bytes} the passed string required to be written. Bytes
written is not necessarily the same as string characters written. See
\href{buffer.md\#static-method-bufferbytelengthstring-encoding}{\texttt{Buffer.byteLength}}.

It is unsafe to use \texttt{fs.write()} multiple times on the same file
without waiting for the callback. For this scenario,
\hyperref[fscreatewritestreampath-options]{\texttt{fs.createWriteStream()}}
is recommended.

On Linux, positional writes don't work when the file is opened in append
mode. The kernel ignores the position argument and always appends the
data to the end of the file.

On Windows, if the file descriptor is connected to the console
(e.g.~\texttt{fd\ ==\ 1} or \texttt{stdout}) a string containing
non-ASCII characters will not be rendered properly by default,
regardless of the encoding used. It is possible to configure the console
to render UTF-8 properly by changing the active codepage with the
\texttt{chcp\ 65001} command. See the
\href{https://ss64.com/nt/chcp.html}{chcp} docs for more details.

\subsubsection{\texorpdfstring{\texttt{fs.writeFile(file,\ data{[},\ options{]},\ callback)}}{fs.writeFile(file, data{[}, options{]}, callback)}}\label{fs.writefilefile-data-options-callback}

\begin{itemize}
\tightlist
\item
  \texttt{file} \{string\textbar Buffer\textbar URL\textbar integer\}
  filename or file descriptor
\item
  \texttt{data}
  \{string\textbar Buffer\textbar TypedArray\textbar DataView\}
\item
  \texttt{options} \{Object\textbar string\}

  \begin{itemize}
  \tightlist
  \item
    \texttt{encoding} \{string\textbar null\} \textbf{Default:}
    \texttt{\textquotesingle{}utf8\textquotesingle{}}
  \item
    \texttt{mode} \{integer\} \textbf{Default:} \texttt{0o666}
  \item
    \texttt{flag} \{string\} See \hyperref[file-system-flags]{support of
    file system \texttt{flags}}. \textbf{Default:}
    \texttt{\textquotesingle{}w\textquotesingle{}}.
  \item
    \texttt{flush} \{boolean\} If all data is successfully written to
    the file, and \texttt{flush} is \texttt{true}, \texttt{fs.fsync()}
    is used to flush the data. \textbf{Default:} \texttt{false}.
  \item
    \texttt{signal} \{AbortSignal\} allows aborting an in-progress
    writeFile
  \end{itemize}
\item
  \texttt{callback} \{Function\}

  \begin{itemize}
  \tightlist
  \item
    \texttt{err} \{Error\textbar AggregateError\}
  \end{itemize}
\end{itemize}

When \texttt{file} is a filename, asynchronously writes data to the
file, replacing the file if it already exists. \texttt{data} can be a
string or a buffer.

When \texttt{file} is a file descriptor, the behavior is similar to
calling \texttt{fs.write()} directly (which is recommended). See the
notes below on using a file descriptor.

The \texttt{encoding} option is ignored if \texttt{data} is a buffer.

The \texttt{mode} option only affects the newly created file. See
\hyperref[fsopenpath-flags-mode-callback]{\texttt{fs.open()}} for more
details.

\begin{Shaded}
\begin{Highlighting}[]
\ImportTok{import}\NormalTok{ \{ writeFile \} }\ImportTok{from} \StringTok{\textquotesingle{}node:fs\textquotesingle{}}\OperatorTok{;}
\ImportTok{import}\NormalTok{ \{ }\BuiltInTok{Buffer}\NormalTok{ \} }\ImportTok{from} \StringTok{\textquotesingle{}node:buffer\textquotesingle{}}\OperatorTok{;}

\KeywordTok{const}\NormalTok{ data }\OperatorTok{=} \KeywordTok{new} \BuiltInTok{Uint8Array}\NormalTok{(}\BuiltInTok{Buffer}\OperatorTok{.}\FunctionTok{from}\NormalTok{(}\StringTok{\textquotesingle{}Hello Node.js\textquotesingle{}}\NormalTok{))}\OperatorTok{;}
\FunctionTok{writeFile}\NormalTok{(}\StringTok{\textquotesingle{}message.txt\textquotesingle{}}\OperatorTok{,}\NormalTok{ data}\OperatorTok{,}\NormalTok{ (err) }\KeywordTok{=\textgreater{}}\NormalTok{ \{}
  \ControlFlowTok{if}\NormalTok{ (err) }\ControlFlowTok{throw}\NormalTok{ err}\OperatorTok{;}
  \BuiltInTok{console}\OperatorTok{.}\FunctionTok{log}\NormalTok{(}\StringTok{\textquotesingle{}The file has been saved!\textquotesingle{}}\NormalTok{)}\OperatorTok{;}
\NormalTok{\})}\OperatorTok{;}
\end{Highlighting}
\end{Shaded}

If \texttt{options} is a string, then it specifies the encoding:

\begin{Shaded}
\begin{Highlighting}[]
\ImportTok{import}\NormalTok{ \{ writeFile \} }\ImportTok{from} \StringTok{\textquotesingle{}node:fs\textquotesingle{}}\OperatorTok{;}

\FunctionTok{writeFile}\NormalTok{(}\StringTok{\textquotesingle{}message.txt\textquotesingle{}}\OperatorTok{,} \StringTok{\textquotesingle{}Hello Node.js\textquotesingle{}}\OperatorTok{,} \StringTok{\textquotesingle{}utf8\textquotesingle{}}\OperatorTok{,}\NormalTok{ callback)}\OperatorTok{;}
\end{Highlighting}
\end{Shaded}

It is unsafe to use \texttt{fs.writeFile()} multiple times on the same
file without waiting for the callback. For this scenario,
\hyperref[fscreatewritestreampath-options]{\texttt{fs.createWriteStream()}}
is recommended.

Similarly to \texttt{fs.readFile} - \texttt{fs.writeFile} is a
convenience method that performs multiple \texttt{write} calls
internally to write the buffer passed to it. For performance sensitive
code consider using
\hyperref[fscreatewritestreampath-options]{\texttt{fs.createWriteStream()}}.

It is possible to use an \{AbortSignal\} to cancel an
\texttt{fs.writeFile()}. Cancelation is ``best effort'', and some amount
of data is likely still to be written.

\begin{Shaded}
\begin{Highlighting}[]
\ImportTok{import}\NormalTok{ \{ writeFile \} }\ImportTok{from} \StringTok{\textquotesingle{}node:fs\textquotesingle{}}\OperatorTok{;}
\ImportTok{import}\NormalTok{ \{ }\BuiltInTok{Buffer}\NormalTok{ \} }\ImportTok{from} \StringTok{\textquotesingle{}node:buffer\textquotesingle{}}\OperatorTok{;}

\KeywordTok{const}\NormalTok{ controller }\OperatorTok{=} \KeywordTok{new} \FunctionTok{AbortController}\NormalTok{()}\OperatorTok{;}
\KeywordTok{const}\NormalTok{ \{ signal \} }\OperatorTok{=}\NormalTok{ controller}\OperatorTok{;}
\KeywordTok{const}\NormalTok{ data }\OperatorTok{=} \KeywordTok{new} \BuiltInTok{Uint8Array}\NormalTok{(}\BuiltInTok{Buffer}\OperatorTok{.}\FunctionTok{from}\NormalTok{(}\StringTok{\textquotesingle{}Hello Node.js\textquotesingle{}}\NormalTok{))}\OperatorTok{;}
\FunctionTok{writeFile}\NormalTok{(}\StringTok{\textquotesingle{}message.txt\textquotesingle{}}\OperatorTok{,}\NormalTok{ data}\OperatorTok{,}\NormalTok{ \{ signal \}}\OperatorTok{,}\NormalTok{ (err) }\KeywordTok{=\textgreater{}}\NormalTok{ \{}
  \CommentTok{// When a request is aborted {-} the callback is called with an AbortError}
\NormalTok{\})}\OperatorTok{;}
\CommentTok{// When the request should be aborted}
\NormalTok{controller}\OperatorTok{.}\FunctionTok{abort}\NormalTok{()}\OperatorTok{;}
\end{Highlighting}
\end{Shaded}

Aborting an ongoing request does not abort individual operating system
requests but rather the internal buffering \texttt{fs.writeFile}
performs.

\paragraph{\texorpdfstring{Using \texttt{fs.writeFile()} with file
descriptors}{Using fs.writeFile() with file descriptors}}\label{using-fs.writefile-with-file-descriptors}

When \texttt{file} is a file descriptor, the behavior is almost
identical to directly calling \texttt{fs.write()} like:

\begin{Shaded}
\begin{Highlighting}[]
\ImportTok{import}\NormalTok{ \{ write \} }\ImportTok{from} \StringTok{\textquotesingle{}node:fs\textquotesingle{}}\OperatorTok{;}
\ImportTok{import}\NormalTok{ \{ }\BuiltInTok{Buffer}\NormalTok{ \} }\ImportTok{from} \StringTok{\textquotesingle{}node:buffer\textquotesingle{}}\OperatorTok{;}

\FunctionTok{write}\NormalTok{(fd}\OperatorTok{,} \BuiltInTok{Buffer}\OperatorTok{.}\FunctionTok{from}\NormalTok{(data}\OperatorTok{,}\NormalTok{ options}\OperatorTok{.}\AttributeTok{encoding}\NormalTok{)}\OperatorTok{,}\NormalTok{ callback)}\OperatorTok{;}
\end{Highlighting}
\end{Shaded}

The difference from directly calling \texttt{fs.write()} is that under
some unusual conditions, \texttt{fs.write()} might write only part of
the buffer and need to be retried to write the remaining data, whereas
\texttt{fs.writeFile()} retries until the data is entirely written (or
an error occurs).

The implications of this are a common source of confusion. In the file
descriptor case, the file is not replaced! The data is not necessarily
written to the beginning of the file, and the file's original data may
remain before and/or after the newly written data.

For example, if \texttt{fs.writeFile()} is called twice in a row, first
to write the string \texttt{\textquotesingle{}Hello\textquotesingle{}},
then to write the string
\texttt{\textquotesingle{},\ World\textquotesingle{}}, the file would
contain \texttt{\textquotesingle{}Hello,\ World\textquotesingle{}}, and
might contain some of the file's original data (depending on the size of
the original file, and the position of the file descriptor). If a file
name had been used instead of a descriptor, the file would be guaranteed
to contain only \texttt{\textquotesingle{},\ World\textquotesingle{}}.

\subsubsection{\texorpdfstring{\texttt{fs.writev(fd,\ buffers{[},\ position{]},\ callback)}}{fs.writev(fd, buffers{[}, position{]}, callback)}}\label{fs.writevfd-buffers-position-callback}

\begin{itemize}
\tightlist
\item
  \texttt{fd} \{integer\}
\item
  \texttt{buffers} \{ArrayBufferView{[}{]}\}
\item
  \texttt{position} \{integer\textbar null\} \textbf{Default:}
  \texttt{null}
\item
  \texttt{callback} \{Function\}

  \begin{itemize}
  \tightlist
  \item
    \texttt{err} \{Error\}
  \item
    \texttt{bytesWritten} \{integer\}
  \item
    \texttt{buffers} \{ArrayBufferView{[}{]}\}
  \end{itemize}
\end{itemize}

Write an array of \texttt{ArrayBufferView}s to the file specified by
\texttt{fd} using \texttt{writev()}.

\texttt{position} is the offset from the beginning of the file where
this data should be written. If
\texttt{typeof\ position\ !==\ \textquotesingle{}number\textquotesingle{}},
the data will be written at the current position.

The callback will be given three arguments: \texttt{err},
\texttt{bytesWritten}, and \texttt{buffers}. \texttt{bytesWritten} is
how many bytes were written from \texttt{buffers}.

If this method is
\href{util.md\#utilpromisifyoriginal}{\texttt{util.promisify()}}ed, it
returns a promise for an \texttt{Object} with \texttt{bytesWritten} and
\texttt{buffers} properties.

It is unsafe to use \texttt{fs.writev()} multiple times on the same file
without waiting for the callback. For this scenario, use
\hyperref[fscreatewritestreampath-options]{\texttt{fs.createWriteStream()}}.

On Linux, positional writes don't work when the file is opened in append
mode. The kernel ignores the position argument and always appends the
data to the end of the file.

\subsection{Synchronous API}\label{synchronous-api}

The synchronous APIs perform all operations synchronously, blocking the
event loop until the operation completes or fails.

\subsubsection{\texorpdfstring{\texttt{fs.accessSync(path{[},\ mode{]})}}{fs.accessSync(path{[}, mode{]})}}\label{fs.accesssyncpath-mode}

\begin{itemize}
\tightlist
\item
  \texttt{path} \{string\textbar Buffer\textbar URL\}
\item
  \texttt{mode} \{integer\} \textbf{Default:}
  \texttt{fs.constants.F\_OK}
\end{itemize}

Synchronously tests a user's permissions for the file or directory
specified by \texttt{path}. The \texttt{mode} argument is an optional
integer that specifies the accessibility checks to be performed.
\texttt{mode} should be either the value \texttt{fs.constants.F\_OK} or
a mask consisting of the bitwise OR of any of
\texttt{fs.constants.R\_OK}, \texttt{fs.constants.W\_OK}, and
\texttt{fs.constants.X\_OK} (e.g.
\texttt{fs.constants.W\_OK\ \textbar{}\ fs.constants.R\_OK}). Check
\hyperref[file-access-constants]{File access constants} for possible
values of \texttt{mode}.

If any of the accessibility checks fail, an \texttt{Error} will be
thrown. Otherwise, the method will return \texttt{undefined}.

\begin{Shaded}
\begin{Highlighting}[]
\ImportTok{import}\NormalTok{ \{ accessSync}\OperatorTok{,}\NormalTok{ constants \} }\ImportTok{from} \StringTok{\textquotesingle{}node:fs\textquotesingle{}}\OperatorTok{;}

\ControlFlowTok{try}\NormalTok{ \{}
  \FunctionTok{accessSync}\NormalTok{(}\StringTok{\textquotesingle{}etc/passwd\textquotesingle{}}\OperatorTok{,}\NormalTok{ constants}\OperatorTok{.}\AttributeTok{R\_OK} \OperatorTok{|}\NormalTok{ constants}\OperatorTok{.}\AttributeTok{W\_OK}\NormalTok{)}\OperatorTok{;}
  \BuiltInTok{console}\OperatorTok{.}\FunctionTok{log}\NormalTok{(}\StringTok{\textquotesingle{}can read/write\textquotesingle{}}\NormalTok{)}\OperatorTok{;}
\NormalTok{\} }\ControlFlowTok{catch}\NormalTok{ (err) \{}
  \BuiltInTok{console}\OperatorTok{.}\FunctionTok{error}\NormalTok{(}\StringTok{\textquotesingle{}no access!\textquotesingle{}}\NormalTok{)}\OperatorTok{;}
\NormalTok{\}}
\end{Highlighting}
\end{Shaded}

\subsubsection{\texorpdfstring{\texttt{fs.appendFileSync(path,\ data{[},\ options{]})}}{fs.appendFileSync(path, data{[}, options{]})}}\label{fs.appendfilesyncpath-data-options}

\begin{itemize}
\tightlist
\item
  \texttt{path} \{string\textbar Buffer\textbar URL\textbar number\}
  filename or file descriptor
\item
  \texttt{data} \{string\textbar Buffer\}
\item
  \texttt{options} \{Object\textbar string\}

  \begin{itemize}
  \tightlist
  \item
    \texttt{encoding} \{string\textbar null\} \textbf{Default:}
    \texttt{\textquotesingle{}utf8\textquotesingle{}}
  \item
    \texttt{mode} \{integer\} \textbf{Default:} \texttt{0o666}
  \item
    \texttt{flag} \{string\} See \hyperref[file-system-flags]{support of
    file system \texttt{flags}}. \textbf{Default:}
    \texttt{\textquotesingle{}a\textquotesingle{}}.
  \item
    \texttt{flush} \{boolean\} If \texttt{true}, the underlying file
    descriptor is flushed prior to closing it. \textbf{Default:}
    \texttt{false}.
  \end{itemize}
\end{itemize}

Synchronously append data to a file, creating the file if it does not
yet exist. \texttt{data} can be a string or a \{Buffer\}.

The \texttt{mode} option only affects the newly created file. See
\hyperref[fsopenpath-flags-mode-callback]{\texttt{fs.open()}} for more
details.

\begin{Shaded}
\begin{Highlighting}[]
\ImportTok{import}\NormalTok{ \{ appendFileSync \} }\ImportTok{from} \StringTok{\textquotesingle{}node:fs\textquotesingle{}}\OperatorTok{;}

\ControlFlowTok{try}\NormalTok{ \{}
  \FunctionTok{appendFileSync}\NormalTok{(}\StringTok{\textquotesingle{}message.txt\textquotesingle{}}\OperatorTok{,} \StringTok{\textquotesingle{}data to append\textquotesingle{}}\NormalTok{)}\OperatorTok{;}
  \BuiltInTok{console}\OperatorTok{.}\FunctionTok{log}\NormalTok{(}\StringTok{\textquotesingle{}The "data to append" was appended to file!\textquotesingle{}}\NormalTok{)}\OperatorTok{;}
\NormalTok{\} }\ControlFlowTok{catch}\NormalTok{ (err) \{}
  \CommentTok{/* Handle the error */}
\NormalTok{\}}
\end{Highlighting}
\end{Shaded}

If \texttt{options} is a string, then it specifies the encoding:

\begin{Shaded}
\begin{Highlighting}[]
\ImportTok{import}\NormalTok{ \{ appendFileSync \} }\ImportTok{from} \StringTok{\textquotesingle{}node:fs\textquotesingle{}}\OperatorTok{;}

\FunctionTok{appendFileSync}\NormalTok{(}\StringTok{\textquotesingle{}message.txt\textquotesingle{}}\OperatorTok{,} \StringTok{\textquotesingle{}data to append\textquotesingle{}}\OperatorTok{,} \StringTok{\textquotesingle{}utf8\textquotesingle{}}\NormalTok{)}\OperatorTok{;}
\end{Highlighting}
\end{Shaded}

The \texttt{path} may be specified as a numeric file descriptor that has
been opened for appending (using \texttt{fs.open()} or
\texttt{fs.openSync()}). The file descriptor will not be closed
automatically.

\begin{Shaded}
\begin{Highlighting}[]
\ImportTok{import}\NormalTok{ \{ openSync}\OperatorTok{,}\NormalTok{ closeSync}\OperatorTok{,}\NormalTok{ appendFileSync \} }\ImportTok{from} \StringTok{\textquotesingle{}node:fs\textquotesingle{}}\OperatorTok{;}

\KeywordTok{let}\NormalTok{ fd}\OperatorTok{;}

\ControlFlowTok{try}\NormalTok{ \{}
\NormalTok{  fd }\OperatorTok{=} \FunctionTok{openSync}\NormalTok{(}\StringTok{\textquotesingle{}message.txt\textquotesingle{}}\OperatorTok{,} \StringTok{\textquotesingle{}a\textquotesingle{}}\NormalTok{)}\OperatorTok{;}
  \FunctionTok{appendFileSync}\NormalTok{(fd}\OperatorTok{,} \StringTok{\textquotesingle{}data to append\textquotesingle{}}\OperatorTok{,} \StringTok{\textquotesingle{}utf8\textquotesingle{}}\NormalTok{)}\OperatorTok{;}
\NormalTok{\} }\ControlFlowTok{catch}\NormalTok{ (err) \{}
  \CommentTok{/* Handle the error */}
\NormalTok{\} }\ControlFlowTok{finally}\NormalTok{ \{}
  \ControlFlowTok{if}\NormalTok{ (fd }\OperatorTok{!==} \KeywordTok{undefined}\NormalTok{)}
    \FunctionTok{closeSync}\NormalTok{(fd)}\OperatorTok{;}
\NormalTok{\}}
\end{Highlighting}
\end{Shaded}

\subsubsection{\texorpdfstring{\texttt{fs.chmodSync(path,\ mode)}}{fs.chmodSync(path, mode)}}\label{fs.chmodsyncpath-mode}

\begin{itemize}
\tightlist
\item
  \texttt{path} \{string\textbar Buffer\textbar URL\}
\item
  \texttt{mode} \{string\textbar integer\}
\end{itemize}

For detailed information, see the documentation of the asynchronous
version of this API:
\hyperref[fschmodpath-mode-callback]{\texttt{fs.chmod()}}.

See the POSIX chmod(2) documentation for more detail.

\subsubsection{\texorpdfstring{\texttt{fs.chownSync(path,\ uid,\ gid)}}{fs.chownSync(path, uid, gid)}}\label{fs.chownsyncpath-uid-gid}

\begin{itemize}
\tightlist
\item
  \texttt{path} \{string\textbar Buffer\textbar URL\}
\item
  \texttt{uid} \{integer\}
\item
  \texttt{gid} \{integer\}
\end{itemize}

Synchronously changes owner and group of a file. Returns
\texttt{undefined}. This is the synchronous version of
\hyperref[fschownpath-uid-gid-callback]{\texttt{fs.chown()}}.

See the POSIX chown(2) documentation for more detail.

\subsubsection{\texorpdfstring{\texttt{fs.closeSync(fd)}}{fs.closeSync(fd)}}\label{fs.closesyncfd}

\begin{itemize}
\tightlist
\item
  \texttt{fd} \{integer\}
\end{itemize}

Closes the file descriptor. Returns \texttt{undefined}.

Calling \texttt{fs.closeSync()} on any file descriptor (\texttt{fd})
that is currently in use through any other \texttt{fs} operation may
lead to undefined behavior.

See the POSIX close(2) documentation for more detail.

\subsubsection{\texorpdfstring{\texttt{fs.copyFileSync(src,\ dest{[},\ mode{]})}}{fs.copyFileSync(src, dest{[}, mode{]})}}\label{fs.copyfilesyncsrc-dest-mode}

\begin{itemize}
\tightlist
\item
  \texttt{src} \{string\textbar Buffer\textbar URL\} source filename to
  copy
\item
  \texttt{dest} \{string\textbar Buffer\textbar URL\} destination
  filename of the copy operation
\item
  \texttt{mode} \{integer\} modifiers for copy operation.
  \textbf{Default:} \texttt{0}.
\end{itemize}

Synchronously copies \texttt{src} to \texttt{dest}. By default,
\texttt{dest} is overwritten if it already exists. Returns
\texttt{undefined}. Node.js makes no guarantees about the atomicity of
the copy operation. If an error occurs after the destination file has
been opened for writing, Node.js will attempt to remove the destination.

\texttt{mode} is an optional integer that specifies the behavior of the
copy operation. It is possible to create a mask consisting of the
bitwise OR of two or more values (e.g.
\texttt{fs.constants.COPYFILE\_EXCL\ \textbar{}\ fs.constants.COPYFILE\_FICLONE}).

\begin{itemize}
\tightlist
\item
  \texttt{fs.constants.COPYFILE\_EXCL}: The copy operation will fail if
  \texttt{dest} already exists.
\item
  \texttt{fs.constants.COPYFILE\_FICLONE}: The copy operation will
  attempt to create a copy-on-write reflink. If the platform does not
  support copy-on-write, then a fallback copy mechanism is used.
\item
  \texttt{fs.constants.COPYFILE\_FICLONE\_FORCE}: The copy operation
  will attempt to create a copy-on-write reflink. If the platform does
  not support copy-on-write, then the operation will fail.
\end{itemize}

\begin{Shaded}
\begin{Highlighting}[]
\ImportTok{import}\NormalTok{ \{ copyFileSync}\OperatorTok{,}\NormalTok{ constants \} }\ImportTok{from} \StringTok{\textquotesingle{}node:fs\textquotesingle{}}\OperatorTok{;}

\CommentTok{// destination.txt will be created or overwritten by default.}
\FunctionTok{copyFileSync}\NormalTok{(}\StringTok{\textquotesingle{}source.txt\textquotesingle{}}\OperatorTok{,} \StringTok{\textquotesingle{}destination.txt\textquotesingle{}}\NormalTok{)}\OperatorTok{;}
\BuiltInTok{console}\OperatorTok{.}\FunctionTok{log}\NormalTok{(}\StringTok{\textquotesingle{}source.txt was copied to destination.txt\textquotesingle{}}\NormalTok{)}\OperatorTok{;}

\CommentTok{// By using COPYFILE\_EXCL, the operation will fail if destination.txt exists.}
\FunctionTok{copyFileSync}\NormalTok{(}\StringTok{\textquotesingle{}source.txt\textquotesingle{}}\OperatorTok{,} \StringTok{\textquotesingle{}destination.txt\textquotesingle{}}\OperatorTok{,}\NormalTok{ constants}\OperatorTok{.}\AttributeTok{COPYFILE\_EXCL}\NormalTok{)}\OperatorTok{;}
\end{Highlighting}
\end{Shaded}

\subsubsection{\texorpdfstring{\texttt{fs.cpSync(src,\ dest{[},\ options{]})}}{fs.cpSync(src, dest{[}, options{]})}}\label{fs.cpsyncsrc-dest-options}

\begin{quote}
Stability: 1 - Experimental
\end{quote}

\begin{itemize}
\tightlist
\item
  \texttt{src} \{string\textbar URL\} source path to copy.
\item
  \texttt{dest} \{string\textbar URL\} destination path to copy to.
\item
  \texttt{options} \{Object\}

  \begin{itemize}
  \tightlist
  \item
    \texttt{dereference} \{boolean\} dereference symlinks.
    \textbf{Default:} \texttt{false}.
  \item
    \texttt{errorOnExist} \{boolean\} when \texttt{force} is
    \texttt{false}, and the destination exists, throw an error.
    \textbf{Default:} \texttt{false}.
  \item
    \texttt{filter} \{Function\} Function to filter copied
    files/directories. Return \texttt{true} to copy the item,
    \texttt{false} to ignore it. When ignoring a directory, all of its
    contents will be skipped as well. \textbf{Default:}
    \texttt{undefined}

    \begin{itemize}
    \tightlist
    \item
      \texttt{src} \{string\} source path to copy.
    \item
      \texttt{dest} \{string\} destination path to copy to.
    \item
      Returns: \{boolean\}
    \end{itemize}
  \item
    \texttt{force} \{boolean\} overwrite existing file or directory. The
    copy operation will ignore errors if you set this to false and the
    destination exists. Use the \texttt{errorOnExist} option to change
    this behavior. \textbf{Default:} \texttt{true}.
  \item
    \texttt{mode} \{integer\} modifiers for copy operation.
    \textbf{Default:} \texttt{0}. See \texttt{mode} flag of
    \hyperref[fscopyfilesyncsrc-dest-mode]{\texttt{fs.copyFileSync()}}.
  \item
    \texttt{preserveTimestamps} \{boolean\} When \texttt{true}
    timestamps from \texttt{src} will be preserved. \textbf{Default:}
    \texttt{false}.
  \item
    \texttt{recursive} \{boolean\} copy directories recursively
    \textbf{Default:} \texttt{false}
  \item
    \texttt{verbatimSymlinks} \{boolean\} When \texttt{true}, path
    resolution for symlinks will be skipped. \textbf{Default:}
    \texttt{false}
  \end{itemize}
\end{itemize}

Synchronously copies the entire directory structure from \texttt{src} to
\texttt{dest}, including subdirectories and files.

When copying a directory to another directory, globs are not supported
and behavior is similar to \texttt{cp\ dir1/\ dir2/}.

\subsubsection{\texorpdfstring{\texttt{fs.existsSync(path)}}{fs.existsSync(path)}}\label{fs.existssyncpath}

\begin{itemize}
\tightlist
\item
  \texttt{path} \{string\textbar Buffer\textbar URL\}
\item
  Returns: \{boolean\}
\end{itemize}

Returns \texttt{true} if the path exists, \texttt{false} otherwise.

For detailed information, see the documentation of the asynchronous
version of this API:
\hyperref[fsexistspath-callback]{\texttt{fs.exists()}}.

\texttt{fs.exists()} is deprecated, but \texttt{fs.existsSync()} is not.
The \texttt{callback} parameter to \texttt{fs.exists()} accepts
parameters that are inconsistent with other Node.js callbacks.
\texttt{fs.existsSync()} does not use a callback.

\begin{Shaded}
\begin{Highlighting}[]
\ImportTok{import}\NormalTok{ \{ existsSync \} }\ImportTok{from} \StringTok{\textquotesingle{}node:fs\textquotesingle{}}\OperatorTok{;}

\ControlFlowTok{if}\NormalTok{ (}\FunctionTok{existsSync}\NormalTok{(}\StringTok{\textquotesingle{}/etc/passwd\textquotesingle{}}\NormalTok{))}
  \BuiltInTok{console}\OperatorTok{.}\FunctionTok{log}\NormalTok{(}\StringTok{\textquotesingle{}The path exists.\textquotesingle{}}\NormalTok{)}\OperatorTok{;}
\end{Highlighting}
\end{Shaded}

\subsubsection{\texorpdfstring{\texttt{fs.fchmodSync(fd,\ mode)}}{fs.fchmodSync(fd, mode)}}\label{fs.fchmodsyncfd-mode}

\begin{itemize}
\tightlist
\item
  \texttt{fd} \{integer\}
\item
  \texttt{mode} \{string\textbar integer\}
\end{itemize}

Sets the permissions on the file. Returns \texttt{undefined}.

See the POSIX fchmod(2) documentation for more detail.

\subsubsection{\texorpdfstring{\texttt{fs.fchownSync(fd,\ uid,\ gid)}}{fs.fchownSync(fd, uid, gid)}}\label{fs.fchownsyncfd-uid-gid}

\begin{itemize}
\tightlist
\item
  \texttt{fd} \{integer\}
\item
  \texttt{uid} \{integer\} The file's new owner's user id.
\item
  \texttt{gid} \{integer\} The file's new group's group id.
\end{itemize}

Sets the owner of the file. Returns \texttt{undefined}.

See the POSIX fchown(2) documentation for more detail.

\subsubsection{\texorpdfstring{\texttt{fs.fdatasyncSync(fd)}}{fs.fdatasyncSync(fd)}}\label{fs.fdatasyncsyncfd}

\begin{itemize}
\tightlist
\item
  \texttt{fd} \{integer\}
\end{itemize}

Forces all currently queued I/O operations associated with the file to
the operating system's synchronized I/O completion state. Refer to the
POSIX fdatasync(2) documentation for details. Returns
\texttt{undefined}.

\subsubsection{\texorpdfstring{\texttt{fs.fstatSync(fd{[},\ options{]})}}{fs.fstatSync(fd{[}, options{]})}}\label{fs.fstatsyncfd-options}

\begin{itemize}
\tightlist
\item
  \texttt{fd} \{integer\}
\item
  \texttt{options} \{Object\}

  \begin{itemize}
  \tightlist
  \item
    \texttt{bigint} \{boolean\} Whether the numeric values in the
    returned \{fs.Stats\} object should be \texttt{bigint}.
    \textbf{Default:} \texttt{false}.
  \end{itemize}
\item
  Returns: \{fs.Stats\}
\end{itemize}

Retrieves the \{fs.Stats\} for the file descriptor.

See the POSIX fstat(2) documentation for more detail.

\subsubsection{\texorpdfstring{\texttt{fs.fsyncSync(fd)}}{fs.fsyncSync(fd)}}\label{fs.fsyncsyncfd}

\begin{itemize}
\tightlist
\item
  \texttt{fd} \{integer\}
\end{itemize}

Request that all data for the open file descriptor is flushed to the
storage device. The specific implementation is operating system and
device specific. Refer to the POSIX fsync(2) documentation for more
detail. Returns \texttt{undefined}.

\subsubsection{\texorpdfstring{\texttt{fs.ftruncateSync(fd{[},\ len{]})}}{fs.ftruncateSync(fd{[}, len{]})}}\label{fs.ftruncatesyncfd-len}

\begin{itemize}
\tightlist
\item
  \texttt{fd} \{integer\}
\item
  \texttt{len} \{integer\} \textbf{Default:} \texttt{0}
\end{itemize}

Truncates the file descriptor. Returns \texttt{undefined}.

For detailed information, see the documentation of the asynchronous
version of this API:
\hyperref[fsftruncatefd-len-callback]{\texttt{fs.ftruncate()}}.

\subsubsection{\texorpdfstring{\texttt{fs.futimesSync(fd,\ atime,\ mtime)}}{fs.futimesSync(fd, atime, mtime)}}\label{fs.futimessyncfd-atime-mtime}

\begin{itemize}
\tightlist
\item
  \texttt{fd} \{integer\}
\item
  \texttt{atime} \{number\textbar string\textbar Date\}
\item
  \texttt{mtime} \{number\textbar string\textbar Date\}
\end{itemize}

Synchronous version of
\hyperref[fsfutimesfd-atime-mtime-callback]{\texttt{fs.futimes()}}.
Returns \texttt{undefined}.

\subsubsection{\texorpdfstring{\texttt{fs.lchmodSync(path,\ mode)}}{fs.lchmodSync(path, mode)}}\label{fs.lchmodsyncpath-mode}

\begin{itemize}
\tightlist
\item
  \texttt{path} \{string\textbar Buffer\textbar URL\}
\item
  \texttt{mode} \{integer\}
\end{itemize}

Changes the permissions on a symbolic link. Returns \texttt{undefined}.

This method is only implemented on macOS.

See the POSIX lchmod(2) documentation for more detail.

\subsubsection{\texorpdfstring{\texttt{fs.lchownSync(path,\ uid,\ gid)}}{fs.lchownSync(path, uid, gid)}}\label{fs.lchownsyncpath-uid-gid}

\begin{itemize}
\tightlist
\item
  \texttt{path} \{string\textbar Buffer\textbar URL\}
\item
  \texttt{uid} \{integer\} The file's new owner's user id.
\item
  \texttt{gid} \{integer\} The file's new group's group id.
\end{itemize}

Set the owner for the path. Returns \texttt{undefined}.

See the POSIX lchown(2) documentation for more details.

\subsubsection{\texorpdfstring{\texttt{fs.lutimesSync(path,\ atime,\ mtime)}}{fs.lutimesSync(path, atime, mtime)}}\label{fs.lutimessyncpath-atime-mtime}

\begin{itemize}
\tightlist
\item
  \texttt{path} \{string\textbar Buffer\textbar URL\}
\item
  \texttt{atime} \{number\textbar string\textbar Date\}
\item
  \texttt{mtime} \{number\textbar string\textbar Date\}
\end{itemize}

Change the file system timestamps of the symbolic link referenced by
\texttt{path}. Returns \texttt{undefined}, or throws an exception when
parameters are incorrect or the operation fails. This is the synchronous
version of
\hyperref[fslutimespath-atime-mtime-callback]{\texttt{fs.lutimes()}}.

\subsubsection{\texorpdfstring{\texttt{fs.linkSync(existingPath,\ newPath)}}{fs.linkSync(existingPath, newPath)}}\label{fs.linksyncexistingpath-newpath}

\begin{itemize}
\tightlist
\item
  \texttt{existingPath} \{string\textbar Buffer\textbar URL\}
\item
  \texttt{newPath} \{string\textbar Buffer\textbar URL\}
\end{itemize}

Creates a new link from the \texttt{existingPath} to the
\texttt{newPath}. See the POSIX link(2) documentation for more detail.
Returns \texttt{undefined}.

\subsubsection{\texorpdfstring{\texttt{fs.lstatSync(path{[},\ options{]})}}{fs.lstatSync(path{[}, options{]})}}\label{fs.lstatsyncpath-options}

\begin{itemize}
\tightlist
\item
  \texttt{path} \{string\textbar Buffer\textbar URL\}
\item
  \texttt{options} \{Object\}

  \begin{itemize}
  \tightlist
  \item
    \texttt{bigint} \{boolean\} Whether the numeric values in the
    returned \{fs.Stats\} object should be \texttt{bigint}.
    \textbf{Default:} \texttt{false}.
  \item
    \texttt{throwIfNoEntry} \{boolean\} Whether an exception will be
    thrown if no file system entry exists, rather than returning
    \texttt{undefined}. \textbf{Default:} \texttt{true}.
  \end{itemize}
\item
  Returns: \{fs.Stats\}
\end{itemize}

Retrieves the \{fs.Stats\} for the symbolic link referred to by
\texttt{path}.

See the POSIX lstat(2) documentation for more details.

\subsubsection{\texorpdfstring{\texttt{fs.mkdirSync(path{[},\ options{]})}}{fs.mkdirSync(path{[}, options{]})}}\label{fs.mkdirsyncpath-options}

\begin{itemize}
\tightlist
\item
  \texttt{path} \{string\textbar Buffer\textbar URL\}
\item
  \texttt{options} \{Object\textbar integer\}

  \begin{itemize}
  \tightlist
  \item
    \texttt{recursive} \{boolean\} \textbf{Default:} \texttt{false}
  \item
    \texttt{mode} \{string\textbar integer\} Not supported on Windows.
    \textbf{Default:} \texttt{0o777}.
  \end{itemize}
\item
  Returns: \{string\textbar undefined\}
\end{itemize}

Synchronously creates a directory. Returns \texttt{undefined}, or if
\texttt{recursive} is \texttt{true}, the first directory path created.
This is the synchronous version of
\hyperref[fsmkdirpath-options-callback]{\texttt{fs.mkdir()}}.

See the POSIX mkdir(2) documentation for more details.

\subsubsection{\texorpdfstring{\texttt{fs.mkdtempSync(prefix{[},\ options{]})}}{fs.mkdtempSync(prefix{[}, options{]})}}\label{fs.mkdtempsyncprefix-options}

\begin{itemize}
\tightlist
\item
  \texttt{prefix} \{string\textbar Buffer\textbar URL\}
\item
  \texttt{options} \{string\textbar Object\}

  \begin{itemize}
  \tightlist
  \item
    \texttt{encoding} \{string\} \textbf{Default:}
    \texttt{\textquotesingle{}utf8\textquotesingle{}}
  \end{itemize}
\item
  Returns: \{string\}
\end{itemize}

Returns the created directory path.

For detailed information, see the documentation of the asynchronous
version of this API:
\hyperref[fsmkdtempprefix-options-callback]{\texttt{fs.mkdtemp()}}.

The optional \texttt{options} argument can be a string specifying an
encoding, or an object with an \texttt{encoding} property specifying the
character encoding to use.

\subsubsection{\texorpdfstring{\texttt{fs.opendirSync(path{[},\ options{]})}}{fs.opendirSync(path{[}, options{]})}}\label{fs.opendirsyncpath-options}

\begin{itemize}
\tightlist
\item
  \texttt{path} \{string\textbar Buffer\textbar URL\}
\item
  \texttt{options} \{Object\}

  \begin{itemize}
  \tightlist
  \item
    \texttt{encoding} \{string\textbar null\} \textbf{Default:}
    \texttt{\textquotesingle{}utf8\textquotesingle{}}
  \item
    \texttt{bufferSize} \{number\} Number of directory entries that are
    buffered internally when reading from the directory. Higher values
    lead to better performance but higher memory usage.
    \textbf{Default:} \texttt{32}
  \item
    \texttt{recursive} \{boolean\} \textbf{Default:} \texttt{false}
  \end{itemize}
\item
  Returns: \{fs.Dir\}
\end{itemize}

Synchronously open a directory. See opendir(3).

Creates an \{fs.Dir\}, which contains all further functions for reading
from and cleaning up the directory.

The \texttt{encoding} option sets the encoding for the \texttt{path}
while opening the directory and subsequent read operations.

\subsubsection{\texorpdfstring{\texttt{fs.openSync(path{[},\ flags{[},\ mode{]}{]})}}{fs.openSync(path{[}, flags{[}, mode{]}{]})}}\label{fs.opensyncpath-flags-mode}

\begin{itemize}
\tightlist
\item
  \texttt{path} \{string\textbar Buffer\textbar URL\}
\item
  \texttt{flags} \{string\textbar number\} \textbf{Default:}
  \texttt{\textquotesingle{}r\textquotesingle{}}. See
  \hyperref[file-system-flags]{support of file system \texttt{flags}}.
\item
  \texttt{mode} \{string\textbar integer\} \textbf{Default:}
  \texttt{0o666}
\item
  Returns: \{number\}
\end{itemize}

Returns an integer representing the file descriptor.

For detailed information, see the documentation of the asynchronous
version of this API:
\hyperref[fsopenpath-flags-mode-callback]{\texttt{fs.open()}}.

\subsubsection{\texorpdfstring{\texttt{fs.readdirSync(path{[},\ options{]})}}{fs.readdirSync(path{[}, options{]})}}\label{fs.readdirsyncpath-options}

\begin{itemize}
\tightlist
\item
  \texttt{path} \{string\textbar Buffer\textbar URL\}
\item
  \texttt{options} \{string\textbar Object\}

  \begin{itemize}
  \tightlist
  \item
    \texttt{encoding} \{string\} \textbf{Default:}
    \texttt{\textquotesingle{}utf8\textquotesingle{}}
  \item
    \texttt{withFileTypes} \{boolean\} \textbf{Default:} \texttt{false}
  \item
    \texttt{recursive} \{boolean\} If \texttt{true}, reads the contents
    of a directory recursively. In recursive mode, it will list all
    files, sub files, and directories. \textbf{Default:} \texttt{false}.
  \end{itemize}
\item
  Returns: \{string{[}{]}\textbar Buffer{[}{]}\textbar fs.Dirent{[}{]}\}
\end{itemize}

Reads the contents of the directory.

See the POSIX readdir(3) documentation for more details.

The optional \texttt{options} argument can be a string specifying an
encoding, or an object with an \texttt{encoding} property specifying the
character encoding to use for the filenames returned. If the
\texttt{encoding} is set to
\texttt{\textquotesingle{}buffer\textquotesingle{}}, the filenames
returned will be passed as \{Buffer\} objects.

If \texttt{options.withFileTypes} is set to \texttt{true}, the result
will contain \{fs.Dirent\} objects.

\subsubsection{\texorpdfstring{\texttt{fs.readFileSync(path{[},\ options{]})}}{fs.readFileSync(path{[}, options{]})}}\label{fs.readfilesyncpath-options}

\begin{itemize}
\tightlist
\item
  \texttt{path} \{string\textbar Buffer\textbar URL\textbar integer\}
  filename or file descriptor
\item
  \texttt{options} \{Object\textbar string\}

  \begin{itemize}
  \tightlist
  \item
    \texttt{encoding} \{string\textbar null\} \textbf{Default:}
    \texttt{null}
  \item
    \texttt{flag} \{string\} See \hyperref[file-system-flags]{support of
    file system \texttt{flags}}. \textbf{Default:}
    \texttt{\textquotesingle{}r\textquotesingle{}}.
  \end{itemize}
\item
  Returns: \{string\textbar Buffer\}
\end{itemize}

Returns the contents of the \texttt{path}.

For detailed information, see the documentation of the asynchronous
version of this API:
\hyperref[fsreadfilepath-options-callback]{\texttt{fs.readFile()}}.

If the \texttt{encoding} option is specified then this function returns
a string. Otherwise it returns a buffer.

Similar to
\hyperref[fsreadfilepath-options-callback]{\texttt{fs.readFile()}}, when
the path is a directory, the behavior of \texttt{fs.readFileSync()} is
platform-specific.

\begin{Shaded}
\begin{Highlighting}[]
\ImportTok{import}\NormalTok{ \{ readFileSync \} }\ImportTok{from} \StringTok{\textquotesingle{}node:fs\textquotesingle{}}\OperatorTok{;}

\CommentTok{// macOS, Linux, and Windows}
\FunctionTok{readFileSync}\NormalTok{(}\StringTok{\textquotesingle{}\textless{}directory\textgreater{}\textquotesingle{}}\NormalTok{)}\OperatorTok{;}
\CommentTok{// =\textgreater{} [Error: EISDIR: illegal operation on a directory, read \textless{}directory\textgreater{}]}

\CommentTok{//  FreeBSD}
\FunctionTok{readFileSync}\NormalTok{(}\StringTok{\textquotesingle{}\textless{}directory\textgreater{}\textquotesingle{}}\NormalTok{)}\OperatorTok{;} \CommentTok{// =\textgreater{} \textless{}data\textgreater{}}
\end{Highlighting}
\end{Shaded}

\subsubsection{\texorpdfstring{\texttt{fs.readlinkSync(path{[},\ options{]})}}{fs.readlinkSync(path{[}, options{]})}}\label{fs.readlinksyncpath-options}

\begin{itemize}
\tightlist
\item
  \texttt{path} \{string\textbar Buffer\textbar URL\}
\item
  \texttt{options} \{string\textbar Object\}

  \begin{itemize}
  \tightlist
  \item
    \texttt{encoding} \{string\} \textbf{Default:}
    \texttt{\textquotesingle{}utf8\textquotesingle{}}
  \end{itemize}
\item
  Returns: \{string\textbar Buffer\}
\end{itemize}

Returns the symbolic link's string value.

See the POSIX readlink(2) documentation for more details.

The optional \texttt{options} argument can be a string specifying an
encoding, or an object with an \texttt{encoding} property specifying the
character encoding to use for the link path returned. If the
\texttt{encoding} is set to
\texttt{\textquotesingle{}buffer\textquotesingle{}}, the link path
returned will be passed as a \{Buffer\} object.

\subsubsection{\texorpdfstring{\texttt{fs.readSync(fd,\ buffer,\ offset,\ length{[},\ position{]})}}{fs.readSync(fd, buffer, offset, length{[}, position{]})}}\label{fs.readsyncfd-buffer-offset-length-position}

\begin{itemize}
\tightlist
\item
  \texttt{fd} \{integer\}
\item
  \texttt{buffer} \{Buffer\textbar TypedArray\textbar DataView\}
\item
  \texttt{offset} \{integer\}
\item
  \texttt{length} \{integer\}
\item
  \texttt{position} \{integer\textbar bigint\textbar null\}
  \textbf{Default:} \texttt{null}
\item
  Returns: \{number\}
\end{itemize}

Returns the number of \texttt{bytesRead}.

For detailed information, see the documentation of the asynchronous
version of this API:
\hyperref[fsreadfd-buffer-offset-length-position-callback]{\texttt{fs.read()}}.

\subsubsection{\texorpdfstring{\texttt{fs.readSync(fd,\ buffer{[},\ options{]})}}{fs.readSync(fd, buffer{[}, options{]})}}\label{fs.readsyncfd-buffer-options}

\begin{itemize}
\tightlist
\item
  \texttt{fd} \{integer\}
\item
  \texttt{buffer} \{Buffer\textbar TypedArray\textbar DataView\}
\item
  \texttt{options} \{Object\}

  \begin{itemize}
  \tightlist
  \item
    \texttt{offset} \{integer\} \textbf{Default:} \texttt{0}
  \item
    \texttt{length} \{integer\} \textbf{Default:}
    \texttt{buffer.byteLength\ -\ offset}
  \item
    \texttt{position} \{integer\textbar bigint\textbar null\}
    \textbf{Default:} \texttt{null}
  \end{itemize}
\item
  Returns: \{number\}
\end{itemize}

Returns the number of \texttt{bytesRead}.

Similar to the above \texttt{fs.readSync} function, this version takes
an optional \texttt{options} object. If no \texttt{options} object is
specified, it will default with the above values.

For detailed information, see the documentation of the asynchronous
version of this API:
\hyperref[fsreadfd-buffer-offset-length-position-callback]{\texttt{fs.read()}}.

\subsubsection{\texorpdfstring{\texttt{fs.readvSync(fd,\ buffers{[},\ position{]})}}{fs.readvSync(fd, buffers{[}, position{]})}}\label{fs.readvsyncfd-buffers-position}

\begin{itemize}
\tightlist
\item
  \texttt{fd} \{integer\}
\item
  \texttt{buffers} \{ArrayBufferView{[}{]}\}
\item
  \texttt{position} \{integer\textbar null\} \textbf{Default:}
  \texttt{null}
\item
  Returns: \{number\} The number of bytes read.
\end{itemize}

For detailed information, see the documentation of the asynchronous
version of this API:
\hyperref[fsreadvfd-buffers-position-callback]{\texttt{fs.readv()}}.

\subsubsection{\texorpdfstring{\texttt{fs.realpathSync(path{[},\ options{]})}}{fs.realpathSync(path{[}, options{]})}}\label{fs.realpathsyncpath-options}

\begin{itemize}
\tightlist
\item
  \texttt{path} \{string\textbar Buffer\textbar URL\}
\item
  \texttt{options} \{string\textbar Object\}

  \begin{itemize}
  \tightlist
  \item
    \texttt{encoding} \{string\} \textbf{Default:}
    \texttt{\textquotesingle{}utf8\textquotesingle{}}
  \end{itemize}
\item
  Returns: \{string\textbar Buffer\}
\end{itemize}

Returns the resolved pathname.

For detailed information, see the documentation of the asynchronous
version of this API:
\hyperref[fsrealpathpath-options-callback]{\texttt{fs.realpath()}}.

\subsubsection{\texorpdfstring{\texttt{fs.realpathSync.native(path{[},\ options{]})}}{fs.realpathSync.native(path{[}, options{]})}}\label{fs.realpathsync.nativepath-options}

\begin{itemize}
\tightlist
\item
  \texttt{path} \{string\textbar Buffer\textbar URL\}
\item
  \texttt{options} \{string\textbar Object\}

  \begin{itemize}
  \tightlist
  \item
    \texttt{encoding} \{string\} \textbf{Default:}
    \texttt{\textquotesingle{}utf8\textquotesingle{}}
  \end{itemize}
\item
  Returns: \{string\textbar Buffer\}
\end{itemize}

Synchronous realpath(3).

Only paths that can be converted to UTF8 strings are supported.

The optional \texttt{options} argument can be a string specifying an
encoding, or an object with an \texttt{encoding} property specifying the
character encoding to use for the path returned. If the
\texttt{encoding} is set to
\texttt{\textquotesingle{}buffer\textquotesingle{}}, the path returned
will be passed as a \{Buffer\} object.

On Linux, when Node.js is linked against musl libc, the procfs file
system must be mounted on \texttt{/proc} in order for this function to
work. Glibc does not have this restriction.

\subsubsection{\texorpdfstring{\texttt{fs.renameSync(oldPath,\ newPath)}}{fs.renameSync(oldPath, newPath)}}\label{fs.renamesyncoldpath-newpath}

\begin{itemize}
\tightlist
\item
  \texttt{oldPath} \{string\textbar Buffer\textbar URL\}
\item
  \texttt{newPath} \{string\textbar Buffer\textbar URL\}
\end{itemize}

Renames the file from \texttt{oldPath} to \texttt{newPath}. Returns
\texttt{undefined}.

See the POSIX rename(2) documentation for more details.

\subsubsection{\texorpdfstring{\texttt{fs.rmdirSync(path{[},\ options{]})}}{fs.rmdirSync(path{[}, options{]})}}\label{fs.rmdirsyncpath-options}

\begin{itemize}
\tightlist
\item
  \texttt{path} \{string\textbar Buffer\textbar URL\}
\item
  \texttt{options} \{Object\}

  \begin{itemize}
  \tightlist
  \item
    \texttt{maxRetries} \{integer\} If an \texttt{EBUSY},
    \texttt{EMFILE}, \texttt{ENFILE}, \texttt{ENOTEMPTY}, or
    \texttt{EPERM} error is encountered, Node.js retries the operation
    with a linear backoff wait of \texttt{retryDelay} milliseconds
    longer on each try. This option represents the number of retries.
    This option is ignored if the \texttt{recursive} option is not
    \texttt{true}. \textbf{Default:} \texttt{0}.
  \item
    \texttt{recursive} \{boolean\} If \texttt{true}, perform a recursive
    directory removal. In recursive mode, operations are retried on
    failure. \textbf{Default:} \texttt{false}. \textbf{Deprecated.}
  \item
    \texttt{retryDelay} \{integer\} The amount of time in milliseconds
    to wait between retries. This option is ignored if the
    \texttt{recursive} option is not \texttt{true}. \textbf{Default:}
    \texttt{100}.
  \end{itemize}
\end{itemize}

Synchronous rmdir(2). Returns \texttt{undefined}.

Using \texttt{fs.rmdirSync()} on a file (not a directory) results in an
\texttt{ENOENT} error on Windows and an \texttt{ENOTDIR} error on POSIX.

To get a behavior similar to the \texttt{rm\ -rf} Unix command, use
\hyperref[fsrmsyncpath-options]{\texttt{fs.rmSync()}} with options
\texttt{\{\ recursive:\ true,\ force:\ true\ \}}.

\subsubsection{\texorpdfstring{\texttt{fs.rmSync(path{[},\ options{]})}}{fs.rmSync(path{[}, options{]})}}\label{fs.rmsyncpath-options}

\begin{itemize}
\tightlist
\item
  \texttt{path} \{string\textbar Buffer\textbar URL\}
\item
  \texttt{options} \{Object\}

  \begin{itemize}
  \tightlist
  \item
    \texttt{force} \{boolean\} When \texttt{true}, exceptions will be
    ignored if \texttt{path} does not exist. \textbf{Default:}
    \texttt{false}.
  \item
    \texttt{maxRetries} \{integer\} If an \texttt{EBUSY},
    \texttt{EMFILE}, \texttt{ENFILE}, \texttt{ENOTEMPTY}, or
    \texttt{EPERM} error is encountered, Node.js will retry the
    operation with a linear backoff wait of \texttt{retryDelay}
    milliseconds longer on each try. This option represents the number
    of retries. This option is ignored if the \texttt{recursive} option
    is not \texttt{true}. \textbf{Default:} \texttt{0}.
  \item
    \texttt{recursive} \{boolean\} If \texttt{true}, perform a recursive
    directory removal. In recursive mode operations are retried on
    failure. \textbf{Default:} \texttt{false}.
  \item
    \texttt{retryDelay} \{integer\} The amount of time in milliseconds
    to wait between retries. This option is ignored if the
    \texttt{recursive} option is not \texttt{true}. \textbf{Default:}
    \texttt{100}.
  \end{itemize}
\end{itemize}

Synchronously removes files and directories (modeled on the standard
POSIX \texttt{rm} utility). Returns \texttt{undefined}.

\subsubsection{\texorpdfstring{\texttt{fs.statSync(path{[},\ options{]})}}{fs.statSync(path{[}, options{]})}}\label{fs.statsyncpath-options}

\begin{itemize}
\tightlist
\item
  \texttt{path} \{string\textbar Buffer\textbar URL\}
\item
  \texttt{options} \{Object\}

  \begin{itemize}
  \tightlist
  \item
    \texttt{bigint} \{boolean\} Whether the numeric values in the
    returned \{fs.Stats\} object should be \texttt{bigint}.
    \textbf{Default:} \texttt{false}.
  \item
    \texttt{throwIfNoEntry} \{boolean\} Whether an exception will be
    thrown if no file system entry exists, rather than returning
    \texttt{undefined}. \textbf{Default:} \texttt{true}.
  \end{itemize}
\item
  Returns: \{fs.Stats\}
\end{itemize}

Retrieves the \{fs.Stats\} for the path.

\subsubsection{\texorpdfstring{\texttt{fs.statfsSync(path{[},\ options{]})}}{fs.statfsSync(path{[}, options{]})}}\label{fs.statfssyncpath-options}

\begin{itemize}
\tightlist
\item
  \texttt{path} \{string\textbar Buffer\textbar URL\}
\item
  \texttt{options} \{Object\}

  \begin{itemize}
  \tightlist
  \item
    \texttt{bigint} \{boolean\} Whether the numeric values in the
    returned \{fs.StatFs\} object should be \texttt{bigint}.
    \textbf{Default:} \texttt{false}.
  \end{itemize}
\item
  Returns: \{fs.StatFs\}
\end{itemize}

Synchronous statfs(2). Returns information about the mounted file system
which contains \texttt{path}.

In case of an error, the \texttt{err.code} will be one of
\href{errors.md\#common-system-errors}{Common System Errors}.

\subsubsection{\texorpdfstring{\texttt{fs.symlinkSync(target,\ path{[},\ type{]})}}{fs.symlinkSync(target, path{[}, type{]})}}\label{fs.symlinksynctarget-path-type}

\begin{itemize}
\tightlist
\item
  \texttt{target} \{string\textbar Buffer\textbar URL\}
\item
  \texttt{path} \{string\textbar Buffer\textbar URL\}
\item
  \texttt{type} \{string\textbar null\} \textbf{Default:} \texttt{null}
\end{itemize}

Returns \texttt{undefined}.

For detailed information, see the documentation of the asynchronous
version of this API:
\hyperref[fssymlinktarget-path-type-callback]{\texttt{fs.symlink()}}.

\subsubsection{\texorpdfstring{\texttt{fs.truncateSync(path{[},\ len{]})}}{fs.truncateSync(path{[}, len{]})}}\label{fs.truncatesyncpath-len}

\begin{itemize}
\tightlist
\item
  \texttt{path} \{string\textbar Buffer\textbar URL\}
\item
  \texttt{len} \{integer\} \textbf{Default:} \texttt{0}
\end{itemize}

Truncates the file. Returns \texttt{undefined}. A file descriptor can
also be passed as the first argument. In this case,
\texttt{fs.ftruncateSync()} is called.

Passing a file descriptor is deprecated and may result in an error being
thrown in the future.

\subsubsection{\texorpdfstring{\texttt{fs.unlinkSync(path)}}{fs.unlinkSync(path)}}\label{fs.unlinksyncpath}

\begin{itemize}
\tightlist
\item
  \texttt{path} \{string\textbar Buffer\textbar URL\}
\end{itemize}

Synchronous unlink(2). Returns \texttt{undefined}.

\subsubsection{\texorpdfstring{\texttt{fs.utimesSync(path,\ atime,\ mtime)}}{fs.utimesSync(path, atime, mtime)}}\label{fs.utimessyncpath-atime-mtime}

\begin{itemize}
\tightlist
\item
  \texttt{path} \{string\textbar Buffer\textbar URL\}
\item
  \texttt{atime} \{number\textbar string\textbar Date\}
\item
  \texttt{mtime} \{number\textbar string\textbar Date\}
\end{itemize}

Returns \texttt{undefined}.

For detailed information, see the documentation of the asynchronous
version of this API:
\hyperref[fsutimespath-atime-mtime-callback]{\texttt{fs.utimes()}}.

\subsubsection{\texorpdfstring{\texttt{fs.writeFileSync(file,\ data{[},\ options{]})}}{fs.writeFileSync(file, data{[}, options{]})}}\label{fs.writefilesyncfile-data-options}

\begin{itemize}
\tightlist
\item
  \texttt{file} \{string\textbar Buffer\textbar URL\textbar integer\}
  filename or file descriptor
\item
  \texttt{data}
  \{string\textbar Buffer\textbar TypedArray\textbar DataView\}
\item
  \texttt{options} \{Object\textbar string\}

  \begin{itemize}
  \tightlist
  \item
    \texttt{encoding} \{string\textbar null\} \textbf{Default:}
    \texttt{\textquotesingle{}utf8\textquotesingle{}}
  \item
    \texttt{mode} \{integer\} \textbf{Default:} \texttt{0o666}
  \item
    \texttt{flag} \{string\} See \hyperref[file-system-flags]{support of
    file system \texttt{flags}}. \textbf{Default:}
    \texttt{\textquotesingle{}w\textquotesingle{}}.
  \item
    \texttt{flush} \{boolean\} If all data is successfully written to
    the file, and \texttt{flush} is \texttt{true},
    \texttt{fs.fsyncSync()} is used to flush the data.
  \end{itemize}
\end{itemize}

Returns \texttt{undefined}.

The \texttt{mode} option only affects the newly created file. See
\hyperref[fsopenpath-flags-mode-callback]{\texttt{fs.open()}} for more
details.

For detailed information, see the documentation of the asynchronous
version of this API:
\hyperref[fswritefilefile-data-options-callback]{\texttt{fs.writeFile()}}.

\subsubsection{\texorpdfstring{\texttt{fs.writeSync(fd,\ buffer,\ offset{[},\ length{[},\ position{]}{]})}}{fs.writeSync(fd, buffer, offset{[}, length{[}, position{]}{]})}}\label{fs.writesyncfd-buffer-offset-length-position}

\begin{itemize}
\tightlist
\item
  \texttt{fd} \{integer\}
\item
  \texttt{buffer} \{Buffer\textbar TypedArray\textbar DataView\}
\item
  \texttt{offset} \{integer\} \textbf{Default:} \texttt{0}
\item
  \texttt{length} \{integer\} \textbf{Default:}
  \texttt{buffer.byteLength\ -\ offset}
\item
  \texttt{position} \{integer\textbar null\} \textbf{Default:}
  \texttt{null}
\item
  Returns: \{number\} The number of bytes written.
\end{itemize}

For detailed information, see the documentation of the asynchronous
version of this API:
\hyperref[fswritefd-buffer-offset-length-position-callback]{\texttt{fs.write(fd,\ buffer...)}}.

\subsubsection{\texorpdfstring{\texttt{fs.writeSync(fd,\ buffer{[},\ options{]})}}{fs.writeSync(fd, buffer{[}, options{]})}}\label{fs.writesyncfd-buffer-options}

\begin{itemize}
\tightlist
\item
  \texttt{fd} \{integer\}
\item
  \texttt{buffer} \{Buffer\textbar TypedArray\textbar DataView\}
\item
  \texttt{options} \{Object\}

  \begin{itemize}
  \tightlist
  \item
    \texttt{offset} \{integer\} \textbf{Default:} \texttt{0}
  \item
    \texttt{length} \{integer\} \textbf{Default:}
    \texttt{buffer.byteLength\ -\ offset}
  \item
    \texttt{position} \{integer\} \textbf{Default:} \texttt{null}
  \end{itemize}
\item
  Returns: \{number\} The number of bytes written.
\end{itemize}

For detailed information, see the documentation of the asynchronous
version of this API:
\hyperref[fswritefd-buffer-offset-length-position-callback]{\texttt{fs.write(fd,\ buffer...)}}.

\subsubsection{\texorpdfstring{\texttt{fs.writeSync(fd,\ string{[},\ position{[},\ encoding{]}{]})}}{fs.writeSync(fd, string{[}, position{[}, encoding{]}{]})}}\label{fs.writesyncfd-string-position-encoding}

\begin{itemize}
\tightlist
\item
  \texttt{fd} \{integer\}
\item
  \texttt{string} \{string\}
\item
  \texttt{position} \{integer\textbar null\} \textbf{Default:}
  \texttt{null}
\item
  \texttt{encoding} \{string\} \textbf{Default:}
  \texttt{\textquotesingle{}utf8\textquotesingle{}}
\item
  Returns: \{number\} The number of bytes written.
\end{itemize}

For detailed information, see the documentation of the asynchronous
version of this API:
\hyperref[fswritefd-string-position-encoding-callback]{\texttt{fs.write(fd,\ string...)}}.

\subsubsection{\texorpdfstring{\texttt{fs.writevSync(fd,\ buffers{[},\ position{]})}}{fs.writevSync(fd, buffers{[}, position{]})}}\label{fs.writevsyncfd-buffers-position}

\begin{itemize}
\tightlist
\item
  \texttt{fd} \{integer\}
\item
  \texttt{buffers} \{ArrayBufferView{[}{]}\}
\item
  \texttt{position} \{integer\textbar null\} \textbf{Default:}
  \texttt{null}
\item
  Returns: \{number\} The number of bytes written.
\end{itemize}

For detailed information, see the documentation of the asynchronous
version of this API:
\hyperref[fswritevfd-buffers-position-callback]{\texttt{fs.writev()}}.

\subsection{Common Objects}\label{common-objects}

The common objects are shared by all of the file system API variants
(promise, callback, and synchronous).

\subsubsection{\texorpdfstring{Class:
\texttt{fs.Dir}}{Class: fs.Dir}}\label{class-fs.dir}

A class representing a directory stream.

Created by
\hyperref[fsopendirpath-options-callback]{\texttt{fs.opendir()}},
\hyperref[fsopendirsyncpath-options]{\texttt{fs.opendirSync()}}, or
\hyperref[fspromisesopendirpath-options]{\texttt{fsPromises.opendir()}}.

\begin{Shaded}
\begin{Highlighting}[]
\ImportTok{import}\NormalTok{ \{ opendir \} }\ImportTok{from} \StringTok{\textquotesingle{}node:fs/promises\textquotesingle{}}\OperatorTok{;}

\ControlFlowTok{try}\NormalTok{ \{}
  \KeywordTok{const}\NormalTok{ dir }\OperatorTok{=} \ControlFlowTok{await} \FunctionTok{opendir}\NormalTok{(}\StringTok{\textquotesingle{}./\textquotesingle{}}\NormalTok{)}\OperatorTok{;}
  \ControlFlowTok{for} \ControlFlowTok{await}\NormalTok{ (}\KeywordTok{const}\NormalTok{ dirent }\KeywordTok{of}\NormalTok{ dir)}
    \BuiltInTok{console}\OperatorTok{.}\FunctionTok{log}\NormalTok{(dirent}\OperatorTok{.}\AttributeTok{name}\NormalTok{)}\OperatorTok{;}
\NormalTok{\} }\ControlFlowTok{catch}\NormalTok{ (err) \{}
  \BuiltInTok{console}\OperatorTok{.}\FunctionTok{error}\NormalTok{(err)}\OperatorTok{;}
\NormalTok{\}}
\end{Highlighting}
\end{Shaded}

When using the async iterator, the \{fs.Dir\} object will be
automatically closed after the iterator exits.

\paragraph{\texorpdfstring{\texttt{dir.close()}}{dir.close()}}\label{dir.close}

\begin{itemize}
\tightlist
\item
  Returns: \{Promise\}
\end{itemize}

Asynchronously close the directory's underlying resource handle.
Subsequent reads will result in errors.

A promise is returned that will be fulfilled after the resource has been
closed.

\paragraph{\texorpdfstring{\texttt{dir.close(callback)}}{dir.close(callback)}}\label{dir.closecallback}

\begin{itemize}
\tightlist
\item
  \texttt{callback} \{Function\}

  \begin{itemize}
  \tightlist
  \item
    \texttt{err} \{Error\}
  \end{itemize}
\end{itemize}

Asynchronously close the directory's underlying resource handle.
Subsequent reads will result in errors.

The \texttt{callback} will be called after the resource handle has been
closed.

\paragraph{\texorpdfstring{\texttt{dir.closeSync()}}{dir.closeSync()}}\label{dir.closesync}

Synchronously close the directory's underlying resource handle.
Subsequent reads will result in errors.

\paragraph{\texorpdfstring{\texttt{dir.path}}{dir.path}}\label{dir.path}

\begin{itemize}
\tightlist
\item
  \{string\}
\end{itemize}

The read-only path of this directory as was provided to
\hyperref[fsopendirpath-options-callback]{\texttt{fs.opendir()}},
\hyperref[fsopendirsyncpath-options]{\texttt{fs.opendirSync()}}, or
\hyperref[fspromisesopendirpath-options]{\texttt{fsPromises.opendir()}}.

\paragraph{\texorpdfstring{\texttt{dir.read()}}{dir.read()}}\label{dir.read}

\begin{itemize}
\tightlist
\item
  Returns: \{Promise\} containing \{fs.Dirent\textbar null\}
\end{itemize}

Asynchronously read the next directory entry via readdir(3) as an
\{fs.Dirent\}.

A promise is returned that will be fulfilled with an \{fs.Dirent\}, or
\texttt{null} if there are no more directory entries to read.

Directory entries returned by this function are in no particular order
as provided by the operating system's underlying directory mechanisms.
Entries added or removed while iterating over the directory might not be
included in the iteration results.

\paragraph{\texorpdfstring{\texttt{dir.read(callback)}}{dir.read(callback)}}\label{dir.readcallback}

\begin{itemize}
\tightlist
\item
  \texttt{callback} \{Function\}

  \begin{itemize}
  \tightlist
  \item
    \texttt{err} \{Error\}
  \item
    \texttt{dirent} \{fs.Dirent\textbar null\}
  \end{itemize}
\end{itemize}

Asynchronously read the next directory entry via readdir(3) as an
\{fs.Dirent\}.

After the read is completed, the \texttt{callback} will be called with
an \{fs.Dirent\}, or \texttt{null} if there are no more directory
entries to read.

Directory entries returned by this function are in no particular order
as provided by the operating system's underlying directory mechanisms.
Entries added or removed while iterating over the directory might not be
included in the iteration results.

\paragraph{\texorpdfstring{\texttt{dir.readSync()}}{dir.readSync()}}\label{dir.readsync}

\begin{itemize}
\tightlist
\item
  Returns: \{fs.Dirent\textbar null\}
\end{itemize}

Synchronously read the next directory entry as an \{fs.Dirent\}. See the
POSIX readdir(3) documentation for more detail.

If there are no more directory entries to read, \texttt{null} will be
returned.

Directory entries returned by this function are in no particular order
as provided by the operating system's underlying directory mechanisms.
Entries added or removed while iterating over the directory might not be
included in the iteration results.

\paragraph{\texorpdfstring{\texttt{dir{[}Symbol.asyncIterator{]}()}}{dir{[}Symbol.asyncIterator{]}()}}\label{dirsymbol.asynciterator}

\begin{itemize}
\tightlist
\item
  Returns: \{AsyncIterator\} of \{fs.Dirent\}
\end{itemize}

Asynchronously iterates over the directory until all entries have been
read. Refer to the POSIX readdir(3) documentation for more detail.

Entries returned by the async iterator are always an \{fs.Dirent\}. The
\texttt{null} case from \texttt{dir.read()} is handled internally.

See \{fs.Dir\} for an example.

Directory entries returned by this iterator are in no particular order
as provided by the operating system's underlying directory mechanisms.
Entries added or removed while iterating over the directory might not be
included in the iteration results.

\subsubsection{\texorpdfstring{Class:
\texttt{fs.Dirent}}{Class: fs.Dirent}}\label{class-fs.dirent}

A representation of a directory entry, which can be a file or a
subdirectory within the directory, as returned by reading from an
\{fs.Dir\}. The directory entry is a combination of the file name and
file type pairs.

Additionally, when
\hyperref[fsreaddirpath-options-callback]{\texttt{fs.readdir()}} or
\hyperref[fsreaddirsyncpath-options]{\texttt{fs.readdirSync()}} is
called with the \texttt{withFileTypes} option set to \texttt{true}, the
resulting array is filled with \{fs.Dirent\} objects, rather than
strings or \{Buffer\}s.

\paragraph{\texorpdfstring{\texttt{dirent.isBlockDevice()}}{dirent.isBlockDevice()}}\label{dirent.isblockdevice}

\begin{itemize}
\tightlist
\item
  Returns: \{boolean\}
\end{itemize}

Returns \texttt{true} if the \{fs.Dirent\} object describes a block
device.

\paragraph{\texorpdfstring{\texttt{dirent.isCharacterDevice()}}{dirent.isCharacterDevice()}}\label{dirent.ischaracterdevice}

\begin{itemize}
\tightlist
\item
  Returns: \{boolean\}
\end{itemize}

Returns \texttt{true} if the \{fs.Dirent\} object describes a character
device.

\paragraph{\texorpdfstring{\texttt{dirent.isDirectory()}}{dirent.isDirectory()}}\label{dirent.isdirectory}

\begin{itemize}
\tightlist
\item
  Returns: \{boolean\}
\end{itemize}

Returns \texttt{true} if the \{fs.Dirent\} object describes a file
system directory.

\paragraph{\texorpdfstring{\texttt{dirent.isFIFO()}}{dirent.isFIFO()}}\label{dirent.isfifo}

\begin{itemize}
\tightlist
\item
  Returns: \{boolean\}
\end{itemize}

Returns \texttt{true} if the \{fs.Dirent\} object describes a
first-in-first-out (FIFO) pipe.

\paragraph{\texorpdfstring{\texttt{dirent.isFile()}}{dirent.isFile()}}\label{dirent.isfile}

\begin{itemize}
\tightlist
\item
  Returns: \{boolean\}
\end{itemize}

Returns \texttt{true} if the \{fs.Dirent\} object describes a regular
file.

\paragraph{\texorpdfstring{\texttt{dirent.isSocket()}}{dirent.isSocket()}}\label{dirent.issocket}

\begin{itemize}
\tightlist
\item
  Returns: \{boolean\}
\end{itemize}

Returns \texttt{true} if the \{fs.Dirent\} object describes a socket.

\paragraph{\texorpdfstring{\texttt{dirent.isSymbolicLink()}}{dirent.isSymbolicLink()}}\label{dirent.issymboliclink}

\begin{itemize}
\tightlist
\item
  Returns: \{boolean\}
\end{itemize}

Returns \texttt{true} if the \{fs.Dirent\} object describes a symbolic
link.

\paragraph{\texorpdfstring{\texttt{dirent.name}}{dirent.name}}\label{dirent.name}

\begin{itemize}
\tightlist
\item
  \{string\textbar Buffer\}
\end{itemize}

The file name that this \{fs.Dirent\} object refers to. The type of this
value is determined by the \texttt{options.encoding} passed to
\hyperref[fsreaddirpath-options-callback]{\texttt{fs.readdir()}} or
\hyperref[fsreaddirsyncpath-options]{\texttt{fs.readdirSync()}}.

\paragraph{\texorpdfstring{\texttt{dirent.parentPath}}{dirent.parentPath}}\label{dirent.parentpath}

\begin{quote}
Stability: 1 -- Experimental
\end{quote}

\begin{itemize}
\tightlist
\item
  \{string\}
\end{itemize}

The path to the parent directory of the file this \{fs.Dirent\} object
refers to.

\paragraph{\texorpdfstring{\texttt{dirent.path}}{dirent.path}}\label{dirent.path}

\begin{quote}
Stability: 0 - Deprecated: Use
\hyperref[direntparentpath]{\texttt{dirent.parentPath}} instead.
\end{quote}

\begin{itemize}
\tightlist
\item
  \{string\}
\end{itemize}

Alias for \texttt{dirent.parentPath}.

\subsubsection{\texorpdfstring{Class:
\texttt{fs.FSWatcher}}{Class: fs.FSWatcher}}\label{class-fs.fswatcher}

\begin{itemize}
\tightlist
\item
  Extends \{EventEmitter\}
\end{itemize}

A successful call to
\hyperref[fswatchfilename-options-listener]{\texttt{fs.watch()}} method
will return a new \{fs.FSWatcher\} object.

All \{fs.FSWatcher\} objects emit a
\texttt{\textquotesingle{}change\textquotesingle{}} event whenever a
specific watched file is modified.

\paragraph{\texorpdfstring{Event:
\texttt{\textquotesingle{}change\textquotesingle{}}}{Event: \textquotesingle change\textquotesingle{}}}\label{event-change}

\begin{itemize}
\tightlist
\item
  \texttt{eventType} \{string\} The type of change event that has
  occurred
\item
  \texttt{filename} \{string\textbar Buffer\} The filename that changed
  (if relevant/available)
\end{itemize}

Emitted when something changes in a watched directory or file. See more
details in
\hyperref[fswatchfilename-options-listener]{\texttt{fs.watch()}}.

The \texttt{filename} argument may not be provided depending on
operating system support. If \texttt{filename} is provided, it will be
provided as a \{Buffer\} if \texttt{fs.watch()} is called with its
\texttt{encoding} option set to
\texttt{\textquotesingle{}buffer\textquotesingle{}}, otherwise
\texttt{filename} will be a UTF-8 string.

\begin{Shaded}
\begin{Highlighting}[]
\ImportTok{import}\NormalTok{ \{ watch \} }\ImportTok{from} \StringTok{\textquotesingle{}node:fs\textquotesingle{}}\OperatorTok{;}
\CommentTok{// Example when handled through fs.watch() listener}
\FunctionTok{watch}\NormalTok{(}\StringTok{\textquotesingle{}./tmp\textquotesingle{}}\OperatorTok{,}\NormalTok{ \{ }\DataTypeTok{encoding}\OperatorTok{:} \StringTok{\textquotesingle{}buffer\textquotesingle{}}\NormalTok{ \}}\OperatorTok{,}\NormalTok{ (eventType}\OperatorTok{,}\NormalTok{ filename) }\KeywordTok{=\textgreater{}}\NormalTok{ \{}
  \ControlFlowTok{if}\NormalTok{ (filename) \{}
    \BuiltInTok{console}\OperatorTok{.}\FunctionTok{log}\NormalTok{(filename)}\OperatorTok{;}
    \CommentTok{// Prints: \textless{}Buffer ...\textgreater{}}
\NormalTok{  \}}
\NormalTok{\})}\OperatorTok{;}
\end{Highlighting}
\end{Shaded}

\paragraph{\texorpdfstring{Event:
\texttt{\textquotesingle{}close\textquotesingle{}}}{Event: \textquotesingle close\textquotesingle{}}}\label{event-close-1}

Emitted when the watcher stops watching for changes. The closed
\{fs.FSWatcher\} object is no longer usable in the event handler.

\paragraph{\texorpdfstring{Event:
\texttt{\textquotesingle{}error\textquotesingle{}}}{Event: \textquotesingle error\textquotesingle{}}}\label{event-error}

\begin{itemize}
\tightlist
\item
  \texttt{error} \{Error\}
\end{itemize}

Emitted when an error occurs while watching the file. The errored
\{fs.FSWatcher\} object is no longer usable in the event handler.

\paragraph{\texorpdfstring{\texttt{watcher.close()}}{watcher.close()}}\label{watcher.close}

Stop watching for changes on the given \{fs.FSWatcher\}. Once stopped,
the \{fs.FSWatcher\} object is no longer usable.

\paragraph{\texorpdfstring{\texttt{watcher.ref()}}{watcher.ref()}}\label{watcher.ref}

\begin{itemize}
\tightlist
\item
  Returns: \{fs.FSWatcher\}
\end{itemize}

When called, requests that the Node.js event loop \emph{not} exit so
long as the \{fs.FSWatcher\} is active. Calling \texttt{watcher.ref()}
multiple times will have no effect.

By default, all \{fs.FSWatcher\} objects are ``ref'ed'', making it
normally unnecessary to call \texttt{watcher.ref()} unless
\texttt{watcher.unref()} had been called previously.

\paragraph{\texorpdfstring{\texttt{watcher.unref()}}{watcher.unref()}}\label{watcher.unref}

\begin{itemize}
\tightlist
\item
  Returns: \{fs.FSWatcher\}
\end{itemize}

When called, the active \{fs.FSWatcher\} object will not require the
Node.js event loop to remain active. If there is no other activity
keeping the event loop running, the process may exit before the
\{fs.FSWatcher\} object's callback is invoked. Calling
\texttt{watcher.unref()} multiple times will have no effect.

\subsubsection{\texorpdfstring{Class:
\texttt{fs.StatWatcher}}{Class: fs.StatWatcher}}\label{class-fs.statwatcher}

\begin{itemize}
\tightlist
\item
  Extends \{EventEmitter\}
\end{itemize}

A successful call to \texttt{fs.watchFile()} method will return a new
\{fs.StatWatcher\} object.

\paragraph{\texorpdfstring{\texttt{watcher.ref()}}{watcher.ref()}}\label{watcher.ref-1}

\begin{itemize}
\tightlist
\item
  Returns: \{fs.StatWatcher\}
\end{itemize}

When called, requests that the Node.js event loop \emph{not} exit so
long as the \{fs.StatWatcher\} is active. Calling \texttt{watcher.ref()}
multiple times will have no effect.

By default, all \{fs.StatWatcher\} objects are ``ref'ed'', making it
normally unnecessary to call \texttt{watcher.ref()} unless
\texttt{watcher.unref()} had been called previously.

\paragraph{\texorpdfstring{\texttt{watcher.unref()}}{watcher.unref()}}\label{watcher.unref-1}

\begin{itemize}
\tightlist
\item
  Returns: \{fs.StatWatcher\}
\end{itemize}

When called, the active \{fs.StatWatcher\} object will not require the
Node.js event loop to remain active. If there is no other activity
keeping the event loop running, the process may exit before the
\{fs.StatWatcher\} object's callback is invoked. Calling
\texttt{watcher.unref()} multiple times will have no effect.

\subsubsection{\texorpdfstring{Class:
\texttt{fs.ReadStream}}{Class: fs.ReadStream}}\label{class-fs.readstream}

\begin{itemize}
\tightlist
\item
  Extends: \{stream.Readable\}
\end{itemize}

Instances of \{fs.ReadStream\} are created and returned using the
\hyperref[fscreatereadstreampath-options]{\texttt{fs.createReadStream()}}
function.

\paragraph{\texorpdfstring{Event:
\texttt{\textquotesingle{}close\textquotesingle{}}}{Event: \textquotesingle close\textquotesingle{}}}\label{event-close-2}

Emitted when the \{fs.ReadStream\}'s underlying file descriptor has been
closed.

\paragraph{\texorpdfstring{Event:
\texttt{\textquotesingle{}open\textquotesingle{}}}{Event: \textquotesingle open\textquotesingle{}}}\label{event-open}

\begin{itemize}
\tightlist
\item
  \texttt{fd} \{integer\} Integer file descriptor used by the
  \{fs.ReadStream\}.
\end{itemize}

Emitted when the \{fs.ReadStream\}'s file descriptor has been opened.

\paragraph{\texorpdfstring{Event:
\texttt{\textquotesingle{}ready\textquotesingle{}}}{Event: \textquotesingle ready\textquotesingle{}}}\label{event-ready}

Emitted when the \{fs.ReadStream\} is ready to be used.

Fires immediately after
\texttt{\textquotesingle{}open\textquotesingle{}}.

\paragraph{\texorpdfstring{\texttt{readStream.bytesRead}}{readStream.bytesRead}}\label{readstream.bytesread}

\begin{itemize}
\tightlist
\item
  \{number\}
\end{itemize}

The number of bytes that have been read so far.

\paragraph{\texorpdfstring{\texttt{readStream.path}}{readStream.path}}\label{readstream.path}

\begin{itemize}
\tightlist
\item
  \{string\textbar Buffer\}
\end{itemize}

The path to the file the stream is reading from as specified in the
first argument to \texttt{fs.createReadStream()}. If \texttt{path} is
passed as a string, then \texttt{readStream.path} will be a string. If
\texttt{path} is passed as a \{Buffer\}, then \texttt{readStream.path}
will be a \{Buffer\}. If \texttt{fd} is specified, then
\texttt{readStream.path} will be \texttt{undefined}.

\paragraph{\texorpdfstring{\texttt{readStream.pending}}{readStream.pending}}\label{readstream.pending}

\begin{itemize}
\tightlist
\item
  \{boolean\}
\end{itemize}

This property is \texttt{true} if the underlying file has not been
opened yet, i.e.~before the
\texttt{\textquotesingle{}ready\textquotesingle{}} event is emitted.

\subsubsection{\texorpdfstring{Class:
\texttt{fs.Stats}}{Class: fs.Stats}}\label{class-fs.stats}

A \{fs.Stats\} object provides information about a file.

Objects returned from
\hyperref[fsstatpath-options-callback]{\texttt{fs.stat()}},
\hyperref[fslstatpath-options-callback]{\texttt{fs.lstat()}},
\hyperref[fsfstatfd-options-callback]{\texttt{fs.fstat()}}, and their
synchronous counterparts are of this type. If \texttt{bigint} in the
\texttt{options} passed to those methods is true, the numeric values
will be \texttt{bigint} instead of \texttt{number}, and the object will
contain additional nanosecond-precision properties suffixed with
\texttt{Ns}.

\begin{Shaded}
\begin{Highlighting}[]
\NormalTok{Stats \{}
\NormalTok{  dev: 2114,}
\NormalTok{  ino: 48064969,}
\NormalTok{  mode: 33188,}
\NormalTok{  nlink: 1,}
\NormalTok{  uid: 85,}
\NormalTok{  gid: 100,}
\NormalTok{  rdev: 0,}
\NormalTok{  size: 527,}
\NormalTok{  blksize: 4096,}
\NormalTok{  blocks: 8,}
\NormalTok{  atimeMs: 1318289051000.1,}
\NormalTok{  mtimeMs: 1318289051000.1,}
\NormalTok{  ctimeMs: 1318289051000.1,}
\NormalTok{  birthtimeMs: 1318289051000.1,}
\NormalTok{  atime: Mon, 10 Oct 2011 23:24:11 GMT,}
\NormalTok{  mtime: Mon, 10 Oct 2011 23:24:11 GMT,}
\NormalTok{  ctime: Mon, 10 Oct 2011 23:24:11 GMT,}
\NormalTok{  birthtime: Mon, 10 Oct 2011 23:24:11 GMT \}}
\end{Highlighting}
\end{Shaded}

\texttt{bigint} version:

\begin{Shaded}
\begin{Highlighting}[]
\NormalTok{BigIntStats \{}
\NormalTok{  dev: 2114n,}
\NormalTok{  ino: 48064969n,}
\NormalTok{  mode: 33188n,}
\NormalTok{  nlink: 1n,}
\NormalTok{  uid: 85n,}
\NormalTok{  gid: 100n,}
\NormalTok{  rdev: 0n,}
\NormalTok{  size: 527n,}
\NormalTok{  blksize: 4096n,}
\NormalTok{  blocks: 8n,}
\NormalTok{  atimeMs: 1318289051000n,}
\NormalTok{  mtimeMs: 1318289051000n,}
\NormalTok{  ctimeMs: 1318289051000n,}
\NormalTok{  birthtimeMs: 1318289051000n,}
\NormalTok{  atimeNs: 1318289051000000000n,}
\NormalTok{  mtimeNs: 1318289051000000000n,}
\NormalTok{  ctimeNs: 1318289051000000000n,}
\NormalTok{  birthtimeNs: 1318289051000000000n,}
\NormalTok{  atime: Mon, 10 Oct 2011 23:24:11 GMT,}
\NormalTok{  mtime: Mon, 10 Oct 2011 23:24:11 GMT,}
\NormalTok{  ctime: Mon, 10 Oct 2011 23:24:11 GMT,}
\NormalTok{  birthtime: Mon, 10 Oct 2011 23:24:11 GMT \}}
\end{Highlighting}
\end{Shaded}

\paragraph{\texorpdfstring{\texttt{stats.isBlockDevice()}}{stats.isBlockDevice()}}\label{stats.isblockdevice}

\begin{itemize}
\tightlist
\item
  Returns: \{boolean\}
\end{itemize}

Returns \texttt{true} if the \{fs.Stats\} object describes a block
device.

\paragraph{\texorpdfstring{\texttt{stats.isCharacterDevice()}}{stats.isCharacterDevice()}}\label{stats.ischaracterdevice}

\begin{itemize}
\tightlist
\item
  Returns: \{boolean\}
\end{itemize}

Returns \texttt{true} if the \{fs.Stats\} object describes a character
device.

\paragraph{\texorpdfstring{\texttt{stats.isDirectory()}}{stats.isDirectory()}}\label{stats.isdirectory}

\begin{itemize}
\tightlist
\item
  Returns: \{boolean\}
\end{itemize}

Returns \texttt{true} if the \{fs.Stats\} object describes a file system
directory.

If the \{fs.Stats\} object was obtained from
\hyperref[fslstatpath-options-callback]{\texttt{fs.lstat()}}, this
method will always return \texttt{false}. This is because
\hyperref[fslstatpath-options-callback]{\texttt{fs.lstat()}} returns
information about a symbolic link itself and not the path it resolves
to.

\paragraph{\texorpdfstring{\texttt{stats.isFIFO()}}{stats.isFIFO()}}\label{stats.isfifo}

\begin{itemize}
\tightlist
\item
  Returns: \{boolean\}
\end{itemize}

Returns \texttt{true} if the \{fs.Stats\} object describes a
first-in-first-out (FIFO) pipe.

\paragraph{\texorpdfstring{\texttt{stats.isFile()}}{stats.isFile()}}\label{stats.isfile}

\begin{itemize}
\tightlist
\item
  Returns: \{boolean\}
\end{itemize}

Returns \texttt{true} if the \{fs.Stats\} object describes a regular
file.

\paragraph{\texorpdfstring{\texttt{stats.isSocket()}}{stats.isSocket()}}\label{stats.issocket}

\begin{itemize}
\tightlist
\item
  Returns: \{boolean\}
\end{itemize}

Returns \texttt{true} if the \{fs.Stats\} object describes a socket.

\paragraph{\texorpdfstring{\texttt{stats.isSymbolicLink()}}{stats.isSymbolicLink()}}\label{stats.issymboliclink}

\begin{itemize}
\tightlist
\item
  Returns: \{boolean\}
\end{itemize}

Returns \texttt{true} if the \{fs.Stats\} object describes a symbolic
link.

This method is only valid when using
\hyperref[fslstatpath-options-callback]{\texttt{fs.lstat()}}.

\paragraph{\texorpdfstring{\texttt{stats.dev}}{stats.dev}}\label{stats.dev}

\begin{itemize}
\tightlist
\item
  \{number\textbar bigint\}
\end{itemize}

The numeric identifier of the device containing the file.

\paragraph{\texorpdfstring{\texttt{stats.ino}}{stats.ino}}\label{stats.ino}

\begin{itemize}
\tightlist
\item
  \{number\textbar bigint\}
\end{itemize}

The file system specific ``Inode'' number for the file.

\paragraph{\texorpdfstring{\texttt{stats.mode}}{stats.mode}}\label{stats.mode}

\begin{itemize}
\tightlist
\item
  \{number\textbar bigint\}
\end{itemize}

A bit-field describing the file type and mode.

\paragraph{\texorpdfstring{\texttt{stats.nlink}}{stats.nlink}}\label{stats.nlink}

\begin{itemize}
\tightlist
\item
  \{number\textbar bigint\}
\end{itemize}

The number of hard-links that exist for the file.

\paragraph{\texorpdfstring{\texttt{stats.uid}}{stats.uid}}\label{stats.uid}

\begin{itemize}
\tightlist
\item
  \{number\textbar bigint\}
\end{itemize}

The numeric user identifier of the user that owns the file (POSIX).

\paragraph{\texorpdfstring{\texttt{stats.gid}}{stats.gid}}\label{stats.gid}

\begin{itemize}
\tightlist
\item
  \{number\textbar bigint\}
\end{itemize}

The numeric group identifier of the group that owns the file (POSIX).

\paragraph{\texorpdfstring{\texttt{stats.rdev}}{stats.rdev}}\label{stats.rdev}

\begin{itemize}
\tightlist
\item
  \{number\textbar bigint\}
\end{itemize}

A numeric device identifier if the file represents a device.

\paragraph{\texorpdfstring{\texttt{stats.size}}{stats.size}}\label{stats.size}

\begin{itemize}
\tightlist
\item
  \{number\textbar bigint\}
\end{itemize}

The size of the file in bytes.

If the underlying file system does not support getting the size of the
file, this will be \texttt{0}.

\paragraph{\texorpdfstring{\texttt{stats.blksize}}{stats.blksize}}\label{stats.blksize}

\begin{itemize}
\tightlist
\item
  \{number\textbar bigint\}
\end{itemize}

The file system block size for i/o operations.

\paragraph{\texorpdfstring{\texttt{stats.blocks}}{stats.blocks}}\label{stats.blocks}

\begin{itemize}
\tightlist
\item
  \{number\textbar bigint\}
\end{itemize}

The number of blocks allocated for this file.

\paragraph{\texorpdfstring{\texttt{stats.atimeMs}}{stats.atimeMs}}\label{stats.atimems}

\begin{itemize}
\tightlist
\item
  \{number\textbar bigint\}
\end{itemize}

The timestamp indicating the last time this file was accessed expressed
in milliseconds since the POSIX Epoch.

\paragraph{\texorpdfstring{\texttt{stats.mtimeMs}}{stats.mtimeMs}}\label{stats.mtimems}

\begin{itemize}
\tightlist
\item
  \{number\textbar bigint\}
\end{itemize}

The timestamp indicating the last time this file was modified expressed
in milliseconds since the POSIX Epoch.

\paragraph{\texorpdfstring{\texttt{stats.ctimeMs}}{stats.ctimeMs}}\label{stats.ctimems}

\begin{itemize}
\tightlist
\item
  \{number\textbar bigint\}
\end{itemize}

The timestamp indicating the last time the file status was changed
expressed in milliseconds since the POSIX Epoch.

\paragraph{\texorpdfstring{\texttt{stats.birthtimeMs}}{stats.birthtimeMs}}\label{stats.birthtimems}

\begin{itemize}
\tightlist
\item
  \{number\textbar bigint\}
\end{itemize}

The timestamp indicating the creation time of this file expressed in
milliseconds since the POSIX Epoch.

\paragraph{\texorpdfstring{\texttt{stats.atimeNs}}{stats.atimeNs}}\label{stats.atimens}

\begin{itemize}
\tightlist
\item
  \{bigint\}
\end{itemize}

Only present when \texttt{bigint:\ true} is passed into the method that
generates the object. The timestamp indicating the last time this file
was accessed expressed in nanoseconds since the POSIX Epoch.

\paragraph{\texorpdfstring{\texttt{stats.mtimeNs}}{stats.mtimeNs}}\label{stats.mtimens}

\begin{itemize}
\tightlist
\item
  \{bigint\}
\end{itemize}

Only present when \texttt{bigint:\ true} is passed into the method that
generates the object. The timestamp indicating the last time this file
was modified expressed in nanoseconds since the POSIX Epoch.

\paragraph{\texorpdfstring{\texttt{stats.ctimeNs}}{stats.ctimeNs}}\label{stats.ctimens}

\begin{itemize}
\tightlist
\item
  \{bigint\}
\end{itemize}

Only present when \texttt{bigint:\ true} is passed into the method that
generates the object. The timestamp indicating the last time the file
status was changed expressed in nanoseconds since the POSIX Epoch.

\paragraph{\texorpdfstring{\texttt{stats.birthtimeNs}}{stats.birthtimeNs}}\label{stats.birthtimens}

\begin{itemize}
\tightlist
\item
  \{bigint\}
\end{itemize}

Only present when \texttt{bigint:\ true} is passed into the method that
generates the object. The timestamp indicating the creation time of this
file expressed in nanoseconds since the POSIX Epoch.

\paragraph{\texorpdfstring{\texttt{stats.atime}}{stats.atime}}\label{stats.atime}

\begin{itemize}
\tightlist
\item
  \{Date\}
\end{itemize}

The timestamp indicating the last time this file was accessed.

\paragraph{\texorpdfstring{\texttt{stats.mtime}}{stats.mtime}}\label{stats.mtime}

\begin{itemize}
\tightlist
\item
  \{Date\}
\end{itemize}

The timestamp indicating the last time this file was modified.

\paragraph{\texorpdfstring{\texttt{stats.ctime}}{stats.ctime}}\label{stats.ctime}

\begin{itemize}
\tightlist
\item
  \{Date\}
\end{itemize}

The timestamp indicating the last time the file status was changed.

\paragraph{\texorpdfstring{\texttt{stats.birthtime}}{stats.birthtime}}\label{stats.birthtime}

\begin{itemize}
\tightlist
\item
  \{Date\}
\end{itemize}

The timestamp indicating the creation time of this file.

\paragraph{Stat time values}\label{stat-time-values}

The \texttt{atimeMs}, \texttt{mtimeMs}, \texttt{ctimeMs},
\texttt{birthtimeMs} properties are numeric values that hold the
corresponding times in milliseconds. Their precision is platform
specific. When \texttt{bigint:\ true} is passed into the method that
generates the object, the properties will be
\href{https://tc39.github.io/proposal-bigint}{bigints}, otherwise they
will be
\href{https://developer.mozilla.org/en-US/docs/Web/JavaScript/Data_structures\#Number_type}{numbers}.

The \texttt{atimeNs}, \texttt{mtimeNs}, \texttt{ctimeNs},
\texttt{birthtimeNs} properties are
\href{https://tc39.github.io/proposal-bigint}{bigints} that hold the
corresponding times in nanoseconds. They are only present when
\texttt{bigint:\ true} is passed into the method that generates the
object. Their precision is platform specific.

\texttt{atime}, \texttt{mtime}, \texttt{ctime}, and \texttt{birthtime}
are
\href{https://developer.mozilla.org/en-US/docs/Web/JavaScript/Reference/Global_Objects/Date}{\texttt{Date}}
object alternate representations of the various times. The \texttt{Date}
and number values are not connected. Assigning a new number value, or
mutating the \texttt{Date} value, will not be reflected in the
corresponding alternate representation.

The times in the stat object have the following semantics:

\begin{itemize}
\tightlist
\item
  \texttt{atime} ``Access Time'': Time when file data last accessed.
  Changed by the mknod(2), utimes(2), and read(2) system calls.
\item
  \texttt{mtime} ``Modified Time'': Time when file data last modified.
  Changed by the mknod(2), utimes(2), and write(2) system calls.
\item
  \texttt{ctime} ``Change Time'': Time when file status was last changed
  (inode data modification). Changed by the chmod(2), chown(2), link(2),
  mknod(2), rename(2), unlink(2), utimes(2), read(2), and write(2)
  system calls.
\item
  \texttt{birthtime} ``Birth Time'': Time of file creation. Set once
  when the file is created. On file systems where birthtime is not
  available, this field may instead hold either the \texttt{ctime} or
  \texttt{1970-01-01T00:00Z} (ie, Unix epoch timestamp \texttt{0}). This
  value may be greater than \texttt{atime} or \texttt{mtime} in this
  case. On Darwin and other FreeBSD variants, also set if the
  \texttt{atime} is explicitly set to an earlier value than the current
  \texttt{birthtime} using the utimes(2) system call.
\end{itemize}

Prior to Node.js 0.12, the \texttt{ctime} held the \texttt{birthtime} on
Windows systems. As of 0.12, \texttt{ctime} is not ``creation time'',
and on Unix systems, it never was.

\subsubsection{\texorpdfstring{Class:
\texttt{fs.StatFs}}{Class: fs.StatFs}}\label{class-fs.statfs}

Provides information about a mounted file system.

Objects returned from
\hyperref[fsstatfspath-options-callback]{\texttt{fs.statfs()}} and its
synchronous counterpart are of this type. If \texttt{bigint} in the
\texttt{options} passed to those methods is \texttt{true}, the numeric
values will be \texttt{bigint} instead of \texttt{number}.

\begin{Shaded}
\begin{Highlighting}[]
\NormalTok{StatFs \{}
\NormalTok{  type: 1397114950,}
\NormalTok{  bsize: 4096,}
\NormalTok{  blocks: 121938943,}
\NormalTok{  bfree: 61058895,}
\NormalTok{  bavail: 61058895,}
\NormalTok{  files: 999,}
\NormalTok{  ffree: 1000000}
\NormalTok{\}}
\end{Highlighting}
\end{Shaded}

\texttt{bigint} version:

\begin{Shaded}
\begin{Highlighting}[]
\NormalTok{StatFs \{}
\NormalTok{  type: 1397114950n,}
\NormalTok{  bsize: 4096n,}
\NormalTok{  blocks: 121938943n,}
\NormalTok{  bfree: 61058895n,}
\NormalTok{  bavail: 61058895n,}
\NormalTok{  files: 999n,}
\NormalTok{  ffree: 1000000n}
\NormalTok{\}}
\end{Highlighting}
\end{Shaded}

\paragraph{\texorpdfstring{\texttt{statfs.bavail}}{statfs.bavail}}\label{statfs.bavail}

\begin{itemize}
\tightlist
\item
  \{number\textbar bigint\}
\end{itemize}

Free blocks available to unprivileged users.

\paragraph{\texorpdfstring{\texttt{statfs.bfree}}{statfs.bfree}}\label{statfs.bfree}

\begin{itemize}
\tightlist
\item
  \{number\textbar bigint\}
\end{itemize}

Free blocks in file system.

\paragraph{\texorpdfstring{\texttt{statfs.blocks}}{statfs.blocks}}\label{statfs.blocks}

\begin{itemize}
\tightlist
\item
  \{number\textbar bigint\}
\end{itemize}

Total data blocks in file system.

\paragraph{\texorpdfstring{\texttt{statfs.bsize}}{statfs.bsize}}\label{statfs.bsize}

\begin{itemize}
\tightlist
\item
  \{number\textbar bigint\}
\end{itemize}

Optimal transfer block size.

\paragraph{\texorpdfstring{\texttt{statfs.ffree}}{statfs.ffree}}\label{statfs.ffree}

\begin{itemize}
\tightlist
\item
  \{number\textbar bigint\}
\end{itemize}

Free file nodes in file system.

\paragraph{\texorpdfstring{\texttt{statfs.files}}{statfs.files}}\label{statfs.files}

\begin{itemize}
\tightlist
\item
  \{number\textbar bigint\}
\end{itemize}

Total file nodes in file system.

\paragraph{\texorpdfstring{\texttt{statfs.type}}{statfs.type}}\label{statfs.type}

\begin{itemize}
\tightlist
\item
  \{number\textbar bigint\}
\end{itemize}

Type of file system.

\subsubsection{\texorpdfstring{Class:
\texttt{fs.WriteStream}}{Class: fs.WriteStream}}\label{class-fs.writestream}

\begin{itemize}
\tightlist
\item
  Extends \{stream.Writable\}
\end{itemize}

Instances of \{fs.WriteStream\} are created and returned using the
\hyperref[fscreatewritestreampath-options]{\texttt{fs.createWriteStream()}}
function.

\paragraph{\texorpdfstring{Event:
\texttt{\textquotesingle{}close\textquotesingle{}}}{Event: \textquotesingle close\textquotesingle{}}}\label{event-close-3}

Emitted when the \{fs.WriteStream\}'s underlying file descriptor has
been closed.

\paragraph{\texorpdfstring{Event:
\texttt{\textquotesingle{}open\textquotesingle{}}}{Event: \textquotesingle open\textquotesingle{}}}\label{event-open-1}

\begin{itemize}
\tightlist
\item
  \texttt{fd} \{integer\} Integer file descriptor used by the
  \{fs.WriteStream\}.
\end{itemize}

Emitted when the \{fs.WriteStream\}'s file is opened.

\paragraph{\texorpdfstring{Event:
\texttt{\textquotesingle{}ready\textquotesingle{}}}{Event: \textquotesingle ready\textquotesingle{}}}\label{event-ready-1}

Emitted when the \{fs.WriteStream\} is ready to be used.

Fires immediately after
\texttt{\textquotesingle{}open\textquotesingle{}}.

\paragraph{\texorpdfstring{\texttt{writeStream.bytesWritten}}{writeStream.bytesWritten}}\label{writestream.byteswritten}

The number of bytes written so far. Does not include data that is still
queued for writing.

\paragraph{\texorpdfstring{\texttt{writeStream.close({[}callback{]})}}{writeStream.close({[}callback{]})}}\label{writestream.closecallback}

\begin{itemize}
\tightlist
\item
  \texttt{callback} \{Function\}

  \begin{itemize}
  \tightlist
  \item
    \texttt{err} \{Error\}
  \end{itemize}
\end{itemize}

Closes \texttt{writeStream}. Optionally accepts a callback that will be
executed once the \texttt{writeStream} is closed.

\paragraph{\texorpdfstring{\texttt{writeStream.path}}{writeStream.path}}\label{writestream.path}

The path to the file the stream is writing to as specified in the first
argument to
\hyperref[fscreatewritestreampath-options]{\texttt{fs.createWriteStream()}}.
If \texttt{path} is passed as a string, then \texttt{writeStream.path}
will be a string. If \texttt{path} is passed as a \{Buffer\}, then
\texttt{writeStream.path} will be a \{Buffer\}.

\paragraph{\texorpdfstring{\texttt{writeStream.pending}}{writeStream.pending}}\label{writestream.pending}

\begin{itemize}
\tightlist
\item
  \{boolean\}
\end{itemize}

This property is \texttt{true} if the underlying file has not been
opened yet, i.e.~before the
\texttt{\textquotesingle{}ready\textquotesingle{}} event is emitted.

\subsubsection{\texorpdfstring{\texttt{fs.constants}}{fs.constants}}\label{fs.constants}

\begin{itemize}
\tightlist
\item
  \{Object\}
\end{itemize}

Returns an object containing commonly used constants for file system
operations.

\paragraph{FS constants}\label{fs-constants}

The following constants are exported by \texttt{fs.constants} and
\texttt{fsPromises.constants}.

Not every constant will be available on every operating system; this is
especially important for Windows, where many of the POSIX specific
definitions are not available. For portable applications it is
recommended to check for their presence before use.

To use more than one constant, use the bitwise OR \texttt{\textbar{}}
operator.

Example:

\begin{Shaded}
\begin{Highlighting}[]
\ImportTok{import}\NormalTok{ \{ open}\OperatorTok{,}\NormalTok{ constants \} }\ImportTok{from} \StringTok{\textquotesingle{}node:fs\textquotesingle{}}\OperatorTok{;}

\KeywordTok{const}\NormalTok{ \{}
\NormalTok{  O\_RDWR}\OperatorTok{,}
\NormalTok{  O\_CREAT}\OperatorTok{,}
\NormalTok{  O\_EXCL}\OperatorTok{,}
\NormalTok{\} }\OperatorTok{=}\NormalTok{ constants}\OperatorTok{;}

\FunctionTok{open}\NormalTok{(}\StringTok{\textquotesingle{}/path/to/my/file\textquotesingle{}}\OperatorTok{,}\NormalTok{ O\_RDWR }\OperatorTok{|}\NormalTok{ O\_CREAT }\OperatorTok{|}\NormalTok{ O\_EXCL}\OperatorTok{,}\NormalTok{ (err}\OperatorTok{,}\NormalTok{ fd) }\KeywordTok{=\textgreater{}}\NormalTok{ \{}
  \CommentTok{// ...}
\NormalTok{\})}\OperatorTok{;}
\end{Highlighting}
\end{Shaded}

\subparagraph{File access constants}\label{file-access-constants}

The following constants are meant for use as the \texttt{mode} parameter
passed to
\hyperref[fspromisesaccesspath-mode]{\texttt{fsPromises.access()}},
\hyperref[fsaccesspath-mode-callback]{\texttt{fs.access()}}, and
\hyperref[fsaccesssyncpath-mode]{\texttt{fs.accessSync()}}.

Constant

Description

F\_OK

Flag indicating that the file is visible to the calling process. This is
useful for determining if a file exists, but says nothing about rwx
permissions. Default if no mode is specified.

R\_OK

Flag indicating that the file can be read by the calling process.

W\_OK

Flag indicating that the file can be written by the calling process.

X\_OK

Flag indicating that the file can be executed by the calling process.
This has no effect on Windows (will behave like fs.constants.F\_OK).

The definitions are also available on Windows.

\subparagraph{File copy constants}\label{file-copy-constants}

The following constants are meant for use with
\hyperref[fscopyfilesrc-dest-mode-callback]{\texttt{fs.copyFile()}}.

Constant

Description

COPYFILE\_EXCL

If present, the copy operation will fail with an error if the
destination path already exists.

COPYFILE\_FICLONE

If present, the copy operation will attempt to create a copy-on-write
reflink. If the underlying platform does not support copy-on-write, then
a fallback copy mechanism is used.

COPYFILE\_FICLONE\_FORCE

If present, the copy operation will attempt to create a copy-on-write
reflink. If the underlying platform does not support copy-on-write, then
the operation will fail with an error.

The definitions are also available on Windows.

\subparagraph{File open constants}\label{file-open-constants}

The following constants are meant for use with \texttt{fs.open()}.

Constant

Description

O\_RDONLY

Flag indicating to open a file for read-only access.

O\_WRONLY

Flag indicating to open a file for write-only access.

O\_RDWR

Flag indicating to open a file for read-write access.

O\_CREAT

Flag indicating to create the file if it does not already exist.

O\_EXCL

Flag indicating that opening a file should fail if the O\_CREAT flag is
set and the file already exists.

O\_NOCTTY

Flag indicating that if path identifies a terminal device, opening the
path shall not cause that terminal to become the controlling terminal
for the process (if the process does not already have one).

O\_TRUNC

Flag indicating that if the file exists and is a regular file, and the
file is opened successfully for write access, its length shall be
truncated to zero.

O\_APPEND

Flag indicating that data will be appended to the end of the file.

O\_DIRECTORY

Flag indicating that the open should fail if the path is not a
directory.

O\_NOATIME

Flag indicating reading accesses to the file system will no longer
result in an update to the atime information associated with the file.
This flag is available on Linux operating systems only.

O\_NOFOLLOW

Flag indicating that the open should fail if the path is a symbolic
link.

O\_SYNC

Flag indicating that the file is opened for synchronized I/O with write
operations waiting for file integrity.

O\_DSYNC

Flag indicating that the file is opened for synchronized I/O with write
operations waiting for data integrity.

O\_SYMLINK

Flag indicating to open the symbolic link itself rather than the
resource it is pointing to.

O\_DIRECT

When set, an attempt will be made to minimize caching effects of file
I/O.

O\_NONBLOCK

Flag indicating to open the file in nonblocking mode when possible.

UV\_FS\_O\_FILEMAP

When set, a memory file mapping is used to access the file. This flag is
available on Windows operating systems only. On other operating systems,
this flag is ignored.

On Windows, only \texttt{O\_APPEND}, \texttt{O\_CREAT},
\texttt{O\_EXCL}, \texttt{O\_RDONLY}, \texttt{O\_RDWR},
\texttt{O\_TRUNC}, \texttt{O\_WRONLY}, and \texttt{UV\_FS\_O\_FILEMAP}
are available.

\subparagraph{File type constants}\label{file-type-constants}

The following constants are meant for use with the \{fs.Stats\} object's
\texttt{mode} property for determining a file's type.

Constant

Description

S\_IFMT

Bit mask used to extract the file type code.

S\_IFREG

File type constant for a regular file.

S\_IFDIR

File type constant for a directory.

S\_IFCHR

File type constant for a character-oriented device file.

S\_IFBLK

File type constant for a block-oriented device file.

S\_IFIFO

File type constant for a FIFO/pipe.

S\_IFLNK

File type constant for a symbolic link.

S\_IFSOCK

File type constant for a socket.

On Windows, only \texttt{S\_IFCHR}, \texttt{S\_IFDIR},
\texttt{S\_IFLNK}, \texttt{S\_IFMT}, and \texttt{S\_IFREG}, are
available.

\subparagraph{File mode constants}\label{file-mode-constants}

The following constants are meant for use with the \{fs.Stats\} object's
\texttt{mode} property for determining the access permissions for a
file.

Constant

Description

S\_IRWXU

File mode indicating readable, writable, and executable by owner.

S\_IRUSR

File mode indicating readable by owner.

S\_IWUSR

File mode indicating writable by owner.

S\_IXUSR

File mode indicating executable by owner.

S\_IRWXG

File mode indicating readable, writable, and executable by group.

S\_IRGRP

File mode indicating readable by group.

S\_IWGRP

File mode indicating writable by group.

S\_IXGRP

File mode indicating executable by group.

S\_IRWXO

File mode indicating readable, writable, and executable by others.

S\_IROTH

File mode indicating readable by others.

S\_IWOTH

File mode indicating writable by others.

S\_IXOTH

File mode indicating executable by others.

On Windows, only \texttt{S\_IRUSR} and \texttt{S\_IWUSR} are available.

\subsection{Notes}\label{notes}

\subsubsection{Ordering of callback and promise-based
operations}\label{ordering-of-callback-and-promise-based-operations}

Because they are executed asynchronously by the underlying thread pool,
there is no guaranteed ordering when using either the callback or
promise-based methods.

For example, the following is prone to error because the
\texttt{fs.stat()} operation might complete before the
\texttt{fs.rename()} operation:

\begin{Shaded}
\begin{Highlighting}[]
\KeywordTok{const}\NormalTok{ fs }\OperatorTok{=} \PreprocessorTok{require}\NormalTok{(}\StringTok{\textquotesingle{}node:fs\textquotesingle{}}\NormalTok{)}\OperatorTok{;}

\NormalTok{fs}\OperatorTok{.}\FunctionTok{rename}\NormalTok{(}\StringTok{\textquotesingle{}/tmp/hello\textquotesingle{}}\OperatorTok{,} \StringTok{\textquotesingle{}/tmp/world\textquotesingle{}}\OperatorTok{,}\NormalTok{ (err) }\KeywordTok{=\textgreater{}}\NormalTok{ \{}
  \ControlFlowTok{if}\NormalTok{ (err) }\ControlFlowTok{throw}\NormalTok{ err}\OperatorTok{;}
  \BuiltInTok{console}\OperatorTok{.}\FunctionTok{log}\NormalTok{(}\StringTok{\textquotesingle{}renamed complete\textquotesingle{}}\NormalTok{)}\OperatorTok{;}
\NormalTok{\})}\OperatorTok{;}
\NormalTok{fs}\OperatorTok{.}\FunctionTok{stat}\NormalTok{(}\StringTok{\textquotesingle{}/tmp/world\textquotesingle{}}\OperatorTok{,}\NormalTok{ (err}\OperatorTok{,}\NormalTok{ stats) }\KeywordTok{=\textgreater{}}\NormalTok{ \{}
  \ControlFlowTok{if}\NormalTok{ (err) }\ControlFlowTok{throw}\NormalTok{ err}\OperatorTok{;}
  \BuiltInTok{console}\OperatorTok{.}\FunctionTok{log}\NormalTok{(}\VerbatimStringTok{\textasciigrave{}stats: }\SpecialCharTok{$\{}\BuiltInTok{JSON}\OperatorTok{.}\FunctionTok{stringify}\NormalTok{(stats)}\SpecialCharTok{\}}\VerbatimStringTok{\textasciigrave{}}\NormalTok{)}\OperatorTok{;}
\NormalTok{\})}\OperatorTok{;}
\end{Highlighting}
\end{Shaded}

It is important to correctly order the operations by awaiting the
results of one before invoking the other:

\begin{Shaded}
\begin{Highlighting}[]
\ImportTok{import}\NormalTok{ \{ rename}\OperatorTok{,}\NormalTok{ stat \} }\ImportTok{from} \StringTok{\textquotesingle{}node:fs/promises\textquotesingle{}}\OperatorTok{;}

\KeywordTok{const}\NormalTok{ oldPath }\OperatorTok{=} \StringTok{\textquotesingle{}/tmp/hello\textquotesingle{}}\OperatorTok{;}
\KeywordTok{const}\NormalTok{ newPath }\OperatorTok{=} \StringTok{\textquotesingle{}/tmp/world\textquotesingle{}}\OperatorTok{;}

\ControlFlowTok{try}\NormalTok{ \{}
  \ControlFlowTok{await} \FunctionTok{rename}\NormalTok{(oldPath}\OperatorTok{,}\NormalTok{ newPath)}\OperatorTok{;}
  \KeywordTok{const}\NormalTok{ stats }\OperatorTok{=} \ControlFlowTok{await} \FunctionTok{stat}\NormalTok{(newPath)}\OperatorTok{;}
  \BuiltInTok{console}\OperatorTok{.}\FunctionTok{log}\NormalTok{(}\VerbatimStringTok{\textasciigrave{}stats: }\SpecialCharTok{$\{}\BuiltInTok{JSON}\OperatorTok{.}\FunctionTok{stringify}\NormalTok{(stats)}\SpecialCharTok{\}}\VerbatimStringTok{\textasciigrave{}}\NormalTok{)}\OperatorTok{;}
\NormalTok{\} }\ControlFlowTok{catch}\NormalTok{ (error) \{}
  \BuiltInTok{console}\OperatorTok{.}\FunctionTok{error}\NormalTok{(}\StringTok{\textquotesingle{}there was an error:\textquotesingle{}}\OperatorTok{,}\NormalTok{ error}\OperatorTok{.}\AttributeTok{message}\NormalTok{)}\OperatorTok{;}
\NormalTok{\}}
\end{Highlighting}
\end{Shaded}

\begin{Shaded}
\begin{Highlighting}[]
\KeywordTok{const}\NormalTok{ \{ rename}\OperatorTok{,}\NormalTok{ stat \} }\OperatorTok{=} \PreprocessorTok{require}\NormalTok{(}\StringTok{\textquotesingle{}node:fs/promises\textquotesingle{}}\NormalTok{)}\OperatorTok{;}

\NormalTok{(}\KeywordTok{async} \KeywordTok{function}\NormalTok{(oldPath}\OperatorTok{,}\NormalTok{ newPath) \{}
  \ControlFlowTok{try}\NormalTok{ \{}
    \ControlFlowTok{await} \FunctionTok{rename}\NormalTok{(oldPath}\OperatorTok{,}\NormalTok{ newPath)}\OperatorTok{;}
    \KeywordTok{const}\NormalTok{ stats }\OperatorTok{=} \ControlFlowTok{await} \FunctionTok{stat}\NormalTok{(newPath)}\OperatorTok{;}
    \BuiltInTok{console}\OperatorTok{.}\FunctionTok{log}\NormalTok{(}\VerbatimStringTok{\textasciigrave{}stats: }\SpecialCharTok{$\{}\BuiltInTok{JSON}\OperatorTok{.}\FunctionTok{stringify}\NormalTok{(stats)}\SpecialCharTok{\}}\VerbatimStringTok{\textasciigrave{}}\NormalTok{)}\OperatorTok{;}
\NormalTok{  \} }\ControlFlowTok{catch}\NormalTok{ (error) \{}
    \BuiltInTok{console}\OperatorTok{.}\FunctionTok{error}\NormalTok{(}\StringTok{\textquotesingle{}there was an error:\textquotesingle{}}\OperatorTok{,}\NormalTok{ error}\OperatorTok{.}\AttributeTok{message}\NormalTok{)}\OperatorTok{;}
\NormalTok{  \}}
\NormalTok{\})(}\StringTok{\textquotesingle{}/tmp/hello\textquotesingle{}}\OperatorTok{,} \StringTok{\textquotesingle{}/tmp/world\textquotesingle{}}\NormalTok{)}\OperatorTok{;}
\end{Highlighting}
\end{Shaded}

Or, when using the callback APIs, move the \texttt{fs.stat()} call into
the callback of the \texttt{fs.rename()} operation:

\begin{Shaded}
\begin{Highlighting}[]
\ImportTok{import}\NormalTok{ \{ rename}\OperatorTok{,}\NormalTok{ stat \} }\ImportTok{from} \StringTok{\textquotesingle{}node:fs\textquotesingle{}}\OperatorTok{;}

\FunctionTok{rename}\NormalTok{(}\StringTok{\textquotesingle{}/tmp/hello\textquotesingle{}}\OperatorTok{,} \StringTok{\textquotesingle{}/tmp/world\textquotesingle{}}\OperatorTok{,}\NormalTok{ (err) }\KeywordTok{=\textgreater{}}\NormalTok{ \{}
  \ControlFlowTok{if}\NormalTok{ (err) }\ControlFlowTok{throw}\NormalTok{ err}\OperatorTok{;}
  \FunctionTok{stat}\NormalTok{(}\StringTok{\textquotesingle{}/tmp/world\textquotesingle{}}\OperatorTok{,}\NormalTok{ (err}\OperatorTok{,}\NormalTok{ stats) }\KeywordTok{=\textgreater{}}\NormalTok{ \{}
    \ControlFlowTok{if}\NormalTok{ (err) }\ControlFlowTok{throw}\NormalTok{ err}\OperatorTok{;}
    \BuiltInTok{console}\OperatorTok{.}\FunctionTok{log}\NormalTok{(}\VerbatimStringTok{\textasciigrave{}stats: }\SpecialCharTok{$\{}\BuiltInTok{JSON}\OperatorTok{.}\FunctionTok{stringify}\NormalTok{(stats)}\SpecialCharTok{\}}\VerbatimStringTok{\textasciigrave{}}\NormalTok{)}\OperatorTok{;}
\NormalTok{  \})}\OperatorTok{;}
\NormalTok{\})}\OperatorTok{;}
\end{Highlighting}
\end{Shaded}

\begin{Shaded}
\begin{Highlighting}[]
\KeywordTok{const}\NormalTok{ \{ rename}\OperatorTok{,}\NormalTok{ stat \} }\OperatorTok{=} \PreprocessorTok{require}\NormalTok{(}\StringTok{\textquotesingle{}node:fs/promises\textquotesingle{}}\NormalTok{)}\OperatorTok{;}

\FunctionTok{rename}\NormalTok{(}\StringTok{\textquotesingle{}/tmp/hello\textquotesingle{}}\OperatorTok{,} \StringTok{\textquotesingle{}/tmp/world\textquotesingle{}}\OperatorTok{,}\NormalTok{ (err) }\KeywordTok{=\textgreater{}}\NormalTok{ \{}
  \ControlFlowTok{if}\NormalTok{ (err) }\ControlFlowTok{throw}\NormalTok{ err}\OperatorTok{;}
  \FunctionTok{stat}\NormalTok{(}\StringTok{\textquotesingle{}/tmp/world\textquotesingle{}}\OperatorTok{,}\NormalTok{ (err}\OperatorTok{,}\NormalTok{ stats) }\KeywordTok{=\textgreater{}}\NormalTok{ \{}
    \ControlFlowTok{if}\NormalTok{ (err) }\ControlFlowTok{throw}\NormalTok{ err}\OperatorTok{;}
    \BuiltInTok{console}\OperatorTok{.}\FunctionTok{log}\NormalTok{(}\VerbatimStringTok{\textasciigrave{}stats: }\SpecialCharTok{$\{}\BuiltInTok{JSON}\OperatorTok{.}\FunctionTok{stringify}\NormalTok{(stats)}\SpecialCharTok{\}}\VerbatimStringTok{\textasciigrave{}}\NormalTok{)}\OperatorTok{;}
\NormalTok{  \})}\OperatorTok{;}
\NormalTok{\})}\OperatorTok{;}
\end{Highlighting}
\end{Shaded}

\subsubsection{File paths}\label{file-paths}

Most \texttt{fs} operations accept file paths that may be specified in
the form of a string, a \{Buffer\}, or a \{URL\} object using the
\texttt{file:} protocol.

\paragraph{String paths}\label{string-paths}

String paths are interpreted as UTF-8 character sequences identifying
the absolute or relative filename. Relative paths will be resolved
relative to the current working directory as determined by calling
\texttt{process.cwd()}.

Example using an absolute path on POSIX:

\begin{Shaded}
\begin{Highlighting}[]
\ImportTok{import}\NormalTok{ \{ open \} }\ImportTok{from} \StringTok{\textquotesingle{}node:fs/promises\textquotesingle{}}\OperatorTok{;}

\KeywordTok{let}\NormalTok{ fd}\OperatorTok{;}
\ControlFlowTok{try}\NormalTok{ \{}
\NormalTok{  fd }\OperatorTok{=} \ControlFlowTok{await} \FunctionTok{open}\NormalTok{(}\StringTok{\textquotesingle{}/open/some/file.txt\textquotesingle{}}\OperatorTok{,} \StringTok{\textquotesingle{}r\textquotesingle{}}\NormalTok{)}\OperatorTok{;}
  \CommentTok{// Do something with the file}
\NormalTok{\} }\ControlFlowTok{finally}\NormalTok{ \{}
  \ControlFlowTok{await}\NormalTok{ fd}\OperatorTok{?.}\FunctionTok{close}\NormalTok{()}\OperatorTok{;}
\NormalTok{\}}
\end{Highlighting}
\end{Shaded}

Example using a relative path on POSIX (relative to
\texttt{process.cwd()}):

\begin{Shaded}
\begin{Highlighting}[]
\ImportTok{import}\NormalTok{ \{ open \} }\ImportTok{from} \StringTok{\textquotesingle{}node:fs/promises\textquotesingle{}}\OperatorTok{;}

\KeywordTok{let}\NormalTok{ fd}\OperatorTok{;}
\ControlFlowTok{try}\NormalTok{ \{}
\NormalTok{  fd }\OperatorTok{=} \ControlFlowTok{await} \FunctionTok{open}\NormalTok{(}\StringTok{\textquotesingle{}file.txt\textquotesingle{}}\OperatorTok{,} \StringTok{\textquotesingle{}r\textquotesingle{}}\NormalTok{)}\OperatorTok{;}
  \CommentTok{// Do something with the file}
\NormalTok{\} }\ControlFlowTok{finally}\NormalTok{ \{}
  \ControlFlowTok{await}\NormalTok{ fd}\OperatorTok{?.}\FunctionTok{close}\NormalTok{()}\OperatorTok{;}
\NormalTok{\}}
\end{Highlighting}
\end{Shaded}

\paragraph{File URL paths}\label{file-url-paths}

For most \texttt{node:fs} module functions, the \texttt{path} or
\texttt{filename} argument may be passed as a \{URL\} object using the
\texttt{file:} protocol.

\begin{Shaded}
\begin{Highlighting}[]
\ImportTok{import}\NormalTok{ \{ readFileSync \} }\ImportTok{from} \StringTok{\textquotesingle{}node:fs\textquotesingle{}}\OperatorTok{;}

\FunctionTok{readFileSync}\NormalTok{(}\KeywordTok{new} \FunctionTok{URL}\NormalTok{(}\StringTok{\textquotesingle{}file:///tmp/hello\textquotesingle{}}\NormalTok{))}\OperatorTok{;}
\end{Highlighting}
\end{Shaded}

\texttt{file:} URLs are always absolute paths.

\subparagraph{Platform-specific
considerations}\label{platform-specific-considerations}

On Windows, \texttt{file:} \{URL\}s with a host name convert to UNC
paths, while \texttt{file:} \{URL\}s with drive letters convert to local
absolute paths. \texttt{file:} \{URL\}s with no host name and no drive
letter will result in an error:

\begin{Shaded}
\begin{Highlighting}[]
\ImportTok{import}\NormalTok{ \{ readFileSync \} }\ImportTok{from} \StringTok{\textquotesingle{}node:fs\textquotesingle{}}\OperatorTok{;}
\CommentTok{// On Windows :}

\CommentTok{// {-} WHATWG file URLs with hostname convert to UNC path}
\CommentTok{// file://hostname/p/a/t/h/file =\textgreater{} \textbackslash{}\textbackslash{}hostname\textbackslash{}p\textbackslash{}a\textbackslash{}t\textbackslash{}h\textbackslash{}file}
\FunctionTok{readFileSync}\NormalTok{(}\KeywordTok{new} \FunctionTok{URL}\NormalTok{(}\StringTok{\textquotesingle{}file://hostname/p/a/t/h/file\textquotesingle{}}\NormalTok{))}\OperatorTok{;}

\CommentTok{// {-} WHATWG file URLs with drive letters convert to absolute path}
\CommentTok{// file:///C:/tmp/hello =\textgreater{} C:\textbackslash{}tmp\textbackslash{}hello}
\FunctionTok{readFileSync}\NormalTok{(}\KeywordTok{new} \FunctionTok{URL}\NormalTok{(}\StringTok{\textquotesingle{}file:///C:/tmp/hello\textquotesingle{}}\NormalTok{))}\OperatorTok{;}

\CommentTok{// {-} WHATWG file URLs without hostname must have a drive letters}
\FunctionTok{readFileSync}\NormalTok{(}\KeywordTok{new} \FunctionTok{URL}\NormalTok{(}\StringTok{\textquotesingle{}file:///notdriveletter/p/a/t/h/file\textquotesingle{}}\NormalTok{))}\OperatorTok{;}
\FunctionTok{readFileSync}\NormalTok{(}\KeywordTok{new} \FunctionTok{URL}\NormalTok{(}\StringTok{\textquotesingle{}file:///c/p/a/t/h/file\textquotesingle{}}\NormalTok{))}\OperatorTok{;}
\CommentTok{// TypeError [ERR\_INVALID\_FILE\_URL\_PATH]: File URL path must be absolute}
\end{Highlighting}
\end{Shaded}

\texttt{file:} \{URL\}s with drive letters must use \texttt{:} as a
separator just after the drive letter. Using another separator will
result in an error.

On all other platforms, \texttt{file:} \{URL\}s with a host name are
unsupported and will result in an error:

\begin{Shaded}
\begin{Highlighting}[]
\ImportTok{import}\NormalTok{ \{ readFileSync \} }\ImportTok{from} \StringTok{\textquotesingle{}node:fs\textquotesingle{}}\OperatorTok{;}
\CommentTok{// On other platforms:}

\CommentTok{// {-} WHATWG file URLs with hostname are unsupported}
\CommentTok{// file://hostname/p/a/t/h/file =\textgreater{} throw!}
\FunctionTok{readFileSync}\NormalTok{(}\KeywordTok{new} \FunctionTok{URL}\NormalTok{(}\StringTok{\textquotesingle{}file://hostname/p/a/t/h/file\textquotesingle{}}\NormalTok{))}\OperatorTok{;}
\CommentTok{// TypeError [ERR\_INVALID\_FILE\_URL\_PATH]: must be absolute}

\CommentTok{// {-} WHATWG file URLs convert to absolute path}
\CommentTok{// file:///tmp/hello =\textgreater{} /tmp/hello}
\FunctionTok{readFileSync}\NormalTok{(}\KeywordTok{new} \FunctionTok{URL}\NormalTok{(}\StringTok{\textquotesingle{}file:///tmp/hello\textquotesingle{}}\NormalTok{))}\OperatorTok{;}
\end{Highlighting}
\end{Shaded}

A \texttt{file:} \{URL\} having encoded slash characters will result in
an error on all platforms:

\begin{Shaded}
\begin{Highlighting}[]
\ImportTok{import}\NormalTok{ \{ readFileSync \} }\ImportTok{from} \StringTok{\textquotesingle{}node:fs\textquotesingle{}}\OperatorTok{;}

\CommentTok{// On Windows}
\FunctionTok{readFileSync}\NormalTok{(}\KeywordTok{new} \FunctionTok{URL}\NormalTok{(}\StringTok{\textquotesingle{}file:///C:/p/a/t/h/\%2F\textquotesingle{}}\NormalTok{))}\OperatorTok{;}
\FunctionTok{readFileSync}\NormalTok{(}\KeywordTok{new} \FunctionTok{URL}\NormalTok{(}\StringTok{\textquotesingle{}file:///C:/p/a/t/h/\%2f\textquotesingle{}}\NormalTok{))}\OperatorTok{;}
\CommentTok{/* TypeError [ERR\_INVALID\_FILE\_URL\_PATH]: File URL path must not include encoded}
\CommentTok{\textbackslash{} or / characters */}

\CommentTok{// On POSIX}
\FunctionTok{readFileSync}\NormalTok{(}\KeywordTok{new} \FunctionTok{URL}\NormalTok{(}\StringTok{\textquotesingle{}file:///p/a/t/h/\%2F\textquotesingle{}}\NormalTok{))}\OperatorTok{;}
\FunctionTok{readFileSync}\NormalTok{(}\KeywordTok{new} \FunctionTok{URL}\NormalTok{(}\StringTok{\textquotesingle{}file:///p/a/t/h/\%2f\textquotesingle{}}\NormalTok{))}\OperatorTok{;}
\CommentTok{/* TypeError [ERR\_INVALID\_FILE\_URL\_PATH]: File URL path must not include encoded}
\CommentTok{/ characters */}
\end{Highlighting}
\end{Shaded}

On Windows, \texttt{file:} \{URL\}s having encoded backslash will result
in an error:

\begin{Shaded}
\begin{Highlighting}[]
\ImportTok{import}\NormalTok{ \{ readFileSync \} }\ImportTok{from} \StringTok{\textquotesingle{}node:fs\textquotesingle{}}\OperatorTok{;}

\CommentTok{// On Windows}
\FunctionTok{readFileSync}\NormalTok{(}\KeywordTok{new} \FunctionTok{URL}\NormalTok{(}\StringTok{\textquotesingle{}file:///C:/path/\%5C\textquotesingle{}}\NormalTok{))}\OperatorTok{;}
\FunctionTok{readFileSync}\NormalTok{(}\KeywordTok{new} \FunctionTok{URL}\NormalTok{(}\StringTok{\textquotesingle{}file:///C:/path/\%5c\textquotesingle{}}\NormalTok{))}\OperatorTok{;}
\CommentTok{/* TypeError [ERR\_INVALID\_FILE\_URL\_PATH]: File URL path must not include encoded}
\CommentTok{\textbackslash{} or / characters */}
\end{Highlighting}
\end{Shaded}

\paragraph{Buffer paths}\label{buffer-paths}

Paths specified using a \{Buffer\} are useful primarily on certain POSIX
operating systems that treat file paths as opaque byte sequences. On
such systems, it is possible for a single file path to contain
sub-sequences that use multiple character encodings. As with string
paths, \{Buffer\} paths may be relative or absolute:

Example using an absolute path on POSIX:

\begin{Shaded}
\begin{Highlighting}[]
\ImportTok{import}\NormalTok{ \{ open \} }\ImportTok{from} \StringTok{\textquotesingle{}node:fs/promises\textquotesingle{}}\OperatorTok{;}
\ImportTok{import}\NormalTok{ \{ }\BuiltInTok{Buffer}\NormalTok{ \} }\ImportTok{from} \StringTok{\textquotesingle{}node:buffer\textquotesingle{}}\OperatorTok{;}

\KeywordTok{let}\NormalTok{ fd}\OperatorTok{;}
\ControlFlowTok{try}\NormalTok{ \{}
\NormalTok{  fd }\OperatorTok{=} \ControlFlowTok{await} \FunctionTok{open}\NormalTok{(}\BuiltInTok{Buffer}\OperatorTok{.}\FunctionTok{from}\NormalTok{(}\StringTok{\textquotesingle{}/open/some/file.txt\textquotesingle{}}\NormalTok{)}\OperatorTok{,} \StringTok{\textquotesingle{}r\textquotesingle{}}\NormalTok{)}\OperatorTok{;}
  \CommentTok{// Do something with the file}
\NormalTok{\} }\ControlFlowTok{finally}\NormalTok{ \{}
  \ControlFlowTok{await}\NormalTok{ fd}\OperatorTok{?.}\FunctionTok{close}\NormalTok{()}\OperatorTok{;}
\NormalTok{\}}
\end{Highlighting}
\end{Shaded}

\paragraph{Per-drive working directories on
Windows}\label{per-drive-working-directories-on-windows}

On Windows, Node.js follows the concept of per-drive working directory.
This behavior can be observed when using a drive path without a
backslash. For example
\texttt{fs.readdirSync(\textquotesingle{}C:\textbackslash{}\textbackslash{}\textquotesingle{})}
can potentially return a different result than
\texttt{fs.readdirSync(\textquotesingle{}C:\textquotesingle{})}. For
more information, see
\href{https://docs.microsoft.com/en-us/windows/desktop/FileIO/naming-a-file\#fully-qualified-vs-relative-paths}{this
MSDN page}.

\subsubsection{File descriptors}\label{file-descriptors-1}

On POSIX systems, for every process, the kernel maintains a table of
currently open files and resources. Each open file is assigned a simple
numeric identifier called a \emph{file descriptor}. At the system-level,
all file system operations use these file descriptors to identify and
track each specific file. Windows systems use a different but
conceptually similar mechanism for tracking resources. To simplify
things for users, Node.js abstracts away the differences between
operating systems and assigns all open files a numeric file descriptor.

The callback-based \texttt{fs.open()}, and synchronous
\texttt{fs.openSync()} methods open a file and allocate a new file
descriptor. Once allocated, the file descriptor may be used to read data
from, write data to, or request information about the file.

Operating systems limit the number of file descriptors that may be open
at any given time so it is critical to close the descriptor when
operations are completed. Failure to do so will result in a memory leak
that will eventually cause an application to crash.

\begin{Shaded}
\begin{Highlighting}[]
\ImportTok{import}\NormalTok{ \{ open}\OperatorTok{,}\NormalTok{ close}\OperatorTok{,}\NormalTok{ fstat \} }\ImportTok{from} \StringTok{\textquotesingle{}node:fs\textquotesingle{}}\OperatorTok{;}

\KeywordTok{function} \FunctionTok{closeFd}\NormalTok{(fd) \{}
  \FunctionTok{close}\NormalTok{(fd}\OperatorTok{,}\NormalTok{ (err) }\KeywordTok{=\textgreater{}}\NormalTok{ \{}
    \ControlFlowTok{if}\NormalTok{ (err) }\ControlFlowTok{throw}\NormalTok{ err}\OperatorTok{;}
\NormalTok{  \})}\OperatorTok{;}
\NormalTok{\}}

\FunctionTok{open}\NormalTok{(}\StringTok{\textquotesingle{}/open/some/file.txt\textquotesingle{}}\OperatorTok{,} \StringTok{\textquotesingle{}r\textquotesingle{}}\OperatorTok{,}\NormalTok{ (err}\OperatorTok{,}\NormalTok{ fd) }\KeywordTok{=\textgreater{}}\NormalTok{ \{}
  \ControlFlowTok{if}\NormalTok{ (err) }\ControlFlowTok{throw}\NormalTok{ err}\OperatorTok{;}
  \ControlFlowTok{try}\NormalTok{ \{}
    \FunctionTok{fstat}\NormalTok{(fd}\OperatorTok{,}\NormalTok{ (err}\OperatorTok{,}\NormalTok{ stat) }\KeywordTok{=\textgreater{}}\NormalTok{ \{}
      \ControlFlowTok{if}\NormalTok{ (err) \{}
        \FunctionTok{closeFd}\NormalTok{(fd)}\OperatorTok{;}
        \ControlFlowTok{throw}\NormalTok{ err}\OperatorTok{;}
\NormalTok{      \}}

      \CommentTok{// use stat}

      \FunctionTok{closeFd}\NormalTok{(fd)}\OperatorTok{;}
\NormalTok{    \})}\OperatorTok{;}
\NormalTok{  \} }\ControlFlowTok{catch}\NormalTok{ (err) \{}
    \FunctionTok{closeFd}\NormalTok{(fd)}\OperatorTok{;}
    \ControlFlowTok{throw}\NormalTok{ err}\OperatorTok{;}
\NormalTok{  \}}
\NormalTok{\})}\OperatorTok{;}
\end{Highlighting}
\end{Shaded}

The promise-based APIs use a \{FileHandle\} object in place of the
numeric file descriptor. These objects are better managed by the system
to ensure that resources are not leaked. However, it is still required
that they are closed when operations are completed:

\begin{Shaded}
\begin{Highlighting}[]
\ImportTok{import}\NormalTok{ \{ open \} }\ImportTok{from} \StringTok{\textquotesingle{}node:fs/promises\textquotesingle{}}\OperatorTok{;}

\KeywordTok{let}\NormalTok{ file}\OperatorTok{;}
\ControlFlowTok{try}\NormalTok{ \{}
\NormalTok{  file }\OperatorTok{=} \ControlFlowTok{await} \FunctionTok{open}\NormalTok{(}\StringTok{\textquotesingle{}/open/some/file.txt\textquotesingle{}}\OperatorTok{,} \StringTok{\textquotesingle{}r\textquotesingle{}}\NormalTok{)}\OperatorTok{;}
  \KeywordTok{const}\NormalTok{ stat }\OperatorTok{=} \ControlFlowTok{await}\NormalTok{ file}\OperatorTok{.}\FunctionTok{stat}\NormalTok{()}\OperatorTok{;}
  \CommentTok{// use stat}
\NormalTok{\} }\ControlFlowTok{finally}\NormalTok{ \{}
  \ControlFlowTok{await}\NormalTok{ file}\OperatorTok{.}\FunctionTok{close}\NormalTok{()}\OperatorTok{;}
\NormalTok{\}}
\end{Highlighting}
\end{Shaded}

\subsubsection{Threadpool usage}\label{threadpool-usage}

All callback and promise-based file system APIs (with the exception of
\texttt{fs.FSWatcher()}) use libuv's threadpool. This can have
surprising and negative performance implications for some applications.
See the
\href{cli.md\#uv_threadpool_sizesize}{\texttt{UV\_THREADPOOL\_SIZE}}
documentation for more information.

\subsubsection{File system flags}\label{file-system-flags}

The following flags are available wherever the \texttt{flag} option
takes a string.

\begin{itemize}
\item
  \texttt{\textquotesingle{}a\textquotesingle{}}: Open file for
  appending. The file is created if it does not exist.
\item
  \texttt{\textquotesingle{}ax\textquotesingle{}}: Like
  \texttt{\textquotesingle{}a\textquotesingle{}} but fails if the path
  exists.
\item
  \texttt{\textquotesingle{}a+\textquotesingle{}}: Open file for reading
  and appending. The file is created if it does not exist.
\item
  \texttt{\textquotesingle{}ax+\textquotesingle{}}: Like
  \texttt{\textquotesingle{}a+\textquotesingle{}} but fails if the path
  exists.
\item
  \texttt{\textquotesingle{}as\textquotesingle{}}: Open file for
  appending in synchronous mode. The file is created if it does not
  exist.
\item
  \texttt{\textquotesingle{}as+\textquotesingle{}}: Open file for
  reading and appending in synchronous mode. The file is created if it
  does not exist.
\item
  \texttt{\textquotesingle{}r\textquotesingle{}}: Open file for reading.
  An exception occurs if the file does not exist.
\item
  \texttt{\textquotesingle{}rs\textquotesingle{}}: Open file for reading
  in synchronous mode. An exception occurs if the file does not exist.
\item
  \texttt{\textquotesingle{}r+\textquotesingle{}}: Open file for reading
  and writing. An exception occurs if the file does not exist.
\item
  \texttt{\textquotesingle{}rs+\textquotesingle{}}: Open file for
  reading and writing in synchronous mode. Instructs the operating
  system to bypass the local file system cache.

  This is primarily useful for opening files on NFS mounts as it allows
  skipping the potentially stale local cache. It has a very real impact
  on I/O performance so using this flag is not recommended unless it is
  needed.

  This doesn't turn \texttt{fs.open()} or \texttt{fsPromises.open()}
  into a synchronous blocking call. If synchronous operation is desired,
  something like \texttt{fs.openSync()} should be used.
\item
  \texttt{\textquotesingle{}w\textquotesingle{}}: Open file for writing.
  The file is created (if it does not exist) or truncated (if it
  exists).
\item
  \texttt{\textquotesingle{}wx\textquotesingle{}}: Like
  \texttt{\textquotesingle{}w\textquotesingle{}} but fails if the path
  exists.
\item
  \texttt{\textquotesingle{}w+\textquotesingle{}}: Open file for reading
  and writing. The file is created (if it does not exist) or truncated
  (if it exists).
\item
  \texttt{\textquotesingle{}wx+\textquotesingle{}}: Like
  \texttt{\textquotesingle{}w+\textquotesingle{}} but fails if the path
  exists.
\end{itemize}

\texttt{flag} can also be a number as documented by open(2); commonly
used constants are available from \texttt{fs.constants}. On Windows,
flags are translated to their equivalent ones where applicable,
e.g.~\texttt{O\_WRONLY} to \texttt{FILE\_GENERIC\_WRITE}, or
\texttt{O\_EXCL\textbar{}O\_CREAT} to \texttt{CREATE\_NEW}, as accepted
by \texttt{CreateFileW}.

The exclusive flag \texttt{\textquotesingle{}x\textquotesingle{}}
(\texttt{O\_EXCL} flag in open(2)) causes the operation to return an
error if the path already exists. On POSIX, if the path is a symbolic
link, using \texttt{O\_EXCL} returns an error even if the link is to a
path that does not exist. The exclusive flag might not work with network
file systems.

On Linux, positional writes don't work when the file is opened in append
mode. The kernel ignores the position argument and always appends the
data to the end of the file.

Modifying a file rather than replacing it may require the \texttt{flag}
option to be set to \texttt{\textquotesingle{}r+\textquotesingle{}}
rather than the default \texttt{\textquotesingle{}w\textquotesingle{}}.

The behavior of some flags are platform-specific. As such, opening a
directory on macOS and Linux with the
\texttt{\textquotesingle{}a+\textquotesingle{}} flag, as in the example
below, will return an error. In contrast, on Windows and FreeBSD, a file
descriptor or a \texttt{FileHandle} will be returned.

\begin{Shaded}
\begin{Highlighting}[]
\CommentTok{// macOS and Linux}
\NormalTok{fs}\OperatorTok{.}\FunctionTok{open}\NormalTok{(}\StringTok{\textquotesingle{}\textless{}directory\textgreater{}\textquotesingle{}}\OperatorTok{,} \StringTok{\textquotesingle{}a+\textquotesingle{}}\OperatorTok{,}\NormalTok{ (err}\OperatorTok{,}\NormalTok{ fd) }\KeywordTok{=\textgreater{}}\NormalTok{ \{}
  \CommentTok{// =\textgreater{} [Error: EISDIR: illegal operation on a directory, open \textless{}directory\textgreater{}]}
\NormalTok{\})}\OperatorTok{;}

\CommentTok{// Windows and FreeBSD}
\NormalTok{fs}\OperatorTok{.}\FunctionTok{open}\NormalTok{(}\StringTok{\textquotesingle{}\textless{}directory\textgreater{}\textquotesingle{}}\OperatorTok{,} \StringTok{\textquotesingle{}a+\textquotesingle{}}\OperatorTok{,}\NormalTok{ (err}\OperatorTok{,}\NormalTok{ fd) }\KeywordTok{=\textgreater{}}\NormalTok{ \{}
  \CommentTok{// =\textgreater{} null, \textless{}fd\textgreater{}}
\NormalTok{\})}\OperatorTok{;}
\end{Highlighting}
\end{Shaded}

On Windows, opening an existing hidden file using the
\texttt{\textquotesingle{}w\textquotesingle{}} flag (either through
\texttt{fs.open()}, \texttt{fs.writeFile()}, or
\texttt{fsPromises.open()}) will fail with \texttt{EPERM}. Existing
hidden files can be opened for writing with the
\texttt{\textquotesingle{}r+\textquotesingle{}} flag.

A call to \texttt{fs.ftruncate()} or \texttt{filehandle.truncate()} can
be used to reset the file contents.
